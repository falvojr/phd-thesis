\section{Considerações Iniciais}

A fundamentação teórica deste doutorado tem como base um MS, cujo objetivo foi identificar a interseção entre as TICs e as línguas de sinais no contexto educacional. A \autoref{section:foundation:sm} oferece um resumo deste MS, enquanto análises mais detalhadas dos estudos primários, que abrangem tanto o panorama nacional quanto internacional, estão disponíveis em uma série de publicações \cite{FalvoJr2020_FIE, FalvoJr2020_SBIE, FalvoJr2021_RENOTE}, que foram essenciais para o desenvolvimento do trabalho.

A realização deste estudo permitiu identificar \textit{gaps} tecnológicos no uso de línguas de sinais no ensino-aprendizagem, destacando a necessidade de novas pesquisas na literatura. De modo complementar ao MS, um levantamento bibliográfico foi conduzido para explorar conceitos e TICs promissoras que possam enfrentar esses desafios. As próximas seções aprofundam as temáticas de Arquiteturas de Software, OAs e ASR. Esses tópicos fornecem a base teórica para a \textit{Speech2Learning}, uma arquitetura detalhada no \autoref{chapter3}, projetada para tornar OAs audíveis mais acessíveis por meio de ASR.

\section{Mapeamento Sistemático: TICs e Línguas de Sinais na Educação}
\label{section:foundation:sm}

Para definir o escopo deste projeto, foi realizado um estudo sistemático de literatura para identificar lacunas e oportunidades tecnológicas no processo de ensino-aprendizagem com línguas de sinais. De acordo com \citeonline{Kitchenham2007}, existem duas abordagens principais para este tipo de estudo: revisão ou mapeamento sistemático. Optou-se pelo MS devido à sua capacidade de apresentar evidências de um domínio de estudo em um alto nível de granularidade, agrupando-as em áreas de similaridade e identificando tendências emergentes.

O protocolo de pesquisa para o MS foi cuidadosamente definido com base em diretrizes formais  \cite{Kitchenham2007, Nakagawa2010, Zhang2011, Petersen2015}. A abordagem de \citeonline{Zhang2011} foi particularmente relevante, pois orientou a estratégia de busca e os critérios de qualidade adotados no estudo. Essa estratégia foi adaptada para aumentar o rigor do processo de pesquisa, incorporando o \sigla{QGS}{\textit{Quasi-Gold Standard}} e seguindo boas práticas recomendadas na literatura (\autoref{ms:zhang-approach}).

\begin{figure}[htb]
\centering 
\caption{Busca Sistemática Baseada em QGS.}
\label{ms:zhang-approach}
\includegraphics[width=0.675\textwidth]{images/chapter2-sm-zhang-approach.png}
\fadaptada{Zhang2011}
\end{figure}

\subsection{Definição do Escopo e Critérios de Seleção}
\label{ms:conducao-escopo}

As \sigla{QP}{Questões de Pesquisa} são essenciais para definir o escopo e identificar possíveis palavras-chave em um estudo sistemático de literatura \cite{Kitchenham2007,Petersen2015}. Neste contexto, uma abordagem comum se dá através da aplicação dos critérios de PICO \cite{Petticrew2008}. O \autoref{quadro:c2:pico} representa o PICO, que derivaram as seguintes QP que definem o escopo deste MS.

\begin{quadro}[htb]
\centering
\caption{Critérios de PICO.}
\label{quadro:c2:pico}
\begin{tabularx}{\textwidth}{l|X} \hline
\textit{\textbf{P}opulation} & Aprendizes/Educadores interessados em línguas de sinais. \\ \hline
\textit{\textbf{I}ntervention} & TICs relevantes no processo de ensino-aprendizagem com línguas de sinais. \\ \hline
\textit{\textbf{C}omparison} & Não se aplica. \\ \hline
\textit{\textbf{O}utcome} & Panorama tecnológico sobre o ensino-aprendizagem com línguas de sinais. \\ \hline
\end{tabularx}
\end{quadro}

\begin{itemize}
    \setlength\itemsep{0em}
    \item \textbf{QP1}: Quais soluções tecnológicas vêm sendo propostas no processo de ensino-aprendizagem com línguas de sinais?
    % \begin{itemize}
    %     \item Quais são os tipos de soluções propostas (software ou hardware ou teóricas)?
    %     \item Quais tecnologias foram usadas?
    %     \item Quais métodos de avaliação foram aplicados?
    % \end{itemize}
    \item \textbf{QP2}: Quais tópicos educacionais são abordados?
    \item \textbf{QP3}: Quais línguas de sinais são abordadas?
    % \begin{itemize}
    %     \item Quais estudos abordam múltiplas línguas de sinais?
    % \end{itemize}
\end{itemize}

Segundo \citeonline{Kitchenham2007,Petersen2015}, os estudos sistemáticos requerem critérios explícitos de inclusão e exclusão para avaliar seus potenciais estudos primários. Assim, foram definidos os seguintes critérios de seleção (\autoref{quadro:c2:criterios-selecao}):

\begin{quadro}[htb]
\centering
\caption{Critérios de Inclusão (CI) e Exclusão (CE).}
\label{quadro:c2:criterios-selecao}
\begin{tabularx}{\textwidth}{l|X} \hline
\textbf{CI1} & Os estudos apresentam contribuições (software ou hardware ou teóricas) para o ensino e a aprendizagem de línguas de sinais. \\ \hline
\textbf{CE1} & Estudos que não foram publicados no período de 2000 a 2019, seguindo um racional semelhante à \citeonline{Radermacher2013,Scatalon2019}, os quais sugerem que estudos anteriores a 2000 não representam as abordagens educacionais atuais, especialmente considerando o contexto de tecnologia. \\ \hline
\textbf{CE2} & Estudos classificados como resumos, resumos de conferências/editoriais, literatura cinza ou capítulos de livros. \\ \hline
\textbf{CE3} & Estudos não apresentados em inglês ou português. \\ \hline
\textbf{CE4} & Estudos não acessíveis em texto completo. \\ \hline
\textbf{CE5} & Estudos duplicados ou superficialmente complementares de outros estudos. \\ \hline
\end{tabularx}
\end{quadro}

\subsection{Condução das Buscas Manual e Automatizada}
\label{ms:conducao-busca-manual}

No contexto das buscas manuais, a \autoref{table:c2:busca-manual-nacional} lista as conferências e periódicos nacionais analisados. No entanto, os estudos dessas fontes não foram incluídos na composição do QGS devido à limitação de indexação nos mecanismos de busca internacionais, o que poderia comprometer a eficácia da abordagem sistemática baseada em QGS \cite{Zhang2011}. Apesar disso, \textbf{46 estudos primários de fontes brasileiras foram selecionados} e discutidos nos resultados do MS. 

Por sua vez, a \autoref{table:c2:busca-manual-internacional} apresenta as conferências e periódicos internacionais selecionados durante a busca manual, resultando em 19 estudos primários que compõem o QGS deste MS. As fontes relevantes foram utilizadas para a busca automatizada, garantindo uma sinergia maior com o QGS, conforme recomendado por \citeonline{Zhang2011}.

\begin{table}[htb]
\centering
\caption{Busca Manual Nacional.}
\label{table:c2:busca-manual-nacional}
\begin{tabular}{l|c|c} \hline
\textbf{Conferências/Periódicos} & \textbf{Fonte} & \textbf{Selecionados} \\ \hline
DesafIE                          & CEIE           & 0                     \\
JAIE                             & CEIE           & 0                     \\
RBIE                             & CEIE           & 2                     \\
RENOTE                           & CINTED         & 13                    \\
SBIE                             & CEIE           & 13                    \\
WAVE2                            & CEIE           & 0                     \\
WCBIE                            & CEIE           & 11                    \\
WIE                              & CEIE           & 7                     \\ \hline
\multicolumn{2}{l}{\textbf{Total}}                & \textbf{46}           \\ \hline
\end{tabular}
\end{table}

\begin{table}[htb]
\centering
\caption{Busca Manual Internacional (Equivalente ao QGS).}
\label{table:c2:busca-manual-internacional}
\begin{tabular}{l|c|c} \hline
\textbf{Conferência/Periódico} & \textbf{Fonte}     & \textbf{Selecionados (QGS)} \\ \hline
ACM TOCE                       & ACM                & 0            \\ 
Computers \& Education         & Elsevier           & 5            \\ 
FIE                            & IEEE               & 0            \\ 
HCI International              & Springer           & 5            \\ 
ICALT                          & IEEE               & 5            \\ 
IEEE ToE                       & IEEE               & 1            \\ 
IEEE TLT                       & IEEE               & 0            \\ 
Informatics in Education       & Vilnius University & 0            \\ 
ITiCSE                         & ACM                & 2            \\ 
Learning @ Scale               & ACM                & 0            \\ 
SIGCSE                         & ACM                & 1            \\ \hline
\multicolumn{2}{l}{\textbf{Total}}                  & \textbf{19}  \\ \hline
\end{tabular}
\end{table}

Tendo em vista a busca automatizada, duas estratégias para identificação de palavras-chave foram utilizadas em conjunto para a string de busca: (i) análise do PICO e suas respectivas QP; (ii) importação da tripla \textit{title-abstract-keywords} em um software de análise de frequência. Os resultados desse processo produziram a seguinte string de busca (\autoref{codigo:string_busca_ms}).

\begin{codigo}[caption={String de Busca do MS}, label={codigo:string_busca_ms}]
    (learn OR learning OR teach OR teaching) AND
    ("sign language" OR "signed language") AND
    (technology OR technologies)
\end{codigo}

A \autoref{table:c2:automated-search} resume os resultados da busca automatizada, onde a seleção dos estudos seguiu o mesmo racional apresentado na busca manual. Além disso, a busca automatizada retornou a maioria dos estudos selecionados pela busca manual (QGS), o que sugere uma boa sensibilidade da string de busca. Nesse sentido, \citeonline{Zhang2011} propõem o conceito de \textit{quasi-sensibility}, uma derivação da sensibilidade tradicional que incorpora o QGS como critério de qualidade (\autoref{method:equation:quasi-sensitivity}).

\begin{table}[htb]
\centering
\caption{Resultados da Busca Automatizada.}
\label{table:c2:automated-search}
\begin{tabular}{ll|lll} \hline
 &  & Busca Final &                 &                   \\ \cline{3-5} 
Base de Dados & \textbf{QGS} & Recuperados    & \textbf{no QGS} & \textbf{Relevantes} \\ \hline
ACM DigitalLibrary & 3            & 922          & 3               & 47                \\
IEEE Xplore        & 6            & 359          & 5               & 59                \\
ScienceDirect      & 5            & 1,961        & 5               & 20                \\
SpringerLink       & 5            & 4,980        & 5               & 36                \\ \hline
\multicolumn{1}{l}{\textbf{Total}}   & \textbf{19}  & 8,222        & \textbf{18}     & \textbf{162}      \\ \hline
\end{tabular}
\end{table}

\begin{equation}
\label{method:equation:quasi-sensitivity}
\text{\textit{quasi-sensibility}} = \frac{\text{\textit{Estudos relevantes recuperados (\textbf{no QGS})}}}{\text{\textit{Total de estudos relevantes (\textbf{QGS})}}}
\end{equation}

Como resultado, a \textit{quasi-sensitivity} calculada foi de 94,74\% (18/19), um desempenho adequado segundo \citeonline{Zhang2011}. Portanto, os 163 artigos selecionados pelas buscas (manual internacional e automatizada) foram considerados estudos primários em potencial. Nesta etapa, 24 estudos foram excluídos de acordo com os critérios de inclusão e exclusão pré-estabelecidos. Sendo assim, a \autoref{method:figure:evaluation-refinement} organiza os \textbf{139 estudos primários selecionados pela busca sistemática baseada em QGS}.

\begin{figure}[htb]
\centering 
\caption{Resultados da Busca Sistemática Baseada em QGS.}
\label{method:figure:evaluation-refinement}
\includegraphics[width=.9\textwidth]{images/chapter2-sm-qgs-search.png}
\fautor
\end{figure}

Para extrair as informações relevantes dos estudos primários identificados, um formulário de extração de dados foi criado. O \autoref{quadro:c2:data-extraction} representa o modelo que descreve as informações extraídas e apresenta seu relacionamento com cada QP, quando aplicável.

\sigla*{SWEBOK}{\textit{Software Engineering Body of Knowledge}}
\sigla*{ES}{Engenharia de Software}

\begin{quadro}[htb]
\centering
\caption{Formulário de Extração de Dados.}
\label{quadro:c2:data-extraction}
\begin{tabular}{l|l|l} \hline
\textbf{Informações Gerais} & \multicolumn{2}{l}{\textbf{Descrição}} \\ \hline
ID & \multicolumn{2}{l}{Identificador (prefixos \textit{INT} ou \textit{BRA}).} \\
Título & \multicolumn{2}{l}{Título do estudo.} \\
Autores & \multicolumn{2}{l}{Nomes dos autores.} \\
Ano & \multicolumn{2}{l}{Ano de publicação do artigo.} \\
Conferência/Periódico & \multicolumn{2}{l}{Nome do meio de publicação.} \\
Tipo de busca & \multicolumn{2}{l}{Manual; Automatizada; Ambas.} \\
Língua & \multicolumn{2}{l}{Inglês; Português.} \\
País & \multicolumn{2}{l}{País da afiliação do primeiro autor.} \\ \hline
\textbf{Informações Específicas} & \textbf{Descrição} & \textbf{QP} \\ \hline
Área da Eng. de Software (ES) & Área de conhecimento da ES (SWEBOK). & QP1 \\
Tipo de solução & Software; Hardware; Teórica. & QP1 \\
Estratégia empírica & Quais estratégias empíricas foram encontradas. & QP1 \\
Tópico educacional & Quais tópicos educacionais foram encontrados. & QP2 \\
Línguas de sinais & Quais línguas de sinais foram encontradas. & QP3 \\ \hline
\end{tabular}
\end{quadro}

\subsection{Resultados e Discussões}
\label{ms:resultados}

O MS contou com 185 estudos primários selecionados: 46 da busca manual nacional e 139 da busca sistemática baseada em QGS. Lembrando que, as informações mais relevantes para responder cada QP foram obtidas por meio do formulário de extração de dados. Primeiramente, considerando a quantidade de publicações por ano, uma linha de tendência linear crescente foi identificada (\autoref{results:figure:publications-year}). Portanto, é estatisticamente possível que este domínio de pesquisa esteja em ascensão globalmente.


\begin{figure}[htb]
\centering 
\caption{Linha de Tendência Linear Crescente de Publicações por Ano.}
\label{results:figure:publications-year}
\includegraphics[width=1\textwidth]{images/chapter2-sm-publications-timeline.png}
\fautor
\end{figure}

\simbolo{R^2}{Linha de Tendência Linear}

No que diz respeito às conferências, periódicos e fontes das publicações, esses dados também podem compor um racional interessante para futuras replicações. Sendo assim, todos os estudos primários deste MS foram ordenados pela quantidade de estudos selecionados (\autoref{table:c2:publication-venues}). No contexto internacional, a presença de eventos identificados durante as buscas manuais (em \textbf{\textit{destaque}} na \autoref{table:c2:publication-venues}) sugere uma execução efetiva dessa fase considerando o protocolo de busca adotado.

\begin{table}[htb]
\caption{Conferências/Periódicos mais relevantes.}
\label{table:c2:publication-venues}
\centering
\begin{tabular}{lcc|lcc} \hline
\multicolumn{3}{c|}{\textbf{Internacionais (\textit{INT})}} & \multicolumn{3}{c}{\textbf{Nacionais (\textit{BRA})}} \\ \hline
\textbf{Nome} & \textbf{Fonte} & \textbf{Estudos} & \textbf{Nome} & \textbf{Fonte} & \textbf{Estudos} \\ \hline
\textit{\textbf{HCI International}} & \textit{\textbf{Springer}} & \textit{\textbf{12}} & RENOTE & CINTED & 13 \\ 
ICCHP & Springer & 8 & SBIE & CEIE & 13 \\ 
\textit{\textbf{ICALT}} & \textit{\textbf{IEEE}} & \textit{\textbf{6}} & WCBIE & CEIE & 11 \\ 
ASSETS & ACM & 6 & WIE & CEIE & 7 \\ 
\textit{\textbf{Computers \& Education}} & \textit{\textbf{Elsevier}} & \textit{\textbf{5}} & RBIE & CEIE & 2 \\ 
Procedia Computer Science & Elsevier & 5 & - & - & - \\ 
Outros & - & 97 & - & - & - \\ \hline
\multicolumn{2}{l}{\textbf{Total}} & \textbf{139} & \multicolumn{2}{l}{\textbf{Total}} & \textbf{46} \\ \hline
\end{tabular}
\end{table}

A seguir são discutidos os principais resultados deste estudo, de modo a responder cada QP definida no escopo do MS. Adicionalmente, com o objetivo de organizar os estudos primários, eles foram classificados com relação à sua origem: Internacional (\textit{INT})\footnote{Formulário de extração de dados Internacionais (INT): \url{https://bit.ly/SM-DataExtraction-INT}} ou Nacional (\textit{BRA})\footnote{Formulário de extração de dados Nacionais (BRA): \url{https://bit.ly/SM-DataExtraction-BRA}}. Com isso, os resultados podem ser analisados de forma isolada, o que facilita o planejamento e a condução de trabalhos futuros.

\subsubsection{QP1: Quais soluções tecnológicas vêm sendo propostas no processo de ensino-aprendizagem com línguas de sinais?}

As áreas presentes na \autoref{table:c2:se-areas} destacam a importância intrínseca das arquiteturas de software na construção de recursos e serviços de TA para línguas de sinais. Embora haja uma concentração significativa nas etapas de ``Construção'' e ``Projeto'', poucas soluções se mostraram realmente replicáveis ou adaptáveis a diferentes contextos educacionais, principalmente pela falta de detalhes técnicos.

\begin{table}[htb]
\caption{QP1: Áreas da ES no SWEBOK \cite{Bourque2014}.}
\label{table:c2:se-areas}
\centering
\begin{tabular}{l|cc|cc} \hline
 & \multicolumn{2}{c|}{\textit{\textbf{INT}}} & \multicolumn{2}{c}{\textit{\textbf{BRA}}} \\ \cline{2-5} 
\textbf{Área da ES} & \textbf{Estudos} & \textbf{\%} & \textbf{Estudos} & \textbf{\%} \\ \hline
Construção de Software & 65 & 47\% & 23 & 50\% \\
Projeto de Software & 47 & 34\% & 5 & 11\% \\
Fundamentos da Engenharia & 24 & 17\% & 9 & 19\% \\
Qualidade de Software & 3 & 2\% & 9 & 19\% \\ \hline
\textbf{Total} & \textbf{139} & \textbf{100\%} & \textbf{46} & \textbf{100\%} \\ \hline
\end{tabular}
\end{table}

Tecnicamente, a maioria das soluções é baseada em plataformas Web, Mobile ou Desktop, evidenciando uma preocupação genuína em criar recursos de TA para diferentes plataformas de ensino-aprendizagem. No entanto, poucos estudos foram estruturados de forma a facilitar o seu reuso e extensibilidade. 

Além disso, menos da metade dos estudos apresentou avaliações empíricas formais, como \textit{Surveys}, Experimentos e Estudos de Caso, indicando uma falta de rigor científico em parte das pesquisas \cite{Pressman2016, Sommerville2015}.

Em contrapartida, avatares de línguas de sinais baseados em texto, como o \textit{Hand Talk}\footnote{Mais informações em \url{https://handtalk.me}} e o \textit{VLibras}\footnote{Mais informações em \url{https://gov.br/governodigital/pt-br/vlibras}}, destacam-se ao transformar texto em língua de sinais, evidenciando o potencial de soluções de TA bem arquitetadas para potencializar a acessibilidade de conteúdos em diversos contextos educacionais. Portanto, a discussão sobre Arquiteturas de Software na \autoref{section:foundation:arch} será fundamental para compreender como essas soluções podem ser aprimoradas para desenvolver serviços de TA verdadeiramente escaláveis.

\subsubsection{QP2: Quais tópicos educacionais são abordados?}

A \autoref{table:c2:educational-topics} apresenta uma ampla diversidade de tópicos educacionais tendo em vista os OAs analisados, evidenciando o uso do conceito de TA para línguas de sinais em diversos contextos. Isso representa um esforço consciente em abordar diferentes temas no processo de ensino-aprendizagem, promovendo uma acessibilidade digital mais ampla e personalizada. 

\begin{table}[htb]
\caption{QP2: Tópicos Educacionais.}
\label{table:c2:educational-topics}
\centering
\begin{tabular}{l|cc|cc} \hline
 & \multicolumn{2}{c|}{\textit{\textbf{INT}}} & \multicolumn{2}{c}{\textit{\textbf{BRA}}} \\ \cline{2-5} 
\textbf{Tópico Educacional} & \textbf{Estudos} & \textbf{\%} & \textbf{Estudos} & \textbf{\%} \\ \hline
Línguas de Sinais & 59 & 42,5\% & 19 & 41,3\% \\
Geral & 48 & 34,5\% & 10 & 21,7\% \\
Língua de Sinais Escrita & 10 & 7,2\% & 3 & 6,5\% \\
Matemática & 7 & 5,0\% & - & - \\
Alfabeto & 6 & 4,3\% & 1 & 2,2\% \\
Ciência da Computação & 4 & 2,9\% & 4 & 8,7\% \\
Língua Falada do País & 2 & 1,4\% & 9 & 19,6\% \\
Outros & 3 & 2,2\% & - & - \\ \hline
\textbf{Total} & \textbf{139} & \textbf{100\%} & \textbf{46} & \textbf{100\%} \\ \hline
\end{tabular}
\end{table}

Com uma vasta gama de OAs inclusivos e adaptáveis, os educadores podem proporcionar experiências de aprendizado mais imersivas e eficazes, garantindo oportunidades igualitárias de desenvolvimento para todos os alunos, independentemente de suas habilidades ou desafios individuais. Tais resultados estabelecem a base para uma discussão mais aprofundada sobre OAs na \autoref{section:foundation:lo}, onde são explorados como esses recursos podem ser projetados para atender demandas educacionais diversas.

\subsubsection{QP3: Quais línguas de sinais são abordadas?}

\sigla*{IAGen}{Inteligências Artificiais Generativas}

As línguas de sinais mais comuns, destacadas na \autoref{table:c2:sign-languages}, juntamente com o crescente interesse em ASR multi-idiomas, abrem caminho para avanços significativos de acessibilidade. A capacidade das IAs Generativas (IAGen) de transcrever e traduzir fala em texto em múltiplas línguas viabiliza a integração de avatares de línguas de sinais baseados em texto, resultando em OAs mais inclusivos e versáteis.

\begin{table}[htb]
\caption{QP3: Línguas de Sinais.}
\label{table:c2:sign-languages}
\centering
\begin{tabular}{l|cc|cc} \hline
 & \multicolumn{2}{c|}{\textit{\textbf{INT}}} & \multicolumn{2}{c}{\textit{\textbf{BRA}}} \\ \cline{2-5} 
\textbf{Língua de Sinais} & \textbf{Estudos} & \textbf{\%} & \textbf{Estudo} & \textbf{\%} \\ \hline
ASL & 21 & 15.11\% & - & - \\
Libras & 16 & 11.51\% & 44 & 95.65\% \\
Geral & 15 & 10.79\% & - & - \\
SignWriting & 10 & 7.19\% & 2 & 4.35\% \\
ArSL & 10 & 7.19\% & - & - \\
PSL & 6 & 4.32\% & - & - \\
BSL & 6 & 4.32\% & - & - \\
MySL & 6 & 4.32\% & - & - \\
ISL & 5 & 3.60\% & - & - \\
Outras & 44 & 31.65\% & - & - \\ \hline
\textbf{Total} & \textbf{139} & \textbf{100\%} & \textbf{46} & \textbf{100\%} \\ \hline
\end{tabular}
\end{table}

Nesse cenário, línguas de sinais como a \textit{American Sign Language} (ASL) e a Libras, além de sistemas de escrita como o \textit{SignWriting}, podem se beneficiar dessas tecnologias, ampliando o acesso a conteúdos educacionais antes restritos aos formatos de áudio e vídeo. Esses resultados ressaltam a importância do ASR, que será aprofundado na \autoref{section:foundation:asr}.

A análise dos resultados obtidos pelas QP fornece um panorama do uso das TICs no ensino e aprendizado com línguas de sinais, revelando tanto lacunas quanto oportunidades para avanços significativos na criação de soluções em TA ainda mais robustas. Essas descobertas abrem caminho para uma exploração detalhada de temas cruciais, como Arquiteturas de Software (\autoref{section:foundation:arch}), OAs (\autoref{section:foundation:lo}) e ASR (\autoref{section:foundation:asr}). Cada seção subsequente destaca como suas temáticas podem ajudar a superar os desafios identificados e a capitalizar nas oportunidades emergentes, contribuindo para um processo de ensino-aprendizagem mais inclusivo e acessível.

\section{Arquiteturas de Software: Bases para Tecnologias Assistivas}
\label{section:foundation:arch}

Uma das principais lacunas identificadas no MS foi a carência de padrões e boas práticas que permitam o reuso e a adaptação das soluções no ensino e aprendizagem com línguas de sinais. Muitos dos estudos primários apresentaram contribuições técnicas relevantes, mas não detalharam as arquiteturas de software utilizadas, dificultando a evolução e derivação dessas soluções para outros contextos e domínios de aplicação. Por isso, nesta seção é discutido como as arquiteturas podem contribuir para o desenvolvimento de TA replicável, flexível e independente de tecnologia, seguindo alguns princípios e diretrizes da ES.

Uma arquitetura de software pode ser definida como o conjunto de estruturas necessárias para o entendimento de um sistema, compreendendo desde seus componentes de software e hardware até suas relações e propriedades internas e externas \cite{Bass2021}. Essa definição enfatiza que a arquitetura inclui todas as decisões que moldam a estrutura do projeto e suas interações, não se limitando apenas às decisões iniciais. A arquitetura é fundamental para a criação de sistemas complexos e facilita a análise de requisitos não funcionais, como desempenho, segurança e escalabilidade \cite{Pressman2016, Sommerville2015}.

Conforme \citeonline{Bass2021}, a arquitetura abrange estruturas que permitem o raciocínio e a análise do sistema, oferecendo uma compreensão ampla e flexível, considerando suas múltiplas dimensões e aspectos envolvidos no desenvolvimento e manutenção do software. A seguir, são exploradas as diferentes estruturas e visões arquiteturais, bem como os critérios que definem uma ``boa'' arquitetura.

\subsection{Estruturas e Visões Arquiteturais}

Resumidamente, uma arquitetura pode ser vista como um conjunto de estruturas que proporcionam múltiplas perspectivas sobre o sistema, cada uma com seu foco específico, o qual pode ser necessário em diferentes fases do ciclo de vida do software. \citeonline{Bass2021} propõem três tipos principais de estruturas arquiteturais, que formam as principais visões neste contexto:

\begin{itemize}

    \item \textbf{Estruturas de Componentes e Conectores (C\&C)}: Estas estruturas focam nas interações em tempo de execução entre os componentes que realizam as funções do sistema. Componentes, que podem ser serviços, clientes, servidores ou filtros, são as unidades principais de computação. Os conectores, por sua vez, são os veículos de comunicação entre esses componentes, facilitando a troca de dados e a sincronização de processos. As estruturas C\&C são cruciais para entender o comportamento em tempo de execução, incluindo a interação entre componentes, a replicação de partes do sistema e a paralelização de tarefas \cite{Bass2021};
    
    \item \textbf{Estruturas de Módulos}: Estas estruturas particionam o sistema em unidades de implementação, conhecidas como módulos, que são responsáveis por funções específicas e são a base para a organização do trabalho de desenvolvimento. Módulos podem representar classes, pacotes ou divisões de funcionalidade, cada um com um papel definido no sistema. As relações entre os módulos, como uso, generalização e composição, ajudam a entender a estrutura estática do sistema e a gerenciar sua evolução e manutenção \cite{Bass2021};
    
    \item \textbf{Estruturas de Alocação}: Essas estruturas estabelecem a correspondência entre os componentes de software e os elementos não-software do sistema, como ambientes de desenvolvimento e execução. Elas respondem a questões críticas sobre onde cada componente será executado, como estão armazenados e como são atribuídos às equipes de desenvolvimento. As estruturas de alocação são essenciais para compreender a distribuição do software e gerenciar recursos durante todo o ciclo de vida do sistema \cite{Bass2021}.
    
\end{itemize}

Essas estruturas arquiteturais podem ser compreendidas de maneira análoga através dos diferentes sistemas fisiológicos humanos, conforme ilustrado na \autoref{chapter2:figure:physiological-structures}. Essa analogia facilita a compreensão de como diferentes visões se complementam para fornecer uma compreensão abrangente do sistema como um todo \cite{Bass2021}. Na figura, os diferentes sistemas fisiológicos representam de maneira análoga as estruturas arquiteturais:

\begin{figure}[htb]
\centering
\caption{Fisiologia Humana: Análoga às Estruturas e Visões Arquiteturais}
\label{chapter2:figure:physiological-structures}
\includegraphics[width=0.70\textwidth]{images/chapter2-arch-physiological-structures.jpeg}
\fdireta{Bass2021}
\end{figure}

\begin{itemize}
    \item \textbf{Esqueleto $\equiv$ Estruturas de Módulos}: Assim como o esqueleto fornece a estrutura e suporte básico para o corpo, as estruturas de módulos organizam e definem a base do software, dividindo-o em partes manejáveis e específicas, como classes e pacotes. Diagramas de classes e pacotes são exemplos de representações visuais que ilustram essas estruturas.

    \item \textbf{Músculos e Sistema Circulatório $\equiv$ Estruturas de C\&C}: Os músculos permitem o movimento e a interação entre as partes do corpo, enquanto o sistema circulatório transporta nutrientes e oxigênio, facilitando a comunicação. De forma similar, as estruturas de componentes e conectores permitem a interação e execução das funcionalidades do sistema, garantindo comunicação eficiente entre os componentes. Diagramas de componentes e de sequência são exemplos de como essas estruturas podem ser representadas visualmente.

    \item \textbf{Sistema Nervoso $\equiv$ Estruturas de Alocação}: O sistema nervoso controla e coordena as ações do corpo, assim como as estruturas de alocação determinam onde e como os componentes de software são executados, garantindo uma distribuição eficiente e gestão de recursos durante todo o ciclo de vida do sistema. Diagramas de implantação e de distribuição ilustram essas estruturas.
\end{itemize}

Essas três categorias de estruturas facilitam a criação de representações visuais que auxiliam na compreensão da arquitetura de software em diferentes etapas do desenvolvimento. A arquitetura \textit{Speech2Learning}, por exemplo, adota essas estruturas para garantir clareza arquitetural desde sua concepção até a avaliação em seus estudos de caso (detalhes nos Capítulos \ref{chapter3} e \ref{chapter4}). Essa abordagem sistematiza e torna mais claro o processo de desenvolvimento de soluções de TA baseadas na \textit{Speech2Learning}.

\subsection{O que Torna uma Arquitetura ``Boa''?}

Na prática, a arquitetura de software é uma abstração que destaca detalhes relevantes para a compreensão e análise do sistema, omitindo informações desnecessárias para o raciocínio sobre ele. A abstração é crucial para gerir a complexidade, permitindo que arquitetos e desenvolvedores se concentrem em aspectos essenciais sem se preocuparem com detalhes de implementação. A arquitetura trata dos elementos públicos do sistema, ou seja, aqueles que interagem entre si através de interfaces, enquanto os detalhes privados de implementação não são considerados \cite{Bass2021}.

Padrões arquiteturais são composições de elementos arquiteturais que foram documentadas e disseminadas devido à sua eficácia em resolver problemas recorrentes em diferentes domínios. Esses padrões fornecem abordagens comprovadas para o \textit{design} de sistemas e são fundamentais para alcançar os atributos de qualidade desejados, como modularidade e facilidade de manutenção \cite{Bass2021}. Por exemplo, o padrão de arquitetura em camadas é amplamente utilizado para sistemas que necessitam de alta modularidade, enquanto o padrão de microsserviços é ideal para sistemas que requerem escalabilidade e resiliência \cite{Pressman2016, Sommerville2015}.

Entretanto, não existe uma arquitetura intrinsecamente ``boa'' ou ``ruim''; a adequação de uma arquitetura depende de como ela atende aos requisitos específicos do sistema. Uma arquitetura projetada para um sistema de comércio eletrônico pode não ser adequada para um sistema de controle de voo, por exemplo. A avaliação da arquitetura em relação a objetivos específicos é crucial para garantir que ela atenda às necessidades do sistema \cite{Pressman2016, Sommerville2015}.

Para orientar o desenvolvimento de uma boa arquitetura de software, \citeonline{Bass2021} propõem algumas boas práticas, categorizadas em recomendações de processo e recomendações estruturais. Primeiramente, as \textbf{recomendações de processo} focam na maneira como a arquitetura deve ser desenvolvida e gerenciada ao longo do ciclo de vida do sistema, garantindo que a integridade conceitual e a qualidade sejam mantidas de forma contínua:

\begin{enumerate}
    \item \textbf{Condução por lideranças técnicas}: É fundamental que a arquitetura seja concebida por um arquiteto ou uma pequena equipe de arquitetos com um líder técnico identificado, assegurando a integridade conceitual e a consistência técnica. 
    
    Esse princípio também se aplica a projetos ágeis e de código aberto, evitando \textit{designs} impraticáveis e desconectados da realidade do desenvolvimento.
    
    \item \textbf{Foco nos requisitos de qualidade}: A arquitetura deve se basear continuamente em uma lista priorizada de requisitos de qualidade bem definidos. Esses requisitos guiam as decisões de \textit{trade-offs}, que sempre ocorrem, sendo mais relevantes do que a funcionalidade em si.
    
    \item \textbf{Documentação por meio de visões arquiteturais}: A arquitetura deve ser documentada através de visões que representem uma ou mais estruturas arquiteturais. Essas visões devem abordar as preocupações dos \textit{stakeholders} mais importantes e apoiar o cronograma do projeto, fornecendo uma documentação que pode ser inicialmente minimalista, mas detalhada posteriormente.
    
    \item \textbf{Avaliação contínua dos atributos de qualidade}: A arquitetura deve ser avaliada quanto à sua capacidade de fornecer os principais atributos de qualidade do sistema. Isso deve ocorrer no início do ciclo de vida, proporcionando os maiores benefícios, e ser repetido conforme necessário para garantir que alterações na arquitetura ou no ambiente não tornem o \textit{design} obsoleto.
    
    \item \textbf{Implementação incremental e adaptativa}: A arquitetura deve permitir a implementação incremental, evitando a integração total de uma só vez, o que raramente funciona. Isso pode ser alcançado através da criação de um sistema esquelético, no qual os caminhos de comunicação são exercidos inicialmente com funcionalidade mínima, permitindo o crescimento incremental do sistema e a refatoração conforme necessário.
\end{enumerate}

Por sua vez, as \textbf{recomendações estruturais} dizem respeito à organização interna da arquitetura, enfatizando a importância da modularidade, da separação de responsabilidades e da flexibilidade na integração dos componentes, visando a criação de um sistema robusto e facilmente evolutivo:

\begin{enumerate}
    \item \textbf{Modularização e separação de preocupações}: A arquitetura deve apresentar módulos bem definidos, cujas responsabilidades funcionais são atribuídas com base nos princípios de ocultação de informações e separação de preocupações. Esses módulos devem encapsular aspectos passíveis de mudança, isolando o software dos efeitos dessas mudanças.
    
    \item \textbf{Uso de padrões arquiteturais bem estabelecidos}: A arquitetura deve alcançar atributos de qualidade usando padrões arquiteturais e táticas bem estabelecidas e específicas para cada atributo. Isso proporciona uma base sólida para o \textit{design}, garantindo que os requisitos de qualidade sejam atendidos de maneira eficaz.
    
    \item \textbf{Flexibilidade em relação a versões de produtos}: A arquitetura nunca deve depender de uma versão específica de um produto comercial ou ferramenta. Se isso for inevitável, deve ser estruturada de forma que a mudança para uma versão diferente seja simples e barata.
    
    \item \textbf{Separação entre componentes produtores e consumidores de dados}: Os módulos que produzem dados devem ser separados dos módulos que consomem esses dados. Isso aumenta a manutenibilidade, permitindo que mudanças sejam confinadas ao lado da produção ou do consumo de dados, facilitando atualizações incrementais.
    
    \item \textbf{Flexibilidade na correspondência entre módulos e componentes}: Não se deve esperar uma correspondência um-para-um entre módulos e componentes. Em sistemas com concorrência, por exemplo, múltiplas instâncias de um componente podem ser executadas em paralelo, cada uma construída a partir do mesmo módulo.
    
    \item \textbf{Alocação flexível de processos}: Projete cada processo para ser executado em qualquer processador, permitindo fácil realocação, inclusive durante a execução. Isso é essencial em ambientes de virtualização e nuvem, onde os recursos computacionais podem variar.
    
    \item \textbf{Consistência e simplicidade nos padrões de interação}: A arquitetura deve conter um pequeno número de padrões simples de interação entre componentes. O sistema deve realizar as mesmas funções da mesma maneira em todas as partes, o que facilita a compreensão, reduz o tempo de desenvolvimento, além de aumentar confiabilidade e manutenibilidade.
    
    \item \textbf{Gestão eficaz de áreas de contenção de recursos}: A arquitetura deve conter um conjunto específico e pequeno de áreas de contenção de recursos, cuja resolução deve ser claramente especificada e mantida. Por exemplo, se a utilização da rede é uma preocupação, o arquiteto deve produzir diretrizes para cada equipe de desenvolvimento que resultem em níveis aceitáveis de tráfego de rede.
\end{enumerate}

Portanto, uma arquitetura de software bem projetada não só atende aos requisitos funcionais imediatos, mas também oferece uma base sólida que permite a evolução e adaptação contínua do sistema, especialmente em áreas críticas como a educação inclusiva. A flexibilidade da arquitetura é essencial para suportar a evolução contínua das TICs e a adaptação às necessidades dos aprendizes, garantindo que as soluções sejam sustentáveis e capazes de atender às necessidades dos alunos a longo prazo.

Nesse contexto, a arquitetura \textit{Speech2Learning} surge como uma proposta para impulsionar a construção de recursos e serviços de TA. Projetada para integrar soluções de ASR, ela visa facilitar a criação de OAs mais acessíveis a uma ampla gama de aprendizes. Nos próximos capítulos, são explorados em detalhes os conceitos de OAs e ASR, aprofundando o entendimento sobre como essas tecnologias se entrelaçam e formam a base da \textit{Speech2Learning}.

\section{Objetos de Aprendizagem: Diversidade em Conteúdos Educacionais}
\label{section:foundation:lo}

A crescente demanda por diversidade em conteúdos educacionais é amplamente reconhecida, conforme evidenciado no MS conduzido. Dessa forma, os OAs emergem como uma solução promissora para atender a essa demanda, permitindo a criação de recursos personalizados e adaptáveis a diferentes contextos e públicos. No âmbito da arquitetura \textit{Speech2Learning}, os OAs desempenham um papel fundamental no acesso a materiais didáticos audíveis, enriquecidos pela tecnologia de ASR para maior acessibilidade \cite{FalvoJr2023_HICSS}.

Os OAs abrangem uma vasta gama de recursos digitais projetados para enriquecer o processo de ensino-aprendizagem. Eles transcendem a mera entrega de conteúdo, proporcionando uma experiência mais rica e interativa \cite{Wiley2000}. A importância da multimídia no aprendizado é destacada por \citeonline{Mayer2021}, que argumentam que a combinação eficaz de texto, áudio, vídeo e elementos interativos pode potencializar a educação.

\sigla*{IEEE}{\textit{Institute of Electrical and Electronics Engineers}}

O \citeonline{LOM2000} define os OAs como entidades, digitais ou não-digitais, que podem ser usadas, reutilizadas ou referenciadas durante o ensino com suporte tecnológico. Essa definição abrange uma vasta gama de recursos, incluindo conteúdos multimídia, software instrucional, eventos educacionais, entre outros. \citeonline{Wiley2000} simplifica essa concepção ao descrever os OAs como recursos digitais que podem ser reutilizados para facilitar a aprendizagem, destacando a adaptabilidade e a reusabilidade como características centrais dos OAs.

Segundo \citeonline{Tarouco2021}, a essência dos OAs está na criação de pequenos módulos instrucionais reutilizáveis, combináveis de diferentes maneiras para atender às necessidades específicas de aprendizagem. Essa abordagem permite que educadores personalizem o ensino, adaptando materiais didáticos às suas metas pedagógicas individuais. O resultado é um processo de ensino-aprendizagem mais dinâmico, onde diferentes recursos se conectam para formar um todo coeso (\autoref{chapter2:figure:lo-mindmap}).

\begin{figure}[htb]
\centering
\caption{Mapa Conceitual Sobre Objetos de Aprendizagem}
\label{chapter2:figure:lo-mindmap}
\includegraphics[width=1\textwidth]{images/chapter2-lo-mindmap.jpg}
\fadaptada{Tarouco2021}
\end{figure}

\subsection{Estratégias de Identificação e Utilização de OAs}

O uso e reuso de OAs envolve várias estratégias que facilitam sua adaptação a diferentes contextos educacionais. Conforme discutido por \citeonline{Tarouco2021}, os OAs podem variar em tamanho, escopo e nível de granularidade, afetando diretamente sua reusabilidade (\autoref{chapter2:figure:lo-granularity}). OAs com alta granularidade, como imagens ou pequenos vídeos, são mais fáceis de reutilizar devido à sua simplicidade e especificidade. Em contraste, objetos de baixa granularidade, como cursos, oferecem uma experiência educacional mais completa e integrada, mas são mais difíceis de adaptar a novos contextos de ensino-aprendizagem. O equilíbrio da granularidade é fundamental para que os OAs atinjam seus objetivos educacionais \cite{Tarouco2021}.

\begin{figure}[htb]
\centering
\caption{Granularidade de Objetos de Aprendizagem}
\label{chapter2:figure:lo-granularity}
\includegraphics[width=0.64\textwidth]{images/chapter2-lo-granularity.jpg}
\fadaptada{Tarouco2021}
\end{figure}



A granularidade está estreitamente relacionada à intencionalidade pedagógica, que se refere à finalidade educacional para a qual o OA foi criado. A eficácia de um objeto depende da clareza de seus objetivos pedagógicos e da adequação às necessidades dos alunos \cite{Bloom1984}. Portanto, a escolha de OAs deve considerar tanto a granularidade quanto a intencionalidade pedagógica para garantir uma aprendizagem eficaz e significativa.

Nesse sentido, a adoção de padrões de metadados é essencial para a organização, indexação e reutilização de OAs. \citeonline{Santana2023} destacam a importância desses padrões e sua aplicabilidade no contexto da ES experimental, cujo compartilhamento de OAs é vital para replicações e pesquisas futuras. O \autoref{quadro:c2:lo-metadata} apresenta uma comparação entre vários padrões de metadados, destacando suas características principais:

\begin{itemize}
    \item \textit{Dublin Core}\footnote{Mais informações em \url{https://dublincore.org}}: Um padrão internacional que fornece um conjunto simples e padronizado de termos para descrever recursos. O Dublin Core é conhecido por sua simplicidade e extensibilidade, permitindo sua aplicação em diversos contextos, desde bibliotecas digitais até sistemas de informação corporativos.
    \item \sigla{SCORM}{\textit{Sharable Content Object Reference Model}}\footnote{Mais informações em \url{https://adlnet.gov/scorm}}: Um conjunto de padrões e especificações para e-learning que define a comunicação entre o conteúdo de aprendizado online e os Sistemas de Gerenciamento de Aprendizado (LMS). Desenvolvido pela \textit{Advanced Distributed Learning} (ADL), o SCORM facilita a interoperabilidade e a reutilização de conteúdos educacionais em diferentes plataformas de aprendizado.
    \item \textit{Motion Imagery Standard Board} (MISB)\footnote{Mais informações em \url{https://nsgreg.nga.mil/misb.jsp}}: Padrão desenvolvido para a gestão e utilização de imagens em movimento, particularmente em contextos que exigem alta precisão e interoperabilidade, como vigilância e análise de vídeo. O MISB, parte da \textit{National Geospatial-Intelligence Agency} (NGA), assegura que os dados de vídeo sejam consistentes e compatíveis em diferentes sistemas.
    \item \sigla{LOM}{\textit{Learning Object Metadata}}\footnote{Mais informações em \url{https://ieeexplore.ieee.org/document/9262118}}: Um padrão para a descrição de metadados de OAs, abrangendo aspectos como a finalidade educacional, estrutura, nível de agregação e condições de uso. Publicado pelo IEEE, o LOM é amplamente utilizado para descrever e categorizar recursos educacionais digitais, facilitando sua descoberta e reutilização.
    \item \sigla{RDF}{\textit{Resource Description Framework}}\footnote{Mais informações em \url{https://w3.org/rdf}}: Uma especificação da W3C que fornece uma base para a descrição de recursos da Web. O RDF é utilizado para modelagem de informações, permitindo a interoperabilidade entre diferentes sistemas de informação e facilitando a integração de dados de diversas fontes.
\end{itemize}

\begin{quadro}[htb]
\caption{Comparação Entre os Padrões de Metadados} 
\label{quadro:c2:lo-metadata}
\begin{tabular}{P{3.5cm}|P{3cm}|P{3cm}|P{3cm}}\hline
\textbf{Padrão de Metadados} & \textbf{Documentação Completa} & \textbf{Processamento Automatizado} & \textbf{Flexibilidade para ES} \\ \hline
Dublin Core & X & X & X \\  \hline
SCORM & X & X & X \\ \hline
MISB & X & X & \\ \hline
LOM & X & X & X \\ \hline
RDF & X & X & X \\ \hline
\end{tabular}
\fadaptada{Santana2023}
\end{quadro}

Segundo \citeonline{Santana2023}, o Dublin Core se destaca por sua simplicidade, documentação abrangente e capacidade de processamento automatizado, permitindo sua aplicação em diferentes linhas de pesquisa na área da ES. No entanto, no contexto deste trabalho, qualquer padrão de metadados pode ser adequado, visto que o papel de uma arquitetura de software não é o de definir ``detalhes de implementação''. 

Por outro lado, o LOM merece destaque pelo seu foco intrínseco em aspectos educacionais, demonstrando uma sinergia natural com o conceito de OAs. Além disso, com exceção do MISB, todos os padrões são tão robustos quanto o Dublin Core, considerando os critérios de comparação do \autoref{quadro:c2:lo-metadata}.

Ao conhecer os conceitos de granularidade e padrões de metadados, fica mais simples entendermos como os OAs podem ser encontrados. Nesse sentido, os \sigla{ROAs}{Repositórios de OAs} são plataformas que armazenam e disponibilizam OAs para educadores, estudantes e desenvolvedores. Eles desempenham um papel crucial na disseminação de recursos educacionais e na promoção do uso e reuso de OAs. No Brasil, de acordo com \citeonline{Tarouco2021}, alguns dos principais ROAs incluem: \textit{Portal do Professor}\footnote{Mais informações em \url{http://portaldoprofessor.mec.gov.br}}, \textit{Domínio Público}\footnote{Mais informações em \url{http://www.dominiopublico.gov.br}} e \textit{eduCAPES} \footnote{Mais informações em \url{https://educapes.capes.gov.br}}.

A utilização de padrões de metadados, como o LOM, nesses repositórios facilita a indexação e a busca eficiente de OAs, permitindo que educadores encontrem rapidamente os recursos que atendam às suas necessidades pedagógicas. A combinação de granularidade adequada e metadados padronizados assegura que os OAs possam ser reutilizados em diversos contextos educacionais, maximizando seu impacto e alcance.

%Um exemplo prático da adaptabilidade dos OAs é a inclusão de transcrições em materiais audíveis, como videoaulas. Essa prática não apenas flexibiliza o acesso à informação para todos os alunos, mas também destaca a capacidade dos OAs de se adaptarem às necessidades de uma gama diversificada de aprendizes, promovendo uma educação mais inclusiva. O potencial de adaptação dos OAs reforça a importância das TICs, como o ASR, na ampliação da acessibilidade e personalização dos conteúdos educacionais, conforme identificado no MS como um \textit{insight} relevante para a concepção da arquitetura \textit{Speech2Learning} \cite{FalvoJr2023_HICSS}.

A granularidade, os padrões de metadados e os repositórios de OAs se alinham para apoiar a intencionalidade pedagógica. Essa intencionalidade refere-se aos objetivos educacionais específicos para os quais os OAs são desenvolvidos e utilizados. De acordo com \citeonline{Bloom1984}, a eficácia de um OA depende da clareza de seus objetivos pedagógicos e da adequação às necessidades dos estudantes.

A próxima seção discorrerá sobre a intencionalidade pedagógica, exemplificando como a Taxonomia de Bloom e sua revisão podem ser aplicadas na seleção e utilização de OAs para maximizar o processo de ensino-aprendizagem.

\subsection{Intencionalidade Pedagógica}

A intencionalidade pedagógica dos OAs pode ser exemplificada pela Taxonomia de Bloom, que categoriza objetivos educacionais em uma hierarquia de complexidade cognitiva \cite{Bloom1984}. Em um trabalho posterior, \citeonline{Krathwohl2002} propôs uma revisão da taxonomia original, estruturando-a em uma perspectiva bidimensional. A primeira dimensão é a do conhecimento, que se desdobra em quatro tipos:

\begin{itemize}
    \item \textbf{Factual}: conhecimento básico de terminologias e detalhes específicos;
    \item \textbf{Conceitual}: inter-relações entre os elementos básicos em uma estrutura maior;
    \item \textbf{Procedimental}: métodos e critérios para realizar tarefas e resolver problemas;
    \item \textbf{Metaconhecimento}: conhecimento sobre o próprio conhecimento e sua regulação.
\end{itemize}

A segunda dimensão trata dos processos cognitivos, referida como Taxonomia de Bloom Revisada, que propõe seis categorias adaptadas da taxonomia original de \citeonline{Bloom1984}. \citeonline{Krathwohl2002} introduziu mudanças significativas, com destaque para o uso de verbos ativos ao invés de substantivos, além da reestruturação de alguns dos níveis. Como resultado, a nova versão organiza os OAs nas seguintes categorias:

\begin{itemize}
    \item \textbf{Recordar}: Capacidade de reter conhecimento na memória de longo prazo;
    \item \textbf{Entender}: Capacidade de construir significado a partir do material instrucional;
    \item \textbf{Aplicar}: Capacidade de utilizar o(s) procedimento(s) adequado(s) à situação vivenciada;
    \item \textbf{Analisar}: Capacidade de identificar diferentes partes constituintes de um material compreendendo suas inter-relações;
    \item \textbf{Avaliar}: Capacidade de estabelecer julgamentos a partir de critérios e padrões;
    \item \textbf{Criar}: Capacidade de transpor o conhecimento construído para novas situações a partir de produtos originais de autoria do próprio estudante.
\end{itemize}

Dada a representação bidimensional dos OAs proposta pela Taxonomia de Bloom Revisada de \citeonline{Krathwohl2002}, optou-se por utilizar uma tabela de duas dimensões (\autoref{quadro:c2:los-categories}) para a classificação desses objetivos. Na tabela, a dimensão do conhecimento é representada pelo eixo vertical, enquanto a dimensão dos processos cognitivos está no eixo horizontal. 

\begin{quadro}[htbp]
\centering
\resizebox{\textwidth}{!}{
\caption{Categorização de OAs}
\label{quadro:c2:los-categories}
\begin{tabular}{P{2.6cm}|c|c|c|c|c|c}
\hline
Dimensão do & \multicolumn{6}{c}{Dimensão dos Processos Cognitivos} \\ \cline{2-7}
Conhecimento & \textbf{Recordar} & \textbf{Entender} & \textbf{Aplicar} & \textbf{Analisar} & \textbf{Avaliar} & \textbf{Criar} \\ \hline
\textbf{Factual} & Listar & Resumir & Responder & Selecionar & Verificar & Generalizar \\ \hline
\textbf{Conceitual} & Reconhecer & Classificar & Providenciar & Diferenciar & Determinar & Montar \\ \hline
\textbf{Procedimental} & Recomendar & Esclarecer & Executar & Integrar & Julgar & Projetar \\ \hline
\textbf{Metacognitivo} & Identificar & Prever & Usar & Desconstruir & Refletir & Criar \\ \hline
\end{tabular}
}
\fadaptada{Mayer2021}
\end{quadro}

Cada célula do \autoref{quadro:c2:los-categories} relaciona um tipo de conhecimento com uma categoria de processo cognitivo, utilizando verbos para descrever ações esperadas dos alunos e substantivos para definir o conhecimento a ser adquirido. Isso facilita a adaptação dos OAs para diferentes objetivos pedagógicos.

Devido a essa flexibilidade, os OAs podem ser utilizados em diferentes contextos educacionais. Dependendo da intencionalidade pedagógica, alguns OAs podem focar mais no desenvolvimento de competências específicas dentro de uma categoria, enquanto outros podem abranger várias categorias simultaneamente. Dessa forma, os OAs mostram-se valiosos em diversas etapas do processo de ensino-aprendizagem, como:

\begin{itemize}
\item \textbf{Etapa A}: Introdução ao conteúdo a ser estudado;
\item \textbf{Etapa B}: Demonstração da teoria estudada;
\item \textbf{Etapa C}: Exemplo de aplicação do conteúdo estudado;
\item \textbf{Etapa D}: Instrumento de avaliação da aprendizagem.
\end{itemize}

Ademais, os OAs podem ser utilizados individualmente ou coletivamente, dependendo da intencionalidade pedagógica do professor e de sua escolha metodológica. Para simplificar a representação dos OAs nas etapas de ensino-aprendizagem, o \autoref{quadro:c2:samples-bloom-and-steps} foca exclusivamente na dimensão dos processos cognitivos da Taxonomia de Bloom Revisada \cite{Krathwohl2002}. 

Na prática, as etapas do processo de ensino-aprendizagem (introdução, demonstração, aplicação, avaliação) estão mais diretamente relacionadas aos processos cognitivos que os alunos desenvolvem durante essas atividades. Portanto, conectar os processos cognitivos com as etapas do ensino-aprendizagem permite uma visualização mais prática e objetiva dos diferentes tipos de OAs, proporcionando clareza na utilização dos recursos educacionais de acordo com a intencionalidade pedagógica desejada.

\begin{quadro}[htbp]
\centering
\resizebox{\textwidth}{!}{
\caption{Tipos de OAs: Processos Cognitivos e Etapas de Ensino-Aprendizagem}
\label{quadro:c2:samples-bloom-and-steps}
\begin{tabular}{l*{6}{|c}*{4}{|P{0.41cm}}}
\hline
\multirow{2}{*}{\textbf{Tipo de OA}} & \multicolumn{6}{c|}{\textbf{Processo Cognitivo}} & \multicolumn{4}{c}{\textbf{Etapa de Ensino}} \\ \cline{2-11}
 & Recordar & Entender & Aplicar & Analisar & Avaliar & Criar & A & B & C & D \\ \hline
Texto & \faCheckCircle & \faCheckCircle & \faCheckCircle & \faCheckCircle & \faCheckCircle & \faCheckCircle & \faCheck & \faCheck & \faCheck & \faCheck \\ \hline
Jogo & \faCheckCircle & \faCheckCircle & \faCheckCircle & \faCheckCircleO & \faCheckCircleO & \faCheckCircleO & \faCheck & \faCheck & \faCheck & \faCheck \\ \hline
Simulação & \faCheckCircle & \faCheckCircle & \faCheckCircle & \faCheckCircleO & \faCheckCircleO & \faCheckCircleO & \faCheck & \faCheck & \faCheck & \faCheck \\ \hline
Áudio/Vídeo & \faCheckCircle & \faCheckCircle & \faCheckCircleO & \faCheckCircle & \faCheckCircleO & \faCheckCircleO & \faCheck & \faCheck & \faCheck & \faCheck \\ \hline
Slides & \faCheckCircle & \faCheckCircle & \faCheckCircleO & \faCheckCircle & \faCheckCircleO & \faCheckCircleO & \faCheck & \faCheck & \faCheck & \faCheck \\ \hline
Exercícios & \faCheckCircle & \faCheckCircle & \faCheckCircle & \faCheckCircle & \faCheckCircleO & \faCheckCircleO & \faCheck & \faTimes & \faTimes & \faCheck \\ \hline
Mapa Mental & \faCheckCircle & \faCheckCircle & \faCheckCircle & \faCheckCircle & \faCheckCircleO & \faCheckCircle & \faCheck & \faCheck & \faCheck & \faCheck \\ \hline
Experimento & \faCheckCircle & \faCheckCircle & \faCheckCircle & \faCheckCircle & \faCheckCircle & \faCheckCircleO & \faCheck & \faCheck & \faCheck & \faCheck \\ \hline
Infográfico & \faCheckCircle & \faCheckCircle & \faCheckCircle & \faCheckCircle & \faCheckCircle & \faCheckCircle & \faCheck & \faCheck & \faCheck & \faCheck \\ \hline
\end{tabular}
}
\fadaptada{Mayer2021}
\nota{Legenda: \faCheckCircle~(uso comum); \faCheckCircleO~(uso em potencial); \faCheck~(aplicado); \faTimes~(não aplicado).}
\end{quadro}

Interpretando o \autoref{quadro:c2:samples-bloom-and-steps}, fica evidente a flexibilidade dos OAs do tipo texto no processo de ensino-aprendizagem atual. Eles são comumente utilizados nas seis categorias da taxonomia revisada e são relevantes em todas as etapas de ensino-aprendizagem. Essa característica foi fundamental para delimitar o escopo do \textit{Speech2Learning}, que se concentrou em tornar os OAs audíveis mais acessíveis através de ASR, transformando conteúdos de áudio/vídeo em texto e, assim, aumentando o alcance e a acessibilidade desses tipos de OAs.

De acordo com \citeonline{Tarouco2021}, o uso de novas tecnologias é um fator determinante para potencializar o desempenho de aprendizagem e o engajamento dos aprendizes. Nesse sentido, a \autoref{figure:chapter2-lo-engagement} demonstra que a tutoria individualizada oferece resultados superiores, mas os autores ressaltam que ela muitas vezes é inviável devido aos custos e recursos necessários. No entanto, novas TICs, como ASR ou IAGen, podem proporcionar melhorias significativas no processo de ensino-aprendizagem, alcançando resultados comparáveis aos de uma tutoria 1:1.

\begin{figure}[htb]
\centering
\caption{Impacto de Diferentes Estratégias de Interatividade na Aprendizagem}
\label{figure:chapter2-lo-engagement}
\includegraphics[width=.96\textwidth]{images/chapter2-lo-engagement.jpg}
\fadaptada{Tarouco2021}
\end{figure}

Na seção a seguir, discute-se como o reconhecimento de fala pode promover a acessibilidade digital por meio do ASR, abordando intrinsecamente subáreas fundamentais da IAGen, como \sigla{NLP}{Processamento de Linguagem Natural} e \sigla{LLMs}{Grandes Modelos de Linguagem}. Através dessas tecnologias disruptivas, pretende-se potencializar ainda mais o alcance e a qualidade dos OAs nos mais diversos ambientes educacionais.

\section{Reconhecimento de Fala: Promovendo Acessibilidade Digital com IA}
\label{section:foundation:asr}

Na era digital, a educação está em constante evolução à medida que as tecnologias emergentes remodelam as abordagens pedagógicas tradicionais. Neste contexto, o ASR, interpretado neste trabalho como um sinônimo de STT, emerge como uma ferramenta poderosa.

O ASR não apenas amplia a acessibilidade de conteúdos por meio de transcrições e legendas, mas também representa um avanço significativo rumo a uma educação mais inclusiva. Corroborando essa visão, \citeonline{Homburg2019} enfatizam o potencial do ASR em soluções de TA para surdos, como os avatares de línguas de sinais baseados em texto, explorados no MS conduzido como parte deste trabalho \cite{FalvoJr2020_FIE, FalvoJr2020_SBIE, FalvoJr2021_RENOTE}.

De acordo com \citeonline{Jurafsky2024}, a função do ASR é converter ondas sonoras da fala em uma sequência de palavras correspondentes. Embora a transcrição automática de fala de qualquer locutor e em qualquer ambiente ainda apresente desafios, a tecnologia de ASR já está suficientemente avançada para ser aplicada em diversas tarefas práticas, como a geração automática de legendas para áudio e vídeo, transcrição de conversas e comunicação assistiva para PcD, facilitando a interação entre computadores e humanos.

\citeonline{fleischmann2021} ressaltam que o crescimento do ensino remoto, intensificado pela pandemia da COVID-19, impulsionou a busca por métodos inovadores para criar e compartilhar conteúdo educacional. Nesse contexto, o ASR tem se mostrado uma ferramenta promissora, especialmente em ambientes colaborativos e conferências online.

Soluções baseadas em ASR desempenham um papel vital ao quebrar barreiras linguísticas, otimizando a comunicação entre falantes de diferentes línguas. Este argumento é reforçado por \citeonline{Homburg2019}, que destaca a relevância da tradução de voz para línguas de sinais visando promover a inclusão da comunidade surda no processo de ensino-aprendizagem.

Apesar de seu potencial, o ASR enfrenta diversos obstáculos e desafios de pesquisa. O trabalho de \citeonline{Koenecke2020} destaca alguns deles, apontando para disparidades raciais e as sutilezas de características linguísticas, como sotaques e peculiaridades regionais, assim como identificado no MS para a Libras. Essas constatações reforçam a relevância de promover soluções baseadas em ASR que sejam verdadeiramente inclusivas e que contemplem um espectro mais amplo de considerações sociolinguísticas em seu projeto e implementação.

\citeonline{Mayer2021} destacam a importância de tecnologias disruptivas, como o ASR, para expandir o alcance dos OAs, tornando o conteúdo educacional acessível a mais alunos. Complementando essa perspectiva, \citeonline{Parakh2022} define OAs como unidades digitais reutilizáveis, frequentemente integradas a projetos \textit{open-source}, que desempenham um papel fundamental na criação de experiências de ensino-aprendizagem adaptáveis, democráticas e contextualizadas.

Estas recentes perspectivas reforçam e expandem as percepções obtidas no MS, que enfatizou a importância das inovações tecnológicas no processo de ensino e aprendizagem de línguas de sinais, revelando lacunas interessantes. Nesse sentido, foi possível observar a falta de padrões de projeto, além de boas práticas de código e reuso, o que compromete a qualidade, o compartilhamento e o potencial de impacto desses OAs.

Diante destas lacunas e das tendências emergentes citadas, foi projetada a Arquitetura \textit{Speech2Learning}. Essa abstração propõe diretrizes de desenvolvimento para a criação de soluções baseadas em ASR, promovendo maior acessibilidade de seus OAs, em especial os audíveis.

A evolução e sofisticação do ASR estão intimamente ligadas aos avanços em \sigla{ML}{\textit{Machine Learning}} e IA. A aplicação de modelos de ML em ASR permite a otimização contínua e a adaptação a diferentes contextos linguísticos e acústicos. Isso é alcançado através de técnicas de treinamento supervisionado e não supervisionado, utilizando vastas quantidades de dados de fala para melhorar a precisão e a robustez dos sistemas de reconhecimento.

O fluxo de processamento de fala em sistemas ASR é ilustrado na \autoref{figure:chapter2-asr-diagram}. O processo começa com a captura do sinal de fala, que é submetido a uma etapa de pré-processamento. Nessa fase, o sinal é filtrado para remover ruídos e normalizado para padrões específicos. Em seguida, ocorre a extração de características, onde as propriedades acústicas relevantes da fala são convertidas em vetores de características. Esses vetores representam os componentes principais da fala que serão utilizados nas próximas etapas.

\begin{figure}[htb]
\centering
\caption{Fluxo do Processamento de Fala em Sistemas ASR}
\label{figure:chapter2-asr-diagram}
\includegraphics[width=0.95\textwidth]{images/chapter2-asr-diagram.png}
\fadaptada{Li2018}
\end{figure}

Após a extração de características, os vetores resultantes são enviados a um decodificador. O decodificador utiliza modelos acústicos, um dicionário de pronúncia e um modelo de linguagem para interpretar os vetores de características e mapear palavras/frases correspondentes. Os modelos acústicos são responsáveis por capturar as nuances dos sons da fala, enquanto o dicionário de pronúncia fornece a correspondência entre as sequências de fonemas e palavras.

O modelo de linguagem, que pode ser um LLM, ajuda a prever a sequência mais provável de palavras com base no contexto linguístico. Esta abordagem estruturada e baseada em modelos de ML e IA permite que sistemas de ASR sejam precisos e adaptáveis a uma ampla gama de variabilidades na fala humana.

A distinção entre ASR e STT é muitas vezes sutil e pode ser usada de forma intercambiável; neste trabalho, por motivo de simplificação, optou-se por essa abordagem. Enquanto ASR geralmente se refere ao campo de estudo e tecnologia de reconhecimento automático de fala, STT descreve a função específica de converter fala em texto. Ambos os termos são fundamentais para o desenvolvimento de TA e têm aplicações que se sobrepõem consideravelmente.

Para garantir que os sistemas de ASR cumpram seu papel de maneira eficaz e inclusiva, é essencial utilizar métodos robustos de avaliação. A próxima seção discutirá os métodos de avaliação de reconhecimento de fala, incluindo a \sigla{WER}{\textit{Word Error Rate}} e métodos de similaridade léxica, fundamentais para uma análise abrangente da qualidade desses sistemas.

\subsection{Métodos de Avaliação de Reconhecimento de Fala}

A avaliação da precisão dos sistemas baseados em ASR é essencial para o aprimoramento e a implementação eficaz dessas tecnologias. Um dos métodos mais tradicionais e amplamente utilizados é a WER, que mede a discrepância entre a transcrição gerada com ASR e uma transcrição de referência. Segundo \citeonline{Jurafsky2024}, o WER é definido como:

\[
\text{WER} = \frac{\text{Inserções} + \text{Substituições} + \text{Deleções}}{\text{Total de Palavras na Transcrição Correta}} \times 100
\]

O WER quantifica o número total de palavras que foram inseridas, substituídas ou deletadas na transcrição de ASR em comparação com uma referência, expressando esse total como uma porcentagem do número total de palavras na transcrição correta. Embora seja amplamente utilizado, o WER tem limitações, particularmente ao lidar com variações lexicais e semânticas, pois considera todas as palavras igualmente importantes e não distingue entre diferentes tipos de erro, como erros semânticos e sintáticos.

Para complementar a análise de WER, métodos de similaridade léxica são frequentemente utilizados para fornecer uma visão mais detalhada da qualidade das transcrições geradas por ASR. Esses métodos avaliam a similaridade entre as transcrições com base em diversos critérios, como alterações lexicais e proximidade semântica. De acordo com \citeonline{Majumdar2022}, alguns dos métodos de similaridade léxica incluem:

\begin{itemize}
\item \textbf{Distância de Levenshtein}: Mede o número mínimo de operações necessárias para transformar uma palavra na outra. Este método é útil para capturar a similaridade em termos de alterações literais entre as transcrições automática e de referência \cite{levens-1,levens-2}.
\item \textbf{Índice de Jaccard}: Avalia a similaridade entre conjuntos de palavras, calculando a razão entre o tamanho da interseção e o tamanho da união dos conjuntos de palavras de duas transcrições. É particularmente útil para medir a presença de palavras comuns e a diversidade de vocabulário \cite{jaccard-1,jaccard-2}.
\item \textbf{Similaridade de Cosseno}: Mede a similaridade entre vetores de palavras, considerando o cosseno do ângulo entre eles. Este método é frequentemente usado para comparar representações vetoriais de frases ou textos, capturando a similaridade semântica além da simples correspondência de palavras \cite{cosseno-1,cosseno-2,cosseno-3}.
\end{itemize}

Esses métodos de similaridade léxica oferecem uma análise complementar à WER, proporcionando uma visão mais rica e detalhada da precisão dos sistemas de ASR. Eles são especialmente úteis para capturar nuances linguísticas e semânticas nas transcrições, que podem não ser refletidas de forma adequada pela WER. Nos estudos de caso conduzidos, optou-se por utilizar métodos de similaridade léxica para obter uma avaliação mais holística e detalhada da qualidade das transcrições automáticas.

Tais abordagens permitem uma avaliação mais abrangente dos sistemas de ASR, destacando não apenas a precisão em termos de correspondência palavra a palavra, mas também a preservação do significado e a fluidez das transcrições. Isso é fundamental para aplicações que exigem uma compreensão precisa e contextual do conteúdo falado.

Em resumo, o reconhecimento de fala, por meio de tecnologias como ASR e STT, suportado por avanços em ML e IA, tem o potencial de transformar significativamente a acessibilidade digital. Estas tecnologias não só facilitam a criação de conteúdos mais inclusivos, mas também permitem a adaptação e personalização da educação para atender a uma diversidade maior de aprendizes. A integração de ASR, portanto, representa um marco crucial na busca por uma educação mais acessível e equitativa.

Dessa forma, ao reconhecer a importância das novas TICs, especialmente aquelas baseadas em IA, reafirma-se o compromisso com uma educação mais inclusiva e flexível. Combinando boas práticas arquiteturais, OAs e ASR, propõe-se neste trabalho de doutorado a Arquitetura \textit{Speech2Learning}, detalhada no próximo capítulo.

\section{Considerações Finais}

As análises e discussões apresentadas nesta fundamentação teórica permitiram uma compreensão abrangente das interseções entre TICs e línguas de sinais no contexto educacional. O MS realizado, detalhado na \autoref{section:foundation:sm}, revelou importantes lacunas tecnológicas e de pesquisa na utilização de línguas de sinais para o ensino-aprendizagem, destacando áreas onde inovações são necessárias. Nesse sentido, esse estudo sistemático resultou na publicação de uma série de artigos, os quais discutem múltiplas perspectivas:

\begin{itemize}
    \item \fullcite{\textbf{FALVOJR, V.}; MARTINS FALVO, C.; SCATALON, L.; BARBOSA, E.}{Tecnologias Aplicadas ao Ensino e Aprendizagem de LIBRAS: Um Mapeamento Sistemático}{Simpósio Brasileiro de Informática na Educação (SBIE)}{2020}{Disponível em \url{doi.org/10.5753/cbie.sbie.2020.812}}

    \item \fullcite{\textbf{FALVOJR, V.}; SCATALON, L.; BARBOSA, E.}{The Role of Technology to Teaching and Learning Sign Languages: A Systematic Mapping}{Frontiers in Education Conference (FIE)}{2020}{Disponível em \url{doi.org/10.1109/FIE44824.2020.9274169}}

    \item \fullcite{\textbf{FALVOJR, V.}; MARTINS FALVO, C.; SCATALON, L.; BARBOSA, E.}{Tecnologias Aplicadas ao Ensino e Aprendizagem com Línguas de Sinais: Um Mapeamento Sistemático Sob as Perspectivas Nacional e Internacional}{Revista Novas Tecnologias na Educação (RENOTE)}{2021}{Disponível em \url{doi.org/10.22456/1679-1916.110217}}
\end{itemize}

As lacunas tecnológicas identificadas foram essenciais para orientar a condução de um levantamento bibliográfico adicional, visando identificar e explorar TICs promissoras que pudessem mitigar os desafios identificados e promover uma educação mais inclusiva.

No decorrer desta fundamentação teórica, explorou-se de forma aprofundada diversos aspectos cruciais, incluindo Arquiteturas de Software, OAs e ASR. A \autoref{section:foundation:arch} discutiu arquiteturas de software inovadoras que suportam a integração eficiente de TICs no processo educacional, enquanto a \autoref{section:foundation:lo} concentrou-se nos OAs, destacando sua importância na criação de materiais educacionais adaptáveis e acessíveis. Particularmente, a \autoref{section:foundation:asr} demonstrou como as tecnologias de ASR podem ser aplicadas para melhorar a acessibilidade de conteúdos educacionais, proporcionando uma base teórica robusta para a arquitetura \textit{Speech2Learning}.

Através da síntese destas temáticas, esta fundamentação teórica estabelece os alicerces para a proposição da \textit{Speech2Learning}, uma arquitetura inovadora que visa tornar os Objetos de Aprendizagem audíveis mais acessíveis. Definida em detalhes no \autoref{chapter3}, a \textit{Speech2Learning} representa uma proposta baseada no uso de ASR para educação inclusiva, aproveitando as últimas inovações em TICs e IA para criar soluções educacionais que atendam às necessidades de uma diversidade maior de aprendizes. Esta fundamentação teórica, portanto, não apenas considera as lacunas existentes, mas também propõe caminhos concretos para superá-las, contribuindo para o avanço da pesquisa e prática educacional inclusiva.