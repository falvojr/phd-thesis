\section{Considerações Iniciais}

Nossa fundamentação teórica começa com um Mapeamento Sistemático, desenvolvido com o propósito de identificar a intersecção entre TICs e línguas de sinais no processo de ensino-aprendizagem. A Subseção \ref{section:foundation:sm} oferece um resumo desse estudo sistemático, cujas análises aprofundadas, abrangendo tanto o panorama nacional quanto o internacional, estão detalhadas em uma série de publicações \cite{FalvoJr2020_FIE, FalvoJr2020_SBIE, FalvoJr2020_RENOTE}.

A realização deste estudo nos permitiu identificar lacunas e oportunidades no uso de línguas de sinais no ensino e aprendizagem, incentivando o desenvolvimento de novas pesquisas na literatura. Essas investigações adicionais, que possuem um caráter mais exploratório, se concentram em conceitos e tecnologias promissores para endereçar esses desafios. Tais temas são abordados com mais detalhes nas Subseções \ref{section:foundation:arch} (Arquiteturas de Software), \ref{section:foundation:lo} (Objetos de Aprendizagem) e \ref{section:foundation:asr} (Reconhecimento Automático de Fala).

\section{Mapeamento Sistemático: Lacunas e Oportunidades no Processo de Ensino-Aprendizagem com Línguas de Sinais}
\label{section:foundation:sm}



\section{Arquiteturas de Software: Bases Sólidas para Tecnologias Assistivas}
\label{section:foundation:arch}

\section{Objetos de Aprendizagem: Diversidade em Conteúdos Educacionais}
\label{section:foundation:lo}

\section{Reconhecimento de Fala: Promovendo Acessibilidade Digital com IA}
\label{section:foundation:asr}

\section{Considerações Finais}