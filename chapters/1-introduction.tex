\section{Contexto e Motivação}

A acessibilidade digital é um componente essencial para a inclusão social, especialmente no contexto educacional, no qual a variedade de perfis e necessidades dos alunos demanda soluções genuinamente inclusivas. Segundo dados do \sigla{IBGE}{Instituto Brasileiro de Geografia e Estatística}, no Brasil, aproximadamente 8,9\% da população com mais de 2 anos, o que corresponde a 18,6 milhões de pessoas, possui algum tipo de deficiência \cite{IBGE2023}.

Nesse cenário, a área do conhecimento de \sigla{TA}{Tecnologia Assistiva} apresenta-se como uma resposta natural. De acordo com \citeonline{Cook2020}, soluções de TA englobam uma ampla gama de recursos, desde dispositivos e serviços até práticas voltadas para melhorar as capacidades funcionais e a qualidade de vida das \sigla{PcD}{Pessoas com Deficiência}. A TA é uma área interdisciplinar que combina produtos, metodologias e estratégias, todas direcionadas a aumentar a autonomia e a participação das PcD em suas atividades cotidianas \cite{Cat2009}.

Além disso, o \sigla{CAT}{Comitê de Ajudas Técnicas} do Brasil destaca que a TA desempenha um papel essencial na promoção da autonomia, independência e inclusão social dessas pessoas. As soluções de TA vão além de artefatos tecnológicos, abrangendo também serviços e práticas adaptadas às necessidades específicas dos usuários em diferentes contextos \cite{Cat2009}. Dessa forma, a TA tem o potencial de ampliar o acesso a recursos educacionais, considerando as particularidades de cada aprendiz \cite{UNESCO2023, GovBr2023}.

O uso de \sigla{TICs}{Tecnologias de Informação e Comunicação}, que podem incluir recursos e serviços de TA, desempenha um papel essencial nesse contexto. Os relatórios da \citeonline{OMS2011, OMS2018} enfatizam que a promoção da acessibilidade digital por meio das TICs é vital para garantir a participação plena de todas as pessoas na sociedade e proporcionar oportunidades igualitárias de desenvolvimento educacional e profissional.

Ainda no âmbito das TICs, as \sigla{IAs}{Inteligências Artificiais} surgem como ferramentas poderosas para promover a acessibilidade. A \sigla{UNESCO}{Organização das Nações Unidas para a Educação, a Ciência e a Cultura} destaca o potencial transformador das TICs na educação e ressalta a importância do \textit{Design} Universal na criação de soluções acessíveis a todos, independentemente de suas habilidades ou deficiências \cite{UNESCO2023, GovBr2023}. Nesse contexto, as IAs podem revolucionar a educação, fornecendo ferramentas que democratizam o acesso ao conhecimento e aprimoram a experiência de aprendizagem para todos os estudantes \cite{Holmes2019,UNESCO2024}.

No entanto, apesar do potencial das TICs e IAs na área do conhecimento de TA, o Brasil ainda enfrenta muitos desafios para a inclusão de PcD no ambiente educacional. Dados de 2022 indicam que 19,5\% das PcD estão fora da escola, uma taxa significativamente maior em comparação aos 4,1\% entre pessoas sem deficiência \cite{IBGE2023}. Essa disparidade é agravada pela carência de recursos de TA, evidenciando a urgência de investimentos em soluções que promovam o acesso e garantam a permanência de PcD em ambientes educacionais, idealmente inclusivos.

\textcolor{red}{Nesse contexto, um \sigla{MS}{Mapeamento Sistemático}, focado na intersecção entre línguas de sinais e TICs, revelou desafios significativos, como a complexidade e a falta de padronização no desenvolvimento de soluções tecnológicas voltadas à acessibilidade educacional \cite{FalvoJr2020_FIE, FalvoJr2020_SBIE, FalvoJr2021_RENOTE}. Embora as línguas de sinais apresentem necessidades específicas, o MS evidenciou a viabilidade de integrar a tecnologia de \sigla{ASR}{\textit{Automatic Speech Recognition}} como recurso assistivo, ampliando o acesso a \sigla{OAs}{Objetos de Aprendizagem} audíveis em diversos contextos.}

\textcolor{red}{Sob essa perspectiva, este trabalho de doutorado propõe a \textit{Speech2Learning}, uma arquitetura de software baseada em ASR, com o objetivo de facilitar a criação de soluções de TA para OAs mais acessíveis, utilizando reconhecimento de fala. Segundo \citeonline{Wiley2000}, OAs são recursos digitais reutilizáveis projetados para apoiar o processo de ensino-aprendizagem. Já o ASR, também conhecido como \sigla{STT}{\textit{Speech-to-Text}}, é uma subárea da IA que converte fala em texto \cite{Jurafsky2024}, proporcionando maior acessibilidade a OAs audíveis, como videoaulas. A integração desses conceitos na \textit{Speech2Learning} não apenas apoia PcD no acesso à educação, mas também beneficia qualquer aprendiz que utilize legendas ou transcrições automáticas.}

\textcolor{red}{Dessa forma, a motivação deste estudo origina-se nas lacunas identificadas pelo MS, que destacaram a necessidade de soluções replicáveis e alinhadas às novas tendências tecnológicas para promover uma educação mais inclusiva \cite{FalvoJr2021_RENOTE}. As reflexões resultantes apontaram o ASR como uma abordagem promissora para ampliar a acessibilidade de OAs audíveis em diversos contextos educacionais. Com os avanços em IA, especialmente em modelos de reconhecimento de fala, tornou-se possível projetar uma arquitetura de software capaz de atender a essas demandas.}

\textcolor{red}{A \textit{Speech2Learning} é uma resposta prática a esse cenário, utilizando o ASR para superar barreiras no acesso à informação e expandir a acessibilidade de OAs audíveis para um público de aprendizes mais diverso. Ao longo deste trabalho, foram criadas duas instâncias concretas da \textit{Speech2Learning}, que permitiram avaliar a arquitetura por meio de estudos de caso aplicados na indústria. Essas instâncias, além de servirem como uma avaliação prática da arquitetura, estabeleceram-se como recursos de TA ao expandirem o alcance dos OAs audíveis, contribuindo para um processo de ensino-aprendizagem mais acessível.}

\section{Objetivos e Questões da Pesquisa}
\label{chapter1:research-questions}

%O principal objetivo deste trabalho de doutorado é desenvolver e avaliar uma arquitetura de software que promova a melhoria na acessibilidade de OAs audíveis, promovendo o desenvolvimento de recursos e serviços de TA. Para alcançar este objetivo, foram definidas as seguintes questões de pesquisa:
\textcolor{red}{O principal objetivo desta pesquisa de doutorado é investigar como soluções de TA baseadas em ASR podem ampliar a acessibilidade de OAs audíveis, contribuindo para a inclusão de diferentes grupos de aprendizes. Nesse contexto, a \textit{Speech2Learning} é proposta como uma arquitetura que busca padronizar e simplificar o desenvolvimento de recursos e serviços de TA, respondendo à necessidade de ampliar o acesso a OAs em cenários educacionais diversos. Para atingir esse objetivo, foi definida a seguinte Questão de Pesquisa (QP), que orienta este estudo como um todo:}

\begin{itemize}
%\item \textbf{QP1:} Como uma arquitetura de software voltada para a acessibilidade de Objetos de Aprendizagem (OAs) audíveis pode ser desenvolvida para apoiar a inclusão educacional de diferentes grupos de usuários, promovendo maior igualdade de acesso à educação em diversos contextos?
\item \textcolor{red}{\textbf{QP Principal:} Como soluções de Tecnologia Assistiva (TA) baseadas em Reconhecimento Automático de Fala (ASR) podem ser projetadas e desenvolvidas para ampliar o acesso à educação e apoiar a inclusão de aprendizes com diferentes perfis?}
\end{itemize}

\textcolor{red}{Adicionalmente, para aprofundar questões específicas que sustentam a investigação da \textit{QP Principal}, foram formuladas duas questões complementares. Essas questões exploram aspectos técnicos e práticos por meio de estudos de caso, cujos detalhes são apresentados na \autoref{chapter1:methodological-path}, fornecendo insumos essenciais para responder ao problema central:}

\begin{itemize}
    \item \textcolor{red}{\textbf{QP do Estudo de Caso 1:}} De que maneira as tecnologias de ASR podem contribuir para melhorar a acessibilidade educacional?
    \begin{itemize}
        \item \textbf{Questão Avaliativa:} Qual é o nível de precisão dos serviços de ASR, oferecidos pelos principais provedores do mundo, nos processos de transcrição e legendagem de videoaulas?
    \end{itemize}
    
    \item \textcolor{red}{\textbf{QP do Estudo de Caso 2:}} Como as tecnologias de ASR podem ser adaptadas para atender às necessidades de acessibilidade de usuários da Libras?
    \begin{itemize}
        \item \textbf{Questão Avaliativa:} Qual é a percepção dos intérpretes de Libras em relação à precisão dos avatares de línguas de sinais baseados em texto, integrados às transcrições automáticas em um \textit{player} de vídeo com \textit{Design} Universal?
    \end{itemize}
\end{itemize}

\subsection*{Objetivos Específicos:}

\begin{enumerate}
% \item Pesquisar e identificar os requisitos fundamentais para a proposição de uma arquitetura de software voltada para a acessibilidade de OAs audíveis, considerando diferentes contextos educacionais e grupos de usuários.
\item \textcolor{red}{Pesquisar e identificar os requisitos fundamentais para o desenvolvimento de soluções baseadas em ASR que ampliem a acessibilidade de OAs audíveis, promovendo a inclusão educacional em diferentes contextos. Com base nesses requisitos, propor uma arquitetura de software que padronize e simplifique o desenvolvimento de soluções de TA baseadas em ASR.}
% \item Desenvolver \sigla{POCs}{Provas de Conceito} na indústria, visando aferir a viabilidade prática da arquitetura de software. Dentre as POCs, incluem-se:
\item Desenvolver \sigla{POCs}{Provas de Conceito} na indústria que demonstrem a viabilidade prática da arquitetura proposta, incluindo:
\begin{itemize}
\item Uma API para a transcrição e legendagem automática de videoaulas, com uma interface padronizada e de fácil integração a diferentes aplicações educacionais.
\item Um \textit{player} de vídeo baseado nos princípios do \textit{Design} Universal, capaz de integrar avatares de Libras baseados em texto, conectados às transcrições automáticas.
\end{itemize}
\item Avaliar as soluções desenvolvidas como POCs, incluindo as transcrições/legendas automáticas e a integração com avatares de Libras, por meio de estudos de caso que utilizem métodos quantitativos e qualitativos.
\item Investigar o potencial das soluções desenvolvidas, com base nos resultados e valor demonstrado pelos estudos de caso, visando aprimorar e expandir a arquitetura proposta para atender a mais contextos educacionais.
\end{enumerate}

\section{Percurso Metodológico}
\label{chapter1:methodological-path}

A proposta da \textit{Speech2Learning} foi consolidada após a condução de um MS focado em estudos sobre línguas de sinais e TICs, que evidenciou uma carência de soluções padronizadas e reutilizáveis em múltiplos contextos educacionais. Este estudo revelou uma oportunidade para o desenvolvimento de uma arquitetura de software que transcendesse as limitações específicas das línguas de sinais, promovendo uma estrutura genérica voltada para o desenvolvimento de recursos e/ou serviços de TA baseados em ASR, com o intuito de tornar os OAs audíveis mais acessíveis \cite{FalvoJr2020_FIE, FalvoJr2020_SBIE, FalvoJr2021_RENOTE}.

Dentro desse contexto, a arquitetura \textit{Speech2Learning} foi concebida como uma diretriz de software robusta e flexível, voltada para a transcrição automática e a geração de legendas de conteúdos educacionais, com o objetivo de melhorar a acessibilidade de OAs. \textcolor{red}{Essa abordagem buscou não apenas suprir a lacuna identificada no MS, mas também explorar o potencial do ASR como uma solução versátil e escalável, capaz de atender a diferentes perfis de aprendizes e contextos educacionais.}

A arquitetura foi avaliada por meio de dois estudos de caso realizados em parceria com a \textit{EdTech} brasileira DIO (\url{https://dio.me}). A DIO é uma plataforma de ensino com mais de 1 milhão de usuários, dedicada a capacitar profissionais em tecnologia, conectando-os com as empresas mais inovadoras do mundo por meio de uma metodologia educacional com foco em empregabilidade. Essa colaboração proporcionou acesso aos OAs e à infraestrutura necessários para implementar e avaliar as soluções propostas, que foram previamente validadas por meio de POCs conduzidas dentro da própria \textit{EdTech}. A seguir, os detalhes dos estudos de caso, incluindo suas especificidades metodológicas:

\begin{itemize}
\item \textbf{\textit{Estudo de Caso 1 -- Legendas Automáticas de Videoaulas}}: Implementação de uma API para a transcrição e legendagem de videoaulas. Essa API, desenvolvida conforme as diretrizes da \textit{Speech2Learning}, integrou serviços de ASR baseados em IA oferecidos por empresas líderes do setor, segundo o \citeonline{Gartner2023}: Amazon, Google, IBM, Microsoft e OpenAI (essa última devido ao seu destaque em soluções extremamente difundidas atualmente, como o ChatGPT). A precisão e qualidade das transcrições automáticas fornecidas por cada provedor foram avaliadas utilizando algoritmos de similaridade léxica e um \textit{Survey} que capturou as percepções dos usuários sobre a acurácia das legendas geradas. A combinação desses dados quantitativos com uma análise documental, que forneceu uma perspectiva qualitativa, foi essencial para uma abordagem de triangulação de dados, possibilitando uma avaliação abrangente das transcrições automáticas a partir de múltiplas perspectivas \cite{FalvoJr2023_HICSS, FalvoJr2024_FIE}.

\item \textbf{\textit{Estudo de Caso 2 -- Player de Vídeo com Avatar de Libras}}: Desenvolvimento de um \textit{player} de vídeo aderente ao conceito de \textit{Design} Universal \cite{GovBr2023}, projetado para a integração com avatares de línguas de sinais, como a Libras. A \sigla{Libras}{Língua Brasileira de Sinais} é uma língua gestual utilizada pela comunidade surda no Brasil, reconhecida legalmente desde 2002 \cite{Quadros2017, Quadros2019, Honora2021}. Nesta segunda instância da \textit{Speech2Learning}, o ASR foi combinado com avatares de Libras baseados em texto, demonstrando a sinergia dessas soluções de TA para tornar os OAs mais acessíveis. O potencial desta solução foi avaliado por meio de um \textit{Survey} e de entrevistas com intérpretes de Libras. Esses métodos permitiram a coleta de dados quantitativos, como a avaliação dos intérpretes sobre a qualidade das sinalizações dos avatares, e de dados qualitativos, que forneceram percepções detalhadas sobre a relevância do \textit{Player} de Vídeo em contextos educacionais para usuários da Libras. Assim, o estudo possibilitou uma análise interessante sobre a eficácia dos melhores avatares de Libras quando aplicados a videoaulas transcritas automaticamente.
\end{itemize}

A condução de estudos de caso, conforme definido por \citeonline{Sommerville2015, Pressman2016}, mostrou-se apropriada neste contexto, pois permitiu uma análise detalhada e contextualizada da aplicação prática da \textit{Speech2Learning} na plataforma educacional da DIO. Essa abordagem proporcionou uma compreensão aprofundada dos impactos, desafios e potencialidades da arquitetura na implementação de soluções acessíveis, oferecendo percepções valiosas para o desenvolvimento de TA adaptável a diversos contextos educacionais.

Os estudos de caso desta tese foram aprovados pelo \sigla{CEP}{Comitê de Ética e Pesquisa}, sob o CAAE 78381524.3.0000.5390. As questões éticas foram consideradas em todas as etapas do trabalho, com especial atenção às análises qualitativas que envolveram entrevistas com intérpretes de Libras no Estudo de Caso 2. Ressalta-se que ambos os estudos foram planejados e conduzidos com o rigor ético necessário para garantir a integridade e a relevância dos resultados.

Este trabalho, portanto, não apenas aborda um problema social relevante, mas também contribui para o campo da acessibilidade digital, fornecendo uma base sólida para o desenvolvimento de soluções em TA adaptáveis a diferentes contextos educacionais. Através da arquitetura \textit{Speech2Learning} e de suas instâncias implementadas, espera-se promover um impacto significativo na acessibilidade educacional, abrindo novos caminhos para a inclusão e a igualdade no acesso aos OAs.

\section{Organização}

Esta tese está organizada em cinco capítulos, além das referências e apêndices. Após esta introdução, que contextualiza a pesquisa e define seus objetivos, o \autoref{chapter2} explora a fundamentação teórica, apresentando os principais conceitos e estudos relacionados a este trabalho. O \autoref{chapter3} detalha a arquitetura \textit{Speech2Learning}, apresentando desde sua concepção até suas camadas e respectivas responsabilidades para promover OAs mais acessíveis. O \autoref{chapter4} apresenta a aplicação prática da arquitetura \textit{Speech2Learning} em dois estudos de caso, demonstrando suas possibilidades de implementação. Por fim, o \autoref{chapter5} consolida as conclusões e discute as principais perspectivas para a continuidade da pesquisa em trabalhos futuros.