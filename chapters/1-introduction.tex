\section{Contexto e Motivação}

A acessibilidade digital é um componente essencial para a inclusão social, especialmente no contexto educacional, onde a variedade de perfis e necessidades dos alunos demanda soluções genuinamente inclusivas. Segundo dados do \sigla{IBGE}{Instituto Brasileiro de Geografia e Estatística}, no Brasil, aproximadamente 8,9\% da população com mais de 2 anos, o que corresponde a 18,6 milhões de pessoas, possui algum tipo de deficiência \cite{IBGE2023}.

Nesse cenário, a área do conhecimento de \sigla{TA}{Tecnologia Assistiva} se apresenta como uma resposta natural. De acordo com \citeonline{Cook2020}, soluções de TA englobam uma variedade de dispositivos, serviços, recursos e práticas concebidos para melhorar as capacidades funcionais e a qualidade de vida das \sigla{PcD}{Pessoas com Deficiência}. Assim, a TA tem o potencial de ampliar o acesso a recursos educacionais, levando em conta as necessidades específicas de cada aprendiz \cite{UNESCO2023, GovBr2023}

O uso de \sigla{TICs}{Tecnologias de Informação e Comunicação}, que podem incluir recursos e serviços de TA, desempenha um papel essencial nesse contexto. Os relatórios da \citeonline{OMS2011, OMS2018} enfatizam que a promoção da acessibilidade digital por meio das TICs é vital para garantir a participação plena de todas as pessoas na sociedade e proporcionar oportunidades igualitárias de desenvolvimento educacional e profissional.

\sigla*{UNESCO}{Organização das Nações Unidas para a Educação, a Ciência e a Cultura}

Ainda no âmbito das TICs, as \sigla{IAs}{Inteligências Artificiais} surgem como ferramentas poderosas para promover a acessibilidade. A UNESCO destaca o potencial transformador das TICs na educação e ressalta a importância do \textit{Design} Universal na criação de soluções acessíveis a todos, independentemente de suas habilidades ou deficiências \cite{UNESCO2023, GovBr2023}. Nesse contexto, as IAs podem revolucionar a educação, fornecendo ferramentas que democratizam o acesso ao conhecimento e aprimoram a experiência de aprendizagem para todos os estudantes \cite{Holmes2019,UNESCO2024}.

No entanto, apesar do potencial das TICs e IAs na área do conhecimento de TA, o Brasil ainda enfrenta muitos desafios para a inclusão de PcD no ambiente educacional. Dados de 2022 indicam que 19,5\% das PcD estão fora da escola, uma taxa significativamente maior em comparação aos 4,1\% entre pessoas sem deficiência \cite{IBGE2023}. Essa disparidade é frequentemente agravada pela falta de recursos de TA, destacando a necessidade de investimentos em soluções que ampliem o acesso e a permanência de PcD em ambientes educacionais inclusivos.

Sob esse olhar, neste trabalho de doutorado é proposta a \textit{Speech2Learning}, uma arquitetura de software baseada em \sigla{ASR}{\textit{Automatic Speech Recognition}}, com o objetivo de promover a criação de \sigla{OAs}{Objetos de Aprendizagem} mais acessíveis utilizando reconhecimento de fala. Segundo \citeonline{Wiley2000}, OAs são recursos digitais reutilizáveis projetados para apoiar o processo de ensino-aprendizagem. Já o ASR, também conhecido como \sigla{STT}{\textit{Speech-to-Text}}, é uma subárea da IA que converte fala em texto \cite{Jurafsky2024}, possibilitando a acessibilidade em OAs audíveis, como videoaulas. A integração desses conceitos na \textit{Speech2Learning} não apenas beneficia PcD, mas qualquer aprendiz que possa se beneficiar do uso de legendas ou transcrições automáticas.

\section{Objetivos e Questões da Pesquisa}

O principal objetivo deste trabalho de doutorado é desenvolver e avaliar uma arquitetura de software que apoie a expansão da acessibilidade de objetos de aprendizagem audíveis, promovendo o desenvolvimento de recursos e serviços de TA. Para alcançar este objetivo, foram definidas as seguintes questões de pesquisa:

\begin{itemize}
    \item \textbf{QP1:} Como uma arquitetura de software voltada para a acessibilidade de Objetos de Aprendizagem (OAs) audíveis pode ser desenvolvida para apoiar a inclusão educacional de diferentes grupos de usuários, promovendo maior igualdade de acesso à educação em diversos contextos?
    
    \begin{itemize}
        \item \textbf{QP1.1:} De que maneira as tecnologias de Reconhecimento Automático de Fala (ASR) podem contribuir para melhorar a acessibilidade educacional?
        
        \begin{itemize}
            \item \textbf{Questão Avaliativa:} Qual é o nível de precisão dos serviços de ASR, oferecidos pelos principais provedores do mundo, nos processos de transcrição e legendagem de videoaulas?
        \end{itemize}
        
        \item \textbf{QP1.2:} Como as tecnologias de ASR podem ser adaptadas para atender às necessidades de acessibilidade de usuários da Libras?
        
        \begin{itemize}
            \item \textbf{Questão Avaliativa:} Qual é a percepção dos intérpretes de Libras em relação à precisão dos avatares de línguas de sinais baseados em texto, integrados às transcrições automáticas em um \textit{player} de vídeo com \textit{Design} Universal?
        \end{itemize}
    
    \end{itemize}

\end{itemize}

\subsection*{Objetivos Específicos:}

\begin{enumerate}
\item Investigar e identificar os requisitos fundamentais para a proposição de uma arquitetura de software voltada para a acessibilidade de OAs audíveis, considerando diferentes contextos educacionais e grupos de usuários.
\item Desenvolver \sigla{POCs}{Provas de Conceito} na indústria, visando aferir a viabilidade prática da arquitetura de software. Dentre as POCs, incluem-se:
\begin{itemize}
\item Uma API para a transcrição e legendagem automática de videoaulas, com uma interface padronizada e de fácil integração a diferentes aplicações educacionais.
\item Um \textit{player} de vídeo baseado nos princípios do \textit{Design} Universal, capaz de integrar avatares de Libras baseados em texto, conectados às transcrições automáticas.
\end{itemize}
\item Avaliar as soluções desenvolvidas como POCs, incluindo as transcrições/legendas automáticas e a integração com avatares de Libras, por meio de estudos de caso que utilizem métodos quantitativos e qualitativos.
\item Explorar o potencial de replicação e adaptação das soluções desenvolvidas para diferentes contextos educacionais, destacando a flexibilidade e aplicabilidade da arquitetura proposta para promover inclusão e acessibilidade.
\end{enumerate}

\section{Percurso Metodológico}

A proposta da \textit{Speech2Learning} foi consolidada após um \sigla{MS}{Mapeamento Sistemático} focado em estudos sobre línguas de sinais e TICs, que evidenciou uma carência de soluções padronizadas e reutilizáveis em múltiplos contextos educacionais. Esse estudo identificou a oportunidade de desenvolver uma arquitetura que transcendesse as limitações específicas das línguas de sinais, promovendo uma estrutura de software genérica para o desenvolvimento de recursos e/ou serviços de TA baseados no uso de ASR para OAs audíveis mais acessíveis \cite{FalvoJr2020_FIE, FalvoJr2020_SBIE, FalvoJr2021_RENOTE}.

Nesse contexto, a Arquitetura \textit{Speech2Learning} foi concebida para ser uma diretriz de software robusta e flexível, voltada para a transcrição automática e a geração de legendas de conteúdos educacionais, com o objetivo de melhorar a acessibilidade de OAs. A arquitetura foi avaliada por meio de dois estudos de caso realizados em parceria com a \textit{EdTech} brasileira DIO\footnote{Mais informações em \url{https://dio.me}}. Essa colaboração possibilitou o acesso aos OAs e à infraestrutura necessários para implementar e avaliar as soluções propostas, previamente validadas por meio de POCs conduzidas dentro da própria \textit{EdTech}. A seguir, os detalhes dos estudos de caso:

\begin{itemize}
\item \textbf{\textit{Estudo de Caso 1: Legendas Automáticas de Videoaulas}} - Implementação de uma API para a transcrição e legendagem de videoaulas. Esta API, desenvolvida com base nas diretrizes da \textit{Speech2Learning}, integrou serviços de ASR baseados em IA das empresas mais relevantes do mundo segundo o \citeonline{Gartner2023}: Amazon, Google, IBM, Microsoft e OpenAI. A precisão e qualidade das transcrições automáticas fornecidas por cada provedor foram avaliadas utilizando algoritmos de similaridade léxica e um \textit{Survey} que capturou as percepções dos usuários sobre a acurácia das legendas geradas. Dessa forma, as análises estatísticas incluíram tanto dados objetivos de similaridade léxica quanto dados subjetivos das percepções dos participantes do \textit{Survey}. Isso permitiu uma análise comparativa abrangente, combinando diferentes fontes de dados quantitativos para avaliar as transcrições automáticas de múltiplas perspectivas \cite{FalvoJr2023_HICSS, FalvoJr2024_FIE}.

\item \textbf{\textit{Estudo de Caso 2: Player de Vídeo com Avatar de Libras}} - Desenvolvimento de um \textit{player} de vídeo aderente ao conceito de \textit{Design} Universal \cite{GovBr2023}, projetado para a integração com avatares de línguas de sinais, como a Libras. A \sigla{Libras}{Língua Brasileira de Sinais} é uma língua gestual utilizada pela comunidade surda no Brasil, reconhecida legalmente desde 2002 \cite{Quadros2017, Quadros2019, Honora2021}. Essa segunda instância da \textit{Speech2Learning} combinou ASR com avatares de Libras baseados em texto, demonstrando a sinergia entre essas soluções de TA para tornar OAs mais acessíveis. A eficácia e o impacto desta solução foram avaliados por meio de um \textit{Survey} e de entrevistas com intérpretes de Libras. Estes métodos permitiram a coleta de dados quantitativos, como a percepção dos intérpretes sobre a qualidade das sinalizações dos avatares, e de dados qualitativos, através das entrevistas que forneceram \textit{insights} detalhados sobre a percepção e a usabilidade dos avatares. Assim, o estudo possibilitou uma análise abrangente sobre a relevância e a eficácia dos melhores avatares de Libras quando aplicados a videoaulas transcritas automaticamente.
\end{itemize}

A condução de estudos de caso, conforme definido por \citeonline{Sommerville2015, Pressman2016}, mostrou-se apropriada neste contexto, pois permitiu uma análise detalhada e contextualizada da aplicação prática da arquitetura \textit{Speech2Learning} em situações reais na \textit{EdTech} DIO. Esta abordagem proporcionou uma compreensão aprofundada dos impactos, desafios e potencialidades da arquitetura na implementação de soluções acessíveis, oferecendo percepções valiosas para o desenvolvimento de TA adaptável a diversos contextos educacionais.

Este trabalho, portanto, não apenas aborda um problema social relevante, mas também contribui para o campo da acessibilidade digital, oferecendo uma base sólida para o desenvolvimento de soluções em TA que podem ser construídas e adaptadas em diversos contextos educacionais. Através da Arquitetura \textit{Speech2Learning} e de suas instâncias implementadas, espera-se promover um impacto positivo significativo na acessibilidade educacional, abrindo novos caminhos para a inclusão e a igualdade no acesso à OAs.

\section{Organização}

Esta tese está organizada em cinco capítulos, além das referências e apêndices. Após a introdução, que contextualiza a pesquisa e define os objetivos, a fundamentação teórica é explorada no \autoref{chapter2}, discutindo os principais conceitos e trabalhos relacionados ao tema de pesquisa. O \autoref{chapter3} detalha a Arquitetura \textit{Speech2Learning}, apresentando desde sua concepção até suas camadas e respectivas responsabilidades para promover OAs mais acessíveis. O \autoref{chapter4} apresenta a aplicação prática da Arquitetura \textit{Speech2Learning} em dois estudos de caso, demonstrando suas possibilidades de implementação. Por fim, o \autoref{chapter5} consolida as conclusões e expande algumas discussões com foco em \textit{insights} para trabalhos futuros.