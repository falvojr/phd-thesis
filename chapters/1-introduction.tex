\section{Contexto e Motivação}

A acessibilidade digital é um elemento crucial para a inclusão social, especialmente no ambiente educacional, onde a diversidade de perfis e necessidades dos alunos requer soluções verdadeiramente inclusivas. No Brasil, o Instituto Brasileiro de Geografia e Estatística (IBGE) relata que aproximadamente 8,4\% da população, ou cerca de 18,6 milhões de pessoas, possuem algum tipo de deficiência \cite{IBGE2022}. Este cenário evidencia a necessidade de tecnologias assistivas que possam expandir o acesso a recursos educacionais, considerando as necessidades específicas de cada aprendiz. De acordo com a Organização Mundial de Saúde (OMS), promover a acessibilidade digital é fundamental para garantir que todas as pessoas possam participar plenamente na sociedade e ter oportunidades igualitárias de desenvolvimento educacional e profissional \cite{OMS2011, OMS2018}.

Em um contexto onde as Tecnologias de Informação e Comunicação (TICs) são essenciais, as Inteligências Artificiais (IAs) emergem como ferramentas poderosas para promover a acessibilidade. A Organização das Nações Unidas para a Educação, a Ciência e a Cultura (UNESCO) sublinha o potencial transformador das TICs na educação, destacando a importância do Design Universal para criar soluções acessíveis a todos, independentemente de suas habilidades ou deficiências \cite{UNESCO2023, GovBr2023}. Neste cenário, as IAs, em especial, têm o potencial de revolucionar a educação, proporcionando ferramentas que não apenas democratizam o acesso ao conhecimento, mas também melhoram a experiência de aprendizagem para todos os estudantes, conforme discutido por \citeonline{Holmes2019,UNESCO2024}.

Embora as TICs e IAs ofereçam um enorme potencial na criação de tecnologias assistivas, no Brasil ainda há muitos desafios para a inclusão de Pessoas com Deficiência (PcD) no ambiente educacional. Dados de 2022 mostram que 19,5\% das PcD estão fora da escola, uma taxa significativamente maior em comparação aos 4,1\% entre pessoas sem deficiência \cite{IBGE2022}. Essa disparidade é frequentemente exacerbada pela falta de tecnologias assistivas, reforçando a necessidade de investimentos em soluções que ampliem o acesso e a permanência de PcD em ambientes educacionais inclusivos.

Neste cenário, propomos a \textit{Speech2Learning}, uma Arquitetura de Software baseada em Reconhecimento de Fala Automático (ASR, na sigla em inglês), com o objetivo de promover a criação de Objetos de Aprendizagem (OAs) mais acessíveis. Segundo \citeonline{Wiley2000}, OAs são recursos digitais reutilizáveis projetados para apoiar o ensino e a aprendizagem. Já o ASR, também conhecido como \textit{Speech-to-Text} (STT), é uma subárea da IA que converte fala em texto, possibilitando a acessibilidade em OAs audíveis, como vídeoaulas \cite{Jurafsky2024}. A integração desses conceitos na \textit{Speech2Learning} não apenas beneficia PcD, mas quaisquer aprendizes que possam se beneficiar do uso de legendas e transcrições automáticas.

A proposta da \textit{Speech2Learning} foi consolidada após um Mapeamento Sistemático (MS) focado em estudos sobre línguas de sinais e TICs, que evidenciou uma carência de soluções padronizadas e reutilizáveis em múltiplos contextos educacionais. Esse estudo identificou a oportunidade de desenvolver uma arquitetura que transcendesse as limitações específicas das línguas de sinais, promovendo uma estrutura de software genérica para o desenvolvimento de tecnologias assistivas baseadas no uso de ASR para OAs mais acessíveis \cite{FalvoJr2020_FIE, FalvoJr2020_SBIE, FalvoJr2020_RENOTE}.

A Arquitetura \textit{Speech2Learning} foi concebida para ser uma solução de software robusta e flexível, voltada para a transcrição automática e a geração de legendas de conteúdos educacionais, com o objetivo de melhorar a acessibilidade de OAs. Este projeto de doutorado inclui a avaliação desta arquitetura por meio de dois estudos de caso em parceria com a \textit{EdTech} brasileira DIO (\url{https://dio.me}), que forneceu recursos e infraestrutura para a implementação e teste das soluções propostas:

\begin{itemize}
\item \textbf{\textit{Estudo de Caso 1: Legendas Automáticas de Videoaulas}} - Implementação de uma API REST para a transcrição e legendagem de vídeoaulas. Esta API, desenvolvida sob as diretrizes da \textit{Speech2Learning}, integrou múltiplos serviços de ASR de empresas como Amazon, Google, IBM, Microsoft e OpenAI. A precisão das transcrições automáticas foi avaliada por meio de análises estatísticas detalhadas e de um survey que capturou as percepções dos usuários quanto à qualidade e à acurácia das legendas geradas. A análise quantitativa incluiu medidas de similaridade léxica entre as transcrições e o texto original, oferecendo uma visão clara da eficácia de cada serviço de ASR \cite{FalvoJr2023_HICSS}.

\item \textbf{\textit{Estudo de Caso 2: Player de Vídeo com Avatar de Libras}} - Desenvolvimento de um player de vídeo aderente ao conceito de Design Universal \cite{GovBr2023}, projetado para a integração com avatares de línguas de sinais, como a Libras. A Libras, ou Língua Brasileira de Sinais, é uma língua gestual utilizada pela comunidade surda no Brasil, reconhecida legalmente desde 2002 \cite{Quadros2019, Honora2017}. Essa segunda instância da \textit{Speech2Learning} combinou ASR com avatares de Libras baseados em texto, demonstrando a sinergia entre diferentes tecnologias assistivas para OAs mais acessíveis. A eficácia e o impacto desta solução foram avaliados qualitativamente através de entrevistas com intérpretes de Libras, que forneceram \textit{insights} valiosos sobre a percepção e a usabilidade dos avatares, destacando a importância de soluções que promovam a inclusão da comunidade surda no processo de ensino-aprendizagem.
\end{itemize}

A utilização de estudos de caso, conforme definido por \citeonline{Sommerville2015, Pressman2016}, se mostrou apropriada neste contexto, pois permitiu uma análise detalhada e contextualizada da aplicação prática da arquitetura \textit{Speech2Learning} em situações reais na \textit{EdTech} DIO. Esta abordagem proporcionou uma compreensão aprofundada dos impactos, desafios e potencialidades da arquitetura na implementação de soluções acessíveis, oferecendo \textit{insights} valiosos para o desenvolvimento de tecnologias assistivas adaptáveis a diversos contextos educacionais.

Este trabalho, portanto, não apenas aborda um problema social relevante, mas também contribui para o campo da acessibilidade digital, oferecendo uma base sólida para a criação de tecnologias assistivas que podem ser construídas e adaptadas em diversos contextos educacionais. Através da Arquitetura \textit{Speech2Learning} e de suas instâncias implementadas, espera-se promover um impacto positivo significativo na acessibilidade educacional, abrindo novos caminhos para a inclusão e a igualdade no acesso à educação para todos.

\section{Objetivos e Questões da Pesquisa}

O principal objetivo deste trabalho de doutorado é desenvolver e avaliar a Arquitetura \textit{Speech2Learning}, com o intuito de expandir a acessibilidade de objetos de aprendizagem não apenas para a comunidade surda, mas também para um público mais amplo que necessita de soluções acessíveis. Para alcançar este objetivo, definimos as seguintes questões de pesquisa:

\begin{itemize}
\item \textbf{QP1:} Como a tecnologia de Reconhecimento de Fala Automática (ASR) pode ser utilizada para melhorar a acessibilidade de Objetos de Aprendizagem (OAs)?
\begin{itemize}
\item \textbf{QP1.1:} Qual é o nível de precisão dos serviços de ASR, oferecidos pelos principais provedores do mundo, nos processos de transcrição e legendagem de vídeoaulas?
\end{itemize}
\item \textbf{QP2:} Como um player de vídeo aderente ao conceito de Design Universal e preparado para a integração com avatares de línguas de sinais pode melhorar o processo de ensino-aprendizagem de usuários da Libras?
\begin{itemize}
\item \textbf{QP2.1:} Qual é a percepção dos intérpretes de Libras em relação à eficácia e à precisão dos avatares de línguas de sinais baseados em texto?
\end{itemize}
\item \textbf{QP3:} Como as soluções desenvolvidas a partir da Arquitetura \textit{Speech2Learning} podem ser replicadas e adaptadas para outros contextos educacionais, promovendo inclusão e igualdade no acesso à educação?
\end{itemize}

\textbf{Objetivos Específicos:}

\begin{enumerate}
\item Desenvolver uma instância da Arquitetura \textit{Speech2Learning} como uma API para a transcrição e legendagem automática de vídeoaulas, disponibilizando uma interface padronizada e de fácil integração para outras instâncias da arquitetura.
\item Desenvolver uma instância da Arquitetura \textit{Speech2Learning} como um player de vídeo aderente ao conceito de Design Universal, preparado para a integração com avatares de Libras baseados em texto, utilizando as transcrições e legendas geradas pela API.
\item Avaliar a precisão das transcrições e legendas automáticas, além da eficácia dos avatares de Libras, por meio de estudos de caso formais, utilizando avaliações quantitativas e qualitativas para investigar o impacto das tecnologias assistivas implementadas como instâncias da Arquitetura \textit{Speech2Learning}.
\item Explorar a possibilidade de replicação e adaptação das soluções desenvolvidas para outros contextos educacionais, evidenciando a flexibilidade e aplicabilidade da Arquitetura \textit{Speech2Learning}.
\end{enumerate}

\section{Organização}

Esta tese está organizada em cinco capítulos principais, além de referências e apêndices. Após a introdução, que contextualiza a pesquisa e define seus objetivos, a fundamentação teórica é explorada no \autoref{chapter2}, discutindo os principais conceitos e trabalhos relacionados ao tema de pesquisa. O \autoref{chapter3} detalha a Arquitetura \textit{Speech2Learning}, descrevendo sua concepção e camadas, além de suas respectivas responsabilidades para promover OAs mais acessíveis. O \autoref{chapter4} apresenta a aplicação prática da Arquitetura \textit{Speech2Learning} em dois estudos de caso, demonstrando suas possibilidades de implementação. Por fim, o \autoref{chapter5} consolida as conclusões e expande algumas discussões com foco em \textit{insights} para trabalhos futuros.