
\noindent
\textcolor{red}{
Dúvidas e Pendências:
\begin{itemize}
    \item TODO: Padronizar as Tabelas e Figuras em Português ou Inglês?
    \item TODO: \textit{Screenshots} da DIO no Estudo de Caso 1: Videoaulas com Legendadas?
    \item TODO: Retomar as Hipóteses e QPs nos Resultados dos Estudos de Caso ou na Conclusão?
\end{itemize}
}

\section{Contribuições da Pesquisa}

\section{Limitações da Pesquisa}

\section{Trabalhos Futuros}

\section{Publicações Resultantes}

As principais publicações resultantes das atividades conduzidas nesta pesquisa de doutorado são apresentadas a seguir, ordenadas cronologicamente:

\begin{enumerate}
    
    \item \fullcite{\textbf{FALVOJR, V.}; MARTINS FALVO, C. H.; SCATALON, L. P.; BARBOSA, E. F.}{Tecnologias Aplicadas ao Ensino e Aprendizagem de LIBRAS: Um Mapeamento Sistemático}{Simpósio Brasileiro de Informática na Educação (SBIE)}{2020}{Disponível em \url{doi.org/10.5753/cbie.sbie.2020.812}}

    \item \fullcite{\textbf{FALVOJR, V.}; SCATALON, L. P.; BARBOSA, E. F. }{The Role of Technology to Teaching and Learning Sign Languages: A Systematic Mapping}{Frontiers in Education Conference (FIE)}{2020}{Disponível em \url{doi.org/10.1109/FIE44824.2020.9274169}}

    \item \fullcite{\textbf{FALVOJR, V.}; MARTINS FALVO, C. H.; SCATALON, L. P.; BARBOSA, E. F.}{Tecnologias Aplicadas ao Ensino e Aprendizagem com Línguas de Sinais: Um Mapeamento Sistemático Sob as Perspectivas Nacional e Internacional}{Revista Novas Tecnologias na Educação (RENOTE)}{2021}{Disponível em \url{doi.org/10.22456/1679-1916.110217}}

    \item \fullcite{\textbf{FALVOJR, V.}; MARCOLINO, A.; BRUNO, D.; MARTINS FALVO, C.; OSÓRIO, F.; BARBOSA, E.}{Lexical Analysis of Automatic Transcriptions Using Speech-to-Text Services: A Statistically Evaluated Case Study}{Hawaii International Conference on System Sciences (HICSS)}{2024}{Disponível em \url{hdl.handle.net/10125/107023}}
\end{enumerate}

De modo complementar, outros trabalhos indiretamente relacionados a esta pesquisa foram publicados. Esta colaboração contínua entre pesquisadores, muitas vezes de diferentes instituições, nos levou a descobertas e \textit{insights} fundamentais para a idealização e desenvolvimento deste doutorado \cite{Soad2017_FIE,Oliveira2019_SBIE,FalvoJr2020_JUCS,FalvoJr2023_SMarty}:

\begin{enumerate}\setcounter{enumi}{4}
    
    \item \fullcite{SOAD, G. W.; FIORAVANTI, M. L.; \textbf{FALVOJR, V.}; MARCOLINO, A.; DUARTE FILHO, N. F.; BARBOSA, E. F.}{ReqML-catalog: The Road to a Requirements Catalog for Mobile Learning Applications}{Frontiers in Education Conference (FIE)}{2017}{Disponível em \url{doi.org/10.1109/FIE.2017.8190718}}
    
    \item \fullcite{OLIVEIRA, R.; \textbf{FALVOJR, V.}; BARBOSA, E. F.}{Internet das Coisas aplicada à Educação: Um Mapeamento Sistemático}{Simpósio Brasileiro de Informática na Educação (SBIE)}{2019}{Disponível em \url{doi.org/10.5753/cbie.sbie.2019.499}}
    
    \item \fullcite{\textbf{FALVOJR, V.}; MARCOLINO, A. da S.; FILHO, N. F. D.; OLIVEIRAJR, E.; BARBOSA, E. F.}{Variability-based Improvement of M-Learning Applications Development}{Journal of Universal Computer Science (J.UCS)}{2020}{Disponível em \url{doi.org/10.1007/978-3-031-18556-4_13}}
    
    \item \fullcite{\textbf{FALVOJR, V.}; MARCOLINO, A. S.; DUARTE FILHO, N. F.; OLIVEIRAJR, E.; BARBOSA, E. F.}{A Software Product Line for Mobile Learning Applications}{Capítulo 13 do Livro ``UML-Based Software Product Line Engineering with SMarty''}{2023}{Disponível em \url{doi.org/10.1007/978-3-031-18556-4_13}}
\end{enumerate}
    
Por fim, a seguinte publicação se encontra em processo de revisão:

\begin{enumerate}\setcounter{enumi}{8}
    \item \fullcite{\textbf{FALVOJR, V.}; MARCOLINO, A.; BRUNO, D.; MARTINS FALVO, C.; OSÓRIO, F.; BARBOSA, E.}{Enhancing Learning Objects Accessibility Through Speech-To-Text Based Architecture: A Comprehensive Triangulation Study}{Frontiers in Education (FIE)}{2024}{Submetido em 20/05/2024}
\end{enumerate}