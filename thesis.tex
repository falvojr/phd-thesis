% ------------------------------------------------------------------------
% ------------------------------------------------------------------------
% ICMC: Modelo de Trabalho Acadêmico (tese de doutorado, dissertação de
% mestrado e trabalhos monográficos em geral) em conformidade com 
% ABNT NBR 14724:2011: Informação e documentação - Trabalhos acadêmicos -
% Apresentação
% ------------------------------------------------------------------------
% ------------------------------------------------------------------------

% Opções: 
%   Qualificação          = qualificacao 
%   Curso                 = doutorado/mestrado
%   Situação do trabalho  = pre-defesa/pos-defesa (exceto para qualificação)
%   Versão para impressão = impressao
\documentclass[doutorado, pos-defesa]{packages/icmc}

% ---------------------------------------------------------------------------
% Pacotes Opcionais
% ---------------------------------------------------------------------------
\usepackage{rotating}           % Usado para rotacionar o texto
\usepackage[all,knot,arc,import,poly]{xy}   % Pacote para desenhos gráficos
% Este pacote pode conflitar com outros pacotes gráficos como o ``pictex''
% Então é necessário usar apenas um dos pacotes conflitantes
\newcommand{\VerbL}{0.52\textwidth}
\newcommand{\LatL}{0.42\textwidth}
% ---------------------------------------------------------------------------

% ---------------------------------------------------------------------------
% Pacotes e Comandos Adicionais (FalvoJr)
% ---------------------------------------------------------------------------
\usepackage{caption}
\usepackage{multirow}
\usepackage{fontawesome}

\newcolumntype{C}[1]{>{\centering}m{#1}}
\newcolumntype{P}[1]{>{\centering\arraybackslash}p{#1}}
\newcommand{\fullcite}[5]{%
  #1 \emph{#2}. \textbf{#3}. #4. #5.
}
% ---------------------------------------------------------------------------


% ---
% Informações de dados para CAPA e FOLHA DE ROSTO
% ---
% Tanto na capa quanto nas folhas de rosto apenas a primeira letra da primeira palavra (ou nomes próprios) devem estar em letra maiúscula, todas as demais devem ser em letra minúscula.
\tituloPT{Speech2Learning: Uma Arquitetura de Software Baseada em Reconhecimento de Fala para Promover a Acessibilidade de Objetos de Aprendizagem}
\tituloEN{Speech2Learning: A Speech Recognition-Based Software Architecture to Promote the Accessibility of Learning Objects}
\autor[FalvoJr, V.]{Venilton FalvoJr}
\genero{M} % Gênero do autor (M = Masculino / F = Feminino)
\orientador[Orientadora]{Profa. Dra.}{Ellen Francine Barbosa}
%\coorientador{Prof. Dr.}{Fulano de Tal}
\curso{CCMC}
\data{05}{01}{2025} % Data do depósito
\idioma{PT} % Idioma principal do documento (PT = português / EN = inglês)
% ---


% ---
% RESUMOS
% ---

% Resumo em PORTUGUÊS
% conter no máximo 500 palavras
% conter no mínimo 1 e no máximo 5 palavras-chave
\textoresumo[brazil]{

    \textbf{Introdução:} O acesso a Objetos de Aprendizagem (OAs) audíveis ainda é um desafio, especialmente para aprendizes que dependem de Tecnologia Assistiva (TA). Com o avanço do Reconhecimento Automático de Fala (ASR), surgem novas possibilidades para tornar os OAs mais acessíveis. Este trabalho visa abordar essa questão por meio de uma Arquitetura de Software que facilite a criação de recursos de TA baseados em ASR. \textbf{Objetivo:} O principal objetivo desta pesquisa foi desenvolver e avaliar uma Arquitetura de Software, denominada \textit{Speech2Learning}, destinada a promover soluções baseadas em ASR, visando ampliar a acessibilidade de OAs audíveis para diferentes aprendizes. \textbf{Métodos:} A metodologia incluiu um Mapeamento Sistemático (MS) que fundamentou a definição da \textit{Speech2Learning}. A arquitetura foi avaliada por meio de dois Estudos de Caso aplicados na indústria, em parceria com a \textit{EdTech} DIO. O primeiro estudo de caso investigou a precisão das transcrições automáticas dos principais serviços de ASR em videoaulas, utilizando uma triangulação de dados que combinou análises de similaridade léxica, respostas dos participantes de um \textit{survey} e uma análise documental adicional. O segundo estudo de caso avaliou um \textit{player} de vídeo integrado a avatares de Libras baseados em texto, alimentados por transcrições automáticas. Este \textit{player} foi testado funcionalmente com intérpretes de Libras, que forneceram \textit{feedback} qualitativo especializado. \textbf{Resultados:} No primeiro estudo de caso, os resultados mostraram que o serviço de ASR da OpenAI apresentou a maior precisão nas transcrições automáticas, destacada tanto nas análises estatísticas dos algoritmos de similaridade léxica quanto nas respostas dos participantes do \textit{survey}. A convergência dessas fontes de dados reforçou a relevância do ASR na promoção da acessibilidade de OAs audíveis. O segundo estudo de caso revelou que, embora tecnicamente viável, a integração de avatares de Libras com as transcrições automáticas apresentou desafios significativos, especialmente devido à complexidade cultural e linguística da Libras. \textbf{Conclusões:} Esta pesquisa contribuiu para a definição e avaliação de uma arquitetura genérica e adaptável a diferentes contextos educacionais. A condução dos estudos de caso na indústria agregou complexidade e realismo, gerando \textit{insights} valiosos para potenciais evoluções da \textit{Speech2Learning}. Como trabalhos futuros, estudos adicionais podem ser realizados para ampliar a amostragem e testar a arquitetura em novos contextos, aferindo sua relevância na promoção da acessibilidade educacional por meio de OAs audíveis.

    }{Arquitetura de Software, Reconhecimento Automático de Fala, Objetos de Aprendizagem, Tecnologia Assistiva, Acessibilidade Digital}


% resumo em INGLÊS
% conter no máximo 500 palavras
% conter no mínimo 1 e no máximo 5 palavras-chave
\textoresumo[english]{
    \textbf{Introduction:} Access to audible Learning Objects (LOs) remains a challenge, especially for learners who rely on Assistive Technology (AT). With advancements in Automatic Speech Recognition (ASR), new opportunities arise to make LOs more accessible. This work aims to address this issue through a Software Architecture that facilitates the creation of AT resources based on ASR. \textbf{Objective:} The main objective of this research was to develop and evaluate a Software Architecture, named \textit{Speech2Learning}, designed to promote ASR-based solutions, aiming to expand the accessibility of audible LOs for a diverse range of learners. \textbf{Methods:} The methodology included a Systematic Mapping (SM) that supported the definition of \textit{Speech2Learning}. The architecture was evaluated through two Case Studies conducted in the industry, in partnership with the \textit{EdTech} DIO. The first case study investigated the accuracy of automatic transcriptions from the main ASR services in video lectures, using data triangulation combining lexical similarity analyses, participants' responses from a survey, and additional document analysis. The second case study evaluated a video \textit{player} integrated with Libras avatars based on text, powered by automatic transcriptions. This \textit{player} was functionally tested with Libras interpreters, who provided specialized qualitative feedback. \textbf{Results:} In the first case study, the results showed that OpenAI's ASR service had the highest accuracy in automatic transcriptions, highlighted both in the statistical analyses of the lexical similarity algorithms and in the participants' survey responses. The convergence of these data sources reinforced the relevance of ASR in promoting the accessibility of audible LOs. The second case study revealed that, although technically feasible, the integration of Libras avatars with automatic transcriptions posed significant challenges, especially due to the cultural and linguistic complexity of Libras. \textbf{Conclusions:} This research contributed to the definition and evaluation of a generic and adaptable architecture for different educational contexts. The case studies conducted in the industry added complexity and realism, generating valuable insights for potential evolutions of \textit{Speech2Learning}. Future work may include additional studies to expand the sample size and test the architecture in new contexts, assessing its relevance in promoting educational accessibility through audible LOs. 
    }{Software Architecture, Automatic Speech Recognition, Learning Objects, Assistive Technology, Digital Accessibility}


% ----------------------------------------------------------
% ELEMENTOS PRÉ-TEXTUAIS
% ----------------------------------------------------------

% Inserir a ficha catalográfica
\incluifichacatalografica{tex/pre-textual/ficha-catalografica.pdf}

% DEDICATÓRIA / AGRADECIMENTO / EPÍGRAFE
\textodedicatoria*{tex/pre-textual/dedicatoria}
\textoagradecimentos*{tex/pre-textual/agradecimentos}
\textoepigrafe*{tex/pre-textual/epigrafe}

% Inclui a lista de figuras
\incluilistadefiguras

% Inclui a lista de tabelas
\incluilistadetabelas

% Inclui a lista de quadros
\incluilistadequadros

% Inclui a lista de algoritmos
%\incluilistadealgoritmos

% Inclui a lista de códigos
\incluilistadecodigos

% Inclui a lista de siglas e abreviaturas
\incluilistadesiglas

% Inclui a lista de símbolos
\incluilistadesimbolos

% ----
% Início do documento
% ----
\begin{document}
% ----------------------------------------------------------
% ELEMENTOS TEXTUAIS
% ----------------------------------------------------------
\textual

\chapter{Introdução}
\label{chapter1}
\section{Contexto e Motivação}

A acessibilidade digital é um componente essencial para a inclusão social, especialmente no contexto educacional, no qual a variedade de perfis e necessidades dos alunos demanda soluções genuinamente inclusivas. Segundo dados do \sigla{IBGE}{Instituto Brasileiro de Geografia e Estatística}, no Brasil, aproximadamente 8,9\% da população com mais de 2 anos, o que corresponde a 18,6 milhões de pessoas, possui algum tipo de deficiência \cite{IBGE2023}.

Nesse cenário, a área do conhecimento de \sigla{TA}{Tecnologia Assistiva} apresenta-se como uma resposta natural. De acordo com \citeonline{Cook2020}, soluções de TA englobam uma ampla gama de recursos, desde dispositivos e serviços até práticas voltadas para melhorar as capacidades funcionais e a qualidade de vida das \sigla{PcD}{Pessoas com Deficiência}. A TA é uma área interdisciplinar que combina produtos, metodologias e estratégias, todas direcionadas a aumentar a autonomia e a participação das PcD em suas atividades cotidianas \cite{Cat2009}.

Além disso, o \sigla{CAT}{Comitê de Ajudas Técnicas} do Brasil destaca que a TA desempenha um papel essencial na promoção da autonomia, independência e inclusão social dessas pessoas. As soluções de TA vão além de artefatos tecnológicos, abrangendo também serviços e práticas adaptadas às necessidades específicas dos usuários em diferentes contextos \cite{Cat2009}. Dessa forma, a TA tem o potencial de ampliar o acesso a recursos educacionais, considerando as particularidades de cada aprendiz \cite{UNESCO2023, GovBr2023}.

O uso de \sigla{TICs}{Tecnologias de Informação e Comunicação}, que podem incluir recursos e serviços de TA, desempenha um papel essencial nesse contexto. Os relatórios da \citeonline{OMS2011, OMS2018} enfatizam que a promoção da acessibilidade digital por meio das TICs é vital para garantir a participação plena de todas as pessoas na sociedade e proporcionar oportunidades igualitárias de desenvolvimento educacional e profissional.

Ainda no âmbito das TICs, as \sigla{IAs}{Inteligências Artificiais} surgem como ferramentas poderosas para promover a acessibilidade. A \sigla{UNESCO}{Organização das Nações Unidas para a Educação, a Ciência e a Cultura} destaca o potencial transformador das TICs na educação e ressalta a importância do \textit{Design} Universal na criação de soluções acessíveis a todos, independentemente de suas habilidades ou deficiências \cite{UNESCO2023, GovBr2023}. Nesse contexto, as IAs podem revolucionar a educação, fornecendo ferramentas que democratizam o acesso ao conhecimento e aprimoram a experiência de aprendizagem para todos os estudantes \cite{Holmes2019,UNESCO2024}.

No entanto, apesar do potencial das TICs e IAs na área do conhecimento de TA, o Brasil ainda enfrenta muitos desafios para a inclusão de PcD no ambiente educacional. Dados de 2022 indicam que 19,5\% das PcD estão fora da escola, uma taxa significativamente maior em comparação aos 4,1\% entre pessoas sem deficiência \cite{IBGE2023}. Essa disparidade é agravada pela carência de recursos de TA, evidenciando a urgência de investimentos em soluções que promovam o acesso e garantam a permanência de PcD em ambientes educacionais, idealmente inclusivos.

\textcolor{red}{Nesse contexto, um \sigla{MS}{Mapeamento Sistemático}, focado na intersecção entre línguas de sinais e TICs, revelou desafios significativos, como a complexidade e a falta de padronização no desenvolvimento de soluções tecnológicas voltadas à acessibilidade educacional \cite{FalvoJr2020_FIE, FalvoJr2020_SBIE, FalvoJr2021_RENOTE}. Embora as línguas de sinais apresentem necessidades específicas, o MS evidenciou a viabilidade de integrar a tecnologia de \sigla{ASR}{\textit{Automatic Speech Recognition}} como recurso assistivo, ampliando o acesso a \sigla{OAs}{Objetos de Aprendizagem} audíveis em diversos contextos.}

\textcolor{red}{Sob essa perspectiva, este trabalho de doutorado propõe a \textit{Speech2Learning}, uma arquitetura de software baseada em ASR, com o objetivo de facilitar a criação de soluções de TA para OAs mais acessíveis, utilizando reconhecimento de fala. Segundo \citeonline{Wiley2000}, OAs são recursos digitais reutilizáveis projetados para apoiar o processo de ensino-aprendizagem. Já o ASR, também conhecido como \sigla{STT}{\textit{Speech-to-Text}}, é uma subárea da IA que converte fala em texto \cite{Jurafsky2024}, proporcionando maior acessibilidade a OAs audíveis, como videoaulas. A integração desses conceitos na \textit{Speech2Learning} não apenas apoia PcD no acesso à educação, mas também beneficia qualquer aprendiz que utilize legendas ou transcrições automáticas.}

\textcolor{red}{Dessa forma, a motivação deste estudo origina-se nas lacunas identificadas pelo MS, que destacaram a necessidade de soluções replicáveis e alinhadas às novas tendências tecnológicas para promover uma educação mais inclusiva \cite{FalvoJr2021_RENOTE}. As reflexões resultantes apontaram o ASR como uma abordagem promissora para ampliar a acessibilidade de OAs audíveis em diversos contextos educacionais. Com os avanços em IA, especialmente em modelos de reconhecimento de fala, tornou-se possível projetar uma arquitetura de software capaz de atender a essas demandas.}

\textcolor{red}{A \textit{Speech2Learning} é uma resposta prática a esse cenário, utilizando o ASR para superar barreiras no acesso à informação e expandir a acessibilidade de OAs audíveis para um público de aprendizes mais diverso. Ao longo deste trabalho, foram criadas duas instâncias concretas da \textit{Speech2Learning}, que permitiram avaliar a arquitetura por meio de estudos de caso aplicados na indústria. Essas instâncias, além de servirem como uma avaliação prática da arquitetura, estabeleceram-se como recursos de TA ao expandirem o alcance dos OAs audíveis, contribuindo para um processo de ensino-aprendizagem mais acessível.}

\section{Objetivos e Questões da Pesquisa}
\label{chapter1:research-questions}

%O principal objetivo deste trabalho de doutorado é desenvolver e avaliar uma arquitetura de software que promova a melhoria na acessibilidade de OAs audíveis, promovendo o desenvolvimento de recursos e serviços de TA. Para alcançar este objetivo, foram definidas as seguintes questões de pesquisa:
\textcolor{red}{O principal objetivo desta pesquisa de doutorado é investigar como soluções de TA baseadas em ASR podem ampliar a acessibilidade de OAs audíveis, contribuindo para a inclusão de diferentes grupos de aprendizes. Nesse contexto, a \textit{Speech2Learning} é proposta como uma arquitetura que busca padronizar e simplificar o desenvolvimento de recursos e serviços de TA, respondendo à necessidade de ampliar o acesso a OAs em cenários educacionais diversos. Para atingir esse objetivo, foi definida a seguinte Questão de Pesquisa (QP), que orienta este estudo como um todo:}

\begin{itemize}
%\item \textbf{QP1:} Como uma arquitetura de software voltada para a acessibilidade de Objetos de Aprendizagem (OAs) audíveis pode ser desenvolvida para apoiar a inclusão educacional de diferentes grupos de usuários, promovendo maior igualdade de acesso à educação em diversos contextos?
\item \textcolor{red}{\textbf{QP Principal:} Como soluções de Tecnologia Assistiva (TA) baseadas em Reconhecimento Automático de Fala (ASR) podem ser projetadas e desenvolvidas para ampliar o acesso à educação e apoiar a inclusão de aprendizes com diferentes perfis?}
\end{itemize}

\textcolor{red}{Adicionalmente, para aprofundar questões específicas que sustentam a investigação da \textit{QP Principal}, foram formuladas duas questões complementares. Essas questões exploram aspectos técnicos e práticos por meio de estudos de caso, cujos detalhes são apresentados na \autoref{chapter1:methodological-path}, fornecendo insumos essenciais para responder ao problema central:}

\begin{itemize}
    \item \textcolor{red}{\textbf{QP do Estudo de Caso 1:}} De que maneira as tecnologias de ASR podem contribuir para melhorar a acessibilidade educacional?
    \begin{itemize}
        \item \textbf{Questão Avaliativa:} Qual é o nível de precisão dos serviços de ASR, oferecidos pelos principais provedores do mundo, nos processos de transcrição e legendagem de videoaulas?
    \end{itemize}
    
    \item \textcolor{red}{\textbf{QP do Estudo de Caso 2:}} Como as tecnologias de ASR podem ser adaptadas para atender às necessidades de acessibilidade de usuários da Libras?
    \begin{itemize}
        \item \textbf{Questão Avaliativa:} Qual é a percepção dos intérpretes de Libras em relação à precisão dos avatares de línguas de sinais baseados em texto, integrados às transcrições automáticas em um \textit{player} de vídeo com \textit{Design} Universal?
    \end{itemize}
\end{itemize}

\subsection*{Objetivos Específicos:}

\begin{enumerate}
% \item Pesquisar e identificar os requisitos fundamentais para a proposição de uma arquitetura de software voltada para a acessibilidade de OAs audíveis, considerando diferentes contextos educacionais e grupos de usuários.
\item \textcolor{red}{Pesquisar e identificar os requisitos fundamentais para o desenvolvimento de soluções baseadas em ASR que ampliem a acessibilidade de OAs audíveis, promovendo a inclusão educacional em diferentes contextos. Com base nesses requisitos, propor uma arquitetura de software que padronize e simplifique o desenvolvimento de soluções de TA baseadas em ASR.}
% \item Desenvolver \sigla{POCs}{Provas de Conceito} na indústria, visando aferir a viabilidade prática da arquitetura de software. Dentre as POCs, incluem-se:
\item Desenvolver \sigla{POCs}{Provas de Conceito} na indústria que demonstrem a viabilidade prática da arquitetura proposta, incluindo:
\begin{itemize}
\item Uma API para a transcrição e legendagem automática de videoaulas, com uma interface padronizada e de fácil integração a diferentes aplicações educacionais.
\item Um \textit{player} de vídeo baseado nos princípios do \textit{Design} Universal, capaz de integrar avatares de Libras baseados em texto, conectados às transcrições automáticas.
\end{itemize}
\item Avaliar as soluções desenvolvidas como POCs, incluindo as transcrições/legendas automáticas e a integração com avatares de Libras, por meio de estudos de caso que utilizem métodos quantitativos e qualitativos.
\item Investigar o potencial das soluções desenvolvidas, com base nos resultados e valor demonstrado pelos estudos de caso, visando aprimorar e expandir a arquitetura proposta para atender a mais contextos educacionais.
\end{enumerate}

\section{Percurso Metodológico}
\label{chapter1:methodological-path}

A proposta da \textit{Speech2Learning} foi consolidada após a condução de um MS focado em estudos sobre línguas de sinais e TICs, que evidenciou uma carência de soluções padronizadas e reutilizáveis em múltiplos contextos educacionais. Este estudo revelou uma oportunidade para o desenvolvimento de uma arquitetura de software que transcendesse as limitações específicas das línguas de sinais, promovendo uma estrutura genérica voltada para o desenvolvimento de recursos e/ou serviços de TA baseados em ASR, com o intuito de tornar os OAs audíveis mais acessíveis \cite{FalvoJr2020_FIE, FalvoJr2020_SBIE, FalvoJr2021_RENOTE}.

Dentro desse contexto, a arquitetura \textit{Speech2Learning} foi concebida como uma diretriz de software robusta e flexível, voltada para a transcrição automática e a geração de legendas de conteúdos educacionais, com o objetivo de melhorar a acessibilidade de OAs. \textcolor{red}{Essa abordagem buscou não apenas suprir a lacuna identificada no MS, mas também explorar o potencial do ASR como uma solução versátil e escalável, capaz de atender a diferentes perfis de aprendizes e contextos educacionais.}

A arquitetura foi avaliada por meio de dois estudos de caso realizados em parceria com a \textit{EdTech} brasileira DIO (\url{https://dio.me}). A DIO é uma plataforma de ensino com mais de 1 milhão de usuários, dedicada a capacitar profissionais em tecnologia, conectando-os com as empresas mais inovadoras do mundo por meio de uma metodologia educacional com foco em empregabilidade. Essa colaboração proporcionou acesso aos OAs e à infraestrutura necessários para implementar e avaliar as soluções propostas, que foram previamente validadas por meio de POCs conduzidas dentro da própria \textit{EdTech}. A seguir, os detalhes dos estudos de caso, incluindo suas especificidades metodológicas:

\begin{itemize}
\item \textbf{\textit{Estudo de Caso 1 -- Legendas Automáticas de Videoaulas}}: Implementação de uma API para a transcrição e legendagem de videoaulas. Essa API, desenvolvida conforme as diretrizes da \textit{Speech2Learning}, integrou serviços de ASR baseados em IA oferecidos por empresas líderes do setor, segundo o \citeonline{Gartner2023}: Amazon, Google, IBM, Microsoft e OpenAI (essa última devido ao seu destaque em soluções extremamente difundidas atualmente, como o ChatGPT). A precisão e qualidade das transcrições automáticas fornecidas por cada provedor foram avaliadas utilizando algoritmos de similaridade léxica e um \textit{Survey} que capturou as percepções dos usuários sobre a acurácia das legendas geradas. A combinação desses dados quantitativos com uma análise documental, que forneceu uma perspectiva qualitativa, foi essencial para uma abordagem de triangulação de dados, possibilitando uma avaliação abrangente das transcrições automáticas a partir de múltiplas perspectivas \cite{FalvoJr2023_HICSS, FalvoJr2024_FIE}.

\item \textbf{\textit{Estudo de Caso 2 -- Player de Vídeo com Avatar de Libras}}: Desenvolvimento de um \textit{player} de vídeo aderente ao conceito de \textit{Design} Universal \cite{GovBr2023}, projetado para a integração com avatares de línguas de sinais, como a Libras. A \sigla{Libras}{Língua Brasileira de Sinais} é uma língua gestual utilizada pela comunidade surda no Brasil, reconhecida legalmente desde 2002 \cite{Quadros2017, Quadros2019, Honora2021}. Nesta segunda instância da \textit{Speech2Learning}, o ASR foi combinado com avatares de Libras baseados em texto, demonstrando a sinergia dessas soluções de TA para tornar os OAs mais acessíveis. O potencial desta solução foi avaliado por meio de um \textit{Survey} e de entrevistas com intérpretes de Libras. Esses métodos permitiram a coleta de dados quantitativos, como a avaliação dos intérpretes sobre a qualidade das sinalizações dos avatares, e de dados qualitativos, que forneceram percepções detalhadas sobre a relevância do \textit{Player} de Vídeo em contextos educacionais para usuários da Libras. Assim, o estudo possibilitou uma análise interessante sobre a eficácia dos melhores avatares de Libras quando aplicados a videoaulas transcritas automaticamente.
\end{itemize}

A condução de estudos de caso, conforme definido por \citeonline{Sommerville2015, Pressman2016}, mostrou-se apropriada neste contexto, pois permitiu uma análise detalhada e contextualizada da aplicação prática da \textit{Speech2Learning} na plataforma educacional da DIO. Essa abordagem proporcionou uma compreensão aprofundada dos impactos, desafios e potencialidades da arquitetura na implementação de soluções acessíveis, oferecendo percepções valiosas para o desenvolvimento de TA adaptável a diversos contextos educacionais.

Os estudos de caso desta tese foram aprovados pelo \sigla{CEP}{Comitê de Ética e Pesquisa}, sob o CAAE 78381524.3.0000.5390. As questões éticas foram consideradas em todas as etapas do trabalho, com especial atenção às análises qualitativas que envolveram entrevistas com intérpretes de Libras no Estudo de Caso 2. Ressalta-se que ambos os estudos foram planejados e conduzidos com o rigor ético necessário para garantir a integridade e a relevância dos resultados.

Este trabalho, portanto, não apenas aborda um problema social relevante, mas também contribui para o campo da acessibilidade digital, fornecendo uma base sólida para o desenvolvimento de soluções em TA adaptáveis a diferentes contextos educacionais. Através da arquitetura \textit{Speech2Learning} e de suas instâncias implementadas, espera-se promover um impacto significativo na acessibilidade educacional, abrindo novos caminhos para a inclusão e a igualdade no acesso aos OAs.

\section{Organização}

Esta tese está organizada em cinco capítulos, além das referências e apêndices. Após esta introdução, que contextualiza a pesquisa e define seus objetivos, o \autoref{chapter2} explora a fundamentação teórica, apresentando os principais conceitos e estudos relacionados a este trabalho. O \autoref{chapter3} detalha a arquitetura \textit{Speech2Learning}, apresentando desde sua concepção até suas camadas e respectivas responsabilidades para promover OAs mais acessíveis. O \autoref{chapter4} apresenta a aplicação prática da arquitetura \textit{Speech2Learning} em dois estudos de caso, demonstrando suas possibilidades de implementação. Por fim, o \autoref{chapter5} consolida as conclusões e discute as principais perspectivas para a continuidade da pesquisa em trabalhos futuros.

\chapter{Fundamentação Teórica}
\label{chapter2}
\section{Considerações Iniciais}

A fundamentação teórica deste doutorado tem como base um MS, cujo objetivo foi identificar a interseção entre as TICs e as línguas de sinais no contexto educacional. A \autoref{section:foundation:sm} oferece um resumo deste MS, enquanto análises mais detalhadas dos estudos primários, que abrangem tanto o panorama nacional quanto internacional, estão disponíveis em uma série de publicações \cite{FalvoJr2020_FIE, FalvoJr2020_SBIE, FalvoJr2021_RENOTE}, que foram essenciais para o desenvolvimento do trabalho.

A realização deste estudo permitiu identificar \textit{gaps} tecnológicos no uso de línguas de sinais no ensino-aprendizagem, destacando a necessidade de novas pesquisas na literatura. De modo complementar ao MS, um levantamento bibliográfico foi conduzido para explorar conceitos e TICs promissoras que possam enfrentar esses desafios. As próximas seções aprofundam as temáticas de Arquiteturas de Software, OAs e ASR. Esses tópicos fornecem a base teórica para a \textit{Speech2Learning}, uma arquitetura detalhada no \autoref{chapter3}, projetada para tornar OAs audíveis mais acessíveis por meio de ASR.

\section{Mapeamento Sistemático: TICs e Línguas de Sinais na Educação}
\label{section:foundation:sm}

Para definir o escopo deste projeto, foi realizado um estudo sistemático de literatura para identificar lacunas e oportunidades tecnológicas no processo de ensino-aprendizagem com línguas de sinais. De acordo com \citeonline{Kitchenham2007}, existem duas abordagens principais para este tipo de estudo: revisão ou mapeamento sistemático. Optou-se pelo MS devido à sua capacidade de apresentar evidências de um domínio de estudo em um alto nível de granularidade, agrupando-as em áreas de similaridade e identificando tendências emergentes.

O protocolo de pesquisa para o MS foi cuidadosamente definido com base em diretrizes formais  \cite{Kitchenham2007, Nakagawa2010, Zhang2011, Petersen2015}. A abordagem de \citeonline{Zhang2011} foi particularmente relevante, pois orientou a estratégia de busca e os critérios de qualidade adotados no estudo. Essa estratégia foi adaptada para aumentar o rigor do processo de pesquisa, incorporando o \sigla{QGS}{\textit{Quasi-Gold Standard}} e seguindo boas práticas recomendadas na literatura (\autoref{ms:zhang-approach}).

\begin{figure}[htb]
\centering 
\caption{Busca Sistemática Baseada em QGS.}
\label{ms:zhang-approach}
\includegraphics[width=0.675\textwidth]{images/chapter2-sm-zhang-approach.png}
\fadaptada{Zhang2011}
\end{figure}

\subsection{Definição do Escopo e Critérios de Seleção}
\label{ms:conducao-escopo}

As \sigla{QP}{Questões de Pesquisa} são essenciais para definir o escopo e identificar possíveis palavras-chave em um estudo sistemático de literatura \cite{Kitchenham2007,Petersen2015}. Neste contexto, uma abordagem comum se dá através da aplicação dos critérios de PICO \cite{Petticrew2008}. O \autoref{quadro:c2:pico} representa o PICO, que derivaram as seguintes QP que definem o escopo deste MS.

\begin{quadro}[htb]
\centering
\caption{Critérios de PICO.}
\label{quadro:c2:pico}
\begin{tabularx}{\textwidth}{l|X} \hline
\textit{\textbf{P}opulation} & Aprendizes/Educadores interessados em línguas de sinais. \\ \hline
\textit{\textbf{I}ntervention} & TICs relevantes no processo de ensino-aprendizagem com línguas de sinais. \\ \hline
\textit{\textbf{C}omparison} & Não se aplica. \\ \hline
\textit{\textbf{O}utcome} & Panorama tecnológico sobre o ensino-aprendizagem com línguas de sinais. \\ \hline
\end{tabularx}
\end{quadro}

\begin{itemize}
    \setlength\itemsep{0em}
    \item \textbf{QP1}: Quais soluções tecnológicas vêm sendo propostas no processo de ensino-aprendizagem com línguas de sinais?
    % \begin{itemize}
    %     \item Quais são os tipos de soluções propostas (software ou hardware ou teóricas)?
    %     \item Quais tecnologias foram usadas?
    %     \item Quais métodos de avaliação foram aplicados?
    % \end{itemize}
    \item \textbf{QP2}: Quais tópicos educacionais são abordados?
    \item \textbf{QP3}: Quais línguas de sinais são abordadas?
    % \begin{itemize}
    %     \item Quais estudos abordam múltiplas línguas de sinais?
    % \end{itemize}
\end{itemize}

Segundo \citeonline{Kitchenham2007,Petersen2015}, os estudos sistemáticos requerem critérios explícitos de inclusão e exclusão para avaliar seus potenciais estudos primários. Assim, foram definidos os seguintes critérios de seleção (\autoref{quadro:c2:criterios-selecao}):

\begin{quadro}[htb]
\centering
\caption{Critérios de Inclusão (CI) e Exclusão (CE).}
\label{quadro:c2:criterios-selecao}
\begin{tabularx}{\textwidth}{l|X} \hline
\textbf{CI1} & Os estudos apresentam contribuições (software ou hardware ou teóricas) para o ensino e a aprendizagem de línguas de sinais. \\ \hline
\textbf{CE1} & Estudos que não foram publicados no período de 2000 a 2019, seguindo um racional semelhante à \citeonline{Radermacher2013,Scatalon2019}, os quais sugerem que estudos anteriores a 2000 não representam as abordagens educacionais atuais, especialmente considerando o contexto de tecnologia. \\ \hline
\textbf{CE2} & Estudos classificados como resumos, resumos de conferências/editoriais, literatura cinza ou capítulos de livros. \\ \hline
\textbf{CE3} & Estudos não apresentados em inglês ou português. \\ \hline
\textbf{CE4} & Estudos não acessíveis em texto completo. \\ \hline
\textbf{CE5} & Estudos duplicados ou superficialmente complementares de outros estudos. \\ \hline
\end{tabularx}
\end{quadro}

\subsection{Condução das Buscas Manual e Automatizada}
\label{ms:conducao-busca-manual}

No contexto das buscas manuais, a \autoref{table:c2:busca-manual-nacional} lista as conferências e periódicos nacionais analisados. No entanto, os estudos dessas fontes não foram incluídos na composição do QGS devido à limitação de indexação nos mecanismos de busca internacionais, o que poderia comprometer a eficácia da abordagem sistemática baseada em QGS \cite{Zhang2011}. Apesar disso, \textbf{46 estudos primários de fontes brasileiras foram selecionados} e discutidos nos resultados do MS. 

Por sua vez, a \autoref{table:c2:busca-manual-internacional} apresenta as conferências e periódicos internacionais selecionados durante a busca manual, resultando em 19 estudos primários que compõem o QGS deste MS. As fontes relevantes foram utilizadas para a busca automatizada, garantindo uma sinergia maior com o QGS, conforme recomendado por \citeonline{Zhang2011}.

\begin{table}[htb]
\centering
\caption{Busca Manual Nacional.}
\label{table:c2:busca-manual-nacional}
\begin{tabular}{l|c|c} \hline
\textbf{Conferências/Periódicos} & \textbf{Fonte} & \textbf{Selecionados} \\ \hline
DesafIE                          & CEIE           & 0                     \\
JAIE                             & CEIE           & 0                     \\
RBIE                             & CEIE           & 2                     \\
RENOTE                           & CINTED         & 13                    \\
SBIE                             & CEIE           & 13                    \\
WAVE2                            & CEIE           & 0                     \\
WCBIE                            & CEIE           & 11                    \\
WIE                              & CEIE           & 7                     \\ \hline
\multicolumn{2}{l}{\textbf{Total}}                & \textbf{46}           \\ \hline
\end{tabular}
\end{table}

\begin{table}[htb]
\centering
\caption{Busca Manual Internacional (Equivalente ao QGS).}
\label{table:c2:busca-manual-internacional}
\begin{tabular}{l|c|c} \hline
\textbf{Conferência/Periódico} & \textbf{Fonte}     & \textbf{Selecionados (QGS)} \\ \hline
ACM TOCE                       & ACM                & 0            \\ 
Computers \& Education         & Elsevier           & 5            \\ 
FIE                            & IEEE               & 0            \\ 
HCI International              & Springer           & 5            \\ 
ICALT                          & IEEE               & 5            \\ 
IEEE ToE                       & IEEE               & 1            \\ 
IEEE TLT                       & IEEE               & 0            \\ 
Informatics in Education       & Vilnius University & 0            \\ 
ITiCSE                         & ACM                & 2            \\ 
Learning @ Scale               & ACM                & 0            \\ 
SIGCSE                         & ACM                & 1            \\ \hline
\multicolumn{2}{l}{\textbf{Total}}                  & \textbf{19}  \\ \hline
\end{tabular}
\end{table}

Tendo em vista a busca automatizada, duas estratégias para identificação de palavras-chave foram utilizadas em conjunto para a string de busca: (i) análise do PICO e suas respectivas QP; (ii) importação da tripla \textit{title-abstract-keywords} em um software de análise de frequência. Os resultados desse processo produziram a seguinte string de busca (\autoref{codigo:string_busca_ms}).

\begin{codigo}[caption={String de Busca do MS}, label={codigo:string_busca_ms}]
    (learn OR learning OR teach OR teaching) AND
    ("sign language" OR "signed language") AND
    (technology OR technologies)
\end{codigo}

A \autoref{table:c2:automated-search} resume os resultados da busca automatizada, onde a seleção dos estudos seguiu o mesmo racional apresentado na busca manual. Além disso, a busca automatizada retornou a maioria dos estudos selecionados pela busca manual (QGS), o que sugere uma boa sensibilidade da string de busca. Nesse sentido, \citeonline{Zhang2011} propõem o conceito de \textit{quasi-sensibility}, uma derivação da sensibilidade tradicional que incorpora o QGS como critério de qualidade (\autoref{method:equation:quasi-sensitivity}).

\begin{table}[htb]
\centering
\caption{Resultados da Busca Automatizada.}
\label{table:c2:automated-search}
\begin{tabular}{ll|lll} \hline
 &  & Busca Final &                 &                   \\ \cline{3-5} 
Base de Dados & \textbf{QGS} & Recuperados    & \textbf{no QGS} & \textbf{Relevantes} \\ \hline
ACM DigitalLibrary & 3            & 922          & 3               & 47                \\
IEEE Xplore        & 6            & 359          & 5               & 59                \\
ScienceDirect      & 5            & 1,961        & 5               & 20                \\
SpringerLink       & 5            & 4,980        & 5               & 36                \\ \hline
\multicolumn{1}{l}{\textbf{Total}}   & \textbf{19}  & 8,222        & \textbf{18}     & \textbf{162}      \\ \hline
\end{tabular}
\end{table}

\begin{equation}
\label{method:equation:quasi-sensitivity}
\text{\textit{quasi-sensibility}} = \frac{\text{\textit{Estudos relevantes recuperados (\textbf{no QGS})}}}{\text{\textit{Total de estudos relevantes (\textbf{QGS})}}}
\end{equation}

Como resultado, a \textit{quasi-sensitivity} calculada foi de 94,74\% (18/19), um desempenho adequado segundo \citeonline{Zhang2011}. Portanto, os 163 artigos selecionados pelas buscas (manual internacional e automatizada) foram considerados estudos primários em potencial. Nesta etapa, 24 estudos foram excluídos de acordo com os critérios de inclusão e exclusão pré-estabelecidos. Sendo assim, a \autoref{method:figure:evaluation-refinement} organiza os \textbf{139 estudos primários selecionados pela busca sistemática baseada em QGS}.

\begin{figure}[htb]
\centering 
\caption{Resultados da Busca Sistemática Baseada em QGS.}
\label{method:figure:evaluation-refinement}
\includegraphics[width=.9\textwidth]{images/chapter2-sm-qgs-search.png}
\fautor
\end{figure}

Para extrair as informações relevantes dos estudos primários identificados, um formulário de extração de dados foi criado. O \autoref{quadro:c2:data-extraction} representa o modelo que descreve as informações extraídas e apresenta seu relacionamento com cada QP, quando aplicável.

\sigla*{SWEBOK}{\textit{Software Engineering Body of Knowledge}}
\sigla*{ES}{Engenharia de Software}

\begin{quadro}[htb]
\centering
\caption{Formulário de Extração de Dados.}
\label{quadro:c2:data-extraction}
\begin{tabular}{l|l|l} \hline
\textbf{Informações Gerais} & \multicolumn{2}{l}{\textbf{Descrição}} \\ \hline
ID & \multicolumn{2}{l}{Identificador (prefixos \textit{INT} ou \textit{BRA}).} \\
Título & \multicolumn{2}{l}{Título do estudo.} \\
Autores & \multicolumn{2}{l}{Nomes dos autores.} \\
Ano & \multicolumn{2}{l}{Ano de publicação do artigo.} \\
Conferência/Periódico & \multicolumn{2}{l}{Nome do meio de publicação.} \\
Tipo de busca & \multicolumn{2}{l}{Manual; Automatizada; Ambas.} \\
Língua & \multicolumn{2}{l}{Inglês; Português.} \\
País & \multicolumn{2}{l}{País da afiliação do primeiro autor.} \\ \hline
\textbf{Informações Específicas} & \textbf{Descrição} & \textbf{QP} \\ \hline
Área da Eng. de Software (ES) & Área de conhecimento da ES (SWEBOK). & QP1 \\
Tipo de solução & Software; Hardware; Teórica. & QP1 \\
Estratégia empírica & Quais estratégias empíricas foram encontradas. & QP1 \\
Tópico educacional & Quais tópicos educacionais foram encontrados. & QP2 \\
Línguas de sinais & Quais línguas de sinais foram encontradas. & QP3 \\ \hline
\end{tabular}
\end{quadro}

\subsection{Resultados e Discussões}
\label{ms:resultados}

O MS contou com 185 estudos primários selecionados: 46 da busca manual nacional e 139 da busca sistemática baseada em QGS. Lembrando que, as informações mais relevantes para responder cada QP foram obtidas por meio do formulário de extração de dados. Primeiramente, considerando a quantidade de publicações por ano, uma linha de tendência linear crescente foi identificada (\autoref{results:figure:publications-year}). Portanto, é estatisticamente possível que este domínio de pesquisa esteja em ascensão globalmente.


\begin{figure}[htb]
\centering 
\caption{Linha de Tendência Linear Crescente de Publicações por Ano.}
\label{results:figure:publications-year}
\includegraphics[width=1\textwidth]{images/chapter2-sm-publications-timeline.png}
\fautor
\end{figure}

\simbolo{R^2}{Linha de Tendência Linear}

No que diz respeito às conferências, periódicos e fontes das publicações, esses dados também podem compor um racional interessante para futuras replicações. Sendo assim, todos os estudos primários deste MS foram ordenados pela quantidade de estudos selecionados (\autoref{table:c2:publication-venues}). No contexto internacional, a presença de eventos identificados durante as buscas manuais (em \textbf{\textit{destaque}} na \autoref{table:c2:publication-venues}) sugere uma execução efetiva dessa fase considerando o protocolo de busca adotado.

\begin{table}[htb]
\caption{Conferências/Periódicos mais relevantes.}
\label{table:c2:publication-venues}
\centering
\begin{tabular}{lcc|lcc} \hline
\multicolumn{3}{c|}{\textbf{Internacionais (\textit{INT})}} & \multicolumn{3}{c}{\textbf{Nacionais (\textit{BRA})}} \\ \hline
\textbf{Nome} & \textbf{Fonte} & \textbf{Estudos} & \textbf{Nome} & \textbf{Fonte} & \textbf{Estudos} \\ \hline
\textit{\textbf{HCI International}} & \textit{\textbf{Springer}} & \textit{\textbf{12}} & RENOTE & CINTED & 13 \\ 
ICCHP & Springer & 8 & SBIE & CEIE & 13 \\ 
\textit{\textbf{ICALT}} & \textit{\textbf{IEEE}} & \textit{\textbf{6}} & WCBIE & CEIE & 11 \\ 
ASSETS & ACM & 6 & WIE & CEIE & 7 \\ 
\textit{\textbf{Computers \& Education}} & \textit{\textbf{Elsevier}} & \textit{\textbf{5}} & RBIE & CEIE & 2 \\ 
Procedia Computer Science & Elsevier & 5 & - & - & - \\ 
Outros & - & 97 & - & - & - \\ \hline
\multicolumn{2}{l}{\textbf{Total}} & \textbf{139} & \multicolumn{2}{l}{\textbf{Total}} & \textbf{46} \\ \hline
\end{tabular}
\end{table}

A seguir são discutidos os principais resultados deste estudo, de modo a responder cada QP definida no escopo do MS. Adicionalmente, com o objetivo de organizar os estudos primários, eles foram classificados com relação à sua origem: Internacional (\textit{INT})\footnote{Formulário de extração de dados Internacionais (INT): \url{https://bit.ly/SM-DataExtraction-INT}} ou Nacional (\textit{BRA})\footnote{Formulário de extração de dados Nacionais (BRA): \url{https://bit.ly/SM-DataExtraction-BRA}}. Com isso, os resultados podem ser analisados de forma isolada, o que facilita o planejamento e a condução de trabalhos futuros.

\subsubsection{QP1: Quais soluções tecnológicas vêm sendo propostas no processo de ensino-aprendizagem com línguas de sinais?}

As áreas presentes na \autoref{table:c2:se-areas} destacam a importância intrínseca das arquiteturas de software na construção de recursos e serviços de TA para línguas de sinais. Embora haja uma concentração significativa nas etapas de ``Construção'' e ``Projeto'', poucas soluções se mostraram realmente replicáveis ou adaptáveis a diferentes contextos educacionais, principalmente pela falta de detalhes técnicos.

\begin{table}[htb]
\caption{QP1: Áreas da ES no SWEBOK \cite{Bourque2014}.}
\label{table:c2:se-areas}
\centering
\begin{tabular}{l|cc|cc} \hline
 & \multicolumn{2}{c|}{\textit{\textbf{INT}}} & \multicolumn{2}{c}{\textit{\textbf{BRA}}} \\ \cline{2-5} 
\textbf{Área da ES} & \textbf{Estudos} & \textbf{\%} & \textbf{Estudos} & \textbf{\%} \\ \hline
Construção de Software & 65 & 47\% & 23 & 50\% \\
Projeto de Software & 47 & 34\% & 5 & 11\% \\
Fundamentos da Engenharia & 24 & 17\% & 9 & 19\% \\
Qualidade de Software & 3 & 2\% & 9 & 19\% \\ \hline
\textbf{Total} & \textbf{139} & \textbf{100\%} & \textbf{46} & \textbf{100\%} \\ \hline
\end{tabular}
\end{table}

Tecnicamente, a maioria das soluções é baseada em plataformas Web, Mobile ou Desktop, evidenciando uma preocupação genuína em criar recursos de TA para diferentes plataformas de ensino-aprendizagem. No entanto, poucos estudos foram estruturados de forma a facilitar o seu reuso e extensibilidade. 

Além disso, menos da metade dos estudos apresentou avaliações empíricas formais, como \textit{Surveys}, Experimentos e Estudos de Caso, indicando uma falta de rigor científico em parte das pesquisas \cite{Pressman2016, Sommerville2015}.

Em contrapartida, avatares de línguas de sinais baseados em texto, como o \textit{Hand Talk}\footnote{Mais informações em \url{https://handtalk.me}} e o \textit{VLibras}\footnote{Mais informações em \url{https://gov.br/governodigital/pt-br/vlibras}}, destacam-se ao transformar texto em língua de sinais, evidenciando o potencial de soluções de TA bem arquitetadas para potencializar a acessibilidade de conteúdos em diversos contextos educacionais. Portanto, a discussão sobre Arquiteturas de Software na \autoref{section:foundation:arch} será fundamental para compreender como essas soluções podem ser aprimoradas para desenvolver serviços de TA verdadeiramente escaláveis.

\subsubsection{QP2: Quais tópicos educacionais são abordados?}

A \autoref{table:c2:educational-topics} apresenta uma ampla diversidade de tópicos educacionais tendo em vista os OAs analisados, evidenciando o uso do conceito de TA para línguas de sinais em diversos contextos. Isso representa um esforço consciente em abordar diferentes temas no processo de ensino-aprendizagem, promovendo uma acessibilidade digital mais ampla e personalizada. 

\begin{table}[htb]
\caption{QP2: Tópicos Educacionais.}
\label{table:c2:educational-topics}
\centering
\begin{tabular}{l|cc|cc} \hline
 & \multicolumn{2}{c|}{\textit{\textbf{INT}}} & \multicolumn{2}{c}{\textit{\textbf{BRA}}} \\ \cline{2-5} 
\textbf{Tópico Educacional} & \textbf{Estudos} & \textbf{\%} & \textbf{Estudos} & \textbf{\%} \\ \hline
Línguas de Sinais & 59 & 42,5\% & 19 & 41,3\% \\
Geral & 48 & 34,5\% & 10 & 21,7\% \\
Língua de Sinais Escrita & 10 & 7,2\% & 3 & 6,5\% \\
Matemática & 7 & 5,0\% & - & - \\
Alfabeto & 6 & 4,3\% & 1 & 2,2\% \\
Ciência da Computação & 4 & 2,9\% & 4 & 8,7\% \\
Língua Falada do País & 2 & 1,4\% & 9 & 19,6\% \\
Outros & 3 & 2,2\% & - & - \\ \hline
\textbf{Total} & \textbf{139} & \textbf{100\%} & \textbf{46} & \textbf{100\%} \\ \hline
\end{tabular}
\end{table}

Com uma vasta gama de OAs inclusivos e adaptáveis, os educadores podem proporcionar experiências de aprendizado mais imersivas e eficazes, garantindo oportunidades igualitárias de desenvolvimento para todos os alunos, independentemente de suas habilidades ou desafios individuais. Tais resultados estabelecem a base para uma discussão mais aprofundada sobre OAs na \autoref{section:foundation:lo}, onde são explorados como esses recursos podem ser projetados para atender demandas educacionais diversas.

\subsubsection{QP3: Quais línguas de sinais são abordadas?}

\sigla*{IAGen}{Inteligências Artificiais Generativas}

As línguas de sinais mais comuns, destacadas na \autoref{table:c2:sign-languages}, juntamente com o crescente interesse em ASR multi-idiomas, abrem caminho para avanços significativos de acessibilidade. A capacidade das IAs Generativas (IAGen) de transcrever e traduzir fala em texto em múltiplas línguas viabiliza a integração de avatares de línguas de sinais baseados em texto, resultando em OAs mais inclusivos e versáteis.

\begin{table}[htb]
\caption{QP3: Línguas de Sinais.}
\label{table:c2:sign-languages}
\centering
\begin{tabular}{l|cc|cc} \hline
 & \multicolumn{2}{c|}{\textit{\textbf{INT}}} & \multicolumn{2}{c}{\textit{\textbf{BRA}}} \\ \cline{2-5} 
\textbf{Língua de Sinais} & \textbf{Estudos} & \textbf{\%} & \textbf{Estudo} & \textbf{\%} \\ \hline
ASL & 21 & 15.11\% & - & - \\
Libras & 16 & 11.51\% & 44 & 95.65\% \\
Geral & 15 & 10.79\% & - & - \\
SignWriting & 10 & 7.19\% & 2 & 4.35\% \\
ArSL & 10 & 7.19\% & - & - \\
PSL & 6 & 4.32\% & - & - \\
BSL & 6 & 4.32\% & - & - \\
MySL & 6 & 4.32\% & - & - \\
ISL & 5 & 3.60\% & - & - \\
Outras & 44 & 31.65\% & - & - \\ \hline
\textbf{Total} & \textbf{139} & \textbf{100\%} & \textbf{46} & \textbf{100\%} \\ \hline
\end{tabular}
\end{table}

Nesse cenário, línguas de sinais como a \textit{American Sign Language} (ASL) e a Libras, além de sistemas de escrita como o \textit{SignWriting}, podem se beneficiar dessas tecnologias, ampliando o acesso a conteúdos educacionais antes restritos aos formatos de áudio e vídeo. Esses resultados ressaltam a importância do ASR, que será aprofundado na \autoref{section:foundation:asr}.

A análise dos resultados obtidos pelas QP fornece um panorama do uso das TICs no ensino e aprendizado com línguas de sinais, revelando tanto lacunas quanto oportunidades para avanços significativos na criação de soluções em TA ainda mais robustas. Essas descobertas abrem caminho para uma exploração detalhada de temas cruciais, como Arquiteturas de Software (\autoref{section:foundation:arch}), OAs (\autoref{section:foundation:lo}) e ASR (\autoref{section:foundation:asr}). Cada seção subsequente destaca como suas temáticas podem ajudar a superar os desafios identificados e a capitalizar nas oportunidades emergentes, contribuindo para um processo de ensino-aprendizagem mais inclusivo e acessível.

\section{Arquiteturas de Software: Bases para Tecnologias Assistivas}
\label{section:foundation:arch}

Uma das principais lacunas identificadas no MS foi a carência de padrões e boas práticas que permitam o reuso e a adaptação das soluções no ensino e aprendizagem com línguas de sinais. Muitos dos estudos primários apresentaram contribuições técnicas relevantes, mas não detalharam as arquiteturas de software utilizadas, dificultando a evolução e derivação dessas soluções para outros contextos e domínios de aplicação. Por isso, nesta seção é discutido como as arquiteturas podem contribuir para o desenvolvimento de TA replicável, flexível e independente de tecnologia, seguindo alguns princípios e diretrizes da ES.

Uma arquitetura de software pode ser definida como o conjunto de estruturas necessárias para o entendimento de um sistema, compreendendo desde seus componentes de software e hardware até suas relações e propriedades internas e externas \cite{Bass2021}. Essa definição enfatiza que a arquitetura inclui todas as decisões que moldam a estrutura do projeto e suas interações, não se limitando apenas às decisões iniciais. A arquitetura é fundamental para a criação de sistemas complexos e facilita a análise de requisitos não funcionais, como desempenho, segurança e escalabilidade \cite{Pressman2016, Sommerville2015}.

Conforme \citeonline{Bass2021}, a arquitetura abrange estruturas que permitem o raciocínio e a análise do sistema, oferecendo uma compreensão ampla e flexível, considerando suas múltiplas dimensões e aspectos envolvidos no desenvolvimento e manutenção do software. A seguir, são exploradas as diferentes estruturas e visões arquiteturais, bem como os critérios que definem uma ``boa'' arquitetura.

\subsection{Estruturas e Visões Arquiteturais}

Resumidamente, uma arquitetura pode ser vista como um conjunto de estruturas que proporcionam múltiplas perspectivas sobre o sistema, cada uma com seu foco específico, o qual pode ser necessário em diferentes fases do ciclo de vida do software. \citeonline{Bass2021} propõem três tipos principais de estruturas arquiteturais, que formam as principais visões neste contexto:

\begin{itemize}

    \item \textbf{Estruturas de Componentes e Conectores (C\&C)}: Estas estruturas focam nas interações em tempo de execução entre os componentes que realizam as funções do sistema. Componentes, que podem ser serviços, clientes, servidores ou filtros, são as unidades principais de computação. Os conectores, por sua vez, são os veículos de comunicação entre esses componentes, facilitando a troca de dados e a sincronização de processos. As estruturas C\&C são cruciais para entender o comportamento em tempo de execução, incluindo a interação entre componentes, a replicação de partes do sistema e a paralelização de tarefas \cite{Bass2021};
    
    \item \textbf{Estruturas de Módulos}: Estas estruturas particionam o sistema em unidades de implementação, conhecidas como módulos, que são responsáveis por funções específicas e são a base para a organização do trabalho de desenvolvimento. Módulos podem representar classes, pacotes ou divisões de funcionalidade, cada um com um papel definido no sistema. As relações entre os módulos, como uso, generalização e composição, ajudam a entender a estrutura estática do sistema e a gerenciar sua evolução e manutenção \cite{Bass2021};
    
    \item \textbf{Estruturas de Alocação}: Essas estruturas estabelecem a correspondência entre os componentes de software e os elementos não-software do sistema, como ambientes de desenvolvimento e execução. Elas respondem a questões críticas sobre onde cada componente será executado, como estão armazenados e como são atribuídos às equipes de desenvolvimento. As estruturas de alocação são essenciais para compreender a distribuição do software e gerenciar recursos durante todo o ciclo de vida do sistema \cite{Bass2021}.
    
\end{itemize}

Essas estruturas arquiteturais podem ser compreendidas de maneira análoga através dos diferentes sistemas fisiológicos humanos, conforme ilustrado na \autoref{chapter2:figure:physiological-structures}. Essa analogia facilita a compreensão de como diferentes visões se complementam para fornecer uma compreensão abrangente do sistema como um todo \cite{Bass2021}. Na figura, os diferentes sistemas fisiológicos representam de maneira análoga as estruturas arquiteturais:

\begin{figure}[htb]
\centering
\caption{Fisiologia Humana: Análoga às Estruturas e Visões Arquiteturais}
\label{chapter2:figure:physiological-structures}
\includegraphics[width=0.70\textwidth]{images/chapter2-arch-physiological-structures.jpeg}
\fdireta{Bass2021}
\end{figure}

\begin{itemize}
    \item \textbf{Esqueleto $\equiv$ Estruturas de Módulos}: Assim como o esqueleto fornece a estrutura e suporte básico para o corpo, as estruturas de módulos organizam e definem a base do software, dividindo-o em partes manejáveis e específicas, como classes e pacotes. Diagramas de classes e pacotes são exemplos de representações visuais que ilustram essas estruturas.

    \item \textbf{Músculos e Sistema Circulatório $\equiv$ Estruturas de C\&C}: Os músculos permitem o movimento e a interação entre as partes do corpo, enquanto o sistema circulatório transporta nutrientes e oxigênio, facilitando a comunicação. De forma similar, as estruturas de componentes e conectores permitem a interação e execução das funcionalidades do sistema, garantindo comunicação eficiente entre os componentes. Diagramas de componentes e de sequência são exemplos de como essas estruturas podem ser representadas visualmente.

    \item \textbf{Sistema Nervoso $\equiv$ Estruturas de Alocação}: O sistema nervoso controla e coordena as ações do corpo, assim como as estruturas de alocação determinam onde e como os componentes de software são executados, garantindo uma distribuição eficiente e gestão de recursos durante todo o ciclo de vida do sistema. Diagramas de implantação e de distribuição ilustram essas estruturas.
\end{itemize}

Essas três categorias de estruturas facilitam a criação de representações visuais que auxiliam na compreensão da arquitetura de software em diferentes etapas do desenvolvimento. A arquitetura \textit{Speech2Learning}, por exemplo, adota essas estruturas para garantir clareza arquitetural desde sua concepção até a avaliação em seus estudos de caso (detalhes nos Capítulos \ref{chapter3} e \ref{chapter4}). Essa abordagem sistematiza e torna mais claro o processo de desenvolvimento de soluções de TA baseadas na \textit{Speech2Learning}.

\subsection{O que Torna uma Arquitetura ``Boa''?}

Na prática, a arquitetura de software é uma abstração que destaca detalhes relevantes para a compreensão e análise do sistema, omitindo informações desnecessárias para o raciocínio sobre ele. A abstração é crucial para gerir a complexidade, permitindo que arquitetos e desenvolvedores se concentrem em aspectos essenciais sem se preocuparem com detalhes de implementação. A arquitetura trata dos elementos públicos do sistema, ou seja, aqueles que interagem entre si através de interfaces, enquanto os detalhes privados de implementação não são considerados \cite{Bass2021}.

Padrões arquiteturais são composições de elementos arquiteturais que foram documentadas e disseminadas devido à sua eficácia em resolver problemas recorrentes em diferentes domínios. Esses padrões fornecem abordagens comprovadas para o \textit{design} de sistemas e são fundamentais para alcançar os atributos de qualidade desejados, como modularidade e facilidade de manutenção \cite{Bass2021}. Por exemplo, o padrão de arquitetura em camadas é amplamente utilizado para sistemas que necessitam de alta modularidade, enquanto o padrão de microsserviços é ideal para sistemas que requerem escalabilidade e resiliência \cite{Pressman2016, Sommerville2015}.

Entretanto, não existe uma arquitetura intrinsecamente ``boa'' ou ``ruim''; a adequação de uma arquitetura depende de como ela atende aos requisitos específicos do sistema. Uma arquitetura projetada para um sistema de comércio eletrônico pode não ser adequada para um sistema de controle de voo, por exemplo. A avaliação da arquitetura em relação a objetivos específicos é crucial para garantir que ela atenda às necessidades do sistema \cite{Pressman2016, Sommerville2015}.

Para orientar o desenvolvimento de uma boa arquitetura de software, \citeonline{Bass2021} propõem algumas boas práticas, categorizadas em recomendações de processo e recomendações estruturais. Primeiramente, as \textbf{recomendações de processo} focam na maneira como a arquitetura deve ser desenvolvida e gerenciada ao longo do ciclo de vida do sistema, garantindo que a integridade conceitual e a qualidade sejam mantidas de forma contínua:

\begin{enumerate}
    \item \textbf{Condução por lideranças técnicas}: É fundamental que a arquitetura seja concebida por um arquiteto ou uma pequena equipe de arquitetos com um líder técnico identificado, assegurando a integridade conceitual e a consistência técnica. 
    
    Esse princípio também se aplica a projetos ágeis e de código aberto, evitando \textit{designs} impraticáveis e desconectados da realidade do desenvolvimento.
    
    \item \textbf{Foco nos requisitos de qualidade}: A arquitetura deve se basear continuamente em uma lista priorizada de requisitos de qualidade bem definidos. Esses requisitos guiam as decisões de \textit{trade-offs}, que sempre ocorrem, sendo mais relevantes do que a funcionalidade em si.
    
    \item \textbf{Documentação por meio de visões arquiteturais}: A arquitetura deve ser documentada através de visões que representem uma ou mais estruturas arquiteturais. Essas visões devem abordar as preocupações dos \textit{stakeholders} mais importantes e apoiar o cronograma do projeto, fornecendo uma documentação que pode ser inicialmente minimalista, mas detalhada posteriormente.
    
    \item \textbf{Avaliação contínua dos atributos de qualidade}: A arquitetura deve ser avaliada quanto à sua capacidade de fornecer os principais atributos de qualidade do sistema. Isso deve ocorrer no início do ciclo de vida, proporcionando os maiores benefícios, e ser repetido conforme necessário para garantir que alterações na arquitetura ou no ambiente não tornem o \textit{design} obsoleto.
    
    \item \textbf{Implementação incremental e adaptativa}: A arquitetura deve permitir a implementação incremental, evitando a integração total de uma só vez, o que raramente funciona. Isso pode ser alcançado através da criação de um sistema esquelético, no qual os caminhos de comunicação são exercidos inicialmente com funcionalidade mínima, permitindo o crescimento incremental do sistema e a refatoração conforme necessário.
\end{enumerate}

Por sua vez, as \textbf{recomendações estruturais} dizem respeito à organização interna da arquitetura, enfatizando a importância da modularidade, da separação de responsabilidades e da flexibilidade na integração dos componentes, visando a criação de um sistema robusto e facilmente evolutivo:

\begin{enumerate}
    \item \textbf{Modularização e separação de preocupações}: A arquitetura deve apresentar módulos bem definidos, cujas responsabilidades funcionais são atribuídas com base nos princípios de ocultação de informações e separação de preocupações. Esses módulos devem encapsular aspectos passíveis de mudança, isolando o software dos efeitos dessas mudanças.
    
    \item \textbf{Uso de padrões arquiteturais bem estabelecidos}: A arquitetura deve alcançar atributos de qualidade usando padrões arquiteturais e táticas bem estabelecidas e específicas para cada atributo. Isso proporciona uma base sólida para o \textit{design}, garantindo que os requisitos de qualidade sejam atendidos de maneira eficaz.
    
    \item \textbf{Flexibilidade em relação a versões de produtos}: A arquitetura nunca deve depender de uma versão específica de um produto comercial ou ferramenta. Se isso for inevitável, deve ser estruturada de forma que a mudança para uma versão diferente seja simples e barata.
    
    \item \textbf{Separação entre componentes produtores e consumidores de dados}: Os módulos que produzem dados devem ser separados dos módulos que consomem esses dados. Isso aumenta a manutenibilidade, permitindo que mudanças sejam confinadas ao lado da produção ou do consumo de dados, facilitando atualizações incrementais.
    
    \item \textbf{Flexibilidade na correspondência entre módulos e componentes}: Não se deve esperar uma correspondência um-para-um entre módulos e componentes. Em sistemas com concorrência, por exemplo, múltiplas instâncias de um componente podem ser executadas em paralelo, cada uma construída a partir do mesmo módulo.
    
    \item \textbf{Alocação flexível de processos}: Projete cada processo para ser executado em qualquer processador, permitindo fácil realocação, inclusive durante a execução. Isso é essencial em ambientes de virtualização e nuvem, onde os recursos computacionais podem variar.
    
    \item \textbf{Consistência e simplicidade nos padrões de interação}: A arquitetura deve conter um pequeno número de padrões simples de interação entre componentes. O sistema deve realizar as mesmas funções da mesma maneira em todas as partes, o que facilita a compreensão, reduz o tempo de desenvolvimento, além de aumentar confiabilidade e manutenibilidade.
    
    \item \textbf{Gestão eficaz de áreas de contenção de recursos}: A arquitetura deve conter um conjunto específico e pequeno de áreas de contenção de recursos, cuja resolução deve ser claramente especificada e mantida. Por exemplo, se a utilização da rede é uma preocupação, o arquiteto deve produzir diretrizes para cada equipe de desenvolvimento que resultem em níveis aceitáveis de tráfego de rede.
\end{enumerate}

Portanto, uma arquitetura de software bem projetada não só atende aos requisitos funcionais imediatos, mas também oferece uma base sólida que permite a evolução e adaptação contínua do sistema, especialmente em áreas críticas como a educação inclusiva. A flexibilidade da arquitetura é essencial para suportar a evolução contínua das TICs e a adaptação às necessidades dos aprendizes, garantindo que as soluções sejam sustentáveis e capazes de atender às necessidades dos alunos a longo prazo.

Nesse contexto, a arquitetura \textit{Speech2Learning} surge como uma proposta para impulsionar a construção de recursos e serviços de TA. Projetada para integrar soluções de ASR, ela visa facilitar a criação de OAs mais acessíveis a uma ampla gama de aprendizes. Nos próximos capítulos, são explorados em detalhes os conceitos de OAs e ASR, aprofundando o entendimento sobre como essas tecnologias se entrelaçam e formam a base da \textit{Speech2Learning}.

\section{Objetos de Aprendizagem: Diversidade em Conteúdos Educacionais}
\label{section:foundation:lo}

A crescente demanda por diversidade em conteúdos educacionais é amplamente reconhecida, conforme evidenciado no MS conduzido. Dessa forma, os OAs emergem como uma solução promissora para atender a essa demanda, permitindo a criação de recursos personalizados e adaptáveis a diferentes contextos e públicos. No âmbito da arquitetura \textit{Speech2Learning}, os OAs desempenham um papel fundamental no acesso a materiais didáticos audíveis, enriquecidos pela tecnologia de ASR para maior acessibilidade \cite{FalvoJr2023_HICSS}.

Os OAs abrangem uma vasta gama de recursos digitais projetados para enriquecer o processo de ensino-aprendizagem. Eles transcendem a mera entrega de conteúdo, proporcionando uma experiência mais rica e interativa \cite{Wiley2000}. A importância da multimídia no aprendizado é destacada por \citeonline{Mayer2021}, que argumentam que a combinação eficaz de texto, áudio, vídeo e elementos interativos pode potencializar a educação.

\sigla*{IEEE}{\textit{Institute of Electrical and Electronics Engineers}}

O \citeonline{LOM2000} define os OAs como entidades, digitais ou não-digitais, que podem ser usadas, reutilizadas ou referenciadas durante o ensino com suporte tecnológico. Essa definição abrange uma vasta gama de recursos, incluindo conteúdos multimídia, software instrucional, eventos educacionais, entre outros. \citeonline{Wiley2000} simplifica essa concepção ao descrever os OAs como recursos digitais que podem ser reutilizados para facilitar a aprendizagem, destacando a adaptabilidade e a reusabilidade como características centrais dos OAs.

Segundo \citeonline{Tarouco2021}, a essência dos OAs está na criação de pequenos módulos instrucionais reutilizáveis, combináveis de diferentes maneiras para atender às necessidades específicas de aprendizagem. Essa abordagem permite que educadores personalizem o ensino, adaptando materiais didáticos às suas metas pedagógicas individuais. O resultado é um processo de ensino-aprendizagem mais dinâmico, onde diferentes recursos se conectam para formar um todo coeso (\autoref{chapter2:figure:lo-mindmap}).

\begin{figure}[htb]
\centering
\caption{Mapa Conceitual Sobre Objetos de Aprendizagem}
\label{chapter2:figure:lo-mindmap}
\includegraphics[width=1\textwidth]{images/chapter2-lo-mindmap.jpg}
\fadaptada{Tarouco2021}
\end{figure}

\subsection{Estratégias de Identificação e Utilização de OAs}

O uso e reuso de OAs envolve várias estratégias que facilitam sua adaptação a diferentes contextos educacionais. Conforme discutido por \citeonline{Tarouco2021}, os OAs podem variar em tamanho, escopo e nível de granularidade, afetando diretamente sua reusabilidade (\autoref{chapter2:figure:lo-granularity}). OAs com alta granularidade, como imagens ou pequenos vídeos, são mais fáceis de reutilizar devido à sua simplicidade e especificidade. Em contraste, objetos de baixa granularidade, como cursos, oferecem uma experiência educacional mais completa e integrada, mas são mais difíceis de adaptar a novos contextos de ensino-aprendizagem. O equilíbrio da granularidade é fundamental para que os OAs atinjam seus objetivos educacionais \cite{Tarouco2021}.

\begin{figure}[htb]
\centering
\caption{Granularidade de Objetos de Aprendizagem}
\label{chapter2:figure:lo-granularity}
\includegraphics[width=0.64\textwidth]{images/chapter2-lo-granularity.jpg}
\fadaptada{Tarouco2021}
\end{figure}



A granularidade está estreitamente relacionada à intencionalidade pedagógica, que se refere à finalidade educacional para a qual o OA foi criado. A eficácia de um objeto depende da clareza de seus objetivos pedagógicos e da adequação às necessidades dos alunos \cite{Bloom1984}. Portanto, a escolha de OAs deve considerar tanto a granularidade quanto a intencionalidade pedagógica para garantir uma aprendizagem eficaz e significativa.

Nesse sentido, a adoção de padrões de metadados é essencial para a organização, indexação e reutilização de OAs. \citeonline{Santana2023} destacam a importância desses padrões e sua aplicabilidade no contexto da ES experimental, cujo compartilhamento de OAs é vital para replicações e pesquisas futuras. O \autoref{quadro:c2:lo-metadata} apresenta uma comparação entre vários padrões de metadados, destacando suas características principais:

\begin{itemize}
    \item \textit{Dublin Core}\footnote{Mais informações em \url{https://dublincore.org}}: Um padrão internacional que fornece um conjunto simples e padronizado de termos para descrever recursos. O Dublin Core é conhecido por sua simplicidade e extensibilidade, permitindo sua aplicação em diversos contextos, desde bibliotecas digitais até sistemas de informação corporativos.
    \item \sigla{SCORM}{\textit{Sharable Content Object Reference Model}}\footnote{Mais informações em \url{https://adlnet.gov/scorm}}: Um conjunto de padrões e especificações para e-learning que define a comunicação entre o conteúdo de aprendizado online e os Sistemas de Gerenciamento de Aprendizado (LMS). Desenvolvido pela \textit{Advanced Distributed Learning} (ADL), o SCORM facilita a interoperabilidade e a reutilização de conteúdos educacionais em diferentes plataformas de aprendizado.
    \item \textit{Motion Imagery Standard Board} (MISB)\footnote{Mais informações em \url{https://nsgreg.nga.mil/misb.jsp}}: Padrão desenvolvido para a gestão e utilização de imagens em movimento, particularmente em contextos que exigem alta precisão e interoperabilidade, como vigilância e análise de vídeo. O MISB, parte da \textit{National Geospatial-Intelligence Agency} (NGA), assegura que os dados de vídeo sejam consistentes e compatíveis em diferentes sistemas.
    \item \sigla{LOM}{\textit{Learning Object Metadata}}\footnote{Mais informações em \url{https://ieeexplore.ieee.org/document/9262118}}: Um padrão para a descrição de metadados de OAs, abrangendo aspectos como a finalidade educacional, estrutura, nível de agregação e condições de uso. Publicado pelo IEEE, o LOM é amplamente utilizado para descrever e categorizar recursos educacionais digitais, facilitando sua descoberta e reutilização.
    \item \sigla{RDF}{\textit{Resource Description Framework}}\footnote{Mais informações em \url{https://w3.org/rdf}}: Uma especificação da W3C que fornece uma base para a descrição de recursos da Web. O RDF é utilizado para modelagem de informações, permitindo a interoperabilidade entre diferentes sistemas de informação e facilitando a integração de dados de diversas fontes.
\end{itemize}

\begin{quadro}[htb]
\caption{Comparação Entre os Padrões de Metadados} 
\label{quadro:c2:lo-metadata}
\begin{tabular}{P{3.5cm}|P{3cm}|P{3cm}|P{3cm}}\hline
\textbf{Padrão de Metadados} & \textbf{Documentação Completa} & \textbf{Processamento Automatizado} & \textbf{Flexibilidade para ES} \\ \hline
Dublin Core & X & X & X \\  \hline
SCORM & X & X & X \\ \hline
MISB & X & X & \\ \hline
LOM & X & X & X \\ \hline
RDF & X & X & X \\ \hline
\end{tabular}
\fadaptada{Santana2023}
\end{quadro}

Segundo \citeonline{Santana2023}, o Dublin Core se destaca por sua simplicidade, documentação abrangente e capacidade de processamento automatizado, permitindo sua aplicação em diferentes linhas de pesquisa na área da ES. No entanto, no contexto deste trabalho, qualquer padrão de metadados pode ser adequado, visto que o papel de uma arquitetura de software não é o de definir ``detalhes de implementação''. 

Por outro lado, o LOM merece destaque pelo seu foco intrínseco em aspectos educacionais, demonstrando uma sinergia natural com o conceito de OAs. Além disso, com exceção do MISB, todos os padrões são tão robustos quanto o Dublin Core, considerando os critérios de comparação do \autoref{quadro:c2:lo-metadata}.

Ao conhecer os conceitos de granularidade e padrões de metadados, fica mais simples entendermos como os OAs podem ser encontrados. Nesse sentido, os \sigla{ROAs}{Repositórios de OAs} são plataformas que armazenam e disponibilizam OAs para educadores, estudantes e desenvolvedores. Eles desempenham um papel crucial na disseminação de recursos educacionais e na promoção do uso e reuso de OAs. No Brasil, de acordo com \citeonline{Tarouco2021}, alguns dos principais ROAs incluem: \textit{Portal do Professor}\footnote{Mais informações em \url{http://portaldoprofessor.mec.gov.br}}, \textit{Domínio Público}\footnote{Mais informações em \url{http://www.dominiopublico.gov.br}} e \textit{eduCAPES} \footnote{Mais informações em \url{https://educapes.capes.gov.br}}.

A utilização de padrões de metadados, como o LOM, nesses repositórios facilita a indexação e a busca eficiente de OAs, permitindo que educadores encontrem rapidamente os recursos que atendam às suas necessidades pedagógicas. A combinação de granularidade adequada e metadados padronizados assegura que os OAs possam ser reutilizados em diversos contextos educacionais, maximizando seu impacto e alcance.

%Um exemplo prático da adaptabilidade dos OAs é a inclusão de transcrições em materiais audíveis, como videoaulas. Essa prática não apenas flexibiliza o acesso à informação para todos os alunos, mas também destaca a capacidade dos OAs de se adaptarem às necessidades de uma gama diversificada de aprendizes, promovendo uma educação mais inclusiva. O potencial de adaptação dos OAs reforça a importância das TICs, como o ASR, na ampliação da acessibilidade e personalização dos conteúdos educacionais, conforme identificado no MS como um \textit{insight} relevante para a concepção da arquitetura \textit{Speech2Learning} \cite{FalvoJr2023_HICSS}.

A granularidade, os padrões de metadados e os repositórios de OAs se alinham para apoiar a intencionalidade pedagógica. Essa intencionalidade refere-se aos objetivos educacionais específicos para os quais os OAs são desenvolvidos e utilizados. De acordo com \citeonline{Bloom1984}, a eficácia de um OA depende da clareza de seus objetivos pedagógicos e da adequação às necessidades dos estudantes.

A próxima seção discorrerá sobre a intencionalidade pedagógica, exemplificando como a Taxonomia de Bloom e sua revisão podem ser aplicadas na seleção e utilização de OAs para maximizar o processo de ensino-aprendizagem.

\subsection{Intencionalidade Pedagógica}

A intencionalidade pedagógica dos OAs pode ser exemplificada pela Taxonomia de Bloom, que categoriza objetivos educacionais em uma hierarquia de complexidade cognitiva \cite{Bloom1984}. Em um trabalho posterior, \citeonline{Krathwohl2002} propôs uma revisão da taxonomia original, estruturando-a em uma perspectiva bidimensional. A primeira dimensão é a do conhecimento, que se desdobra em quatro tipos:

\begin{itemize}
    \item \textbf{Factual}: conhecimento básico de terminologias e detalhes específicos;
    \item \textbf{Conceitual}: inter-relações entre os elementos básicos em uma estrutura maior;
    \item \textbf{Procedimental}: métodos e critérios para realizar tarefas e resolver problemas;
    \item \textbf{Metaconhecimento}: conhecimento sobre o próprio conhecimento e sua regulação.
\end{itemize}

A segunda dimensão trata dos processos cognitivos, referida como Taxonomia de Bloom Revisada, que propõe seis categorias adaptadas da taxonomia original de \citeonline{Bloom1984}. \citeonline{Krathwohl2002} introduziu mudanças significativas, com destaque para o uso de verbos ativos ao invés de substantivos, além da reestruturação de alguns dos níveis. Como resultado, a nova versão organiza os OAs nas seguintes categorias:

\begin{itemize}
    \item \textbf{Recordar}: Capacidade de reter conhecimento na memória de longo prazo;
    \item \textbf{Entender}: Capacidade de construir significado a partir do material instrucional;
    \item \textbf{Aplicar}: Capacidade de utilizar o(s) procedimento(s) adequado(s) à situação vivenciada;
    \item \textbf{Analisar}: Capacidade de identificar diferentes partes constituintes de um material compreendendo suas inter-relações;
    \item \textbf{Avaliar}: Capacidade de estabelecer julgamentos a partir de critérios e padrões;
    \item \textbf{Criar}: Capacidade de transpor o conhecimento construído para novas situações a partir de produtos originais de autoria do próprio estudante.
\end{itemize}

Dada a representação bidimensional dos OAs proposta pela Taxonomia de Bloom Revisada de \citeonline{Krathwohl2002}, optou-se por utilizar uma tabela de duas dimensões (\autoref{quadro:c2:los-categories}) para a classificação desses objetivos. Na tabela, a dimensão do conhecimento é representada pelo eixo vertical, enquanto a dimensão dos processos cognitivos está no eixo horizontal. 

\begin{quadro}[htbp]
\centering
\resizebox{\textwidth}{!}{
\caption{Categorização de OAs}
\label{quadro:c2:los-categories}
\begin{tabular}{P{2.6cm}|c|c|c|c|c|c}
\hline
Dimensão do & \multicolumn{6}{c}{Dimensão dos Processos Cognitivos} \\ \cline{2-7}
Conhecimento & \textbf{Recordar} & \textbf{Entender} & \textbf{Aplicar} & \textbf{Analisar} & \textbf{Avaliar} & \textbf{Criar} \\ \hline
\textbf{Factual} & Listar & Resumir & Responder & Selecionar & Verificar & Generalizar \\ \hline
\textbf{Conceitual} & Reconhecer & Classificar & Providenciar & Diferenciar & Determinar & Montar \\ \hline
\textbf{Procedimental} & Recomendar & Esclarecer & Executar & Integrar & Julgar & Projetar \\ \hline
\textbf{Metacognitivo} & Identificar & Prever & Usar & Desconstruir & Refletir & Criar \\ \hline
\end{tabular}
}
\fadaptada{Mayer2021}
\end{quadro}

Cada célula do \autoref{quadro:c2:los-categories} relaciona um tipo de conhecimento com uma categoria de processo cognitivo, utilizando verbos para descrever ações esperadas dos alunos e substantivos para definir o conhecimento a ser adquirido. Isso facilita a adaptação dos OAs para diferentes objetivos pedagógicos.

Devido a essa flexibilidade, os OAs podem ser utilizados em diferentes contextos educacionais. Dependendo da intencionalidade pedagógica, alguns OAs podem focar mais no desenvolvimento de competências específicas dentro de uma categoria, enquanto outros podem abranger várias categorias simultaneamente. Dessa forma, os OAs mostram-se valiosos em diversas etapas do processo de ensino-aprendizagem, como:

\begin{itemize}
\item \textbf{Etapa A}: Introdução ao conteúdo a ser estudado;
\item \textbf{Etapa B}: Demonstração da teoria estudada;
\item \textbf{Etapa C}: Exemplo de aplicação do conteúdo estudado;
\item \textbf{Etapa D}: Instrumento de avaliação da aprendizagem.
\end{itemize}

Ademais, os OAs podem ser utilizados individualmente ou coletivamente, dependendo da intencionalidade pedagógica do professor e de sua escolha metodológica. Para simplificar a representação dos OAs nas etapas de ensino-aprendizagem, o \autoref{quadro:c2:samples-bloom-and-steps} foca exclusivamente na dimensão dos processos cognitivos da Taxonomia de Bloom Revisada \cite{Krathwohl2002}. 

Na prática, as etapas do processo de ensino-aprendizagem (introdução, demonstração, aplicação, avaliação) estão mais diretamente relacionadas aos processos cognitivos que os alunos desenvolvem durante essas atividades. Portanto, conectar os processos cognitivos com as etapas do ensino-aprendizagem permite uma visualização mais prática e objetiva dos diferentes tipos de OAs, proporcionando clareza na utilização dos recursos educacionais de acordo com a intencionalidade pedagógica desejada.

\begin{quadro}[htbp]
\centering
\resizebox{\textwidth}{!}{
\caption{Tipos de OAs: Processos Cognitivos e Etapas de Ensino-Aprendizagem}
\label{quadro:c2:samples-bloom-and-steps}
\begin{tabular}{l*{6}{|c}*{4}{|P{0.41cm}}}
\hline
\multirow{2}{*}{\textbf{Tipo de OA}} & \multicolumn{6}{c|}{\textbf{Processo Cognitivo}} & \multicolumn{4}{c}{\textbf{Etapa de Ensino}} \\ \cline{2-11}
 & Recordar & Entender & Aplicar & Analisar & Avaliar & Criar & A & B & C & D \\ \hline
Texto & \faCheckCircle & \faCheckCircle & \faCheckCircle & \faCheckCircle & \faCheckCircle & \faCheckCircle & \faCheck & \faCheck & \faCheck & \faCheck \\ \hline
Jogo & \faCheckCircle & \faCheckCircle & \faCheckCircle & \faCheckCircleO & \faCheckCircleO & \faCheckCircleO & \faCheck & \faCheck & \faCheck & \faCheck \\ \hline
Simulação & \faCheckCircle & \faCheckCircle & \faCheckCircle & \faCheckCircleO & \faCheckCircleO & \faCheckCircleO & \faCheck & \faCheck & \faCheck & \faCheck \\ \hline
Áudio/Vídeo & \faCheckCircle & \faCheckCircle & \faCheckCircleO & \faCheckCircle & \faCheckCircleO & \faCheckCircleO & \faCheck & \faCheck & \faCheck & \faCheck \\ \hline
Slides & \faCheckCircle & \faCheckCircle & \faCheckCircleO & \faCheckCircle & \faCheckCircleO & \faCheckCircleO & \faCheck & \faCheck & \faCheck & \faCheck \\ \hline
Exercícios & \faCheckCircle & \faCheckCircle & \faCheckCircle & \faCheckCircle & \faCheckCircleO & \faCheckCircleO & \faCheck & \faTimes & \faTimes & \faCheck \\ \hline
Mapa Mental & \faCheckCircle & \faCheckCircle & \faCheckCircle & \faCheckCircle & \faCheckCircleO & \faCheckCircle & \faCheck & \faCheck & \faCheck & \faCheck \\ \hline
Experimento & \faCheckCircle & \faCheckCircle & \faCheckCircle & \faCheckCircle & \faCheckCircle & \faCheckCircleO & \faCheck & \faCheck & \faCheck & \faCheck \\ \hline
Infográfico & \faCheckCircle & \faCheckCircle & \faCheckCircle & \faCheckCircle & \faCheckCircle & \faCheckCircle & \faCheck & \faCheck & \faCheck & \faCheck \\ \hline
\end{tabular}
}
\fadaptada{Mayer2021}
\nota{Legenda: \faCheckCircle~(uso comum); \faCheckCircleO~(uso em potencial); \faCheck~(aplicado); \faTimes~(não aplicado).}
\end{quadro}

Interpretando o \autoref{quadro:c2:samples-bloom-and-steps}, fica evidente a flexibilidade dos OAs do tipo texto no processo de ensino-aprendizagem atual. Eles são comumente utilizados nas seis categorias da taxonomia revisada e são relevantes em todas as etapas de ensino-aprendizagem. Essa característica foi fundamental para delimitar o escopo do \textit{Speech2Learning}, que se concentrou em tornar os OAs audíveis mais acessíveis através de ASR, transformando conteúdos de áudio/vídeo em texto e, assim, aumentando o alcance e a acessibilidade desses tipos de OAs.

De acordo com \citeonline{Tarouco2021}, o uso de novas tecnologias é um fator determinante para potencializar o desempenho de aprendizagem e o engajamento dos aprendizes. Nesse sentido, a \autoref{figure:chapter2-lo-engagement} demonstra que a tutoria individualizada oferece resultados superiores, mas os autores ressaltam que ela muitas vezes é inviável devido aos custos e recursos necessários. No entanto, novas TICs, como ASR ou IAGen, podem proporcionar melhorias significativas no processo de ensino-aprendizagem, alcançando resultados comparáveis aos de uma tutoria 1:1.

\begin{figure}[htb]
\centering
\caption{Impacto de Diferentes Estratégias de Interatividade na Aprendizagem}
\label{figure:chapter2-lo-engagement}
\includegraphics[width=.96\textwidth]{images/chapter2-lo-engagement.jpg}
\fadaptada{Tarouco2021}
\end{figure}

Na seção a seguir, discute-se como o reconhecimento de fala pode promover a acessibilidade digital por meio do ASR, abordando intrinsecamente subáreas fundamentais da IAGen, como \sigla{NLP}{Processamento de Linguagem Natural} e \sigla{LLMs}{Grandes Modelos de Linguagem}. Através dessas tecnologias disruptivas, pretende-se potencializar ainda mais o alcance e a qualidade dos OAs nos mais diversos ambientes educacionais.

\section{Reconhecimento de Fala: Promovendo Acessibilidade Digital com IA}
\label{section:foundation:asr}

Na era digital, a educação está em constante evolução à medida que as tecnologias emergentes remodelam as abordagens pedagógicas tradicionais. Neste contexto, o ASR, interpretado neste trabalho como um sinônimo de STT, emerge como uma ferramenta poderosa.

O ASR não apenas amplia a acessibilidade de conteúdos por meio de transcrições e legendas, mas também representa um avanço significativo rumo a uma educação mais inclusiva. Corroborando essa visão, \citeonline{Homburg2019} enfatizam o potencial do ASR em soluções de TA para surdos, como os avatares de línguas de sinais baseados em texto, explorados no MS conduzido como parte deste trabalho \cite{FalvoJr2020_FIE, FalvoJr2020_SBIE, FalvoJr2021_RENOTE}.

De acordo com \citeonline{Jurafsky2024}, a função do ASR é converter ondas sonoras da fala em uma sequência de palavras correspondentes. Embora a transcrição automática de fala de qualquer locutor e em qualquer ambiente ainda apresente desafios, a tecnologia de ASR já está suficientemente avançada para ser aplicada em diversas tarefas práticas, como a geração automática de legendas para áudio e vídeo, transcrição de conversas e comunicação assistiva para PcD, facilitando a interação entre computadores e humanos.

\citeonline{fleischmann2021} ressaltam que o crescimento do ensino remoto, intensificado pela pandemia da COVID-19, impulsionou a busca por métodos inovadores para criar e compartilhar conteúdo educacional. Nesse contexto, o ASR tem se mostrado uma ferramenta promissora, especialmente em ambientes colaborativos e conferências online.

Soluções baseadas em ASR desempenham um papel vital ao quebrar barreiras linguísticas, otimizando a comunicação entre falantes de diferentes línguas. Este argumento é reforçado por \citeonline{Homburg2019}, que destaca a relevância da tradução de voz para línguas de sinais visando promover a inclusão da comunidade surda no processo de ensino-aprendizagem.

Apesar de seu potencial, o ASR enfrenta diversos obstáculos e desafios de pesquisa. O trabalho de \citeonline{Koenecke2020} destaca alguns deles, apontando para disparidades raciais e as sutilezas de características linguísticas, como sotaques e peculiaridades regionais, assim como identificado no MS para a Libras. Essas constatações reforçam a relevância de promover soluções baseadas em ASR que sejam verdadeiramente inclusivas e que contemplem um espectro mais amplo de considerações sociolinguísticas em seu projeto e implementação.

\citeonline{Mayer2021} destacam a importância de tecnologias disruptivas, como o ASR, para expandir o alcance dos OAs, tornando o conteúdo educacional acessível a mais alunos. Complementando essa perspectiva, \citeonline{Parakh2022} define OAs como unidades digitais reutilizáveis, frequentemente integradas a projetos \textit{open-source}, que desempenham um papel fundamental na criação de experiências de ensino-aprendizagem adaptáveis, democráticas e contextualizadas.

Estas recentes perspectivas reforçam e expandem as percepções obtidas no MS, que enfatizou a importância das inovações tecnológicas no processo de ensino e aprendizagem de línguas de sinais, revelando lacunas interessantes. Nesse sentido, foi possível observar a falta de padrões de projeto, além de boas práticas de código e reuso, o que compromete a qualidade, o compartilhamento e o potencial de impacto desses OAs.

Diante destas lacunas e das tendências emergentes citadas, foi projetada a Arquitetura \textit{Speech2Learning}. Essa abstração propõe diretrizes de desenvolvimento para a criação de soluções baseadas em ASR, promovendo maior acessibilidade de seus OAs, em especial os audíveis.

A evolução e sofisticação do ASR estão intimamente ligadas aos avanços em \sigla{ML}{\textit{Machine Learning}} e IA. A aplicação de modelos de ML em ASR permite a otimização contínua e a adaptação a diferentes contextos linguísticos e acústicos. Isso é alcançado através de técnicas de treinamento supervisionado e não supervisionado, utilizando vastas quantidades de dados de fala para melhorar a precisão e a robustez dos sistemas de reconhecimento.

O fluxo de processamento de fala em sistemas ASR é ilustrado na \autoref{figure:chapter2-asr-diagram}. O processo começa com a captura do sinal de fala, que é submetido a uma etapa de pré-processamento. Nessa fase, o sinal é filtrado para remover ruídos e normalizado para padrões específicos. Em seguida, ocorre a extração de características, onde as propriedades acústicas relevantes da fala são convertidas em vetores de características. Esses vetores representam os componentes principais da fala que serão utilizados nas próximas etapas.

\begin{figure}[htb]
\centering
\caption{Fluxo do Processamento de Fala em Sistemas ASR}
\label{figure:chapter2-asr-diagram}
\includegraphics[width=0.95\textwidth]{images/chapter2-asr-diagram.png}
\fadaptada{Li2018}
\end{figure}

Após a extração de características, os vetores resultantes são enviados a um decodificador. O decodificador utiliza modelos acústicos, um dicionário de pronúncia e um modelo de linguagem para interpretar os vetores de características e mapear palavras/frases correspondentes. Os modelos acústicos são responsáveis por capturar as nuances dos sons da fala, enquanto o dicionário de pronúncia fornece a correspondência entre as sequências de fonemas e palavras.

O modelo de linguagem, que pode ser um LLM, ajuda a prever a sequência mais provável de palavras com base no contexto linguístico. Esta abordagem estruturada e baseada em modelos de ML e IA permite que sistemas de ASR sejam precisos e adaptáveis a uma ampla gama de variabilidades na fala humana.

A distinção entre ASR e STT é muitas vezes sutil e pode ser usada de forma intercambiável; neste trabalho, por motivo de simplificação, optou-se por essa abordagem. Enquanto ASR geralmente se refere ao campo de estudo e tecnologia de reconhecimento automático de fala, STT descreve a função específica de converter fala em texto. Ambos os termos são fundamentais para o desenvolvimento de TA e têm aplicações que se sobrepõem consideravelmente.

Para garantir que os sistemas de ASR cumpram seu papel de maneira eficaz e inclusiva, é essencial utilizar métodos robustos de avaliação. A próxima seção discutirá os métodos de avaliação de reconhecimento de fala, incluindo a \sigla{WER}{\textit{Word Error Rate}} e métodos de similaridade léxica, fundamentais para uma análise abrangente da qualidade desses sistemas.

\subsection{Métodos de Avaliação de Reconhecimento de Fala}

A avaliação da precisão dos sistemas baseados em ASR é essencial para o aprimoramento e a implementação eficaz dessas tecnologias. Um dos métodos mais tradicionais e amplamente utilizados é a WER, que mede a discrepância entre a transcrição gerada com ASR e uma transcrição de referência. Segundo \citeonline{Jurafsky2024}, o WER é definido como:

\[
\text{WER} = \frac{\text{Inserções} + \text{Substituições} + \text{Deleções}}{\text{Total de Palavras na Transcrição Correta}} \times 100
\]

O WER quantifica o número total de palavras que foram inseridas, substituídas ou deletadas na transcrição de ASR em comparação com uma referência, expressando esse total como uma porcentagem do número total de palavras na transcrição correta. Embora seja amplamente utilizado, o WER tem limitações, particularmente ao lidar com variações lexicais e semânticas, pois considera todas as palavras igualmente importantes e não distingue entre diferentes tipos de erro, como erros semânticos e sintáticos.

Para complementar a análise de WER, métodos de similaridade léxica são frequentemente utilizados para fornecer uma visão mais detalhada da qualidade das transcrições geradas por ASR. Esses métodos avaliam a similaridade entre as transcrições com base em diversos critérios, como alterações lexicais e proximidade semântica. De acordo com \citeonline{Majumdar2022}, alguns dos métodos de similaridade léxica incluem:

\begin{itemize}
\item \textbf{Distância de Levenshtein}: Mede o número mínimo de operações necessárias para transformar uma palavra na outra. Este método é útil para capturar a similaridade em termos de alterações literais entre as transcrições automática e de referência \cite{levens-1,levens-2}.
\item \textbf{Índice de Jaccard}: Avalia a similaridade entre conjuntos de palavras, calculando a razão entre o tamanho da interseção e o tamanho da união dos conjuntos de palavras de duas transcrições. É particularmente útil para medir a presença de palavras comuns e a diversidade de vocabulário \cite{jaccard-1,jaccard-2}.
\item \textbf{Similaridade de Cosseno}: Mede a similaridade entre vetores de palavras, considerando o cosseno do ângulo entre eles. Este método é frequentemente usado para comparar representações vetoriais de frases ou textos, capturando a similaridade semântica além da simples correspondência de palavras \cite{cosseno-1,cosseno-2,cosseno-3}.
\end{itemize}

Esses métodos de similaridade léxica oferecem uma análise complementar à WER, proporcionando uma visão mais rica e detalhada da precisão dos sistemas de ASR. Eles são especialmente úteis para capturar nuances linguísticas e semânticas nas transcrições, que podem não ser refletidas de forma adequada pela WER. Nos estudos de caso conduzidos, optou-se por utilizar métodos de similaridade léxica para obter uma avaliação mais holística e detalhada da qualidade das transcrições automáticas.

Tais abordagens permitem uma avaliação mais abrangente dos sistemas de ASR, destacando não apenas a precisão em termos de correspondência palavra a palavra, mas também a preservação do significado e a fluidez das transcrições. Isso é fundamental para aplicações que exigem uma compreensão precisa e contextual do conteúdo falado.

Em resumo, o reconhecimento de fala, por meio de tecnologias como ASR e STT, suportado por avanços em ML e IA, tem o potencial de transformar significativamente a acessibilidade digital. Estas tecnologias não só facilitam a criação de conteúdos mais inclusivos, mas também permitem a adaptação e personalização da educação para atender a uma diversidade maior de aprendizes. A integração de ASR, portanto, representa um marco crucial na busca por uma educação mais acessível e equitativa.

Dessa forma, ao reconhecer a importância das novas TICs, especialmente aquelas baseadas em IA, reafirma-se o compromisso com uma educação mais inclusiva e flexível. Combinando boas práticas arquiteturais, OAs e ASR, propõe-se neste trabalho de doutorado a Arquitetura \textit{Speech2Learning}, detalhada no próximo capítulo.

\section{Considerações Finais}

As análises e discussões apresentadas nesta fundamentação teórica permitiram uma compreensão abrangente das interseções entre TICs e línguas de sinais no contexto educacional. O MS realizado, detalhado na \autoref{section:foundation:sm}, revelou importantes lacunas tecnológicas e de pesquisa na utilização de línguas de sinais para o ensino-aprendizagem, destacando áreas onde inovações são necessárias. Nesse sentido, esse estudo sistemático resultou na publicação de uma série de artigos, os quais discutem múltiplas perspectivas:

\begin{itemize}
    \item \fullcite{\textbf{FALVOJR, V.}; MARTINS FALVO, C.; SCATALON, L.; BARBOSA, E.}{Tecnologias Aplicadas ao Ensino e Aprendizagem de LIBRAS: Um Mapeamento Sistemático}{Simpósio Brasileiro de Informática na Educação (SBIE)}{2020}{Disponível em \url{doi.org/10.5753/cbie.sbie.2020.812}}

    \item \fullcite{\textbf{FALVOJR, V.}; SCATALON, L.; BARBOSA, E.}{The Role of Technology to Teaching and Learning Sign Languages: A Systematic Mapping}{Frontiers in Education Conference (FIE)}{2020}{Disponível em \url{doi.org/10.1109/FIE44824.2020.9274169}}

    \item \fullcite{\textbf{FALVOJR, V.}; MARTINS FALVO, C.; SCATALON, L.; BARBOSA, E.}{Tecnologias Aplicadas ao Ensino e Aprendizagem com Línguas de Sinais: Um Mapeamento Sistemático Sob as Perspectivas Nacional e Internacional}{Revista Novas Tecnologias na Educação (RENOTE)}{2021}{Disponível em \url{doi.org/10.22456/1679-1916.110217}}
\end{itemize}

As lacunas tecnológicas identificadas foram essenciais para orientar a condução de um levantamento bibliográfico adicional, visando identificar e explorar TICs promissoras que pudessem mitigar os desafios identificados e promover uma educação mais inclusiva.

No decorrer desta fundamentação teórica, explorou-se de forma aprofundada diversos aspectos cruciais, incluindo Arquiteturas de Software, OAs e ASR. A \autoref{section:foundation:arch} discutiu arquiteturas de software inovadoras que suportam a integração eficiente de TICs no processo educacional, enquanto a \autoref{section:foundation:lo} concentrou-se nos OAs, destacando sua importância na criação de materiais educacionais adaptáveis e acessíveis. Particularmente, a \autoref{section:foundation:asr} demonstrou como as tecnologias de ASR podem ser aplicadas para melhorar a acessibilidade de conteúdos educacionais, proporcionando uma base teórica robusta para a arquitetura \textit{Speech2Learning}.

Através da síntese destas temáticas, esta fundamentação teórica estabelece os alicerces para a proposição da \textit{Speech2Learning}, uma arquitetura inovadora que visa tornar os Objetos de Aprendizagem audíveis mais acessíveis. Definida em detalhes no \autoref{chapter3}, a \textit{Speech2Learning} representa uma proposta baseada no uso de ASR para educação inclusiva, aproveitando as últimas inovações em TICs e IA para criar soluções educacionais que atendam às necessidades de uma diversidade maior de aprendizes. Esta fundamentação teórica, portanto, não apenas considera as lacunas existentes, mas também propõe caminhos concretos para superá-las, contribuindo para o avanço da pesquisa e prática educacional inclusiva.

\chapter{Arquitetura Speech2Learning: Potencializando a Acessibilidade em Objetos de Aprendizagem com Reconhecimento de Fala}
\label{chapter3}
\section{Considerações Iniciais}

A Arquitetura \textit{Speech2Learning} foi motivada por um MS prévio, detalhado na \autoref{section:foundation:sm}, cujo objetivo foi identificar o papel da tecnologia no ensino e aprendizagem, particularmente através das línguas de sinais \cite{FalvoJr2020_SBIE, FalvoJr2020_FIE, FalvoJr2021_RENOTE}. Este estudo proporcionou uma visão abrangente sobre o uso das TICs para o desenvolvimento de TA, evidenciando a importância significativa de novas tecnologias para um processo de ensino-aprendizagem mais inclusivo.

Contudo, foi observada uma lacuna importante: a ausência de padrões e boas práticas de desenvolvimento que facilitassem o compartilhamento e reuso de OAs. Como resposta a essa necessidade, a Arquitetura \textit{Speech2Learning} foi estabelecida para promover a criação de soluções estruturalmente preparadas para a inclusão, não apenas de pessoas surdas, mas também de aprendizes que necessitam de maior acessibilidade em conteúdos educacionais audíveis \cite{FalvoJr2023_HICSS}.

Além disso, a \textit{Speech2Learning} adota uma abordagem modular e escalável, permitindo que diferentes componentes de reconhecimento de fala e processamento de áudio sejam integrados de forma flexível. Isso possibilita que a arquitetura se adapte a diversos contextos educacionais e necessidades específicas dos aprendizes. A arquitetura também enfatiza a interoperabilidade e a conformidade com padrões abertos, facilitando a integração com outras plataformas e ferramentas educacionais, promovendo um ecossistema mais colaborativo e acessível.

A evolução da área de IA, especialmente com modelos generativos, tem o potencial de expandir ainda mais as capacidades da \textit{Speech2Learning}. A IAGen, com a habilidade de processar e gerar texto de forma autônoma, pode ser utilizada para aprimorar a acessibilidade e a personalização dos OAs, abrindo novas possibilidades para a criação de conteúdo e interação com os alunos.

Nesse sentido, a \textit{Speech2Learning} não apenas aborda a lacuna identificada em relação à falta de padrões, mas também estabelece um \textit{framework} robusto e inclusivo para o desenvolvimento de soluções educacionais baseadas em reconhecimento automático de fala. Esta arquitetura visa transformar a maneira como os conteúdos educacionais audíveis são criados e acessados, promovendo uma educação mais inclusiva e acessível para todos os aprendizes.

\section{Principais Referências e Inspirações}

Tecnicamente, a \textit{Speech2Learning} propõe um arcabouço genérico que vai além das línguas de sinais, escopo inicial do MS conduzido neste trabalho \cite{FalvoJr2020_SBIE, FalvoJr2020_FIE, FalvoJr2021_RENOTE}. No entanto, ele fornece uma abstração que facilita a acessibilidade de OAs audíveis, permitindo a geração de transcrições e, consequentemente, a sinalização do conteúdo educacional em questão. Nesse contexto, soluções baseadas em avatares, tais como \textit{Hand Talk} ou \textit{VLibras}, podem ser utilizadas com base nas transcrições, tornando os conteúdos acessíveis para usuários das línguas de sinais.

Sendo assim, foi proposta a Arquitetura \textit{Speech2Learning}, uma adaptação da \textit{Clean Architecture} de \citeonline{martin2017}, com o objetivo específico de promover a acessibilidade de OAs por meio do reconhecimento de fala. Segundo \citeonline{martin2017}, a \textit{Clean Architecture} é uma ideia prática que integra algumas das principais referências em ES nas últimas décadas \cite{cockburn2005, freeman2009, palermo2008, coplien2012, reenskaug2009, jacobson1992}. 

Essas iniciativas compartilham a ideia central de separar o código em camadas independentes, com o domínio no núcleo da arquitetura. Isso permite a criação de sistemas altamente testáveis, independentes de tecnologia, e adaptáveis às necessidades específicas de um projeto. A Arquitetura \textit{Speech2Learning}, alinhada a esses princípios, concentra seu domínio de aplicação central nos OAs, conforme representado pela \autoref{fig:chapter3-speech2learning-arch}. 

\begin{figure}[htb]
\centering
\caption{Arquitetura \textit{Speech2Learning}}
\label{fig:chapter3-speech2learning-arch}
\includegraphics[width=0.71\textwidth]{images/chapter3-speech2learning-arch.png}
\fadaptada{FalvoJr2023_HICSS}
\end{figure}

Cada camada da \textit{Speech2Learning} contribui de forma significativa para OAs mais acessíveis por meio dos conceitos de ASR/STT, os quais, em conjunto com os metadados, são cruciais para promover a acessibilidade dos OAs de forma padronizada e consistente. Vale ressaltar que uma arquitetura não tem como objetivo definir detalhes de implementação, mas sim abstrações de software que podem ser adaptadas de acordo com as necessidades de cada domínio de aplicação. A seguir, apresenta-se uma síntese de cada uma das camadas definidas pela Arquitetura \textit{Speech2Learning}:

\begin{itemize}
\item \textbf{Objetos de Aprendizagem (Amarelo)}: Responsável pelos modelos e regras de negócio do domínio. Para garantir que os OAs sejam mais acessíveis e padronizados, as entidades podem incluir transcrição, legendagem e outras capacidades compatíveis com o domínio de aplicação e o padrão de metadados dos OAs audíveis. Além disso, o processo de reconhecimento de fala deve ser, idealmente, multimodal (abrangendo áudio e vídeo) e suportar múltiplos idiomas. Nesse sentido, considerar IAs, especialmente as generativas, é recomendado uma vez que elas resolvem de forma efetiva muitos desses desafios técnicos. Sendo assim, desde o centro da nossa arquitetura, onde estão os OAs, se faz necessária uma visão de projeto arquitetural que tenha sinergia com IAGen;
\item \textbf{Casos de Uso (Vermelho)}: Operações de alto nível e regras de negócio do sistema. Em outras palavras, esta camada é responsável por orquestrar as regras encapsuladas nos OAs, função comumente chamada de ``dança das entidades'';
\item \textbf{Adaptadores (Verde)}: Tem a responsabilidade de converter os dados para um formato conveniente, considerando as necessidades das camadas com as quais faz fronteira;
\item \textbf{Infraestrutura (Azul)}: Abrange questões técnicas e específicas, como \textit{frameworks}, \textit{drivers} e integrações externas. Nesse sentido, conexões com \sigla{BD}{Bancos de Dados}, serviços de ASR/STT ou eventuais integrações com APIs de IAGen fariam parte desta camada. Além disso, ela inclui a \sigla{UI}{Interface do Usuário}, permitindo o desenvolvimento de aplicações educacionais concretas (\textit{d-learning}, \textit{e-learning}, \textit{m-learning} etc) ou repositórios de OAs;
\item \textbf{Principal e Configuração (Cinza)}: Estabelece todas as conexões entre interfaces e suas implementações concretas, além de ser responsável por configurar e executar a aplicação.
\end{itemize}

A ``Regra de Dependência'' define que as dependências de código devem apontar apenas para dentro, em direção aos OAs, nunca para fora. Desse modo, a \textit{Speech2Learning} favorece a adaptabilidade, independência tecnológica e testabilidade, fornecendo uma orientação estrutural para a criação de soluções educacionais mais acessíveis. 

Para alcançar esses objetivos, é fundamental que as responsabilidades e boas práticas de suas camadas sejam bem compreendidas, pois elas estruturam diretrizes de projeto genéricas para a criação de soluções de TA baseadas em reconhecimento de fala. As seções a seguir detalham cada uma dessas camadas.

\section{Camada de Entidades (Objetos de Aprendizagem)}

\begin{figure}[htb]
\centering
\caption{Camada de Entidades (Objetos de Aprendizagem)}
\label{fig:chapter3-speech2learning-layer1}
\includegraphics[width=1\textwidth]{images/chapter3-speech2learning-layer1.png}
\fadaptada{FalvoJr2023_HICSS,Lemos2022}
\end{figure}

A Camada de Entidades na Arquitetura \textit{Speech2Learning} é central para a definição dos modelos e regras de negócio do domínio dos OAs (\autoref{fig:chapter3-speech2learning-layer1}). Esta camada é responsável por encapsular os dados fundamentais e as lógicas que regem a estrutura e o comportamento dos OAs, garantindo sua acessibilidade e padronização. 

Para alcançar esses objetivos, é crucial incorporar padrões de metadados estabelecidos, como Dublin Core, SCORM, MISB, LOM e RDF, que fornecem diretrizes consistentes para a definição e organização dos conteúdos educacionais \cite{Santana2023}. As principais responsabilidades da Camada de Entidades incluem:

\begin{itemize}
    \item \textbf{Modelagem dos OAs}: Definir e estruturar os OAs de forma a garantir sua acessibilidade. Isso inclui a inclusão de elementos como transcrição, legendagem e outros recursos que tornam os conteúdos audíveis acessíveis a uma ampla gama de aprendizes, incluindo aqueles com necessidades especiais.

    \item \textbf{Definição de Regras de Negócio do Domínio}: Estabelecer regras de negócio que governam o comportamento dos OAs, assegurando que os processos de ensino-aprendizagem sejam eficazes e inclusivos. As regras de negócio devem considerar a multimodalidade do reconhecimento de fala, abrangendo áudio e vídeo, e suportando múltiplos idiomas.

    \item \textbf{Padronização e Interoperabilidade}: Garantir que os OAs sejam compatíveis com os padrões de metadados, facilitando seu compartilhamento e reuso em diferentes plataformas educacionais. A conformidade com padrões como Dublin Core, SCORM, MISB, LOM e RDF é essencial para manter a consistência e a qualidade dos OAs.
\end{itemize}

Os padrões de metadados desempenham um papel crucial na definição e organização dos OAs. A seguir, ilustra-se como cada padrão pode ser aplicado na Camada de OAs (Entidades):

\begin{itemize}
    \item \textbf{Dublin Core}: Fornece um conjunto simples e genérico de elementos para descrever recursos digitais. Na Camada de Entidades, Dublin Core pode ser utilizado para definir metadados básicos como título, autor, descrição, data, formato e identificador dos OAs, facilitando sua catalogação e recuperação.

    \item \textbf{SCORM}: É um padrão para LMS que facilita o reuso de conteúdos educacionais. Na \textit{Speech2Learning}, SCORM pode ser utilizado para definir a estrutura modular dos OAs, permitindo que cada unidade de aprendizagem seja facilmente reutilizada e integrada em diferentes cursos e plataformas.

    \item \textbf{MISB}: Embora mais comum em contextos de vídeo e imagem em movimento, MISB pode ser aplicado para garantir que os conteúdos audiovisuais dos OAs sejam padronizados e interoperáveis, especialmente em termos de metadados de vídeo e áudio.

    \item \textbf{LOM}: É um padrão específico para descrever OAs. Na Camada de Entidades, LOM pode ser utilizado para definir metadados detalhados que incluem, além dos elementos básicos, informações pedagógicas como tipo de recurso, nível de dificuldade, tempo de aprendizagem e público-alvo.

    \item \textbf{RDF}: Facilita a interoperabilidade de dados na web. Na \textit{Speech2Learning}, RDF pode ser utilizado para criar descrições estruturadas e semânticas dos OAs, permitindo que os dados sobre os objetos sejam facilmente compartilhados e integrados com outros sistemas e plataformas educacionais.
\end{itemize}

Do ponto de vista prático, o \autoref{codigo:exemplo-camada1} exemplifica o modelo \textit{TranscribedAudio}, o qual estabelece uma conexão com as responsabilidades citadas. Além disso, elementos fundamentais como documentação, convenções e boas práticas se mostram presentes. Outro aspecto fundamental é a independência de \textit{frameworks} e bibliotecas, caracterizando uma entidade limpa.

\begin{codigo}[caption={Exemplo Camada de Entidades: Disponível em \url{https://bit.ly/S2L-Entity}}, label={codigo:exemplo-camada1}, language=Java, breaklines=true]
/**
 * Model that represents an audio file and its transcript.
 * Responsibilities:
 * - Hold properties: ID, name, content, transcript.
 * - Validate integrity and conformity of audio data.
 * Adherence to Clean Architecture:
 * - Central to business logic and rules.
 * - Independent of frameworks or data persistence.
 * 
 * Author: @falvojr
 */
public class TranscribedAudio {
  private String id, name, transcript;
  private InputStream content;

  public void validate() {
    if (empty(name) || empty(transcript) || empty(content)) {
      String msg =  "Name, transcript and content required.";
      throw new EnterpriseBusinessException(msg);
    }
    String ext = FileUtils.getExtension(name).toLowerCase();
    if (!VALID_EXT.contains(ext)) {
      String msg =  "Invalid ext, use %s".formatted(VALID_EXT);
      throw new EnterpriseBusinessException(msg);
    }
  }
}
\end{codigo}

Por fim, para garantir a acessibilidade plena dos OAs, é essencial que esta camada considere o ASR de forma multimodal, abrangendo tanto áudio quanto vídeo. Isso permite que os conteúdos educacionais audíveis sejam mais diversos e flexíveis às diferentes necessidades dos aprendizes. Além disso, a capacidade de suportar múltiplos idiomas é fundamental para a inclusão de aprendizes de diversas origens linguísticas, promovendo uma educação verdadeiramente inclusiva e global.

Adicionalmente, a IAGen pode ser utilizada para enriquecer os metadados dos OAs, gerando automaticamente descrições mais detalhadas e informativas, \textit{tags} relevantes e até mesmo traduções para diferentes idiomas, aumentando a descoberta e a reutilização dos OAs em diferentes contextos e por diferentes públicos.

\section{Camada de Aplicação (Casos de Uso)}

\begin{figure}[htb]
\centering
\caption{Camada de Aplicação (Casos de Uso)}
\label{fig:chapter3-speech2learning-layer2}
\includegraphics[width=1\textwidth]{images/chapter3-speech2learning-layer2.png}
\fadaptada{FalvoJr2023_HICSS, Lemos2022}
\end{figure}

A Camada de Aplicação na arquitetura \textit{Speech2Learning} é responsável pela implementação das regras de negócio da aplicação, ou seja, dos casos de uso. Esta camada orquestra as operações de alto nível, garantindo que as regras encapsuladas nos OAs sejam executadas de maneira eficiente e coerente com os objetivos educacionais e de acessibilidade da arquitetura (\autoref{fig:chapter3-speech2learning-layer2}). As principais responsabilidades da Camada de Aplicação incluem:

\begin{itemize}
    \item \textbf{Orquestração de Regras de Negócio}: Implementar e gerenciar as regras de negócio que definem o comportamento dos OAs, garantindo que as operações de ensino e aprendizagem sejam conduzidas de maneira consistente com os objetivos de acessibilidade e inclusão.

    \item \textbf{Coordenação de Fluxos de Trabalho}: Coordenar os diversos fluxos de trabalho relacionados aos processos de ensino-aprendizagem, incluindo a transcrição de áudio, legendagem de vídeos e geração de metadados, assegurando que todas as etapas sejam integradas de forma harmoniosa.

    \item \textbf{Interação com Outras Camadas}: Facilitar a comunicação entre a Camada de Entidades e as externas (Adaptadores e Infraestrutura), garantindo que as operações sejam executadas corretamente e que os dados sejam processados/transmitidos conforme necessário.
\end{itemize}

Para a \textit{Speech2Learning}, algumas abstrações são essenciais para a implementação eficaz dos casos de uso. Como sugestões de abstrações que fazem sentido para essa arquitetura destacam-se:

\begin{itemize}
    \item \textbf{Controladores de Casos de Uso}: Classes ou componentes que implementam casos de uso específicos, orquestrando as operações de acordo com as regras de negócio. Por exemplo, um controlador pode gerenciar o processo de transcrição de um conteúdo de áudio, interagindo com a Camada de Entidades para armazenar os resultados e atualizar os metadados do OA.

    \item \textbf{Serviços de Aplicação}: Componentes que encapsulam lógica de negócio reutilizável, fornecendo funcionalidades comuns a múltiplos casos de uso. Exemplos incluem serviços para processamento de áudio, reconhecimento de fala ou geração de legendas, que podem ser utilizados por diferentes controladores de casos de uso.

    \item \textbf{Repositórios de Casos de Uso}: Interfaces e implementações que gerenciam a persistência e recuperação de dados necessários para a execução dos casos de uso. Estes repositórios abstraem o acesso aos dados, permitindo que a lógica de negócio permaneça desacoplada das questões de persistência.

    \item \textbf{\sigla{DTO}{\textit{Data Transfer Objects}}}: Objetos de transferência de dados utilizados para encapsular e transportar dados entre as diferentes camadas da arquitetura. Os DTOs garantem que os dados sejam transmitidos de maneira eficiente e coerente, facilitando a comunicação entre controladores de casos de uso e outros componentes do sistema.

    \item \textbf{Interfaces de Serviços Externos}: Abstrações que representam serviços externos necessários para a execução dos casos de uso, como APIs de reconhecimento de fala, serviços de tradução ou plataformas de gestão de conteúdos educacionais. Estas interfaces permitem que a Camada de Aplicação interaja com serviços externos de maneira flexível e desacoplada.
\end{itemize}

Nesse contexto, tem-se como exemplo um caso de uso que orquestra o processo de transcrição de áudio para texto, interagindo com serviços de reconhecimento de fala e atualizando os metadados do OA para incluir a transcrição gerada. Essa implementação fez parte de uma das POCs implementadas ao longo deste trabalho de doutorado e pode ser acessada neste repositório \textit{open-source}: \url{https://bit.ly/S2L-UseCase}.

Tais abstrações e exemplos demonstram como a Camada de Aplicação pode ser estruturada para orquestrar de forma eficiente os OAs aderentes aos princípios da \textit{Speech2Learning}, promovendo a acessibilidade por meio de soluções baseadas em reconhecimento de fala.

A IAGen pode também ser aplicada nesta camada para implementar casos de uso inovadores, como a geração automática de roteiros para videoaulas, a criação de \textit{chatbots} para interação com os aprendizes e a personalização do \textit{feedback} em atividades avaliativas.

\section{Camada de Adaptadores}

\begin{figure}[htb]
\centering
\caption{Camada de Adaptadores}
\label{fig:chapter3-speech2learning-layer3}
\includegraphics[width=.95\textwidth]{images/chapter3-speech2learning-layer3.png}
\fadaptada{FalvoJr2023_HICSS, Lemos2022}
\end{figure}

A Camada de Adaptadores na arquitetura \textit{Speech2Learning} tem a responsabilidade de converter os dados para um formato conveniente, considerando as necessidades das camadas com as quais faz fronteira. Esta camada atua como um intermediário crucial, garantindo que a comunicação entre as diferentes partes do sistema ocorra de maneira eficiente e coerente (\autoref{fig:chapter3-speech2learning-layer3}). As principais responsabilidades da Camada de Adaptadores incluem:

\begin{itemize}
    \item \textbf{Conversão de Dados}: Transformar os dados entre os formatos utilizados pelas camadas de Aplicação, Entidades e Infraestrutura, assegurando que cada camada receba os dados no formato mais adequado para sua funcionalidade.

    \item \textbf{Isolamento de Detalhes Técnicos}: Abstrair os detalhes técnicos de implementação, permitindo que as camadas superiores (Aplicação e Entidades) permaneçam independentes das preocupações técnicas específicas de integração com sistemas externos e infraestrutura.

    \item \textbf{Facilitação da Comunicação}: Garantir que a comunicação entre as diferentes camadas do sistema seja transparente e eficiente, minimizando os impactos de mudanças em uma camada sobre as demais.
\end{itemize}

Na \textit{Speech2Learning}, algumas abstrações são essenciais para a implementação eficaz dos adaptadores. Como sugestões de abstrações que fazem sentido para essa arquitetura destacam-se:

\begin{itemize}
    \item \textbf{Adaptadores de Entrada}: Componentes responsáveis por receber dados de fontes externas (como APIs, bancos de dados, ou interfaces de usuário) e convertê-los para um formato que possa ser processado pela Camada de Aplicação. Por exemplo, um adaptador de entrada pode receber dados de áudio de uma API de reconhecimento de fala e convertê-los em texto para processamento posterior.

    \item \textbf{Adaptadores de Saída}: Componentes que transformam dados gerados pela Camada de Aplicação em um formato que pode ser utilizado por sistemas externos ou armazenado em bancos de dados. Por exemplo, um adaptador de saída pode transformar transcrições de texto em um formato adequado para armazenamento em um ROAs.

    \item \textbf{Interfaces de Conversão de Dados}: Abstrações que definem métodos para converter dados entre diferentes formatos e protocolos. Estas interfaces garantem que a lógica de conversão seja desacoplada das implementações específicas, promovendo reutilização e facilidade de manutenção.

    \item \textbf{Serviços de Mapeamento de Dados}: Componentes que encapsulam a lógica de mapeamento entre diferentes estruturas de dados, facilitando a conversão de objetos complexos e a integração com sistemas heterogêneos. Por exemplo, um serviço de mapeamento pode converter objetos de dados internos para formatos compatíveis com padrões de metadados como SCORM ou LOM.
\end{itemize}

Para ilustrar a aplicação das abstrações sugeridas, considere um exemplo de um adaptador de entrada que implementa um cliente \textit{Open Feign} para interagir com um serviço externo de transcrição, especificamente a API da OpenAI. Este adaptador recebe dados de áudio, converte-os para um formato adequado para a API, e traduz as respostas da API para um formato compreensível pelo domínio. 

Essa implementação foi desenvolvida em colaboração com a DIO e está disponível neste repositório \textit{open-source}: \url{https://bit.ly/S2L-Adapters}. Tais abstrações e exemplos demonstram como a Camada de Adaptadores pode ser estruturada para garantir a conversão eficiente de dados entre diferentes partes do sistema \textit{Speech2Learning}, promovendo a interoperabilidade e a flexibilidade necessárias para uma solução educacional baseada em reconhecimento de fala.

Adicionalmente, a IAGen pode ser integrada aos adaptadores para realizar tarefas como a tradução automática de legendas, a geração de descrições de áudio para pessoas com deficiência visual e a conversão de formatos de dados complexos em formatos mais simples e acessíveis.

\section{Camada de Infraestrutura}

\begin{figure}[htb]
\centering
\caption{Camada de Infraestrutura}
\label{fig:chapter3-speech2learning-layer4}
\includegraphics[width=1\textwidth]{images/chapter3-speech2learning-layer4.png}
\fadaptada{FalvoJr2023_HICSS, Lemos2022}
\end{figure}

A Camada de Infraestrutura na Arquitetura \textit{Speech2Learning} abrange questões técnicas e específicas, como \textit{frameworks}, \textit{drivers} e integrações externas. Nesse sentido, conexões com Bancos de Dados (BD) e serviços de ASR/STT fazem parte desta camada. Além disso, inclui a Interface do Usuário (UI), permitindo o desenvolvimento de aplicações educacionais concretas (\textit{d-learning}, \textit{e-learning}, \textit{m-learning}, etc) ou repositórios de OAs (\autoref{fig:chapter3-speech2learning-layer4}). As principais responsabilidades da Camada de Infraestrutura incluem:

\begin{itemize}
    \item \textbf{Gerenciamento de Dados}: Conectar e interagir com sistemas de gerenciamento de bancos de dados para armazenar, recuperar e manipular os dados necessários para a aplicação, garantindo que os dados sejam acessíveis e seguros.

    \item \textbf{Integrações Externas}: Gerenciar a comunicação com serviços externos, como APIs de reconhecimento de fala e serviços de tradução, garantindo que a aplicação possa utilizar funcionalidades fornecidas por terceiros.

    \item \textbf{Suporte a UI}: Fornecer a infraestrutura necessária para o desenvolvimento e operação das interfaces de usuário, permitindo que os aprendizes interajam com os OAs de maneira eficaz e intuitiva.

    \item \textbf{Manutenção de Infraestrutura Técnica}: Gerenciar componentes técnicos, incluindo \textit{drivers}, bibliotecas e \textit{frameworks}, que são essenciais para o funcionamento da aplicação, assegurando que todos os elementos estejam atualizados e funcionando corretamente.
\end{itemize}

Para a \textit{Speech2Learning}, algumas abstrações são essenciais para a implementação eficaz da infraestrutura. Entre as sugestões de abstrações que fazem sentido para essa arquitetura tem-se:

\begin{itemize}
    \item \textbf{Repositórios de Dados}: Interfaces e implementações responsáveis por gerenciar a persistência dos dados, encapsulando a lógica de acesso a bancos de dados e garantindo a independência da Camada de Aplicação em relação às tecnologias de armazenamento.

    \item \textbf{Serviços de Integração}: Componentes que facilitam a comunicação com serviços externos, fornecendo interfaces padronizadas para interagir com APIs de reconhecimento de fala, serviços de tradução e outras ferramentas externas.

    \item \textbf{Gerenciadores de UI}: Abstrações que suportam o desenvolvimento e a operação das interfaces de usuário, assegurando que a experiência do usuário seja consistente e acessível.

    \item \textbf{\textit{Drivers} e Bibliotecas}: Componentes técnicos que fornecem funcionalidades específicas necessárias para a operação da aplicação, como \textit{drivers} de banco de dados, bibliotecas de manipulação de áudio e ferramentas de desenvolvimento de UI.
\end{itemize}

Para que se tenha uma perspectiva prática, segue um exemplo de um adaptador que implementa a persistência de dados de áudio transcritos utilizando o \textit{MongoDB} e seu sistema de arquivos \textit{GridFS}. Este adaptador lida com operações de \sigla{CRUD}{\textit{Create, Read, Update and Delete}} para entidades de áudio transcritas, gerenciando o armazenamento de conteúdo e metadados de maneira eficiente. Essa implementação está disponível no seguinte repositório \textit{open-source}: \url{https://bit.ly/S2L-Infra}.

Essas abstrações e exemplos demonstram como a Camada de Infraestrutura pode ser estruturada para fornecer o suporte técnico necessário para a Arquitetura \textit{Speech2Learning}, garantindo a interoperabilidade e a eficiência das operações de armazenamento e integração com serviços externos. Além disso, a IAGen pode ser utilizada para otimizar o desempenho da infraestrutura, por meio da análise de dados de uso e da identificação de padrões que permitam a alocação eficiente de recursos e a melhoria da escalabilidade do sistema.

\section{Pseudo-camada ``Principal \& Configuração''}

\begin{figure}[htb]
\centering
\caption{Pseudo-camada Principal \& Configuração}
\label{fig:chapter3-speech2learning-layer5}
\includegraphics[width=1\textwidth]{images/chapter3-speech2learning-layer5.png}
\fadaptada{FalvoJr2023_HICSS, Lemos2022}
\end{figure}

A Pseudo-camada Principal \& Configuração na Arquitetura \textit{Speech2Learning} é assim denominada porque, embora não seja uma camada formal na \textit{Clean Architecture}, desempenha um papel crucial na configuração e inicialização do sistema. Em linhas gerais, esta camada estabelece todas as conexões entre interfaces e suas implementações concretas, além de ser responsável por configurar e executar a aplicação (\autoref{fig:chapter3-speech2learning-layer5}). Além disso, a camada centraliza a configuração e a inicialização, assegurando que todos os componentes estejam devidamente conectados e preparados para operar. As principais responsabilidades desta pseudo-camada incluem:

\begin{itemize}
    \item \textbf{Dependências}: Definir e gerenciar as dependências entre os diferentes componentes do sistema, garantindo que cada componente receba suas dependências de maneira clara e organizada. Isso promove a modularidade e facilita a manutenção do sistema.

    \item \textbf{Inicialização da Aplicação}: Coordenar o processo de inicialização da aplicação, assegurando que todos os componentes sejam configurados e instanciados na ordem correta. Isso inclui a configuração de serviços, repositórios, adaptadores e outros componentes essenciais.

    \item \textbf{Gerenciamento de Ciclo de Vida}: Controlar o ciclo de vida dos componentes, incluindo a criação, a configuração e a destruição de instâncias conforme necessário. Isso assegura que os recursos sejam gerenciados de maneira eficiente e que os componentes estejam sempre em um estado consistente.
\end{itemize}

Para a \textit{Speech2Learning}, algumas abstrações são essenciais para a implementação eficaz da Pseudo-camada Principal \& Configuração. Como sugestões de abstrações que fazem sentido para essa arquitetura tem-se:

\begin{itemize}
    \item \textbf{Módulos de Configuração}: Componentes que centralizam a configuração dos diferentes aspectos da aplicação, como casos de uso, repositórios de dados e serviços de integração. Estes módulos garantem que a configuração seja organizada e fácil de gerenciar.

    \item \textbf{Fábricas de Instâncias}: Abstrações que fornecem métodos para criar e configurar instâncias dos componentes do sistema. As fábricas asseguram que a criação de componentes siga um padrão consistente e que todas as dependências sejam satisfeitas.

    \item \textbf{Controladores de Inicialização}: Componentes responsáveis por coordenar o processo de inicialização da aplicação, garantindo que todos os componentes sejam configurados e preparados para operar. Os controladores de inicialização gerenciam a ordem de inicialização e tratam quaisquer dependências entre os componentes.
\end{itemize}

Para ilustrar a aplicação das abstrações listadas, considere um módulo de configuração que define e gerencia os casos de uso relacionados à transcrição de áudio. Este módulo centraliza a configuração das dependências necessárias para os casos de uso, promovendo a independência e a modularidade dos componentes em relação ao \textit{Spring Framework}. Exemplo em um projeto real disponível no repositório \textit{open-source}: \url{https://bit.ly/S2L-Adapters}.

Estas abstrações e exemplos demonstram como a Pseudo-camada Principal \& Configuração pode ser estruturada para garantir a configuração e a inicialização eficiente dos componentes do sistema \textit{Speech2Learning}. Ao centralizar a configuração e a gestão de dependências, esta camada promove a modularidade e a flexibilidade necessárias para uma solução educacional baseada em reconhecimento de fala.

\section{Considerações Finais}

A Arquitetura \textit{Speech2Learning} foi concebida para promover a inclusão e a acessibilidade no domínio educacional, utilizando tecnologias avançadas de reconhecimento de fala, como o ASR e STT. Ao longo deste trabalho de doutorado, foi discutida a estrutura e as camadas fundamentais da \textit{Speech2Learning}, enfatizando sua modularidade, escalabilidade e conformidade com padrões de mercado. Essas características permitem que a arquitetura seja adaptável a diversos contextos educacionais e atenda às necessidades específicas de diferentes aprendizes, promovendo um ambiente de aprendizagem mais inclusivo e acessível.

A implementação das camadas da \textit{Speech2Learning} demonstrou como a integração de serviços de ASR e STT pode ser realizada de maneira estruturada e eficaz. A arquitetura não só facilita o desenvolvimento de soluções educacionais inclusivas, mas também promove a interoperabilidade entre diferentes plataformas e ferramentas educacionais. Além disso, a adoção de padrões de metadados como Dublin Core, SCORM, MISB, LOM e RDF garante que os OAs sejam estruturados de forma consistente e reutilizável, facilitando o compartilhamento de recursos educacionais.

A sinergia do conceito de IAGen na Arquitetura \textit{Speech2Learning} representa um avanço significativo na busca por soluções educacionais mais inclusivas, personalizadas e eficazes. Ao combinar o poder do ASR/STT com a capacidade de processamento de linguagem natural dos LLMs, a \textit{Speech2Learning} torna-se uma ferramenta poderosa para a criação e o acesso a OAs de alta qualidade, que atendem às necessidades de uma ampla gama de aprendizes.

É importante ressaltar que a adoção de IAGen na \textit{Speech2Learning} não é obrigatória, mas sim uma possibilidade que pode ser explorada para aprimorar ainda mais a arquitetura e suas funcionalidades. A flexibilidade da arquitetura permite que diferentes tecnologias e abordagens sejam integradas de acordo com as necessidades e os recursos disponíveis, garantindo que a \textit{Speech2Learning} continue a evoluir e a se adaptar às demandas do cenário educacional em constante transformação.

No próximo capítulo são apresentados os estudos de caso que ilustram a aplicação prática da \textit{Speech2Learning}. O primeiro estudo de caso avalia uma API para a transcrição e legendagem automática de videoaulas, integrando serviços de ASR de fornecedores líderes do mercado. O segundo estudo de caso explora o desenvolvimento de um player de vídeo com integração de avatares de Libras, destacando a importância do design universal para a inclusão de aprendizes surdos. Essas instâncias práticas da \textit{Speech2Learning} não só demonstram a versatilidade e a eficácia da arquitetura, mas também fornecem percepções valiosas sobre a implementação e a avaliação de soluções educacionais baseadas em reconhecimento de fala.

\chapter{Arquitetura Speech2Learning na Prática: Estudos de Caso na Industria}
\label{chapter4}
\section{Considerações Iniciais}

Conforme discutido anteriormente, a acessibilidade em ambientes de aprendizagem é uma necessidade crescente na era digital, onde as TICs desempenham um papel crucial ao tornar os conteúdos educacionais acessíveis a todos os alunos, independentemente de suas individualidades físicas ou sensoriais \cite{Mayer2021}. Neste contexto, o presente trabalho de doutorado investiga como o enriquecimento de OAs com transcrições e legendas automáticas pode aumentar a inclusão e o engajamento na educação \cite{FalvoJr2023_HICSS, FalvoJr2024_FIE}.

O conceito de OAs é central para as práticas pedagógicas atuais, pois esses recursos digitais flexíveis podem ser reutilizados para apoiar o processo de ensino-aprendizagem \cite{Parakh2022}. A principal contribuição deste trabalho de doutorado é a Arquitetura \textit{Speech2Learning}, que propõe o aprimoramento de OAs audíveis por meio de tecnologias de ASR/STT, promovendo a geração automática de legendas, transcrições ou traduções. Dessa forma, videoaulas podem se tornar acessíveis em diferentes línguas e até mesmo sinalizadas por avatares de línguas de sinais baseados em texto.

Este capítulo detalha a aplicação prática da \textit{Speech2Learning} por meio de dois estudos de caso, concebidos como instâncias concretas da arquitetura. Cada estudo é apresentado individualmente (\autoref{c4:cs1} e \autoref{c4:cs2}) e segue uma estrutura comum de subseções para facilitar a compreensão. As subseções incluem: ``Relação com a Arquitetura \textit{Speech2Learning}'', que descreve como a arquitetura foi implementada e integrada no contexto do estudo; ``Prova de Conceito'', que apresenta a solução desenvolvida como um exemplo prático da \textit{Speech2Learning}; ``Metodologia'', discorre sobre os métodos e procedimentos utilizados na condução do estudo; e ``Resultados e Discussões'', onde são analisados os dados coletados e discutidas as implicações dos achados para a acessibilidade educacional e a arquitetura \textit{Speech2Learning}.

Esses estudos foram conduzidos em colaboração com a \textit{EdTech} brasileira DIO, que desempenhou um papel essencial ao viabilizar avaliações empíricas em OAs de uma plataforma educacional real, marcando um passo significativo na direção de soluções mais acessíveis e inclusivas. Todos os procedimentos éticos relacionados aos estudos de caso foram aprovados pelo CEP, garantindo a conformidade com as diretrizes éticas exigidas.

\section{Estudo de Caso 1: Legendas em Videoaulas}
\label{c4:cs1}

O primeiro estudo de caso buscou investigar a precisão dos principais serviços de ASR, uma vez que essa tecnologia é central na Arquitetura \textit{Speech2Learning}. Na prática, foi utilizado um conjunto de recursos educacionais, disponíveis na plataforma de \textit{e-learning} da DIO, para a construção de uma \sigla{POC}{Prova de Conceito} focada na transcrição e legendagem desses OAs através de serviços de ASR.

Sendo assim, o objetivo do estudo foi expandir a compreensão sobre o papel das transcrições automáticas no processo de ensino-aprendizagem mais acessível. Para isso, foi discutida a convergência dos resultados quantitativos e qualitativos obtidos a partir de três fontes de evidência, seguindo a premissa da metodologia de triangulação de dados \cite{LimaJunior2021}.

Segundo \cite{Farquhar2020}, é possível explorar múltiplas fontes de evidências (quantitativas e/ou qualitativas) e discutir suas respectivas convergências por meio de uma abordagem de triangulação de dados. No contexto deste estudo de caso, as seguintes fontes foram observadas: (i) métodos de similaridade léxica; (ii) respostas de um \textit{survey} anônimo; e (iii) estudos de uma pesquisa documental complementar.

As subseções a seguir exploram o estudo em detalhes, abordando desde a relação com a Arquitetura \textit{Speech2Learning} e a implementação de uma POC em colaboração com a \textit{EdTech} DIO, até os pormenores metodológicos e os resultados alcançados.

\subsection{Relação com a Arquitetura \textit{Speech2Learning}}

No primeiro estudo de caso, a Arquitetura \textit{Speech2Learning} foi implementada como uma API REST, projetada para seguir rigorosamente as responsabilidades de suas camadas e os padrões do estilo arquitetural REST. A adoção da \textit{Speech2Learning} forneceu diretrizes claras para a separação de interesses entre as camadas da API, facilitando a manutenção e a testabilidade do sistema, além de permitir a substituição ágil do serviço de ASR. Esse \textit{design} modular garante que diferentes provedores de ASR possam ser integrados ou trocados sem grandes impactos na estrutura da API, preservando a consistência e a transparência no funcionamento geral.

O diferencial da \textit{Speech2Learning} está na sua aderência a padrões de metadados dos OAs, assegurando que esses recursos sejam enriquecidos com transcrições e legendas de forma padronizada. Essa característica facilita a reutilização dos OAs em diversos contextos educacionais, garantindo acessibilidade e consistência em termos de estrutura e conteúdo.

Embora a avaliação do estudo de caso se concentre na instância concreta da arquitetura, a flexibilidade inerente à \textit{Speech2Learning} possibilitou a criação de uma API robusta, capaz de se conectar a múltiplos serviços de ASR. Esses serviços são implementados por meio de \textit{Gateways} que seguem uma interface comum, permitindo a troca entre componentes de forma transparente e eficiente. Essa abordagem permite uma avaliação abrangente dos provedores de ASR, sem comprometer a integridade da API ou dos OAs gerados.

Simplificando, a relação entre a \textit{Speech2Learning} e este estudo de caso resultou em uma API REST construída em conformidade com os \textit{guidelines} da arquitetura. Na prática, a \textit{Speech2Learning} atende a um requisito essencial deste estudo, que é a necessidade de múltiplos provedores de ASR/STT desacoplados para a geração e avaliação das transcrições automáticas de videoaulas. Para isso, foi iniciada uma POC estratégica com o objetivo de mensurar a complexidade do desenvolvimento e a viabilidade técnica (inicialmente com um único provedor).

A POC é detalhada na seção a seguir e representa a primeira instância concreta da \textit{Speech2Learning}. Sendo assim, essa implementação proporcionou inúmeros aprendizados e explorou a essência da arquitetura, promovendo a acessibilidade por meio do ASR em OAs existentes na plataforma educacional da DIO. A oportunidade de testar a arquitetura em um ecossistema de aprendizagem real foi um passo significativo para a maturidade da \textit{Speech2Learning}.

\subsection{Prova de Conceito: API REST para Legendar Videoaulas}

O desenvolvimento de uma POC, primeira instância da \textit{Speech2Learning}, surgiu de uma necessidade de negócio da \textit{EdTech} DIO de legendar suas videoaulas de forma escalável. Para isso, uma API REST foi implementada para transcrever videoaulas curtas (entre 15 e 30 segundos), algo comum considerando o conceito de \textit{microlearning} adaptado pela DIO para suas necessidades educacionais. 

Com isso, foi possível explorar os padrões de metadados inerentes à Arquitetura \textit{Speech2Learning} e gerar transcrições e legendas para videoaulas em múltiplos idiomas. Nesse sentido, foi definido o escopo da POC aos idiomas suportados na plataforma da DIO (Português, Inglês e Espanhol), sendo possível ampliar a acessibilidade desses OAs por meio de transcrições e legendas geradas automaticamente a priori.

Conforme o diagrama da \autoref{fig:chapter4-cs1-poc-diagram}, a API segue as diretrizes da \textit{Speech2Learning}, bem como apresenta uma estrutura de componentes e conectores \cite{Bass2021}. O esquema de cores dos elementos corresponde ao das camadas da arquitetura, ilustrando uma implementação em conformidade com os limites lógicos/estruturais recomendados para uma API REST.

\begin{figure}[htb]
\centering
\caption{Visão 1ª Instância da \textit{Speech2Learning}: API REST para Legendar Videoaulas}
\label{fig:chapter4-cs1-poc-diagram}
\includegraphics[width=\columnwidth]{images/chapter4-cs1-poc-diagram.png}
\fdireta{FalvoJr2023_HICSS}
\end{figure}

O diagrama define dois \textit{Casos de Uso} relacionados à transcrição de vídeos: \textit{Criar} e \textit{Revisar}. Esta visão é interessante pois esclarece as funcionalidades implementadas na POC. Além disso, as entidades foram projetadas como \textit{OAs Audíveis}, uma vez que são consideradas videoaulas. Para concluir, detalhes sobre a implementação também são expostos claramente na camada de \textit{Infraestrutura}. Por exemplo, o ponto de entrada é uma \textit{API REST}, o \textit{MongoDB} é o banco de dados e \textit{Google STT API} (API externa) é o provedor de STT.

Para oferecer uma perspectiva mais técnica, a POC pode ser consumida sob demanda através de uma requisição HTTP POST para uma API REST reativa. Este \textit{endpoint} aciona um fluxo de trabalho assíncrono para extrair o áudio do vídeo, otimizando assim o consumo de largura de banda antes de invocar a \textit{Google STT API}. Após a resposta dessa requisição, a transcrição é armazenada nos metadados do \textit{OA Audível}, tornando essa transcrição automática elegível para revisão.

Para revisar a transcrição de um OA existente, basta fazer uma requisição HTTP PUT para atualizar este recurso, versionando suas transcrições por meio dos metadados. Vale lembrar que a \textit{Speech2Learning} apenas define diretrizes para a criação de soluções baseadas em STT para promover a acessibilidade de OAs, sem impor decisões de projeto, tecnologias ou aspectos de segurança.

Na prática, o acesso à POC foi restrito à Equipe de Educação da DIO, garantindo que os OAs (videoaulas) fossem criados e revisados com segurança e em um ambiente controlado. Ressalta-se que as transcrições automáticas exigiram uma extensa revisão por especialistas em línguas da empresa, destacando a necessidade de otimizar o processo de transcrição e explorar outros serviços de STT.

Inspirados pelos \textit{insights} e desafios desta POC, procurou-se expandir as percepções por meio de um estudo de caso mais amplo. Por sua vez, esta iniciativa de investigação visa analisar mais profundamente as nuances entre diferentes provedores de STT. Portanto, para o estudo de caso, foram selecionadas 15 videoaulas cujas transcrições foram revisadas por especialistas em idiomas da DIO durante a POC. Esta quantidade e duração de vídeos foram estrategicamente definidas para controle financeiro (custo dos provedores de STT) e, principalmente, viabilizar um \textit{survey} coeso para avaliação da precisão das transcrições automáticas sob novas perspectivas.

Para uma análise mais robusta, procurou-se diversificar as línguas e os professores das videoaulas, de forma a captar diferentes sotaques e regionalidades. O \autoref{quadro:c4:poc-audios-summary} detalha as características linguísticas e técnicas dos áudios extraídos das videoaulas. Este conjunto diversificado de dados servirá como referência, permitindo analisar criticamente o desempenho de outros fornecedores de STT em múltiplos contextos de ensino-aprendizagem.

\begin{quadro}[htb]
\centering
\caption{\textit{Dataset} do Estudo de Caso 1 (Áudios Extraídos das Videoaulas)}
\label{quadro:c4:poc-audios-summary}
\begin{tabular}{c|c|c|c|l|c}
\hline
\textbf{ID} & \textbf{Língua} & \textbf{Sotaque} & \textbf{Gênero} & \textbf{Tópico Educacional} & \textbf{Tempo} \\ \hline
1 & pt-BR & BRA & M & Apps Android & 0:17 \\ \hline
2 & pt-BR & BRA & M & SCRUM & 0:26 \\ \hline
3 & pt-BR & BRA & F & Selenium WebDriver & 0:23 \\ \hline
4 & pt-BR & BRA & F & Blockchain & 0:20 \\ \hline
5 & pt-BR & BRA & M & Kernel Híbrido & 0:24 \\ \hline
6 & en-US & BRA & F & Visto de Trânsito & 0:29 \\ \hline
7 & en-US & USA & Ambos & Entrevista de Emprego & 0:16 \\  \hline
8 & en-US & BRA & F & Oportunidades de Emprego & 0:20 \\  \hline
9 & en-US & BRA & F & Liderança Servidora & 0:15 \\  \hline
10 & en-US & BRA & M & Goroutines & 0:15 \\  \hline
11 & es-AR & ARG & F & Lógica de Programação & 0:12 \\  \hline
12 & es-AR & ARG & F & Linguagens de Programação & 0:21 \\  \hline
13 & es-AR & ARG & F & Tipos de Dados em Python & 0:14 \\  \hline
14 & es-AR & ARG & F & Hello World com Python & 0:17 \\  \hline
15 & es-AR & ARG & F & String Slicing com Python & 0:26 \\ \hline
\end{tabular}
\end{quadro}

Este conjunto de dados (\textit{dataset}) passou por um rigoroso controle de qualidade de áudio, protocolo padrão para todo conteúdo oferecido na plataforma educacional da DIO. Para garantir a transparência dessas informações, foi disponibilizada uma pasta pública\footnote{\textit{Dataset} do Estudo de Caso 1: \url{https://bit.ly/S2L-CS1-AudibleLOs}} que contém todas as amostras de áudio e suas respectivas transcrições revisadas por especialistas em línguas da DIO. 

Tecnicamente, o \textit{dataset} atende ou excede os seguintes critérios de qualidade: canais de áudio duplos, taxa de amostragem de 44,1 kHz e precisão de 16 bits. Tais aspectos técnicos não só garantem excelente qualidade de som, mas também contribuem para transcrições automáticas mais precisas.

Nesse contexto, a realização de um estudo de caso é uma opção adequada para ampliar esta POC e orientar análises e discussões mais profundas. Resumidamente, a POC identificou a necessidade de reduzir o retrabalho exigido na revisão das transcrições automáticas e assim melhorar a eficiência do processo de transcrição. Esta progressão da POC para um estudo de caso reflete uma abordagem comum em pesquisa que permite o desenvolvimento iterativo \cite{Runeson2009}. 

O \textit{dataset} compilado nesta POC permitiu analisar a qualidade das transcrições automáticas em vários provedores de STT, usando as transcrições de referência revisadas pelos especialistas em idiomas da DIO como um controle confiável. A metodologia definida para o estudo de caso decorrente da POC, bem como seus resultados são apresentados a seguir.

\subsection{Metodologia}

Esta seção descreve a metodologia deste estudo de caso, com o objetivo de investigar o papel do ASR e do STT na melhoria da acessibilidade de OAs. Trata-se de uma investigação empírica, apoiada em raciocínio indutivo e pesquisa de campo. Ao contrário dos estudos experimentais, os estudos de caso reúnem informações de diversas fontes por meio de diferentes técnicas de coleta de dados \cite{Sommerville2015}.

Como dito anteriormente, foi estabelecida uma parceria com a \textit{EdTech} DIO, que compartilhou videoaulas disponíveis em seu currículo, oferecendo uma oportunidade de conexão entre a indústria e nossa pesquisa. A DIO concedeu acesso a parte de sua infraestrutura em nuvem, além de disponibilizar especialistas em línguas que revisaram os resultados do ASR durante a POC. Esse processo de revisão foi fundamental para que as fontes de evidência, baseadas na precisão das transcrições e legendas de videoaulas, tivessem referências sólidas para nossas comparações e análises.

Como consequência, tanto os dados de similaridade léxica quanto as respostas do \textit{survey} avaliaram o mesmo conjunto de 15 videoaulas: 5 em inglês, 5 em português e 5 em espanhol. Essa abordagem facilitou a obtenção de duas perspectivas quantitativas distintas sobre a qualidade das transcrições automáticas: uma derivada de algoritmos de similaridade léxica e a outra baseada nas percepções dos aprendizes.

Adicionalmente, a metodologia integra um terceiro aspecto de coleta de dados através de uma pesquisa documental focada em ASR/STT, além de aplicar a Teoria Fundamentada \cite{Charmaz2009} para assegurar um processo de análise de conteúdo. Portanto, a pesquisa documental visa fornecer dados qualitativos sobre o fenômeno observado \cite{LimaJunior2021}.

Ao triangular os dados coletados de similaridade léxica, das respostas do \textit{survey} anônimo e da pesquisa documental complementar (\autoref{fig:chap4:triangulation-sources}), pretende-se oferecer uma compreensão abrangente dos desafios e oportunidades associados à utilização de tecnologias ASR para aumentar a acessibilidade de OAs no domínio educacional. 

\begin{figure}[htb]
\centering
\includegraphics[width=0.75\textwidth]{images/chapter4-cs1-triangulation-sources.png}
\caption{Triangulação de Dados: Fontes de Evidência.}
\label{fig:chap4:triangulation-sources}
\fdireta{FalvoJr2024_FIE}
\end{figure}

A seguir, cada fonte de evidência e suas técnicas de coleta são delineadas, esclarecendo todo o processo de triangulação e sua contribuição para uma compreensão robusta do impacto das tecnologias ASR, como o STT, na acessibilidade educacional.

\subsubsection{1ª Fonte de Evidência: Métodos de Similaridade Léxica}

Esta seção apresenta a metodologia definida para a primeira fonte de evidências da triangulação de dados, onde métodos de similaridade léxica têm como objetivo avaliar e identificar o provedor de ASR/STT mais assertivo para transcrição automática. Nesse sentido, algoritmos de similaridade léxica são métricas particularmente relevantes para a avaliação da precisão de transcrições automáticas, conforme discutido em \citeonline{FalvoJr2023_HICSS}.

Medir a semelhança entre textos é uma prática amplamente investigada em pesquisas acadêmicas, geralmente apoiada por métodos de análise léxica \cite{Majumdar2022}. Para comparar as transcrições automáticas com as de referência, é preciso definir como extrair os dados quantitativos por meio de técnicas de similaridade léxica. No contexto deste trabalho de doutorado, e conforme fundamentado anteriormente, destacam-se as seguintes alternativas:

\begin{itemize}

\item \textbf{\sigla{LD}{Distância de Levenshtein}}: Esta métrica de \textit{string} mede a diferença entre duas strings. Basicamente, a LD entre duas palavras é o número mínimo de edições de um único caractere (inserções, exclusões ou substituições) necessárias para transformar uma palavra na outra. Seu nome é uma homenagem a Vladimir Levenshtein, que considerou essa distância em 1965. A comparação LD é geralmente realizada entre duas palavras, determinando o número mínimo de edições necessárias para alterar uma palavra para outra. Quanto maior o número de edições, maior a diferença entre os textos. Uma edição é definida como a inserção, exclusão ou substituição de um caractere \cite{levens-1,levens-2}.

\item \textbf{\sigla{JI}{Índice de Jaccard}}: Esta métrica é definida como a interseção de dois conjuntos de dados dividida pela união desses conjuntos, medindo a similaridade entre eles. No contexto de textos, pode ser expressa como o número de palavras comuns dividido pelo número total de palavras nos dois textos ou documentos. O JI varia de 0 a 1, onde 0 significa nenhuma similaridade e 1 significa sobreposição completa. O JI é calculado dividindo o número de observações em ambos os conjuntos pelo número de observações em cada conjunto, ou seja, o tamanho da interseção dividido pelo tamanho da união de dois conjuntos \cite{jaccard-1,jaccard-2}.

\item \textbf{\sigla{CS}{Similaridade de Cosseno}}: Esta métrica é usada para medir a similaridade entre dois vetores. Especificamente, ela mede a semelhança na direção ou orientação dos vetores, ignorando as diferenças na sua magnitude ou escala. Ambos os vetores devem pertencer ao mesmo espaço de produto interno, o que significa que devem produzir um escalar ao multiplicar o produto interno. A similaridade de dois vetores é medida pelo cosseno do ângulo entre eles. A métrica CS varia de 0 a 1, onde um valor mais próximo de 0 indica menor similaridade e um valor mais próximo de 1 indica maior similaridade \cite{cosseno-1,cosseno-2,cosseno-3}.

\end{itemize}

Neste estudo, foram aplicados os três métodos mencionados, destacando sua relevância. Primeiramente, foram geradas e revisadas as transcrições de referência usando a instância da \textit{Speech2Learning}, desenvolvida como uma POC. Em seguida, cada OA foi transcrito automaticamente utilizando diferentes serviços de ASR/STT, permitindo a comparação entre as transcrições de referência e as automáticas. Dessa forma, foram utilizados os métodos de similaridade léxica CS, JI e LD como métricas de precisão.

Para essa finalidade, foram integrados os principais provedores de ASR/STT baseados em IA do mercado \cite{Gartner2023}: Amazon, Google, IBM, Microsoft e OpenAI. Ademais, em colaboração com a DIO, garantiu-se que o \textit{dataset} deste estudo de caso, devidamente revisado e sintetizado no \autoref{quadro:c4:poc-audios-summary}, estivesse disponível com todas as transcrições de referência dos OAs.

A partir dos áudios do \textit{dataset}, foram integrados os serviços de ASR de cada provedor e geradas suas respectivas transcrições automáticas. Para uma implementação organizada e documentada, utilizou-se o Google Colab, detalhando todo o processo em um notebook público\footnote{Colab Notebook - Transcrições Automáticas com ASR/STT: \url{https://bit.ly/S2L-CS1-STTServices}}. De forma similar, foram comparadas as 15 transcrições automáticas com as transcrições de referência usando os três métodos de similaridade léxica (CS, JI e LD) em um segundo notebook\footnote{Colab Notebook - Algoritmos de Similaridade Léxica: \url{https://bit.ly/S2L-CS1-LexicalAnalysis}}.

\subsubsection{2ª Fonte de Evidência: Respostas do \textit{Survey}}

Como segunda fonte de evidência, foi conduzido um \textit{survey} para avaliar as percepções dos alunos sobre a qualidade das transcrições automáticas fornecidas por ASR/STT. Esta pesquisa, que contou com 56 respostas, coletou \textit{feedback} quantitativo e qualitativo de participantes com experiência em tecnologia e educação. As respostas foram disponibilizadas em \url{https://bit.ly/S2L-CS1-SurveyResp}, bem como uma cópia do formulário no \autoref{appendix:asr-survey} e outra online (\url{https://bit.ly/S2L-CS1-Survey}), garantindo uma compreensão completa dos dados e da dinâmica de preenchimento do questionário.

O \textit{survey} utilizou uma combinação de perguntas de escala Likert e perguntas abertas. As perguntas de escala Likert visavam avaliar quantitativamente a coerência das transcrições automáticas em três idiomas (inglês, português e espanhol). Esta estratégia permitiu uma análise cruzada entre as avaliações técnicas/léxicas e as centradas no usuário. As perguntas abertas buscavam obter \textit{insights} qualitativos sobre a eficácia percebida das soluções de ASR. No entanto, devido à conformidade ética, os dados qualitativos não serão explorados neste estudo.

Os participantes avaliaram as transcrições do mesmo conjunto de 15 videoaulas (cinco por idioma) e dos mesmos cinco provedores (Amazon, Google, IBM, Microsoft e OpenAI) utilizados no estudo de similaridade léxica. Este alinhamento entre as duas fontes garantiu que as respostas do \textit{survey} complementassem diretamente a análise técnica de similaridade léxica, proporcionando uma visão holística do desempenho das transcrições.

Para mitigar vieses, os provedores foram anonimizados e receberam um identificador numérico aleatório em vez de serem nomeados, garantindo que as avaliações refletissem experiências genuínas dos usuários, sem influência do reconhecimento da empresa/marca. Esta abordagem metodológica está alinhada às melhores práticas em metodologia de pesquisa empírica, conforme defendido por \cite{Sommerville2015}, especialmente na coleta de \textit{feedbacks} e percepções dos usuários em estudos de engenharia de software.

\subsubsection{3ª Fonte de Evidência: Pesquisa Documental}

Para aprofundar a análise dos dados quantitativos, foi realizada uma revisão de literatura que complementa os achados empíricos, identificando estudos relevantes com foco em ASR. Esta revisão oferece uma compreensão mais ampla do potencial e das limitações das tecnologias atuais baseadas em ASR/STT.

Foi utilizada a \textit{Grounded Theory}, ou \sigla{TFD}{Teoria Fundamentada nos Dados}, como um \textit{framework} orientador para a análise documental. A TFD é uma metodologia sistemática e rigorosa utilizada na pesquisa qualitativa para desenvolver teorias emergentes diretamente dos dados \cite{Charmaz2009}. Diferentemente de abordagens que partem de hipóteses preconcebidas, a Teoria Fundamentada nos Dados permite a descoberta de temas e padrões de maneira indutiva, por meio de um processo iterativo de coleta e análise de dados.

Neste estudo, utilizou-se a TFD para examinar os estudos selecionados na pesquisa documental. Este processo incluiu a codificação aberta dos dados, a identificação de temas recorrentes e a integração desses temas em uma teoria coesa. Assim, garantiu-se que as conclusões fossem baseadas diretamente nas evidências coletadas, sem impor noções preconcebidas.

A pesquisa documental envolveu uma revisão complementar da literatura, incluindo artigos, livros e relatórios, focando nas tecnologias ASR e STT no contexto da acessibilidade educacional. Através dessa análise, buscou-se identificar temas recorrentes, diretrizes teóricas e lacunas nas pesquisas existentes, contribuindo para uma compreensão mais detalhada do assunto.

Ao integrar \textit{insights} do conjunto de dados de similaridade lexical, das respostas do \textit{survey} e da pesquisa documental, fornecendo uma visão abrangente e multidimensional do papel do ASR e do STT na melhoria da acessibilidade educacional.

\subsubsection{Triangulação de Dados}

A triangulação, originalmente uma técnica geométrica para determinação de localização, evoluiu para uma metáfora representando métodos de pesquisa que integram diversas abordagens, teorias ou fontes de dados. Essa integração permite uma compreensão mais abrangente de um fenômeno, sendo especialmente útil na pesquisa qualitativa. A triangulação aumenta a validade de um estudo ao empregar vários métodos para verificar ou corroborar um evento, descrição ou fato específico, reforçando assim a credibilidade e a confiabilidade do estudo \cite{Farquhar2020, Yin2015}.

Neste estudo, adotou-se uma abordagem de triangulação que engloba múltiplas fontes de evidência e suas respectivas técnicas. Esta abordagem de métodos mistos combina resultados qualitativos e quantitativos para aprimorar a análise e a interpretação dos dados coletados. Em particular, a triangulação de dados utiliza várias fontes de evidência para corroborar o mesmo fato ou fenômeno \cite{Yin2015}.

Na Engenharia de Software, a triangulação contribui significativamente para o rigor e a validade da pesquisa. A integração de vários métodos, como pesquisas quantitativas, entrevistas qualitativas e análise documental, permite aos pesquisadores verificar os resultados por meio de uma análise cruzada e obter uma compreensão ainda mais aprofundada de fenômenos complexos. Esta abordagem é particularmente valiosa na análise da eficácia das práticas de desenvolvimento de software, experiência do usuário e adoção de tecnologias \cite{Runeson2009}.

A \autoref{fig:chapter4-cs1-triangulation-methodology} representa a metodologia de triangulação de dados utilizada neste trabalho de doutorado. Este diagrama esboça o objeto de estudo, as três fontes distintas de evidência, suas respectivas técnicas de coleta e o processo de convergência de dados que resulta no corpus de dados. Esta abordagem completa e multifacetada garante uma validação robusta e proporciona uma compreensão profunda de como as tecnologias ASR podem melhorar a acessibilidade educacional.

\begin{figure}[htb]
\centering
\caption{Metodologia de Triangulação de Dados Utilizada no Estudo de Caso 1.}
\label{fig:chapter4-cs1-triangulation-methodology}
\includegraphics[width=0.9\textwidth]{images/chapter4-cs1-triangulation-methodology.png}
\fdireta{FalvoJr2024_FIE}
\end{figure}

Para as fontes de evidência quantitativas, similaridade léxica e respostas do \textit{survey}, se estabeleceu dois conjuntos de hipóteses que guiaram as análise. Nesse contexto, hipóteses compartilhadas fazem sentido porque ambas as fontes de evidência exploram as mesmas videoaulas e avaliam os mesmos provedores. Sendo assim, primeiro foi considerada a comparação entre diferentes provedores de transcrição automática, independentemente do idioma. As hipóteses são formuladas da seguinte maneira:

\begin{itemize}
\item \textbf{Hipótese Nula ($H^0$)}: Não há diferença estatisticamente significativa na qualidade das transcrições automáticas entre os provedores.
\item \textbf{Hipótese Alternativa ($H^1$)}: Há uma diferença estatisticamente significativa na qualidade das transcrições automáticas entre os provedores.
\end{itemize}

Em seguida, foi avaliada a qualidade das transcrições automáticas considerando os idiomas Português, Inglês e Espanhol. As hipóteses são delineadas conforme se segue:

\begin{itemize}
\item \textbf{Hipótese Nula ($H^0$)}: Não há diferença estatisticamente significativa na qualidade das transcrições automáticas para Português, Inglês e Espanhol entre os provedores.
\item \textbf{Hipótese Alternativa ($H^1$)}: Há uma diferença estatisticamente significativa na qualidade das transcrições automáticas para Português entre os provedores.
\item \textbf{Hipótese Alternativa ($H^2$)}: Há uma diferença estatisticamente significativa na qualidade das transcrições automáticas para Inglês entre os provedores.
\item \textbf{Hipótese Alternativa ($H^3$)}: Há uma diferença estatisticamente significativa na qualidade das transcrições automáticas para Espanhol entre os provedores.
\end{itemize}

Para a pesquisa documental, adotou-se uma abordagem qualitativa para investigar de maneira mais ampla e exploratória como as tecnologias de reconhecimento de fala para transcrição automática (ASR e STT) influenciam a acessibilidade dos OAs. A questão de pesquisa que orientou a análise documental foi:

\begin{itemize}
\item \textbf{Questão de Pesquisa}: De que maneira as tecnologias de Reconhecimento Automático de Fala (ASR) podem contribuir para melhorar a acessibilidade educacional?
\end{itemize}

Essa combinação de hipóteses quantitativas e uma questão de pesquisa qualitativa permite uma triangulação robusta de dados, fornecendo uma visão abrangente e multidimensional do impacto das tecnologias de ASR/STT na acessibilidade educacional. A análise das respostas do \textit{survey}, juntamente com os métodos de similaridade léxica e a pesquisa documental, permite identificar as convergências entre as fontes de evidências e enriquecer as discussões, fortalecendo a validade do estudo.

Para apresentar uma visão formal, porém simplificada, deste primeiro estudo de caso, o \autoref{quadro:c4:cs1-summary} resume suas principais características. Esta síntese foi elaborada seguindo as diretrizes metodológicas e as perspectivas de pesquisa descritas por \citeonline{CastroFilho2021}, garantindo uma abordagem estruturada e coerente com os padrões de pesquisa científica em informática na educação.

\begin{quadro}[htb]
\centering
\caption{Síntese do Estudo de Caso 1: Legendas Automáticas em Videoaulas}
\label{quadro:c4:cs1-summary}
\begin{tabular}{C{3cm}|m{11.75cm}}\hline
\textbf{Objeto de Estudo} & Serviços de Reconhecimento Automático de Fala (ASR/STT) \\\hline
\textbf{Perspectiva} & Explanatória (Explicativa) \\\hline
\textbf{Característica} & Explicação de causas ou efeitos relacionais ao fenômeno \cite{CastroFilho2021}. \\\hline
\textbf{Objetivos} & \begin{tabular}[c]{@{}m{11.75cm}@{}}1. Avaliar a qualidade das transcrições automáticas fornecidas por diferentes serviços de ASR/STT. \\ 2. Investigar a variação na qualidade das transcrições automáticas em diferentes idiomas (Português, Inglês e Espanhol). \\ 3. Explorar como as transcrições automáticas podem melhorar a acessibilidade educacional.\end{tabular} \\\hline
\textbf{Questões de Pesquisa} & De que maneira as tecnologias de ASR/STT podem contribuir para melhorar a acessibilidade educacional? \\\hline
\textbf{Hipóteses} & \begin{tabular}[c]{@{}m{11.75cm}@{}}1. Existe diferença estatisticamente significativa na precisão das transcrições automáticas entre os provedores de ASR/STT. \\ 2. Existe diferença estatisticamente significativa na precisão das transcrições automáticas para Português, Inglês e Espanhol entre os provedores de ASR/STT.\end{tabular} \\\hline
\textbf{Fontes de Dados} & \begin{tabular}[c]{@{}m{11.75cm}@{}}1. Dados quantitativos de métodos de similaridade léxica (precisão). \\ 2. Respostas quantitativas de um \textit{survey} anônimo (precisão). \\ 3. Dados qualitativos de uma pesquisa documental.\end{tabular} \\\hline
\textbf{Método de Coleta de Dados} & \begin{tabular}[c]{@{}m{11.75cm}@{}}Triangulação de Dados, com as seguintes fontes de evidência: \\ 1. Métodos de similaridade léxica. \\ 2. Respostas do \textit{survey}. \\ 3. Pesquisa documental.\end{tabular} \\\hline
\textbf{Tipo de Análise} & Mista (Quantitativa e Qualitativa) \\\hline
\end{tabular}
\end{quadro}

\subsection{Resultados e Discussões}

\subsubsection{Métodos de Similaridade Léxica}

Com base nas evidências quantitativas, o conjunto de dados projetado para avaliar a qualidade das transcrições automáticas dos principais provedores de ASR (Amazon, Google, IBM, Microsoft e OpenAI) foi analisado por meio de três métricas de similaridade léxica (CS, JI e LD). Dadas as características dos dados coletados, o foco recaiu sobre o JI, única métrica com distribuição normal, demonstrando assim maior robustez estatística. O método de Jaccard, ou Índice de Jaccard, mede a similaridade entre dois conjuntos de dados calculando o tamanho da interseção dividido pelo tamanho da união dos conjuntos de amostras. Logo, ele compara as palavras da transcrição automática com as da transcrição de referência (revisada por especialistas), proporcionando uma medida quantitativa de precisão.

%Sendo assim, vamos apresentar os resultados do JI como métrica principal de similaridade léxica para avaliar os provedores de ASR/STT. 
Os achados iniciais desta fonte de evidência podem ser visualizados por meio do conjunto de gráficos da \autoref{fig:chapter4-cs1-lexical-all}. A figura inclui um Gráfico de Barras (\autoref{fig:chapter4-cs1-lexical-all-barplot}) que exibe o JI médio para cada provedor, um \textit{Boxplot} (\autoref{fig:chapter4-cs1-lexical-all-boxplot}) que detalha a distribuição dos dados e um Gráfico KDE (\autoref{fig:chapter4-cs1-lexical-all-kdeplot}) que ilustra a densidade de probabilidade das pontuações. Essas visualizações destacaram proeminentemente a OpenAI, que demonstrou a maior pontuação, sugerindo seu desempenho superior na captura da similaridade léxica em várias aplicações de transcrição.

\begin{figure}[htbp]
\centering
\caption{Índice de Jaccard: Gráficos dos Provedores Sem Agrupar Por Idioma.}
\label{fig:chapter4-cs1-lexical-all}
\begin{subfigure}[b]{0.8\textwidth}
\centering
\includegraphics[width=\textwidth]{images/chapter4-cs1-lexical-all-barplot.png}
\caption{Gráfico de Barras.}
\label{fig:chapter4-cs1-lexical-all-barplot}
\end{subfigure} ~
\begin{subfigure}[b]{0.8\textwidth}
\centering
\includegraphics[width=\textwidth]{images/chapter4-cs1-lexical-all-boxplot.png}
\caption{Gráfico de Boxplot.}
\label{fig:chapter4-cs1-lexical-all-boxplot}
\end{subfigure} ~
\begin{subfigure}[b]{0.8\textwidth}
\centering
\includegraphics[width=\textwidth]{images/chapter4-cs1-lexical-all-kdeplot.png}
\caption{Gráfico KDE.}
\label{fig:chapter4-cs1-lexical-all-kdeplot}
\end{subfigure}
\fdireta{FalvoJr2023_HICSS}
\end{figure}

As análises estatísticas detalhadas na \autoref{tabble:c4:results-ji-tests} corroboram as impressões iniciais dos gráficos. Os testes de normalidade, o \textit{Kolmogorov-Smirnov} \cite{Kolmogorov1933,Smirnov1948} e o \textit{Shapiro-Wilk} \cite{Shapiro1965}, confirmaram a distribuição normal dos dados, permitindo uma análise adicional por meio da \textit{One-Way ANOVA} \cite{Fisher1925}. 

Por sua vez, o teste de ANOVA revelou diferenças significativas entre os provedores. Portanto, as análises \textit{post-hoc} de \textit{Tukey HSD} \cite{Tukey1949} e de \textit{Bonferroni} \cite{Bonferroni1936} foram aplicadas para identificar essas diferenças explicitamente. 

\begin{table}[htb]
\small
\centering
\caption{Testes Estatísticos do Índice de Jaccard: Provedores Sem Agrupar Por Idioma}
\label{tabble:c4:results-ji-tests} 
\begin{tabular}{lcclc|}
\hline
\multicolumn{5}{|c|}{\textbf{Tests of Normality}} \\ \hline
\multicolumn{1}{|c|}{\multirow{2}{*}{\textbf{Test}}} & \multicolumn{2}{c|}{\textbf{Kolmogorov-Smirnov}} & \multicolumn{2}{c|}{\textbf{Shapiro-Wilk}} \\ \cline{2-5} 
\multicolumn{1}{|c|}{} & \multicolumn{1}{c|}{\textbf{Statistic}} & \multicolumn{1}{c|}{\textbf{Significance \ensuremath{(p)}}} & \multicolumn{1}{c|}{\textbf{Statistic}} & \textbf{\ensuremath{p}} \\ \hline
\multicolumn{1}{|c|}{Jaccard Index} & \multicolumn{1}{c|}{0.057} & \multicolumn{1}{c|}{\textbf{0.200}} & \multicolumn{1}{c|}{0.973} & \textbf{0.111} \\ \hline
\multicolumn{5}{|c|}{\textbf{Interpretation}} \\ \hline
\multicolumn{5}{|l|}{\begin{tabular}[c]{@{}l@{}}Kolmogorov-Smirnov: If \ensuremath{p>0.05}, sample follow the same statistical distribution.\end{tabular}} \\ \hline
\multicolumn{5}{|l|}{Shapiro-wilk: If \ensuremath{p>0.05} is normal.} \\ \hline
\multicolumn{5}{|c|}{\textbf{One-Way ANOVA (One-Way Analysis of Variance)}} \\ \hline
\multicolumn{1}{|c|}{\textbf{Test}} & \multicolumn{2}{c|}{\textbf{F}} & \multicolumn{2}{c|}{\textbf{\ensuremath{p}}} \\ \hline
\multicolumn{1}{|c|}{Jaccard Index} & \multicolumn{2}{c|}{4.562} & \multicolumn{2}{c|}{\textbf{0.002}} \\ \hline
\multicolumn{5}{|c|}{\textbf{Interpretation}} \\ \hline
\multicolumn{5}{|l|}{\begin{tabular}[c]{@{}l@{}}If \ensuremath{p<0.05}, there are significant differences between at least two groups.\end{tabular}} \\ \hline
\multicolumn{5}{|c|}{\textbf{Tukey HSD (Honest Significant Difference)}} \\ \hline
\multicolumn{3}{|l|}{\textbf{Providers (Groups from 1 to 5)}} & \multicolumn{2}{c|}{\textbf{\ensuremath{p}}} \\ \hline
\multicolumn{3}{|l|}{OpenAI (Group 5) to Google (Group 2)} & \multicolumn{2}{c|}{\textbf{0.005}} \\ \hline
\multicolumn{3}{|l|}{OpenAI (Group 5) to IBM (Group 3)} & \multicolumn{2}{c|}{\textbf{0.032}} \\ \hline
\multicolumn{3}{|l|}{Microsoft (Group 4) to Google (Group 2)} & \multicolumn{2}{c|}{\textbf{0.037}} \\ \hline
\multicolumn{5}{|c|}{\textbf{Interpretation}} \\ \hline
\multicolumn{5}{|l|}{\begin{tabular}[c]{@{}l@{}}If \ensuremath{p<>0} and \ensuremath{p<0.05}, there is a significant difference among the groups.\end{tabular}} \\ \hline
\multicolumn{5}{|c|}{\textbf{Bonferroni}} \\ \hline
\multicolumn{3}{|l|}{\textbf{Providers (Groups from 1 to 5)}} & \multicolumn{2}{c|}{\textbf{\ensuremath{p}}} \\ \hline
\multicolumn{3}{|l|}{OpenAI (Group 5) to Google (Group 2)} & \multicolumn{2}{c|}{\textbf{0.006}} \\ \hline
\multicolumn{3}{|l|}{OpenAI (Group 5) to IBM (Group 3)} & \multicolumn{2}{c|}{\textbf{0.041}} \\ \hline
\multicolumn{3}{|l|}{Microsoft (Group 4) to Google (Group 2)} & \multicolumn{2}{c|}{\textbf{0.047}} \\ \hline
\multicolumn{5}{|c|}{\textbf{Interpretation}} \\ \hline
\multicolumn{5}{|l|}{\begin{tabular}[c]{@{}l@{}}If the adjusted \ensuremath{p~(\alpha'=\alpha/n)}, where n is the number of comparisons, is \ensuremath{<0.05},\\then there is a statistical difference among the groups.\end{tabular}} \\ \hline
\end{tabular}
\fdireta{FalvoJr2023_HICSS}
\end{table}

Adicionalmente, tendo em vista os dois conjuntos de hipóteses de pesquisa, as análises foram estendidas para avaliar o desempenho dos provedores em múltiplos idiomas: Português, Inglês e Espanhol (\autoref{fig:chapter4-cs1-lexical-by-lang-barplot}). Esta segmentação linguística revelou disparidades de precisão persistentes, que são críticas para aplicações globais que dependem de transcrições automáticas confiáveis em várias línguas.

\begin{figure}[htb]
\centering
\caption{Índice de Jaccard: Gráfico dos Provedores Agrupados Por Idioma.}
\label{fig:chapter4-cs1-lexical-by-lang-barplot}
\includegraphics[width=0.9\textwidth]{images/chapter4-cs1-lexical-by-lang-barplot.png}
\fdireta{FalvoJr2023_HICSS}
\end{figure}

Como resultado, foram identificadas diferenças estatisticamente significativas entre a OpenAI quando comparada com Google e IBM, além da Microsoft quando comparada ao Google. Essas diferenças evidenciam uma variabilidade na eficiência dos provedores ao lidar com a similaridade léxica em diferentes contextos de transcrição automática, sugerindo uma superioridade da OpenAI e da Microsoft tendo em vista suas medidas de Jaccard.

As avaliações estatísticas correspondentes (\autoref{table:c4:results-ji-by-lang-tests}) incluíram testes de normalidade e \textit{One-Way ANOVA} para cada idioma. Os resultados destacaram diferenças significativas em como os provedores geram transcrições automáticas em inglês. Disparidades específicas entre provedores foram detalhadas através de análises \textit{post-hoc}, como \textit{Tukey HSD} e \textit{Bonferroni} para o idioma inglês, indicando que o desempenho pode variar significativamente dependendo do idioma das transcrições.

\begin{table}[htb]
\small
\centering
\caption{Testes Estatísticos do Índice de Jaccard: Provedores Agrupados Por Idioma}
\label{table:c4:results-ji-by-lang-tests} 
\begin{tabular}{|ccccc|}
\hline
\multicolumn{5}{|c|}{\textbf{Tests of Normality}} \\ \hline
\multicolumn{1}{|c|}{\multirow{2}{*}{\textbf{Test}}} & \multicolumn{2}{c|}{\textbf{Kolmogorov-Smirnov}} & \multicolumn{2}{c|}{\textbf{Shapiro-Wilk}} \\ \cline{2-5} 
\multicolumn{1}{|c|}{} & \multicolumn{1}{c|}{\textbf{Statistic}} & \multicolumn{1}{c|}{\textbf{Significance \ensuremath{(p)}}} & \multicolumn{1}{c|}{\textbf{Statistic}} & \textbf{\ensuremath{p}} \\ \hline
\multicolumn{1}{|c|}{EN} & \multicolumn{1}{c|}{0.121} & \multicolumn{1}{c|}{\textbf{0.200}} & \multicolumn{1}{c|}{0.961} & \textbf{0.427} \\ \hline
\multicolumn{1}{|c|}{PT} & \multicolumn{1}{c|}{0.114} & \multicolumn{1}{c|}{\textbf{0.200}} & \multicolumn{1}{c|}{0.942} & \textbf{0.169} \\ \hline
\multicolumn{1}{|c|}{ES} & \multicolumn{1}{c|}{0.951} & \multicolumn{1}{c|}{\textless 0.001} & \multicolumn{1}{c|}{0.951} & \textbf{0.264} \\ \hline
\multicolumn{5}{|c|}{\textbf{One-Way ANOVA}} \\ \hline
\multicolumn{1}{|c|}{\textbf{Test}} & \multicolumn{2}{c|}{\textbf{F}} & \multicolumn{2}{c|}{\textbf{\ensuremath{p}}} \\ \hline
\multicolumn{1}{|c|}{EN} & \multicolumn{2}{c|}{4.88} & \multicolumn{2}{c|}{\textbf{0.007}} \\ \hline
\multicolumn{1}{|c|}{PT} & \multicolumn{2}{c|}{1.058} & \multicolumn{2}{c|}{0.403} \\ \hline
\multicolumn{1}{|c|}{ES} & \multicolumn{2}{c|}{0,931} & \multicolumn{2}{c|}{0,466} \\ \hline
\multicolumn{5}{|c|}{\textbf{Tukey HSD (Honest Significant Difference) - EN}} \\ \hline
\multicolumn{3}{|l|}{\textbf{Providers (Groups from 1 to 5)}} & \multicolumn{2}{c|}{\textbf{\ensuremath{p}}} \\ \hline
\multicolumn{3}{|l|}{OpenAI (Group 5) to Google (Group 2)} & \multicolumn{2}{c|}{\textbf{0.041}} \\ \hline
\multicolumn{3}{|l|}{OpenAI (Group 5) to IBM (Group 3)} & \multicolumn{2}{c|}{\textbf{0.034}} \\ \hline
\multicolumn{5}{|c|}{\textbf{Bonferroni - EN}} \\ \hline
\multicolumn{3}{|l|}{\textbf{Providers (Groups from 1 to 5)}} & \multicolumn{2}{c|}{\textbf{\ensuremath{p}}} \\ \hline
\multicolumn{3}{|l|}{OpenAI (Group 5) to IBM (Group 3)} & \multicolumn{2}{c|}{\textbf{0.047}} \\ \hline
\end{tabular}
\fdireta{FalvoJr2023_HICSS}
\end{table}

Esses resultados destacam diferenças significativas na qualidade das transcrições automáticas dos provedores avaliados em relação à similaridade léxica. Além disso, abrem caminho para uma exploração mais aprofundada de como essas características podem impactar a experiência do usuário em várias aplicações dessas transcrições. Isso inclui a satisfação do aluno e a percepção sobre a precisão da transcrição em diferentes contextos linguísticos. Tais aspectos são discutidos a seguir, com base nos resultados do \textit{survey}.

\subsubsection{Respostas do \textit{Survey}}

Com base nos \textit{insights} do estudo de similaridade léxica, esta seção apresenta os resultados de uma pesquisa anônima baseada nas percepções dos participantes sobre a qualidade das transcrições automáticas, explorando o \textit{dataset} definido para as fontes de evidência quantitativas. 

Essa pesquisa teve como objetivo comparar os dados objetivos de similaridade léxica com as avaliações subjetivas de pessoas reais sobre a qualidade das transcrições dos diferentes provedores de serviços de ASR/STT.

Os resultados desta pesquisa foram sintetizados na \autoref{fig:chapter4-cs1-survey-all} para ilustrar as avaliações médias e a distribuição das respostas. O gráfico de barras (\autoref{fig:chapter4-cs1-survey-all-barplot}) apresenta as avaliações médias para cada provedor, com a OpenAI recebendo a maior pontuação, o que sugere uma preferência entre os participantes. O gráfico de \textit{boxplot} (\autoref{fig:chapter4-cs1-survey-all-boxplot}) detalha a dispersão e a tendência central das avaliações para cada provedor, evidenciando a variação na satisfação dos usuários, enquanto o gráfico KDE (\autoref{fig:chapter4-cs1-survey-all-kdeplot}) oferece uma estimativa visual da densidade da distribuição das avaliações.

\begin{figure}[htbp]
\centering
\caption{Respostas do \textit{Survey}: Gráficos dos Provedores Sem Agrupar Por Idioma.}
\label{fig:chapter4-cs1-survey-all}
\begin{subfigure}[b]{0.81\textwidth}
\centering
\includegraphics[width=\textwidth]{images/chapter4-cs1-survey-all-barplot.png}
\caption{Gráfico de Barras.}
\label{fig:chapter4-cs1-survey-all-barplot}
\end{subfigure} ~
\begin{subfigure}[b]{0.81\textwidth}
\centering
\includegraphics[width=\textwidth]{images/chapter4-cs1-survey-all-boxplot.png}
\caption{Gráfico de Boxplot.}
\label{fig:chapter4-cs1-survey-all-boxplot}
\end{subfigure} ~
\begin{subfigure}[b]{0.81\textwidth}
\centering
\includegraphics[width=\textwidth]{images/chapter4-cs1-survey-all-kdeplot.png}
\caption{Gráfico KDE.}
\label{fig:chapter4-cs1-survey-all-kdeplot}
\end{subfigure}
\fdireta{FalvoJr2024_FIE}
\end{figure}

Um gráfico adicional detalha as avaliações por idioma, indicando variações significativas na satisfação do usuário com base no idioma da transcrição automática (\autoref{fig:chapter4-cs1-survey-by-lang-barplot}), o que pode sugerir níveis de qualidade distintos para transcrições em Português, Inglês ou Espanhol.

\begin{figure}[htb]
\centering
\caption{Respostas do \textit{Survey}: Gráfico dos Provedores Agrupados Por Idioma.}
\label{fig:chapter4-cs1-survey-by-lang-barplot}
\includegraphics[width=1\textwidth]{images/chapter4-cs1-survey-by-lang-barplot.png}
\fdireta{FalvoJr2024_FIE}
\end{figure}

As análises estatísticas, consolidadas na \autoref{table:c4:results-survey-tests}, indicaram que as respostas do \textit{survey} não seguiram uma distribuição normal, característica confirmada pelos testes de \textit{Kolmogorov-Smirnov} \cite{Kolmogorov1933,Smirnov1948} e \textit{Shapiro-Wilk} \cite{Shapiro1965}. Por isso, fez-se necessário a avaliação e aplicação de testes não paramétricos para as análises subsequentes. O teste de \textit{Kruskal-Wallis} \cite{Kruskal1952}, uma alternativa não paramétrica ao ANOVA, foi empregado para determinar se havia diferenças significativas nas medianas das pontuações entre os diferentes grupos. Este teste mostrou diferenças significativas entre os provedores (\ensuremath{p < 0.05}), sugerindo que as percepções dos participantes variavam significativamente dependendo do provedor.

Ainda analisando a \autoref{table:c4:results-survey-tests}, dada a distribuição não normal dos dados, o teste de \textit{Dunn} \cite{Dunn1964} foi escolhido em detrimento do teste de \textit{Conover} \cite{Conover1999} para a análise \textit{post-hoc} devido às suas suposições menos rigorosas sobre a distribuição dos dados e sua capacidade de lidar efetivamente com dados que possuem \textit{outliers}, embora ambos tenham produzido resultados análogos em testes informais. O teste de \textit{Dunn} comparou pares de provedores, indicando diferenças significativas entre quase todos os pares. Notavelmente, todas as comparações envolvendo OpenAI e outros provedores demonstraram diferenças significativas, sugerindo uma virtual superioridade desse provedor.

\begin{table}[htb]
\small
\centering
\caption{Testes Estatísticos do \textit{Survey}: Provedores Sem Agrupar Por Idioma}
\label{table:c4:results-survey-tests} 
\begin{tabular}{|lcccc|}
\hline
\multicolumn{5}{|c|}{\textbf{Tests of Normality}} \\ \hline
\multicolumn{1}{|c|}{\multirow{2}{*}{\textbf{Test}}} & \multicolumn{2}{c|}{\textbf{Kolmogorov-Smirnov}} & \multicolumn{2}{c|}{\textbf{Shapiro-Wilk}} \\ \cline{2-5} 
\multicolumn{1}{|c|}{} & \multicolumn{1}{c|}{\textbf{Statistic}} & \multicolumn{1}{c|}{\textbf{Significance \ensuremath{(p)}}} & \multicolumn{1}{c|}{\textbf{Statistic}} & \textbf{\ensuremath{p}} \\ \hline
\multicolumn{1}{|c|}{Survey Responses} & \multicolumn{1}{c|}{0.887} & \multicolumn{1}{c|}{\textless 0.001} & \multicolumn{1}{c|}{0.973} & \textless 0.001 \\ \hline
\multicolumn{5}{|c|}{\textbf{Interpretation}} \\ \hline
\multicolumn{5}{|l|}{\begin{tabular}[c]{@{}l@{}}Kolmogorov-Smirnov: If \ensuremath{p>0.05}, sample follow the same  statistical distribution.\end{tabular}} \\ \hline
\multicolumn{5}{|l|}{Shapiro-wilk: If \ensuremath{p>0.05} is normal.} \\ \hline
\multicolumn{5}{|c|}{\textbf{Kruskal-Wallis (Non-Parametric equivalent to ANOVA)}} \\ \hline
\multicolumn{1}{|c|}{\textbf{Test}} & \multicolumn{2}{c|}{\textbf{Statistic}} & \multicolumn{2}{c|}{\textbf{\ensuremath{p}}} \\ \hline
\multicolumn{1}{|c|}{Survey Responses} & \multicolumn{2}{c|}{536.167} & \multicolumn{2}{c|}{\textbf{\textless 0.001}} \\ \hline
\multicolumn{5}{|c|}{\textbf{Interpretation}} \\ \hline
\multicolumn{5}{|l|}{\begin{tabular}[c]{@{}l@{}}If \ensuremath{p<0.05}, there are significant differences among the groups.\end{tabular}} \\ \hline
\multicolumn{5}{|c|}{\textbf{Dunn (Non-Parametric Post-Hoc equivalent to Tukey HSB)}} \\ \hline
\multicolumn{3}{|l|}{\textbf{Providers (Groups from 1 to 5)}} & \multicolumn{2}{c|}{\textbf{\ensuremath{p}}} \\ \hline
\multicolumn{3}{|l|}{Amazon (Group 1) to Google (Group 2)} & \multicolumn{2}{c|}{\textbf{\textless 0.001}} \\ \hline
\multicolumn{3}{|l|}{Amazon (Group 1) to IBM (Group 3)} & \multicolumn{2}{c|}{\textbf{\textless 0.001}} \\ \hline
\multicolumn{3}{|l|}{Amazon (Group 1) to Micorsoft (Group 4)} & \multicolumn{2}{c|}{\textbf{\textless 0.001}} \\ \hline
\multicolumn{3}{|l|}{Amazon (Group 1) to OpenAI (Group 5)} & \multicolumn{2}{c|}{\textbf{\textless 0.001}} \\ \hline
\multicolumn{3}{|l|}{Google (Group 2) to IBM (Group 3)} & \multicolumn{2}{c|}{\textbf{0.010}} \\ \hline
\multicolumn{3}{|l|}{Google (Group 2) to Microsoft (Group 4)} & \multicolumn{2}{c|}{\textbf{\textless 0.001}} \\ \hline
\multicolumn{3}{|l|}{Google (Group 2) to OpenAI (Group 5)} & \multicolumn{2}{c|}{\textbf{\textless 0.001}} \\ \hline
\multicolumn{3}{|l|}{IBM (Group 3) to Microsoft (Group 4)} & \multicolumn{2}{c|}{\textbf{\textless 0.001}} \\ \hline
\multicolumn{3}{|l|}{IBM (Group 3) to OpenAI (Group 5)} & \multicolumn{2}{c|}{\textbf{\textless 0.001}} \\ \hline
\multicolumn{3}{|l|}{Micorsoft (Group 4) to OpenAI (Group 5)} & \multicolumn{2}{c|}{\textbf{\textless 0.001}} \\ \hline
\multicolumn{5}{|c|}{\textbf{Interpretation}} \\ \hline
\multicolumn{5}{|l|}{\begin{tabular}[c]{@{}l@{}}If \ensuremath{p<0.05}, it indicates significant pairwise differences between groups.\end{tabular}} \\ \hline
\end{tabular}
\fdireta{FalvoJr2024_FIE}
\end{table}

Os resultados estratificados por idioma reforçaram essas descobertas, com todos os grupos de idiomas mostrando diferenças significativas entre os provedores na forma como as transcrições automáticas foram avaliadas. Isso sugere que a qualidade percebida pelos participantes não depende apenas do provedor, mas também varia com o idioma, destacando os desafios de fornecer transcrições de alta qualidade de forma uniforme em diferentes contextos linguísticos (\autoref{table:c4:results-survey-by-lang-tests}).

\begin{table}[htb]
\small
\centering
\caption{Testes Estatísticos do \textit{Survey}: Provedores Agrupados Por Idioma}
\label{table:c4:results-survey-by-lang-tests} 
\begin{tabular}{|lcccc|}
\hline
\multicolumn{5}{|c|}{\textbf{Tests of Normality}} \\ \hline
\multicolumn{1}{|c|}{\multirow{2}{*}{\textbf{Test}}} & \multicolumn{2}{c|}{\textbf{Kolmogorov-Smirnov}} & \multicolumn{2}{c|}{\textbf{Shapiro-Wilk}} \\ \cline{2-5} 
\multicolumn{1}{|c|}{} & \multicolumn{1}{c|}{\textbf{Statistic}} & \multicolumn{1}{c|}{\textbf{Significance \ensuremath{(p)}}} & \multicolumn{1}{c|}{\textbf{Statistic}} & \textbf{\ensuremath{p}} \\ \hline
\multicolumn{1}{|c|}{EN} & \multicolumn{1}{c|}{0.897} & \multicolumn{1}{c|}{\textless 0.001} & \multicolumn{1}{c|}{0.897} & \textless 0.001 \\ \hline
\multicolumn{1}{|c|}{PT} & \multicolumn{1}{c|}{0.847} & \multicolumn{1}{c|}{\textless 0.001} & \multicolumn{1}{c|}{0.912} & \textless 0.001 \\ \hline
\multicolumn{1}{|c|}{ES} & \multicolumn{1}{c|}{0.959} & \multicolumn{1}{c|}{\textless 0.001} & \multicolumn{1}{c|}{0.893} & \textless 0.001 \\ \hline
\multicolumn{5}{|c|}{\textbf{Kruskal-Wallis (Non-Parametric)}} \\ \hline
\multicolumn{1}{|c|}{\textbf{Test}} & \multicolumn{2}{c|}{\textbf{Statistic}} & \multicolumn{2}{c|}{\textbf{\ensuremath{p}}} \\ \hline
\multicolumn{1}{|c|}{EN} & \multicolumn{2}{c|}{244.355} & \multicolumn{2}{c|}{\textbf{\textless 0.001}} \\ \hline
\multicolumn{1}{|c|}{PT} & \multicolumn{2}{c|}{355.323} & \multicolumn{2}{c|}{\textbf{\textless 0.001}} \\ \hline
\multicolumn{1}{|c|}{ES} & \multicolumn{2}{c|}{37.615} & \multicolumn{2}{c|}{\textbf{\textless 0.001}} \\ \hline
\multicolumn{5}{|c|}{\textbf{Dunn (Non-Parametric) - EN}} \\ \hline
\multicolumn{3}{|l|}{\textbf{Providers (Groups from 1 to 5)}} & \multicolumn{2}{c|}{\textbf{\ensuremath{p}}} \\ \hline
\multicolumn{3}{|l|}{Amazon (Group 1) to Google (Group 2)} & \multicolumn{2}{c|}{\textbf{\textless 0.001}} \\ \hline
\multicolumn{3}{|l|}{Amazon (Group 1) to IBM (Group 3)} & \multicolumn{2}{c|}{\textbf{\textless 0.001}} \\ \hline
\multicolumn{3}{|l|}{Amazon (Group 1) to OpenAI (Group 5)} & \multicolumn{2}{c|}{\textbf{0.002}} \\ \hline
\multicolumn{3}{|l|}{Google (Group 2) to Microsoft (Group 4)} & \multicolumn{2}{c|}{\textbf{\textless 0.001}} \\ \hline
\multicolumn{3}{|l|}{Google (Group 2) to OpenAI (Group 5)} & \multicolumn{2}{c|}{\textbf{\textless 0.001}} \\ \hline
\multicolumn{3}{|l|}{IBM (Group 3) to Microsoft (Group 4)} & \multicolumn{2}{c|}{\textbf{\textless 0.001}} \\ \hline
\multicolumn{3}{|l|}{IBM (Group 3) to OpenAI (Group 5)} & \multicolumn{2}{c|}{\textbf{\textless 0.001}} \\ \hline
\multicolumn{3}{|l|}{Micorsoft (Group 4) to OpenAI (Group 5)} & \multicolumn{2}{c|}{\textbf{0.005}} \\ \hline
\multicolumn{5}{|c|}{\textbf{Dunn (Non-Parametric) - PT}} \\ \hline
\multicolumn{3}{|l|}{\textbf{Providers (Groups from 1 to 5)}} & \multicolumn{2}{c|}{\textbf{\ensuremath{p}}} \\ \hline
\multicolumn{3}{|l|}{Amazon (Group 1) to Microsoft (Group 4)} & \multicolumn{2}{c|}{\textbf{\textless 0.001}} \\ \hline
\multicolumn{3}{|l|}{Amazon (Group 1) to OpenAI (Group 5)} & \multicolumn{2}{c|}{\textbf{\textless 0.001}} \\ \hline
\multicolumn{3}{|l|}{Google (Group 2) to IBM (Group 3)} & \multicolumn{2}{c|}{\textbf{0.017}} \\ \hline
\multicolumn{3}{|l|}{Google (Group 2) to Microsoft (Group 4)} & \multicolumn{2}{c|}{\textbf{\textless 0.001}} \\ \hline
\multicolumn{3}{|l|}{Google (Group 2) to OpenAI (Group 5)} & \multicolumn{2}{c|}{\textbf{\textless 0.001}} \\ \hline
\multicolumn{3}{|l|}{IBM (Group 3) to Microsoft (Group 4)} & \multicolumn{2}{c|}{\textbf{\textless 0.001}} \\ \hline
\multicolumn{3}{|l|}{IBM (Group 3) to OpenAI (Group 5)} & \multicolumn{2}{c|}{\textbf{\textless 0.001}} \\ \hline
\multicolumn{3}{|l|}{Micorsoft (Group 4) to OpenAI (Group 5)} & \multicolumn{2}{c|}{\textbf{\textless 0.001}} \\ \hline
\multicolumn{5}{|c|}{\textbf{Dunn (Non-Parametric) - ES}} \\ \hline
\multicolumn{3}{|l|}{\textbf{Providers (Groups from 1 to 5)}} & \multicolumn{2}{c|}{\textbf{\ensuremath{p}}} \\ \hline
\multicolumn{3}{|l|}{Amazon (Group 1) to OpenAI (Group 5)} & \multicolumn{2}{c|}{\textbf{\textless 0.001}} \\ \hline
\multicolumn{3}{|l|}{Google (Group 2) to Microsoft (Group 4)} & \multicolumn{2}{c|}{\textbf{0.002}} \\ \hline
\multicolumn{3}{|l|}{Google (Group 2) to OpenAI (Group 5)} & \multicolumn{2}{c|}{\textbf{\textless 0.001}} \\ \hline
\multicolumn{3}{|l|}{IBM (Group 3) to OpenAI (Group 5)} & \multicolumn{2}{c|}{\textbf{\textless 0.001}} \\ \hline
\end{tabular}
\fdireta{FalvoJr2024_FIE}
\end{table}

A análise da pesquisa efetivamente conecta as medidas objetivas de similaridade léxica e as percepções subjetivas da qualidade das transcrições. A correlação entre as pontuações de similaridade léxica e as excelentes avaliações dos participantes, particularmente para a OpenAI, ressalta a relevância prática da precisão léxica na satisfação dos usuários finais. 

Esses \textit{insights} são fundamentais para provedores que buscam otimizar seus serviços de transcrição para conteúdos educacionais, pois ilustram a importância tanto da precisão linguística quanto da percepção dos usuários finais na avaliação da qualidade das transcrições automáticas. As descobertas sugerem um roteiro para futuras melhorias e a potencial customização de serviços para atender de forma mais eficaz às diversas necessidades linguísticas.

\subsubsection{Pesquisa Documental}

Nesta fase, ampliou-se a visão sobre as tecnologias de ASR/STT, examinando a literatura para complementar as descobertas quantitativas com percepções qualitativas adicionais. Esta análise documental aprofunda-se em diversos estudos que contribuíram significativamente para o campo de ASR e STT, proporcionando uma compreensão mais ampla de como essas tecnologias, impulsionadas por avanços em \textit{Machine Learning} (ML) e IA, podem melhorar a acessibilidade de OAs.

O estudo de \citeonline{Ferraro2023} apresenta uma investigação extensa sobre a transcrição de língua falada usando modelos de ML. Sua análise compara serviços de STT de código aberto e pagos, com foco na qualidade das transcrições automáticas e na diversidade dos dados de entrada. A pesquisa utiliza diversos conjuntos de dados de entrevistas, palestras e discursos, empregando métricas como a Taxa de Erro de Palavras (WER) para avaliação. O trabalho de \citeonline{Ferraro2023} fornece um ponto de referência, alternativo aos métodos de similaridade léxica, para avaliar a precisão da transcrição, estabelecendo uma base sólida para trabalhos futuros.

\citeonline{Bengesi2024} oferecem uma revisão abrangente dos avanços recentes em IAGen, destacando suas aplicações potenciais em processos automáticos de transcrição. Embora não se concentrem diretamente na conversão de fala em texto, sua exploração de modelos de ponta, incluindo Redes Adversárias Generativas (GANs), Transformadores Pré-treinados Generativos (GPT), \textit{autoencoders} e modelos de difusão, fornece uma base para entender como a IAGen pode aprimorar a precisão da transcrição.

\citeonline{Homburg2019} exploram o uso de robôs humanoides como avatares para a tradução de língua de sinais, buscando aprimorar a inclusão da comunidade surda. Diferentemente de pesquisas anteriores, que se concentravam apenas no reconhecimento da língua de sinais, este estudo adota uma abordagem inovadora ao utilizar robôs como intermediários na comunicação. Entrevistando 50 participantes surdos, eles avaliam a eficácia percebida dos robôs humanoides na tradução da língua de sinais, oferecendo \textit{insights} inovadores para soluções de acessibilidade.

\citeonline{Alshaikh2024} exploram a integração da IAGen na educação por meio do desenvolvimento e avaliação de um Assistente de Vídeo Educacional com IA. Fundamentado na Teoria Cognitiva da Aprendizagem Multimídia (CTML), seu ferramental, equipado com módulos de Transcrição, Engajamento e Reforço, utiliza tecnologias ASR para aprimorar a experiência de aprendizagem. Ao focar em experiências de aprendizagem multimodal, seu estudo demonstra o potencial das técnicas avançadas de IA, incluindo STT, para melhorar os resultados educacionais.

\citeonline{Cao2023} abordam as limitações das ferramentas tradicionais de transcrição de áudio em salas de aula ruidosas do mundo real. Sua pesquisa enfatiza o papel crucial de sistemas de aprendizagem inteligentes eficazes em ambientes colaborativos, particularmente na análise e compreensão de conversas entre alunos. Ao explorar a influência dos erros de ASR em modelos de conversação, destacam os desafios e oportunidades para melhorar a precisão do ASR em contextos educacionais.

Com base nas contribuições desses estudos, foram aplicados os princípios da TFD \cite{Charmaz2009} para categorizar e analisar os dados qualitativos coletados na pesquisa. A Teoria Fundamentada fornece uma abordagem sistemática para explorar temas emergentes e enriquecer a compreensão dos fatores que influenciam a qualidade das transcrições automáticas.

Ao integrar os resultados coletados nas fontes de evidência, identificaram-se padrões \textit{insights} sobre o uso de tecnologias ASR e STT na melhoria da acessibilidade de OAs. Para facilitar essa análise, os achados de cada estudo foram categorizados, dentro da TFD, em temas distintos, permitindo uma compreensão ampla das implicações tecnológicas e educacionais das ferramentas ASR e STT. Essas categorias incluem:

\begin{itemize}
\item \textbf{Avaliação da Qualidade de Transcrição (AQT)}: Avaliação de desempenho e qualidade em soluções baseadas nas tecnologias de ASR e STT para tarefas de transcrição de fala. Esta categoria foca na precisão das transcrições, utilizando métricas como a Taxa de Erro de Palavras (WER) e comparações entre diferentes serviços de STT, sejam eles de código aberto ou pagos.

\item \textbf{IAGen na Educação (IAGenE)}: Investigação da integração da IAGen em contextos educacionais para aprimorar experiências de aprendizagem. Aqui, a atenção está na aplicação de técnicas de IA, como GANs, GPTs e \textit{autoencoders}, para melhorar processos educacionais, incluindo a transcrição automática de aulas e a personalização da aprendizagem.

\item \textbf{Tecnologia Assistiva (TA)}: Investigação de abordagens inovadoras, como robôs humanoides e ferramentas de aprendizagem multimodal, para melhorar a acessibilidade para diversos alunos. Esta categoria abrange a área do conhecimento de TA, com ênfase no suporte e inclusão de alunos com necessidades especiais, por meio da tradução/sinalização de línguas de sinais, por exemplo.

\item \textbf{Desafios e Oportunidades (DO)}: Identificação de obstáculos e possibilidades para aprimorar a precisão e eficácia do ASR em contextos educacionais. Este tema envolve os desafios na implementação de tecnologias ASR em ambientes ruidosos e colaborativos, buscando soluções para um reconhecimento de fala assertivo mesmo em situações adversas.
\end{itemize}

No \autoref{quadro:c4:grounded-theory-results}, a categorização de cada estudo dentro da Teoria Fundamentada é apresentada, fornecendo uma visão detalhada dos temas abordados e sua relevância para os objetivos desta pesquisa. Esta análise permite estabelecer conexões críticas entre as diferentes vertentes de investigação, revelando como cada estudo contribui para a compreensão das tecnologias ASR e STT em contextos educacionais.

\begin{quadro}
\centering
\caption{Categorização dos Artigos usando Teoria Fundamentada}
\label{quadro:c4:grounded-theory-results} 
\begin{tabular}{c|p{6.8cm}|p{5.7cm}}\hline
\textbf{Categoria} & \textbf{Descrição} & \textbf{Estudos} \\ \hline
\textbf{AQT} & Avalia o desempenho e qualidade do ASR/STT em tarefas de transcrição. & \citeonline{Ferraro2023} \\ \hline
\textbf{IAGenE} & Explora a integração da IAGen em contextos educacionais. & \citeonline{Bengesi2024,Alshaikh2024} \\ \hline
\textbf{TA} & Investiga soluções de TA para maior inclusão e acessibilidade na educação. & \citeonline{Homburg2019,Alshaikh2024} \\ \hline
\textbf{DO} & Identifica desafios e oportunidades no ASR/STT em contextos educacionais. & \citeonline{Cao2023} \\  \hline
\end{tabular}
\end{quadro}

A seguir, será discutida a convergência dos dados provenientes das três fontes de evidência investigadas: métodos de similaridade léxica, respostas do \textit{survey} e pesquisa documental. Ao integrar essas fontes, busca-se uma visão mais robusta sobre o impacto do ASR na acessibilidade dos OAs. Essa abordagem integrada permitirá identificar sinergias e discrepâncias, fornecendo um panorama detalhado e multifacetado dos estados da prática e da arte.

\subsubsection{Convergência de Dados}

Esta pesquisa visa desvendar as complexidades das tecnologias de ASR/STT no aprimoramento da acessibilidade dos OAs por meio da triangulação de dados de três fontes distintas. Cada fonte contribui de forma única para uma compreensão abrangente de como a transcrição automática impacta a interação dos alunos com os OAs.

Primeiramente, o uso de métodos de similaridade léxica destacou variações significativas na precisão das transcrições entre os provedores de ASR avaliados (Amazon, Google, IBM, Microsoft, OpenAI). Esses dados elucidaram não apenas as capacidades técnicas desses serviços, mas também ressaltaram as nuances linguísticas que afetam seu desempenho em diferentes idiomas \cite{FalvoJr2023_HICSS}.

Adicionalmente, o \textit{survey} estende essa análise quantitativa incorporando as percepções dos participantes sobre a qualidade das transcrições. Nesse sentido, as respostas do \textit{survey} se alinham em alguns aspectos aos resultados da similaridade léxica, reforçando a relevância de uma boa métrica de precisão na satisfação do usuário. No entanto, também evidenciam aspectos centrados no usuário dos serviços de ASR (como a facilidade de compreensão e a integração com ambientes de aprendizagem) que não são capturados por algoritmos de similaridade léxica.

Esses dados quantitativos poderosos, baseados na métrica do JI e nas respostas em escala Likert do \textit{survey}, servem como um aspecto fundamental da triangulação de dados, proporcionando uma linha de base para avaliar a precisão e, consequentemente, a qualidade das transcrições automáticas.

Por sua vez, a pesquisa documental enriquece a percepção ao introduzir elementos qualitativos da literatura existente. Esta análise não apenas contextualiza os achados das outras fontes de evidência, mas também explora temas mais amplos, como o impacto educacional das tecnologias de ASR/STT e seu alinhamento com os princípios de acessibilidade. Essa revisão de literatura complementar identificou lacunas e oportunidades interessantes no uso atual do STT, sugerindo áreas para trabalhos futuros tanto no campo tecnológico quanto no pedagógico.

A integração desses três conjuntos de dados (métodos de similaridade léxica, respostas do \textit{survey} e pesquisa documental) proporciona uma visão holística sobre o impacto das TICs, como o ASR/STT, na educação. Essa triangulação de dados não apenas confirma a importância da precisão das transcrições, mas também revela a necessidade de uma abordagem centrada no usuário e consciente do contexto para maximizar a eficácia dessas tecnologias. Segue uma síntese da convergência dos dados triangulados:

\begin{itemize}

\item \textbf{Similaridade Léxica vs. Percepções dos Usuários}: Os dados de similaridade léxica revelaram que diferentes provedores de ASR apresentam diferenças estatisticamente significativas na precisão das transcrições. Notavelmente, a OpenAI demonstrou um desempenho superior (geral e por idioma) quando comparada a outros provedores de peso como Google e IBM. No entanto, os resultados do \textit{survey}, que também focaram em avaliações quantitativas, mas usando escala Likert, indicam que a satisfação do usuário é influenciada por mais do que apenas a precisão léxica. Embora a OpenAI também tenha alcançado a maior satisfação entre os participantes, as diferenças estatísticas nas respostas do \textit{survey} sugerem que as percepções dos usuários são influenciadas por fatores como compreensão do contexto e tolerância a erros, que não são totalmente capturados pelas métricas de similaridade léxica.

\item \textbf{\textit{Insights} Qualitativos da Pesquisa Documental}: A integração da TFD \cite{Charmaz2009} na análise documental permitiu uma compreensão mais profunda dos temas que afetam a qualidade das transcrições automáticas. Estudos relevantes, como os de \citeonline{Ferraro2023, Alshaikh2024}, destacam a importância de considerar a diversidade de dados e os ambientes de aprendizagem ao avaliar tecnologias de ASR/STT. Esses achados sugerem que a eficácia educacional das aplicações de reconhecimento de fala vai além da mera precisão da transcrição, abrangendo também a facilidade de integração, as capacidades de personalização e a adaptabilidade da tecnologia às necessidades educacionais diversificadas.

\item \textbf{Contexto Educacional e Acessibilidade/Inclusão}: Os dados triangulados destacam a necessidade de que as tecnologias de ASR/STT se alinhem aos princípios de acessibilidade e inclusividade. As discrepâncias entre a precisão mecânica das transcrições e a satisfação qualitativa dos usuários apontam para uma lacuna nas capacidades atuais do reconhecimento de fala. Essa lacuna indica a necessidade dos provedores inovarem além das métricas tradicionais de precisão e incorporarem \textit{feedbacks} dos usuários no desenvolvimento de soluções de transcrição mais conscientes do contexto e inclusivas.

\end{itemize}

As convergências identificadas revelam que, embora a precisão léxica das transcrições seja relevante, a satisfação do usuário e a eficácia educacional dependem também de outros fatores, como a facilidade de uso e a capacidade de integração/adaptação a contextos específicos. Este panorama multifacetado sublinha a importância de uma abordagem holística no desenvolvimento e na implementação de tecnologias ASR/STT, conforme as implicações para pesquisas futuras a seguir:

\begin{itemize}
\item \textbf{Aprimoramento do Treinamento de Modelos}: Os achados indicam o potencial de melhorar as tecnologias de ASR/STT treinando modelos em conjuntos de dados linguísticos mais diversos, o que poderia ajudar na compreensão do contexto e na redução de erros nas transcrições automáticas que afetam a satisfação dos usuários finais.

\item \textbf{Customização para Uso Educacional}: Os provedores devem considerar opções de customização que permitam às instituições educacionais adaptar as funcionalidades de ASR/STT às suas necessidades específicas. Alguns exemplos nesse sentido seriam ajustes para diferentes sotaques, dialetos e vocabulário técnico específico de cursos ou disciplinas.

\item \textbf{Abordagens de \textit{Design} Centrado no Usuário}: Incorporar \textit{feedbacks} dos usuários no processo de desenvolvimento pode garantir que as futuras melhorias nas tecnologias de ASR/STT estejam mais alinhadas com as necessidades e expectativas dos usuários finais, particularmente em ambientes educacionais diversificados.

\end{itemize}

A convergência dos dados da análise léxica, dos \textit{surveys} dos usuários e da pesquisa acadêmica ressalta um quadro complexo do estado atual e do potencial das tecnologias de ASR/STT na educação. Embora avanços notáveis em precisão de transcrição tenham sido alcançados, ainda há um trabalho significativo a ser feito para realizar plenamente o potencial dessas tecnologias em melhorar a acessibilidade e inclusão educacional.

Sendo assim, pesquisas futuras podem se concentrar em fechar a lacuna entre a proficiência técnica e a satisfação do usuário, enfatizando o desenvolvimento de sistemas adaptáveis, amigáveis ao usuário e conscientes do contexto que possam suportar uma ampla gama de ambientes e necessidades de aprendizagem.

Seguindo esta linha de investigação, o próximo estudo de caso explorará a POC implementada no primeiro estudo de caso para acelerar o desenvolvimento de um \textit{player} de vídeo aderente ao conceito de \textit{Design} Universal. Esta POC se aproveita das transcrições e legendas para integrar o \textit{player} com avatares de Libras baseados em texto, ampliando ainda mais a acessibilidade e a inclusão em ambientes educacionais.

\section{Estudo de Caso 2: \textit{Player} com Avatar de Libras}
\label{c4:cs2}

O segundo estudo concentrou-se no desenvolvimento de um \textit{player} de vídeo, projetado com base no conceito de \textit{Design} Universal para criar um produto final mais flexível e acessível do ponto de vista educacional \cite{UNESCO2023, GovBr2023}. Este estudo também contou com o apoio da DIO, utilizando a API REST desenvolvida como prova de conceito no primeiro estudo de caso, facilitando o acesso às videoaulas da \textit{EdTech}. Na prática, essa integração permite que o \textit{player} utilize transcrições/legendas como elementos centrais para tornar seus OAs mais acessíveis.

A relevância deste estudo reside em sua capacidade de explorar o impacto de avatares de Libras baseados em texto na acessibilidade educacional para a comunidade surda. Para isso, foram conduzidas avaliações quantitativas por meio de um \textit{survey} com intérpretes de Libras, seguido por entrevistas qualitativas, investigando como essa solução tecnológica pode melhorar a acessibilidade de OAs audíveis.

\subsection{Relação com a Arquitetura \textit{Speech2Learning}}

No segundo estudo de caso, a Arquitetura \textit{Speech2Learning} desempenhou um papel central ao fornecer a base estrutural para a integração entre o \textit{player} de vídeo desenvolvido e a API REST criada no primeiro estudo de caso. Essa API, responsável por enriquecer OAs audíveis com transcrições e legendas, foi utilizada como um serviço de TA, permitindo que o \textit{player} acessasse e utilizasse esses OAs diretamente, funcionando de maneira semelhante a um repositório de OAs.

A principal contribuição da \textit{Speech2Learning} neste estudo de caso é a sua capacidade de oferecer um ecossistema flexível e adaptável para o desenvolvimento de soluções mais acessíveis. A API REST, implementada como um serviço desacoplado, facilitou a integração do \textit{player} de vídeo com avatares de Libras, promovendo uma nova camada de acessibilidade para usuários surdos. Isso foi possível graças à aderência da API aos padrões de metadados, que garantem que as transcrições e legendas estejam prontamente disponíveis e padronizadas para uso em diferentes contextos educacionais.

Além disso, a arquitetura permitiu que o \textit{player} utilizasse os metadados fornecidos pela API para sinalizar os textos em Libras, aproveitando a flexibilidade da \textit{Speech2Learning} para criar um ambiente de aprendizagem mais inclusivo. O \textit{design} modular e a separação de responsabilidades entre as camadas da API e do \textit{player} de vídeo asseguraram que a integração fosse feita de maneira eficiente, sem comprometer a qualidade ou a integridade dos OAs.

De modo geral, a relação entre a \textit{Speech2Learning} e este estudo de caso exemplifica como uma arquitetura bem planejada pode servir como a espinha dorsal para soluções educacionais mais acessíveis, possibilitando a criação de ferramentas que não apenas atendem às necessidades de acessibilidade, mas também exploram novas formas de interação e aprendizado.

\subsection{Prova de Conceito: \textit{Player} de Vídeo com \textit{Design} Universal}

A \autoref{fig:chapter4-cs2-poc-diagram} ilustra, por meio de um diagrama de sequência, a dinâmica do \textit{player} de vídeo desenvolvido, que utiliza as capacidades da arquitetura \textit{Speech2Learning} para ampliar a acessibilidade de videoaulas. Esta representação demonstra como o \textit{player} de vídeo promove novas formas de explorar OAs audíveis, aproveitando os metadados fornecidos pela API desenvolvida anteriormente.

\begin{figure}[htbp]
\centering
\caption{Visão 2ª Instância da \textit{Speech2Learning}: \textit{Player} de Vídeo com Avatar de Libras}
\label{fig:chapter4-cs2-poc-diagram}
\includegraphics[width=1\textwidth]{images/chapter4-cs2-poc-diagram.png}
\fautor
\end{figure}

A integração entre o \textit{player} de vídeo e a API REST ilustra a capacidade da arquitetura \textit{Speech2Learning} de fomentar ambientes educacionais mais inclusivos. Adotando esta abordagem, os desenvolvedores podem criar sistemas robustos, tecnologicamente independentes e flexíveis, capazes de atender às especificidades de cada projeto.

O objetivo central deste estudo de caso é projetar, construir e avaliar um \textit{player} de vídeo que siga as recomendações do ``Guia de Boas Práticas para Acessibilidade Digital'' definidas por \citeonline{GovBr2023}, além dos princípios de \textit{Design} Universal promovidos pela \citeonline{UNESCO2023}. Este \textit{player}, integrado com a arquitetura \textit{Speech2Learning}, visa promover a educação inclusiva para surdos através da Libras.

Tecnicamente, o \textit{player} foi implementado utilizando HTML, CSS e JavaScript de forma ``pura'' (\textit{vanilla}), ou seja, evitando bibliotecas e/ou implementações alternativas para uma solução mais padronizada e manutenível baseada na Web \cite{GovBr2023}. Os \textit{wireframes} preliminares na Figura \ref{fig:chapter4-cs2-poc-wireframes} oferecem uma visão simplificada, destacando características universais e o símbolo ``Acessível em Libras'', sublinhando o compromisso com a educação inclusiva.

\begin{figure}[htbp]
\centering
\caption{\textit{Wireframes} do \textit{Player} de Vídeo Acessível (Modos Claro e Escuro)}
\label{fig:chapter4-cs2-poc-wireframes}
\includegraphics[width=1\textwidth]{images/chapter4-cs2-poc-wireframes.png}
\fautor
\end{figure}

Em particular, esta POC foi conduzida no contexto de dois projetos de \sigla{IC}{Iniciação Científica} vinculados a este trabalho de doutorado, cada um explorando uma faceta essencial da solução proposta:

\begin{itemize}
\item \textbf{``\textit{Design} Acessível de um \textit{Player} de Vídeo com Transcrições em Libras: Uma Aplicação da Arquitetura \textit{Speech2Learning}''}: Este projeto, sob a responsabilidade da aluna Melissa Motoki Nogueira, com financiamento Programa Unificado de Bolsas (PUB) da USP, se dedicou a investigar e aplicar os princípios do \textit{Design} Universal no desenvolvimento do \textit{player}, garantindo que a ferramenta seja acessível e inclusiva para o maior número possível de usuários.
\item \textbf{``Estudo e Desenvolvimento de um \textit{Player} de Vídeo Acessível Para Libras Explorando a Arquitetura \textit{Speech2Learning}''}: O aluno Adriano da Silva de Carvalho, com financiamento do PUB da USP, conduziu este projeto, focado na implementação do \textit{player} de vídeo como uma biblioteca aberta e flexível. Essa abordagem visa facilitar a integração da ferramenta em diferentes plataformas e contextos educacionais, ampliando seu alcance e impacto.
\end{itemize}

\subsection{Metodologia}

Para investigar o impacto e a eficácia do \textit{player} de vídeo com avatares de Libras baseados em texto, adotou-se uma abordagem exploratória, com o objetivo de gerar conhecimento sobre um fenômeno ainda pouco conhecido \cite{CastroFilho2021}. Este estudo de caso visa atingir dois principais objetivos: (i) investigar o impacto de avatares de Libras na acessibilidade de conteúdos para surdos; e (ii) explorar a eficácia desses avatares, integrados à transcrição automática de videoaulas, para a compreensão de OAs audíveis. Para alcançar esses objetivos, formulou-se a seguinte questão de pesquisa, que orienta a investigação:

\begin{itemize}
\item \textbf{Questão de Pesquisa}: Como as tecnologias de ASR/STT podem ser adaptadas para atender às necessidades de acessibilidade de usuários da Libras?
\end{itemize}

Adotou-se uma metodologia de análise mista, integrando dados quantitativos e qualitativos. Inicialmente, disponibilizou-se um \textit{survey} para todos os intérpretes da empresa QS Inclusão (\url{https://qsinclusao.com.br}), totalizando 7 participantes convidados. No \textit{survey}, os intérpretes avaliaram quantitativamente as performances dos avatares \textit{VLibras} e \textit{Hand Talk}, baseados em transcrições automáticas de videoaulas. Para direcionar a análise quantitativa, foram formuladas duas hipóteses de pesquisa:

\begin{itemize}
\item \textbf{H1}: Os avatares de Libras são úteis na compreensão de OAs audíveis transcritos automaticamente.
\item \textbf{H2}: Existe diferença na qualidade de sinalização entre os avatares \textit{VLibras} e \textit{Hand Talk} ao utilizar transcrições automáticas.
\end{itemize}

Além do \textit{survey}, cada intérprete teve a oportunidade de agendar uma entrevista, complementando o \textit{survey} com dados qualitativos. Essas entrevistas foram planejadas para explorar de modo mais aprofundado as percepções e experiências dos intérpretes com o uso do \textit{player} de vídeo e os avatares de Libras. A coleta de dados seguiu um protocolo semi-estruturado, garantindo que todas as áreas relevantes fossem abordadas.

A análise dos dados qualitativos foi realizada utilizando a TFD \cite{Charmaz2009}, uma metodologia que permite o desenvolvimento de teorias emergentes diretamente dos dados coletados. Este processo envolveu a codificação aberta, identificando e categorizando os principais temas nas respostas dos intérpretes de Libras. Essa abordagem assegurou que os temas emergissem naturalmente dos dados, proporcionando uma compreensão aprofundada das percepções e experiências dos participantes, sem preconceitos prévios.

Com o objetivo de ampliar a acessibilidade de conteúdos educacionais para alunos surdos, este estudo investigou a eficácia de um \textit{player} de vídeo acessível, integrado a avatares de Libras baseados em texto, como uma segunda instância da arquitetura \textit{Speech2Learning}. Especificamente, buscou-se:

\begin{itemize}
\item \textbf{Avaliar o impacto dos avatares de Libras na acessibilidade de OAs audíveis}, verificando se a combinação de transcrição automática e representação visual em Libras promove uma experiência de aprendizagem adequada.
\item \textbf{Analisar a viabilidade e a precisão dos avatares de Libras}, considerando-os como um complemento às estratégias de acessibilidade já existentes, alinhadas aos princípios do \textit{Design} Universal.
\item \textbf{Compreender as percepções dos intérpretes de Libras} sobre a usabilidade, a precisão e o impacto dos avatares integrados às transcrições automáticas, fornecendo \textit{insights} valiosos para o aprimoramento da solução.
\end{itemize}

O \autoref{quadro:c4:cs2-summary} resume a abordagem metodológica adotada, que inclui uma análise mista com dados qualitativos e quantitativos. Devido ao número reduzido de participantes, o foco dos resultados e discussões recai sobre os dados qualitativos das entrevistas com os intérpretes de Libras, complementados por \textit{feedbacks} formais obtidos em conferências científicas. Na próxima seção, serão detalhadas as percepções dos intérpretes sobre o uso de avatares de Libras, fornecendo uma visão aprofundada dos desafios e benefícios identificados, essenciais para o desenvolvimento de práticas pedagógicas mais inclusivas e acessíveis.

\begin{quadro}[htb]
\centering
\caption{Síntese do Estudo de Caso 2: \textit{Player} de Vídeo com Avatar de Libras}
\label{quadro:c4:cs2-summary}
\begin{tabular}{C{3cm}|m{11.75cm}}\hline
\textbf{Objeto de Estudo} & \textit{Player} de vídeo com avatar de Libras baseado em texto, integrado à transcrições automáticas. \\\hline
\textbf{Perspectiva} & Exploratório \\\hline
\textbf{Característica} & Elaboração de conhecimentos ou levantamento de informação acerca do fenômeno ainda pouco conhecido \cite{CastroFilho2021}. \\\hline
\textbf{Objetivos} & \begin{tabular}[c]{@{}m{11.75cm}@{}}1. Investigar o impacto de avatares de Libras baseados em texto na acessibilidade de conteúdos para surdos. \\ 2. Explorar a eficácia de avatares de Libras integrados à transcrição automática de videoaulas para a compreensão de OAs audíveis.\end{tabular} \\\hline
\textbf{Questões de Pesquisa} & Como as tecnologias de ASR/STT podem ser adaptadas para atender às necessidades de acessibilidade de usuários da Libras? \\\hline
\textbf{Hipóteses} & \begin{tabular}[c]{@{}m{11.75cm}@{}}1. Os avatares de Libras são úteis na compreensão de OAs audíveis transcritos automaticamente. \\2. Existe diferença na qualidade de sinalização entre os avatares \textit{VLibras} e \textit{Hand Talk} ao utilizar transcrições automáticas.\end{tabular} \\\hline
\textbf{Fontes de Dados} & Dados quantitativos do \textit{survey} e dados qualitativos de entrevistas com intérpretes de Libras. \\\hline
\textbf{Método de Coleta de Dados} & \begin{tabular}[c]{@{}m{11.75cm}@{}}1. \textit{Survey}\\ 2. Entrevistas semi-estruturadas\end{tabular} \\\hline
\textbf{Tipo de Análise} & Mista (Quantitativa e Qualitativa) \\\hline
\end{tabular}
\end{quadro}

\subsection{Resultados e Discussões}

%Os estudos de caso desta tese foram aprovados pelo \sigla{CEP}{Comitê de Ética e Pesquisa}, sob o CAAE 78381524.3.0000.5390. Embora ambos os estudos tenham sido descritos no projeto submetido ao CEP, o foco principal recai sobre o Estudo de Caso 2, que inclui tanto análises quantitativas quanto qualitativas. O Estudo de Caso 1 se concentrou principalmente em análises quantitativas da precisão de transcrições automáticas, o que não exigiu submissão ética individual. Em contraste, o Estudo de Caso 2 envolveu um componente qualitativo mais robusto, justificado pela realização de entrevistas com intérpretes de Libras.

Para explorar o impacto dos avatares de Libras na acessibilidade de conteúdos educacionais, adotou-se uma abordagem mista. Como mencionado anteriormente, sete intérpretes de Libras da empresa QS Inclusão foram convidados, dos quais dois responderam ao \textit{survey} e participaram da entrevista posteriormente. O \textit{survey} foi projetado para capturar as percepções quantitativas dos intérpretes sobre a utilidade e a qualidade dos avatares \textit{VLibras} e \textit{Hand Talk}, com base em transcrições automáticas de videoaulas.

Os resultados detalhados do \textit{survey} podem ser acessados na planilha \url{https://bit.ly/S2L-CS2SurveyResp}, com a cópia do formulário disponível no \autoref{appendix:libras-survey} e também online em \url{https://bit.ly/S2L-CS2Survey}. A análise quantitativa desses dados contribuiu para uma exploração inicial das hipóteses de pesquisa, buscando indícios sobre a utilidade dos avatares na compreensão de OAs e possíveis diferenças na qualidade de sinalização entre os dois avatares.

Para complementar os dados quantitativos do \textit{survey}, os intérpretes foram convidados a participar de entrevistas semi-estruturadas. Essas entrevistas foram planejadas para aprofundar as percepções capturadas pelo \textit{survey}, permitindo uma análise qualitativa rica das experiências e opiniões dos intérpretes sobre o uso de avatares de Libras em conteúdos educacionais.

A análise dos dados, com base nos dois participantes que concluíram o \textit{survey} e a entrevista, revelou \textit{insights} relevantes sobre a precisão e a usabilidade dos avatares de Libras. Embora os resultados quantitativos sejam limitados, eles oferecem uma síntese da percepção dos intérpretes sobre a precisão desses avatares. Em contraste, os dados qualitativos, extraídos das entrevistas, proporcionam uma compreensão mais profunda das nuances e desafios associados à adoção dessas tecnologias.

Os dados quantitativos indicam que tanto o \textit{VLibras} quanto o \textit{Hand Talk} apresentaram limitações significativas na sinalização, com ambos os avatares recebendo avaliações que destacam a presença de erros capazes de comprometer a compreensão completa dos OAs. Esses resultados, embora limitados em escopo, fornecem um plano de fundo importante para a análise dos dados qualitativos, que ofereceram \textit{insights} adicionais sobre os obstáculos e potencialidades no uso dessas ferramentas.

Por sua vez, a análise qualitativa foi conduzida com base na TFD e revelou temas centrais relacionados ao uso dos avatares de Libras. O roteiro completo para as entrevistas com os intérpretes está disponível no \autoref{appendix:libras-interview}, assegurando transparência no processo de coleta de dados e permitindo a replicação dos métodos. Durante as entrevistas, o avatar \textit{VLibras} foi utilizado como a principal ferramenta de sinalização, priorizado em detrimento ao \textit{Hand Talk} devido a fatores como sua gratuidade, natureza \textit{open-source} e o suporte contínuo do governo brasileiro. %Além disso, o \textit{VLibras} oferece configurações avançadas de regionalidade e tem recebido atualizações frequentes nos últimos anos, tornando-o uma escolha adequada para este estudo.

%Embora os resultados das entrevistas ainda estejam sendo conduzidas, a integração dos dados qualitativos e quantitativos fornecerá uma visão mais abrangente e detalhada dos impactos e desafios na adoção dessas tecnologias. O roteiro completo para as entrevistas com os intérpretes está disponível no \autoref{appendix:libras-interview}, assegurando transparência no processo de coleta de dados e permitindo a reprodutibilidade dos métodos.

%Na prática, tendo em vista o nosso planejamento e previsão financeira, utilizamos apenas o \textit{VLibras} como avatar de Libras. Ele foi priorizado em detrimento ao \textit{Hand Talk} por ser gratuito, \textit{open-source} e mantido pelo governo brasileiro. Além disso, possui configurações avançadas relacionadas à regionalidade e, nos últimos anos, voltou a ser atualizado com frequência.

As entrevistas foram transcritas e submetidas a um processo de codificação aberta, axial e seletiva \cite{Charmaz2009}. Durante a codificação aberta, segmentos de texto foram categorizados em conceitos iniciais. Em seguida, na codificação axial, esses conceitos foram organizados em categorias mais amplas, que estabelecem relações entre os diferentes conceitos identificados. Finalmente, a codificação seletiva sintetizou essas categorias em um tema central que reflete a percepção geral dos intérpretes sobre o \textit{player} de vídeo.

Os intérpretes de Libras reconheceram que as transcrições automáticas utilizadas pelo \textit{player} de vídeo apresentaram um alto nível de precisão, um fator já validado no primeiro estudo de caso desta pesquisa. No entanto, a integração dessas transcrições com avatares de Libras revelou desafios significativos. A principal dificuldade mencionada pelos intérpretes está na incapacidade dos avatares de incluir o contexto apropriado em suas sinalizações, o que pode comprometer a transferência de conhecimento. 

A Libras, como uma língua visual-espacial, depende fortemente do contexto para a correta interpretação dos sinais. Quando este contexto não é adequadamente sinalizado pelo avatar, a mensagem pode se tornar incoerente ou incompleta, prejudicando o entendimento do conteúdo \cite{Quadros2017, Quadros2019, Honora2021}.

Outro ponto levantado pelos intérpretes refere-se às barreiras sociais enfrentadas por muitos surdos e pessoas com deficiência auditiva. Grande parte dessa população não tem a oportunidade de ser bilíngue (fluente em Libras e Português) ou oralizada, o que significa que essas pessoas dependem exclusivamente da Libras para acessar o conteúdo educacional. Quando o avatar falha em proporcionar uma sinalização coerente, esses usuários ficam em desvantagem, pois não têm outra forma de acessar a informação. Essa questão ressalta a importância de desenvolver soluções tecnológicas que não só sejam tecnicamente avançadas, mas também culturalmente sensíveis e inclusivas.

Por outro lado, os intérpretes também destacaram o potencial das novas tecnologias de IA para transformar o campo da acessibilidade. Modelos de IA estão sendo treinados especificamente em Libras, o que promete inaugurar uma nova era de acessibilidade para a comunidade surda. 

Um exemplo citado nas entrevistas foi a iniciativa da Lenovo, que desenvolveu uma solução de tradução em tempo real para a Libras impulsionada por IA, mostrando resultados promissores \cite{Lenovo2023}. Esses avanços tecnológicos são vistos como oportunidades para superar as limitações atuais dos avatares de Libras e oferecer soluções mais eficazes e inclusivas no futuro.

A análise das entrevistas indica que, embora o \textit{player} de vídeo desenvolvido como uma instância da Arquitetura \textit{Speech2Learning} seja um passo significativo, há áreas críticas que precisam de melhorias. A principal delas é a integração de avatares de Libras que possam incluir o contexto necessário para uma comunicação assertiva. Além disso, as barreiras sociais e linguísticas enfrentadas por muitos surdos devem ser levadas em consideração no desenvolvimento de futuras soluções. O surgimento de soluções baseadas em IA, como os modelos mencionados pelos intérpretes, deve ser explorada para criar soluções que ofereçam uma tradução mais precisa e culturalmente adequada. O \autoref{quadro:c4:cs2-grounded-theory} sumariza as categorias e temas identificados na análise das entrevistas, utilizando a técnica da TFD.

\begin{quadro}[htb]
\centering
\caption{Categorias e Temas Identificados na Análise das Entrevistas}
\label{quadro:c4:cs2-grounded-theory}
\begin{tabular}{l|p{10cm}}
\hline
\textbf{Categoria/Tema}              & \textbf{Descrição}                                                                 \\ \hline
\textbf{Acessibilidade}         & Avaliação sobre como o \textit{player} facilita o acesso à informação para usuários surdos.  \\ \hline
\textbf{Qualidade Linguística}  & Percepção sobre a exatidão e clareza das transcrições automáticas.                 \\ \hline
\textbf{Experiência do Usuário} & Comentários sobre a interface e facilidade de uso do \textit{player}.                                \\ \hline
\textbf{Integração com Avatares}& Reflexões sobre a compatibilidade cultural e linguística dos avatares de Libras.    \\ \hline
\textbf{Desafios e Limitações}  & Dificuldades técnicas e limitações enfrentadas durante o uso do \textit{player}.            \\ \hline
\textbf{Barreiras Sociais}      & Discussões sobre as barreiras enfrentadas por surdos não bilíngues ou não oralizados.\\ \hline
\textbf{Oportunidades com IA}   & Potencial dos novos modelos de IA especializados em Libras para melhorar a acessibilidade e a inclusão de surdos.                  \\ \hline
\end{tabular}
\end{quadro}

De forma complementar, vale ressaltar que o \textit{player} foi apresentado no ``\textit{Workshop on Opportunities and Challenges of Generative AI in Education}'' da ``\textit{57th Hawaii International Conference on System Sciences}'' (HICSS). A \autoref{fig:chapter4-cs2-poc-demo} mostra as etapas da demonstração: (a) \textit{Player} de vídeo integrado às transcrições automáticas em múltiplos idiomas; (b) Avatar \textit{VLibras} ativado e sinalizando um paragrafo da transcrição; e (c) Configurações de regionalismo disponíveis no \textit{VLibras}. Seguem algumas dúvidas e \textit{feedbacks} relevantes deste evento:

\begin{figure}[htbp]
\centering
\caption{\textit{Screenshots} da Demo do \textit{Player} Apresentada no HICSS.}
\label{fig:chapter4-cs2-poc-demo}
\begin{subfigure}[b]{0.83\textwidth}
\centering
\includegraphics[width=\textwidth]{images/chapter4-cs2-poc-demo1.png}
\caption{\textit{Player} de Vídeo: Transcrições e Legendas.}
\label{fig:chapter4-cs2-poc-demo1}
\end{subfigure} ~
\begin{subfigure}[b]{0.83\textwidth}
\centering
\includegraphics[width=\textwidth]{images/chapter4-cs2-poc-demo2.png}
\caption{\textit{Player} de Vídeo: Avatar de Libras (VLibras) Sinalizando.}
\label{fig:chapter4-cs2-poc-demo2}
\end{subfigure} ~
\begin{subfigure}[b]{0.83\textwidth}
\centering
\includegraphics[width=\textwidth]{images/chapter4-cs2-poc-demo3.png}
\caption{\textit{Player} de Vídeo: Avatar de Libras (VLibras) Regionalismo.}
\label{fig:chapter4-cs2-poc-demo3}
\end{subfigure}
\fautor
\end{figure}

\begin{itemize}
    \item \textbf{Como lidar com as variações regionais das próprias línguas de sinais?}
    É fundamental reconhecer a importância das regionalidades e sotaques tanto na língua falada quanto na sinalizada. A variação regional é uma característica intrínseca das línguas de sinais, que varia conforme a localização geográfica. No estudo, o \textit{VLibras}, que possui configurações de regionalidade para cada estado do Brasil, foi utilizado para que o avatar considerasse essas variações ao sinalizar. Este aspecto é crucial para a aceitação e eficácia do avatar de Libras, pois respeita as particularidades culturais e linguísticas dos usuários.
    \item \textbf{Qual seria o esforço para integrar o \textit{player} de vídeo a avatares de outras línguas de sinais, como a ASL?}
    O \textit{player} foi implementado seguindo as práticas de \textit{Design} Universal, que preconizam que as soluções atendam ao maior número possível de usuários. O \textit{player} é, essencialmente, uma biblioteca HTML, CSS e JavaScript, garantindo que o \textit{player} de vídeo HTML, bem como sua descrição, transcrições e legendas, sigam as boas práticas de semântica, estilos e interações. Dessa forma, o \textit{player} não precisa "conhecer" o avatar de língua de sinais específico, pois está preparado estruturalmente para integrá-lo, facilitando a adaptação para outras línguas de sinais, como a ASL.
    \item \textbf{Como as IAGen estão ajudando no desenvolvimento de projetos relacionados à Arquitetura Speech2Learning?}
    Os modelos de ML têm evoluído consideravelmente nos últimos anos, e hoje é comum ouvirmos falar de GPT e outras siglas comuns na IA. As tecnologias de ASR e STT têm se beneficiado muito dessa revolução. O modelo de reconhecimento de fala da OpenAI, por exemplo, chamado Whisper, é baseado na ideia de \textit{Transformers}, a mesma tecnologia subjacente ao ChatGPT. Notavelmente, o Whisper se destacou em nos testes e análises estatísticas, e foi escolhido como provedor padrão na API de transcrição e legendagem de videoaulas devido à sua precisão e confiabilidade.
\end{itemize}

O estudo de caso realizado sobre o \textit{player} de vídeo com Avatar de Libras ilustra o progresso feito na inclusão educacional para alunos surdos. A implementação da arquitetura \textit{Speech2Learning} e a utilização dos avatares de Libras, como o \textit{VLibras}, têm demonstrado avanços na acessibilidade de conteúdos educacionais, respeitando as particularidades regionais e culturais dos usuários. Os \textit{feedbacks} recebidos no HICSS reforçam a relevância e a aplicabilidade da solução proposta, sublinhando a importância de continuar explorando e aperfeiçoando tecnologias assistivas no contexto educacional.

A abordagem mista adotada para investigar a eficácia dos avatares de Libras envolveu a aplicação de um \textit{survey} e a realização de entrevistas com intérpretes de Libras. A partir dos dados obtidos, as hipóteses formuladas foram: (i) Os avatares de Libras são úteis na compreensão de OAs audíveis transcritos automaticamente; e (ii) Existe uma diferença na qualidade de sinalização entre os avatares \textit{VLibras} e \textit{Hand Talk} ao utilizar transcrições automáticas.

No entanto, a análise dos resultados não fornece uma confirmação conclusiva dessas hipóteses. O \textit{survey} revelou percepções valiosas sobre a utilidade e a qualidade dos avatares, mas a amostra limitada de participantes e a diversidade das respostas obtidas indicam que mais pesquisas são necessárias para validar e generalizar esses achados. A questão de pesquisa, que explora como as tecnologias de reconhecimento de fala (ASR/STT) podem contribuir para uma educação mais acessível para usuários da Libras, continua sendo relevante, mas os dados atuais sugerem que os resultados devem ser interpretados com cautela.

Para concluir, a \textit{Speech2Learning} oferece um arcabouço genérico e aberto às tendências emergentes das IAs, possibilitando uma evolução natural em suas soluções de TA derivadas. Contudo, é fundamental prosseguir com estudos adicionais para refinar e avaliar sua eficácia em contextos variados. A continuidade da pesquisa permitirá aprimorar a precisão dos avatares e a integração de novas abordagens que potencializem ainda mais a acessibilidade educacional.

\section{Considerações Finais}

Os dois estudos de caso apresentados destacam a importância e a viabilidade de utilizar tecnologias de reconhecimento de fala para promover a inclusão educacional. No primeiro estudo de caso, focado em legendas automáticas para videoaulas, observou-se que transcrições automáticas com maior precisão podem aumentar significativamente o potencial de reuso e acessibilidade de OAs audíveis. As contribuições obtidas neste estudo foram detalhadas e discutidas em publicações recentes, destacando a análise lexical das transcrições automáticas e a triangulação de dados para avaliar a acessibilidade dos OAs audíveis:

\begin{itemize}
    \item \fullcite{\textbf{FALVOJR, V.}; MARCOLINO, A.; BRUNO, D.; MARTINS FALVO, C.; OSÓRIO, F.; BARBOSA, E.}{Lexical Analysis of Automatic Transcriptions Using Speech-to-Text Services: A Statistically Evaluated Case Study}{Hawaii International Conference on System Sciences (HICSS)}{2024}{Disponível em \url{hdl.handle.net/10125/107023}}

    \item \fullcite{\textbf{FALVOJR, V.}; MARCOLINO, A.; BRUNO, D.; MARTINS FALVO, C.; OSÓRIO, F.; BARBOSA, E.}{Enhancing Learning Objects Accessibility Through Speech-To-Text Based Architecture: A Comprehensive Triangulation Study}{Frontiers in Education (FIE)}{2024}{Submetido em 20/05/2024 e Aprovado em 23/07/2024}
\end{itemize}

O segundo estudo de caso, que envolveu o desenvolvimento de um \textit{player} de vídeo com \textit{Design} Universal e integração de transcrições automáticas, demonstrou potencial para atender algumas das necessidades da comunidade surda, mas também revelou limitações significativas. Esses desafios sublinham a necessidade de contínuas melhorias e adaptações para alcançar uma inclusão educacional efetiva.

Os resultados obtidos até o momento indicam que a integração de tecnologias como ASR/STT com soluções educacionais pode facilitar uma educação mais inclusiva e acessível. A continuidade deste trabalho incluirá a análise detalhada dos dados coletados nas entrevistas com intérpretes de Libras. Além disso, futuras pesquisas podem explorar a adaptação de avatares de outras línguas de sinais e o uso de novas tecnologias de IA para aprimorar ainda mais as soluções desenvolvidas.

Em síntese, os estudos de caso apresentados neste capítulo representam um avanço significativo na direção de uma educação verdadeiramente inclusiva. A aplicação da arquitetura \textit{Speech2Learning} e o uso de avatares de Libras destacam-se como abordagens promissoras para promover a inclusão educacional, contribuindo para a construção de um ambiente educacional mais acessível.


\chapter{Conclusões}
\label{chapter5}

% \noindent
% \textcolor{red}{
% Dúvidas e Pendências:
% \begin{itemize}
%     \item TODO: Padronizar as Tabelas e Figuras em Português ou Inglês?
%     \item TODO: \textit{Screenshots} da DIO no Estudo de Caso 1: Videoaulas com Legendadas?
%     \item TODO: Retomar as Hipóteses e QPs nos Resultados dos Estudos de Caso ou na Conclusão?
% \end{itemize}
% }

\section{Contribuições da Pesquisa}

Esta pesquisa apresentou a definição e a avaliação da Arquitetura \textit{Speech2Learning}, projetada com o intuito de ampliar a acessibilidade de OAs audíveis, contribuindo para o debate sobre a inclusão educacional em diferentes contextos. A principal contribuição desta tese reside na proposição de uma abordagem que integra tecnologias de ASR para melhorar a acessibilidade educacional. As soluções desenvolvidas foram avaliadas por meio de POCs e estudos de caso, que exploraram a viabilidade prática da arquitetura em ambientes educacionais reais.

Um dos principais achados da pesquisa foi a constatação de que os serviços de ASR, especialmente aqueles oferecidos pelos provedores OpenAI e Microsoft, demonstraram um nível de precisão elevado nas tarefas de transcrição e legendagem automática de videoaulas. Este resultado é particularmente relevante, pois confirma o potencial dessas tecnologias para contribuir de maneira significativa na acessibilidade de OAs audíveis. A precisão elevada das transcrições automáticas permite que os materiais educacionais sejam acessíveis a um público mais amplo de aprendizes, incluindo aqueles que possuem dificuldades auditivas e que dependem de legendas para compreender o conteúdo.

Entretanto, a pesquisa também revelou limitações importantes quanto à aplicação de avatares de Libras como um recurso de TA para a comunidade surda. Apesar da integração de avatares às transcrições automáticas ter sido tecnicamente viável, os intérpretes de Libras expressaram preocupações significativas quanto à eficácia dessa abordagem. A principal crítica refere-se à complexidade inerente à Libras, que envolve nuances culturais, regionalidades e contextos que nem sempre são capturados adequadamente por avatares automatizados. Isso sugere que, embora as tecnologias de ASR possam oferecer suporte relevante, a adaptação dessas tecnologias para atender às necessidades específicas de usuários da Libras exige um aprofundamento maior e um desenvolvimento contínuo.

Outro aspecto importante das contribuições desta pesquisa é a identificação de que nem todas as pessoas surdas ou com deficiência auditiva possuem fluência em português, o que ressalta a importância de soluções bilíngues e culturalmente sensíveis. A adoção de avatares de Libras, embora útil em determinados cenários, não substitui a necessidade de recursos mais completos e adaptáveis que respeitem a diversidade linguística e cultural da comunidade surda. Dessa forma, a pesquisa contribui para a compreensão dos desafios envolvidos na criação de soluções de TA mais inclusivas e eficazes.

As instâncias da Arquitetura \textit{Speech2Learning}, ao integrar ASR e avatares de Libras, também proporcionaram \textit{insights} valiosos sobre como essas tecnologias podem ser adaptadas para melhorar a acessibilidade educacional. A pesquisa mostrou que, enquanto o ASR pode ser altamente eficiente em fornecer transcrições precisas, a integração com avatares de Libras precisa ser repensada e aprimorada para atender às necessidades dos usuários de forma mais eficaz. A combinação dessas tecnologias em um \textit{player} de vídeo com \textit{Design} Universal, apesar das limitações destacadas pelos intérpretes nas entrevistas, oferece um modelo flexível que pode ser adaptado para diferentes contextos educacionais, destacando a potencialidade da arquitetura para ser utilizada em diversas plataformas e aplicações.

De modo geral, esta pesquisa avançou no desenvolvimento de uma arquitetura que pode ser replicada e adaptada para diferentes contextos educacionais, promovendo a inclusão e a acessibilidade de OAs audíveis. As contribuições apresentadas nesta tese são, portanto, significativas para o campo da TA aplicada à educação, oferecendo novas perspectivas e soluções para tornar o aprendizado mais acessível e inclusivo para uma diversidade maior de aprendizes.

\section{Limitações da Pesquisa}

Embora esta pesquisa tenha contribuições significativas, algumas limitações foram identificadas, as quais podem impactar tanto a replicação dos resultados quanto o desenvolvimento de soluções futuras. A seguir, são discutidas as principais limitações identificadas:

\begin{itemize}

    \item \textbf{Evolução das Soluções de ASR Impulsionadas pela IA Generativa:} A rapidez com que as tecnologias de ASR estão evoluindo, especialmente aquelas baseadas em IA, pode tornar os resultados desta pesquisa sobre a precisão dos provedores de ASR rapidamente obsoletos. A cada novo avanço, surgem melhorias nas capacidades de transcrição e legendagem automática, o que pode alterar significativamente os níveis de precisão relatados. Dessa forma, os resultados obtidos precisam ser revisitados em futuros estudos para garantir que reflitam o estado da arte das tecnologias de ASR.

    \item \textbf{Surgimento de Novos Provedores de ASR Relevantes:} Durante o desenvolvimento desta pesquisa, novos provedores de ASR, como a NVIDIA e o Facebook, surgiram com soluções de reconhecimento de fala baseadas em seus modelos mais recentes de IAGen. Essa evolução do mercado é acompanhada por iniciativas como a plataforma HuggingFace, que promove a IA aberta e colaborativa, fornecendo um \textit{Leaderboard}\footnote{Mais informações em \url{https://huggingface.co/spaces/hf-audio/open_asr_leaderboard}} que compara o desempenho de diversos modelos oferecidos por esses e outros provedores de ASR. O surgimento de novos players e a constante atualização dos modelos destacam a necessidade de considerar esses avanços em estudos futuros, uma vez que as comparações realizadas nesta pesquisa refletem apenas os provedores disponíveis no momento da investigação.

    \item \textbf{Uso de Algoritmos de Similaridade Léxica:} A avaliação da precisão dos serviços de ASR nesta pesquisa foi realizada utilizando algoritmos de similaridade léxica, que, embora sejam adequados para o contexto estudado, podem introduzir vieses significativos. Esses algoritmos avaliam a semelhança entre transcrições geradas automaticamente e transcrições de referência, mas podem falhar em capturar nuances contextuais e variações semânticas. Para futuros trabalhos, recomenda-se a combinação desses algoritmos com outras métricas, como o WER, que é utilizado pela HuggingFace em seu \textit{Leaderboard}.

    \item \textbf{\textit{Feedback} Limitado de Intérpretes de Libras:} As entrevistas realizadas com intérpretes de Libras foram limitadas, uma vez que apenas dois dos sete intérpretes convidados participaram até o momento. A falta de um número maior de participantes impede uma avaliação mais ampla e robusta das percepções sobre o uso de avatares de Libras. Isso limita a generalização dos resultados e indica a necessidade de continuar essas entrevistas para obter uma amostra mais representativa.

\end{itemize}

Essas limitações ressaltam a importância de revisitar e atualizar os resultados à medida que novas tecnologias emergem e mais dados se tornam disponíveis.

\section{Trabalhos Futuros}

Com base nas limitações e nos resultados obtidos nesta pesquisa, diversas direções podem ser exploradas em trabalhos futuros. A seguir, são discutidas algumas das principais áreas que merecem atenção:

\begin{itemize}

    \item \textbf{Integração de Agentes na Arquitetura \textit{Speech2Learning}:} Com o crescente interesse no uso de agentes inteligentes, especialmente no contexto de IAGen, uma linha de pesquisa promissora envolve a integração desses agentes na \textit{Speech2Learning}. Agentes poderiam atuar em diversas camadas da arquitetura, oferecendo suporte dinâmico e adaptativo para melhorar a acessibilidade e a personalização dos OAs. A exploração dessas diretrizes pode transformar a maneira como as diferentes camadas da arquitetura, atualmente baseadas na \textit{Clean Architecture}, são observadas e implementadas, permitindo uma evolução mais profunda e adaptativa.

    \item \textbf{Conclusão das Entrevistas com Intérpretes de Libras:} Embora esta pesquisa tenha incluído entrevistas iniciais com dois intérpretes de Libras, é essencial expandir esse conjunto de dados para uma análise mais abrangente. As entrevistas pendentes, envolvendo outros intérpretes, devem ser concluídas para fornecer uma visão mais completa sobre as percepções e desafios enfrentados na aplicação de avatares de Libras como solução de TA. A conclusão dessas entrevistas permitirá uma avaliação mais robusta e pode direcionar melhorias nas soluções desenvolvidas.

    \item \textbf{Integração com Repositórios de OAs:} Um aspecto relevante para o sucesso da Arquitetura \textit{Speech2Learning} é a facilidade com que os OAs podem ser integrados e disponibilizados através da arquitetura proposta. Trabalhos futuros devem focar em simplificar essa integração, criando métodos mais naturais e eficientes para o acoplamento de OAs com a \textit{Speech2Learning}. Isso inclui o desenvolvimento de APIs e conectores que facilitem a disponibilização dos OAs de forma automática e intuitiva, reduzindo a necessidade de intervenções manuais.

    \item \textbf{Melhorias no \textit{Player} de Vídeo com \textit{Design} Acessível:} Baseado nos \textit{feedbacks} recebidos durante esta pesquisa, futuros trabalhos devem focar em aprimorar tecnicamente o \textit{player} de vídeo desenvolvido. Isso inclui não apenas a integração mais fluida com avatares de Libras, mas também melhorias gerais na usabilidade e acessibilidade do \textit{player}. A realização de testes de usabilidade, baseados em heurísticas como as de Nielsen, pode fornecer percepções valiosas para aperfeiçoar a experiência do usuário. Além disso, a disponibilização do \textit{player} como uma biblioteca \textit{open-source} permitirá que a comunidade contribua com seu desenvolvimento e expansão.

    \item \textbf{Documentação e Atualização da \textit{Speech2Learning}:} Embora a evolução das tecnologias de ASR não afete diretamente a \textit{Speech2Learning}, é crucial que a documentação e as definições da arquitetura acompanhem essas evoluções. Isso garantirá que a \textit{Speech2Learning} continue a ser uma ferramenta relevante, mesmo com o surgimento de novas tecnologias e metodologias. A documentação deve ser revisada e atualizada periodicamente para incorporar as melhores práticas e garantir a compatibilidade com os avanços tecnológicos.

\end{itemize}

Essas direções apontam para a necessidade de continuar a evolução da \textit{Speech2Learning} e suas aplicações, garantindo que a arquitetura se mantenha adaptável e relevante em um cenário educacional de constante transformação.

\section{Publicações Resultantes}

As principais publicações resultantes das atividades conduzidas nesta pesquisa de doutorado são apresentadas a seguir, ordenadas cronologicamente:

\begin{enumerate}
    
    \item \fullcite{\textbf{FALVOJR, V.}; MARTINS FALVO, C.; SCATALON, L.; BARBOSA, E.}{Tecnologias Aplicadas ao Ensino e Aprendizagem de LIBRAS: Um Mapeamento Sistemático}{Simpósio Brasileiro de Informática na Educação (SBIE)}{2020}{Disponível em \url{doi.org/10.5753/cbie.sbie.2020.812}}

    \item \fullcite{\textbf{FALVOJR, V.}; SCATALON, L.; BARBOSA, E.}{The Role of Technology to Teaching and Learning Sign Languages: A Systematic Mapping}{Frontiers in Education Conference (FIE)}{2020}{Disponível em \url{doi.org/10.1109/FIE44824.2020.9274169}}

    \item \fullcite{\textbf{FALVOJR, V.}; MARTINS FALVO, C.; SCATALON, L.; BARBOSA, E.}{Tecnologias Aplicadas ao Ensino e Aprendizagem com Línguas de Sinais: Um Mapeamento Sistemático Sob as Perspectivas Nacional e Internacional}{Revista Novas Tecnologias na Educação (RENOTE)}{2021}{Disponível em \url{doi.org/10.22456/1679-1916.110217}}

    \item \fullcite{\textbf{FALVOJR, V.}; MARCOLINO, A.; BRUNO, D.; MARTINS FALVO, C.; OSÓRIO, F.; BARBOSA, E.}{Lexical Analysis of Automatic Transcriptions Using Speech-to-Text Services: A Statistically Evaluated Case Study}{Hawaii International Conference on System Sciences (HICSS)}{2024}{Disponível em \url{hdl.handle.net/10125/107023}}

    \item \fullcite{\textbf{FALVOJR, V.}; MARCOLINO, A.; BRUNO, D.; MARTINS FALVO, C.; OSÓRIO, F.; BARBOSA, E.}{Enhancing Learning Objects Accessibility Through Speech-To-Text Based Architecture: A Comprehensive Triangulation Study}{Frontiers in Education (FIE)}{2024}{Submetido em 20/05/2024 e Aprovado em 23/07/2024}
\end{enumerate}

De modo complementar, outros trabalhos indiretamente relacionados a esta pesquisa foram publicados. Esta colaboração contínua entre pesquisadores, muitas vezes de diferentes instituições, nos levou a descobertas e percepções fundamentais para a idealização e desenvolvimento deste trabalho de doutorado \cite{Soad2017_FIE,Oliveira2019_SBIE,FalvoJr2022_JUCS,FalvoJr2023_SMarty}:

\begin{enumerate}\setcounter{enumi}{4}
    
    \item \fullcite{SOAD, G.; FIORAVANTI, M.; \textbf{FALVOJR, V.}; MARCOLINO, A.; DUARTE FILHO, N.; BARBOSA, E.}{ReqML-catalog: The Road to a Requirements Catalog for Mobile Learning Applications}{Frontiers in Education Conference (FIE)}{2017}{Disponível em \url{doi.org/10.1109/FIE.2017.8190718}}
    
    \item \fullcite{OLIVEIRA, R.; \textbf{FALVOJR, V.}; BARBOSA, E. F.}{Internet das Coisas aplicada à Educação: Um Mapeamento Sistemático}{Simpósio Brasileiro de Informática na Educação (SBIE)}{2019}{Disponível em \url{doi.org/10.5753/cbie.sbie.2019.499}}
    
    \item \fullcite{\textbf{FALVOJR, V.}; MARCOLINO, A.; DUARTE FILHO, N.; OLIVEIRAJR, E.; BARBOSA, E.}{Variability-based Improvement of M-Learning Applications Development}{Journal of Universal Computer Science (J.UCS)}{2022}{Disponível em \url{doi.org/10.3897/jucs.90663}}
    
    \item \fullcite{\textbf{FALVOJR, V.}; MARCOLINO, A.; DUARTE FILHO, N.; OLIVEIRAJR, E.; BARBOSA, E.}{A Software Product Line for Mobile Learning Applications}{Capítulo 13 do Livro ``UML-Based Software Product Line Engineering with SMarty''}{2023}{Disponível em \url{doi.org/10.1007/978-3-031-18556-4_13}}
\end{enumerate}


% ---
% Finaliza a parte no bookmark do PDF, para que se inicie o bookmark na raiz
% ---
\bookmarksetup{startatroot}% 
% ---

% ----------------------------------------------------------
% ELEMENTOS PÓS-TEXTUAIS
% ----------------------------------------------------------
\postextual

% ----------------------------------------------------------
% Referências bibliográficas
% ----------------------------------------------------------
\bibliography{references}

% ---------------------------------------------------------------------
% GLOSSÁRIO
% ---------------------------------------------------------------------

% Arquivo que contém as definições que vão aparecer no glossário
%\newword{Framework}{é uma abstração que une códigos comuns entre vários projetos de \textit{software} provendo uma funcionalidade genérica. \textit{Frameworks} são projetados com a intenção de facilitar o desenvolvimento de \textit{software}, habilitando designers e programadores a gastarem mais tempo determinando as exigências do \textit{software} do que com detalhes de baixo nível do sistema}

\newword{Padrões de Projeto}{ou \textit{Design Pattern}, descreve uma solução geral reutilizável para um problema recorrente no desenvolvimento de sistemas de \textit{software} orientados a objetos. Não é um código final, é uma descrição ou modelo de como resolver o problema do qual trata, que pode ser usada em muitas situações diferentes}

\newword{Web}{Sinônimo mais conhecido de \textit{World Wide Web} (WWW). É a interface gráfica da Internet que torna os serviços disponíveis totalmente transparentes para o usuário e ainda possibilita a manipulação multimídia da informação}

% Comando para incluir todas as definições do arquivo glossario.tex
%\glsaddall
% Impressão do glossário
\printglossaries

% ----------------------------------------------------------
% Apêndices
% ----------------------------------------------------------

% ---
% Inicia os apêndices
% ---
\begin{apendicesenv}

\chapter{Survey Transcrição Automática}
\label{appendix:asr-survey}
\section{Introdução}

\noindent
Caro(a), você está sendo convidado(a) a participar da pesquisa intitulada "Avaliação de 
Soluções Para Transcrição Automática de Videoaulas Sobre Tecnologia". Este estudo 
tem como objetivo avaliar a qualidade da transcrição automática de videoaulas no 
contexto do ensino e aprendizagem de tecnologia em múltiplas línguas (Português, 
Inglês e Espanhol). Para atingir esse objetivo, submeteremos 15 vídeos curtos (5 em 
cada idioma) às seguintes soluções de reconhecimento de fala:

\begin{itemize}
    \item Amazon: \url{https://aws.amazon.com/transcribe}
    \item Google: \url{https://cloud.google.com/speech-to-text}
    \item IBM: \url{https://www.ibm.com/cloud/watson-speech-to-text}
    \item Microsoft: \url{https://azure.microsoft.com/en-us/products/ai-services/ai-speech}
    \item OpenAI: \url{https://openai.com/research/whisper}
\end{itemize}

\noindent
Neste estudo, contamos com a colaboração da edtech DIO (\url{https://dio.me}), que disponibilizou 
trechos de videoaulas (com duração entre 10 e 30 segundos) presentes em sua 
plataforma de ensino de tecnologia. O objetivo dessa parceria é investigar o potencial 
das soluções baseadas em reconhecimento de fala para aprimorar a acessibilidade dos 
conteúdos educacionais audíveis. Tais soluções são especialmente promissoras, visto 
que podem viabilizar a criação de transcrições, legendas e até a sinalização em línguas 
de sinais (por exemplo, utilizando um avatar de LIBRAS baseado em texto).

\noindent
Se você possui alguma experiência nas áreas de Tecnologia da Informação (TI), Letras, 
Linguística ou Educação, está convidado(a) a responder as perguntas a seguir. Além disso, se 
possível, pedimos que encaminhe a possíveis respondentes.

\noindent
Desde já agradecemos sua disponibilidade e participação.

\section{Termo de Consentimento Livre e Esclarecido}

\noindent
Prezado(a) participante: Esta pesquisa é realizada de acordo com as recomendações 
estabelecidas pelo Comitê de Ética da Universidade de São Paulo (USP). Em atendimento às 
normas desse Comitê de Ética e orientações científicas, pedimos que registre sua 
concordância na participação da pesquisa no campo abaixo.

\noindent
OBSERVAÇÃO: O Comitê de Ética em Pesquisa (CEP/EACH) funciona na Av. Arlindo Béttio, 
1000, Ermelino Matarazzo, São Paulo-SP, tel: (11) 3091-1046, e-mail: \href{mailto:cep-each@usp.br}{cep-each@usp.br}.

\noindent
Ressaltamos que na divulgação dos resultados desta pesquisa, a identidade dos participantes 
será mantida no mais rigoroso sigilo. Se precisar de mais informações sobre sua participação 
ou sobre a pesquisa, faça contato para esclarecimentos: 

\noindent
Contato: Venilton FalvoJr \href{mailto:falvojr@usp.br}{falvojr@usp.br}

\noindent
Para CONCORDAR em participar desta pesquisa e preencher o questionário, responda CONCORDO no campo abaixo. Para DESISTIR definitivamente do preenchimento, basta FECHAR SEU NAVEGADOR. Agradecemos pela disponibilidade e atenção!

\noindent
\textit{[Campo de Seleção Única "CONCORDO"]}

\section{Escala de Avaliação e Anonimato dos Provedores}

\noindent
As transcrições automáticas das videoaulas serão avaliadas utilizando uma escala de 5 
alternativas, as quais serão individualmente definidas a fim de minimizar possíveis 
interpretações incorretas: 

\begin{enumerate}
    \item Muito Incoerente: Transcrição sem sentido, tornando a compreensão impossível;
    \item Incoerente: Muitos erros, permitindo apenas uma compreensão parcial;
    \item Útil: Erros presentes, mas sem impedir a compreensão geral;
    \item Coerente: Pequenas imprecisões, mas sem comprometer a compreensão total;
    \item Muito Coerente: Transcrição perfeita, garantindo uma compreensão completa.
\end{enumerate}

\noindent
Para evitar avaliações tendenciosas, não identificaremos os provedores pelo nome 
(Amazon, Google, IBM, Microsoft ou OpenAI), mas sim por um identificador numérico 
gerado aleatoriamente (de 1 a 5) para cada um deles. 

\noindent
Para aqueles interessados em detalhes técnicos, disponibilizamos um projeto no Google Colab contendo o código-fonte para integração com todos os provedores cujas transcrições automáticas são avaliadas nesta pesquisa: \url{https://bit.ly/S2L-STTServices}

\noindent
Antes de iniciarmos, gostaríamos de conhecer sua experiência nas áreas de interesse 
desta pesquisa. Essas informações são importantes para uma análise precisa dos 
resultados. Considere todas as formas de experiência adquirida e, caso não possua 
experiência em alguma delas, responda com 0 (zero).

\noindent
\textbf{Quantos anos de experiência você possui na área de Tecnologia da Informação (TI)?}

\noindent
\textit{[Campo de Texto com Máscara Numérica]}

\noindent
\textbf{Quantos anos de experiência você possui na área de Letras, Linguística ou Educação?}

\noindent
\textit{[Campo de Texto com Máscara Numérica]}

\section{Transcrições Automáticas em Português}

\noindent
Por favor, indique o seu nível de conhecimento em Português, a língua nativa das 
próximas 5 videoaulas. Ao informar sua proficiência, você nos ajudará a conduzir uma 
análise mais precisa e abrangente sobre a qualidade das transcrições automáticas.

\noindent
\textbf{Qual é o seu nível de proficiência em Português?}

\noindent
As alternativas simplificam o Quadro Europeu Comum de Referência para Línguas (QECR)\footnote{Mais informações em: \url{https://cambridgeenglish.org/br/exams-and-tests/cefr}}: ``Básico'' (A1 e A2); ``Intermediário'' (B1 e B2) e ``Avançado'' (C1 e C2). Por outro lado, caso não tenha conhecimento algum em Português, selecione a opção "Sem Proficiência" para não avaliar as videoaulas nesse idioma.

\noindent
\textit{[Grupo de Campos para Seleção Única]}

\noindent
\textit{[``Básico'', ``Intermediário'', ``Avançado'' e ``Sem Proficiência (Pular Vídeos em Português)'']}

\subsection{Vídeo 1/5 - Desafio de Projeto App Android Nativo (pt-BR)}

\noindent
Transcrições Automáticas da videoaula \url{http://youtube.com/watch?v=bZ-PkhuAsGI}:

\begin{itemize}
    \item Provedor 1: O desafio para vocês é que vocês criem. Não é um aplicativo nativo em Android com a temática dos jogos do Brasil na Copa do Mundo, que vai ter ali algumas funcionalidades bem legais, né? Como o consumo da lista de partidas de uma p your este vai ter também é a questão de persistência local de algumas informações.
    \item Provedor 2: O desafio para vocês é que vocês criem né um um aplicativo na ativa em android com a temática dos jogos do brasil na copa do mundo que vai ter ali algumas funcionalidades bem legais né como o consumo da lista de partidas de uma payerest vai ter também a questão de percidência local de algumas informações.
    \item Provedor 3: O desafio para vocês é que vocês criem um aplicativo nativo em Android com a temática dos jogos do Brasil na Copa do Mundo, que vai ter algumas funcionalidades bem legais, como o consumo da lista de partidas de uma Payrest, vai ter também a questão de persistência local de algumas informações.
    \item Provedor 4: O desafio para vocês é que vocês criam um aplicativo nativo em Android com a temática dos jogos do Brasil na Copa do Mundo, que vai ter ali algumas funcionalidades bem legais, como o consumo da lista de partidas de uma PME Oeste Vai ter também a questão de presidência local de algumas informações.
    \item Provedor 5: Desafio para vocês aqui vocês criem é um aplicativo nativo em Android com a temática dos jogos do Brasil na copa do mundo que vai ter ali algumas funcionalidades bem legais né como consumo na lista de partidas de uma pessoa vai ter também a questão de persistência local de algumas informações.
\end{itemize}

\noindent
\textbf{Qual é o nível de coerência das transcrições automáticas apresentadas?}

\noindent
\textit{[Grupo de Campos para Seleção Única (Escala) para Cada Provedor]}

\noindent
\textit{[Provedor 1: ``Muito Incoerente'', ``Incoerente'', ``Útil'', ``Coerente'' e ``Muito Coerente'']}

\noindent
\textit{[Provedor 2: ``Muito Incoerente'', ``Incoerente'', ``Útil'', ``Coerente'' e ``Muito Coerente'']}

\noindent
\textit{[Provedor 3: ``Muito Incoerente'', ``Incoerente'', ``Útil'', ``Coerente'' e ``Muito Coerente'']}

\noindent
\textit{[Provedor 4: ``Muito Incoerente'', ``Incoerente'', ``Útil'', ``Coerente'' e ``Muito Coerente'']}

\noindent
\textit{[Provedor 5: ``Muito Incoerente'', ``Incoerente'', ``Útil'', ``Coerente'' e ``Muito Coerente'']}

\subsection{Vídeo 2/5 - Valores do SCRUM (pt-BR)}

\noindent
Transcrições Automáticas da videoaula \url{http://youtube.com/watch?v=R33d_lFxVMw}:

\begin{itemize}
    \item Provedor 1: Primeiro valor, indivíduos e interações mais que processos e ferramentas. Segundo o valor software em funcionamento, mais do que documentação abrangente. Terceiro valor, colaboração do cliente, mais do que negociação de contratos, quarto valor, responder a mudança mais do que seguir um plano.
    \item Provedor 2: Primeiro valor indivíduos e interações mais que processos e ferramentas. Segundo o valor software em funcionamento mais do que documentação abrangente. Terceiro valor colaboração. Do cliente mais do que negociação de contratos. Quarto valor responder a mudança mais o que seguir um plano.
    \item Provedor 3: Primeiro valor, indivíduos e interações, mais que processos e ferramentas. Segundo valor, software em funcionamento, mais do que documentação abrangente. Terceiro valor, colaboração do cliente, mais do que negociação de contratos. Quarto valor, responder a mudança, mais do que seguir um plano.
    \item Provedor 4: O primeiro valor Indivíduos, Interações mais que processos e ferramentas segundo o valor sofrer em funcionamento mais do que documentação abrangente Terceiro valor Colaboração do cliente Mais do que negociação de contratos quarto valor responder a mudança, mais do que seguir um plano.
    \item Provedor 5: Primeiro valor indivíduos e interações Mais Que processos e ferramentas segundo o valor software em funcionamento mais do que documentos são abrangente terceiro valor colaboração do cliente mais do que negociação de contratos quarto valor responder a dança mais do que seguir um plano.
\end{itemize}

\noindent
\textbf{Qual é o nível de coerência das transcrições automáticas apresentadas?}

\noindent
\textit{[Grupo de Campos para Seleção Única (Escala) para Cada Provedor]}

\noindent
\textit{[Provedor 1: ``Muito Incoerente'', ``Incoerente'', ``Útil'', ``Coerente'' e ``Muito Coerente'']}

\noindent
\textit{[Provedor 2: ``Muito Incoerente'', ``Incoerente'', ``Útil'', ``Coerente'' e ``Muito Coerente'']}

\noindent
\textit{[Provedor 3: ``Muito Incoerente'', ``Incoerente'', ``Útil'', ``Coerente'' e ``Muito Coerente'']}

\noindent
\textit{[Provedor 4: ``Muito Incoerente'', ``Incoerente'', ``Útil'', ``Coerente'' e ``Muito Coerente'']}

\noindent
\textit{[Provedor 5: ``Muito Incoerente'', ``Incoerente'', ``Útil'', ``Coerente'' e ``Muito Coerente'']}

\subsection{Vídeo 3/5 - Design Patterns no Selenium WebDriver (pt-BR)}

\noindent
Transcrições Automáticas da videoaula \url{http://youtube.com/watch?v=kyeiLwq-HQk}:

\begin{itemize}
    \item Provedor 1: Então, primeiramente, eu recomendo, para que você estude um pouco e leia sobre esse design pattern é chamado aí DPS de object. Ele é muito interessante e quando nós estamos trabalhando com automatização de testes, ele é muito utilizado. Tudo bem, então não vou entrar aqui nesse detalhe, porque o foco aqui do nosso treinamento é trabalhar e conhecer as funcionalidades do selênio web drive.
    \item Provedor 2: Então o primeiramento recomendo para que você estude um pouco e leia sobre esse design patrior na chamada de perdi objeto ele é muito interessante e quando nós estamos trabalhando com automatização de testes ele é muito utilizado. Tudo bem então não vou entrar aqui é nesse detalhe porque o foco aqui do nosso treinamento é trabalhar e conhecesse uma finalidades do celenio web drive.
    \item Provedor 3: Então, primeiramente, eu recomendo para que você estude um pouco e leia sobre esse design pattern, chamado de Page Object. Ele é muito interessante e quando nós estamos trabalhando com automatização de testes, ele é muito utilizado. Tudo bem? Então, eu não vou entrar aqui nesse detalhe, porque o foco aqui do nosso treinamento é trabalhar e conhecer as funcionalidades do Selenium WebDriver.
    \item Provedor 4: Então, primeiramente o recomendo para que você estude um pouco leia sobre esse desempate. Na chamada ele é muito interessante. E quando nós estamos trabalhando com a automatização de testes, e ele é muito utilizado, tudo bem, então não foi entrar aqui nesse detalhe, porque o foco aqui no nosso treinamento é trabalhar e conhecer as personalidades do selênio Web Drive.
    \item Provedor 5: Então primeiramente eu recomendo para que você estude um pouco e leia sobre esse design patterns não é chamado aí depois de ouvir ele é muito interessante e quando nós estamos trabalhando com automatização de testes ele é muito utilizado Tudo bem então não vou entrar aqui é desse de tarde porque eu faço para cuidar do nosso treinamento é trabalhar e conhecer as funcionalidades do Selenium web driver.
\end{itemize}

\noindent
\textbf{Qual é o nível de coerência das transcrições automáticas apresentadas?}

\noindent
\textit{[Grupo de Campos para Seleção Única (Escala) para Cada Provedor]}

\noindent
\textit{[Provedor 1: ``Muito Incoerente'', ``Incoerente'', ``Útil'', ``Coerente'' e ``Muito Coerente'']}

\noindent
\textit{[Provedor 2: ``Muito Incoerente'', ``Incoerente'', ``Útil'', ``Coerente'' e ``Muito Coerente'']}

\noindent
\textit{[Provedor 3: ``Muito Incoerente'', ``Incoerente'', ``Útil'', ``Coerente'' e ``Muito Coerente'']}

\noindent
\textit{[Provedor 4: ``Muito Incoerente'', ``Incoerente'', ``Útil'', ``Coerente'' e ``Muito Coerente'']}

\noindent
\textit{[Provedor 5: ``Muito Incoerente'', ``Incoerente'', ``Útil'', ``Coerente'' e ``Muito Coerente'']}

\subsection{Vídeo 4/5 - Características da Blockchain (pt-BR)}

\noindent
Transcrições Automáticas da videoaula \url{http://youtube.com/watch?v=DDmAIpo9EpA}:

\begin{itemize}
    \item Provedor 1: A grande sacada da blockchain é, não existe entidade centralizadora que está intermediando os acordos que acontecem os eventos que acontecem na rede, e sim toda a sua regra, toda sua lógica por trás da aplicação é o que define como que as coisas vão acontecer da autenticidade e a confiabilidade de todos os eventos.
    \item Provedor 2: A grande sacada da block tinha não existe entidade centralizadora que está intermediando os acordos que acontecem os eventos que acontecem na rede e sim toda sua regra da sua lógica por trás da aplicação é o que define como que as coisas vão acontecer da autenticidade e a confiabilidade de todos os eventos.
    \item Provedor 3: A grande sacada da blockchain é que não existe entidade centralizadora que está intermediando os acordos que acontecem, os eventos que acontecem na rede. E sim, toda a sua regra, toda a sua lógica por trás da aplicação é o que define como que as coisas vão acontecer. A autenticidade e a confiabilidade de todos os eventos.
    \item Provedor 4: A grande sacada do bloco tinha. Não existe entidades centralizadoras que está intermediando os acordos que acontecem, os eventos que acontecem na rede, e sim toda sua regra da sua lógica por trás da aplicação. É o que define como que as coisas vão acontecer a autenticidade e a confiabilidade de todos os eventos.
    \item Provedor 5: A grande tacada login e não existe entidade centralizadora que está intermediando os acordos que acontecem os eventos que acontecem na rede e sim toda sua regra dos Faróis traz aplicação é o que define como que as coisas vão acontecer a autenticidade e a confiabilidade e todos os.
\end{itemize}

\noindent
\textbf{Qual é o nível de coerência das transcrições automáticas apresentadas?}

\noindent
\textit{[Grupo de Campos para Seleção Única (Escala) para Cada Provedor]}

\noindent
\textit{[Provedor 1: ``Muito Incoerente'', ``Incoerente'', ``Útil'', ``Coerente'' e ``Muito Coerente'']}

\noindent
\textit{[Provedor 2: ``Muito Incoerente'', ``Incoerente'', ``Útil'', ``Coerente'' e ``Muito Coerente'']}

\noindent
\textit{[Provedor 3: ``Muito Incoerente'', ``Incoerente'', ``Útil'', ``Coerente'' e ``Muito Coerente'']}

\noindent
\textit{[Provedor 4: ``Muito Incoerente'', ``Incoerente'', ``Útil'', ``Coerente'' e ``Muito Coerente'']}

\noindent
\textit{[Provedor 5: ``Muito Incoerente'', ``Incoerente'', ``Útil'', ``Coerente'' e ``Muito Coerente'']}

\subsection{Vídeo 5/5 - SOs de Kernel Híbrido (pt-BR)}

\noindent
Transcrições Automáticas da videoaula \url{http://youtube.com/watch?v=3-8F2J8pzPQ}:

\begin{itemize}
    \item Provedor 1: Bom, o Kernel híbrido é utilizado nos sistemas operacionais da Apple. Não é uma que OSO Windows, né? Windows 10 aí no minix o minix, pra quem não sabe, é um sistema, um Mini sistema operacional, desenvolvido pelo professor André ator é também Ball, né? E para estudos, e ele utilizou como base é a base do Linux, tá Jóia.
    \item Provedor 2: Bom oker no é utilizado nos sistemas operacionais da apple né o macos um windows né um e dois dez e no minix o minix para quem não sabe é um sistema mini sistema operacional desenvolvido pelo prof androton tanembal né e pra estudos e ele utilizou como base a base do linux tá joia.
    \item Provedor 3: Bom, o kernel híbrido é utilizado nos sistemas operacionais da Apple, o MacOS, o Windows 10 e no Minix. O Minix, pra quem não sabe, é um mini sistema operacional desenvolvido pelo professor André Tannenbaum pra estudos e ele utilizou a base do Linux.
    \item Provedor 4: Bom o quer no livro utilizados nos sistemas operacionais da época. Como é que um Windows Windows dez e no mínimo um mínimos para quem não sabe, é um sistema operacional desenvolvido pelo professor doutor também bol e para estudos, e ele utilizou como base a base do Linux.
    \item Provedor 5: Bom o que é utilizado nos sistemas operacionais da Apple né O Michael é 100 Windows na Windows 10 ilumine omnix para quem não sabe é um sistema operacional desenvolvido pelo professor Antônio Aníbal né E para estudos e ele utilizou como base a base do Linux.
\end{itemize}

\noindent
\textbf{Qual é o nível de coerência das transcrições automáticas apresentadas?}

\noindent
\textit{[Grupo de Campos para Seleção Única (Escala) para Cada Provedor]}

\noindent
\textit{[Provedor 1: ``Muito Incoerente'', ``Incoerente'', ``Útil'', ``Coerente'' e ``Muito Coerente'']}

\noindent
\textit{[Provedor 2: ``Muito Incoerente'', ``Incoerente'', ``Útil'', ``Coerente'' e ``Muito Coerente'']}

\noindent
\textit{[Provedor 3: ``Muito Incoerente'', ``Incoerente'', ``Útil'', ``Coerente'' e ``Muito Coerente'']}

\noindent
\textit{[Provedor 4: ``Muito Incoerente'', ``Incoerente'', ``Útil'', ``Coerente'' e ``Muito Coerente'']}

\noindent
\textit{[Provedor 5: ``Muito Incoerente'', ``Incoerente'', ``Útil'', ``Coerente'' e ``Muito Coerente'']}

\subsection{NPS}

\noindent
O NPS é uma métrica de satisfação que utiliza uma escala de 0 a 10, onde: 

\begin{itemize}
    \item Notas 9 e 10 são promotoras e indicam um alto nível de satisfação; 
    \item Notas 7 e 8 são neutras e indicam satisfação moderada e sem entusiasmo; 
    \item Notas de 0 a 6 são detratoras e indicam insatisfação.
\end{itemize}

\noindent
\textbf{Considerando as transcrições dos vídeos em Português, o quão satisfeito você está com os provedores de reconhecimento de fala?}

\noindent
\textit{[Grupo de Campos para Seleção Única (Escala de 0 a 10)]}

\noindent
\textbf{Se desejar, compartilhe observações, sugestões ou feedbacks sobre as transcrições automáticas dos vídeos em Português.}

\noindent
\textit{[Campo de Texto Longo]}

\section{Transcrições Automáticas em Inglês}

\noindent
Por favor, indique o seu nível de conhecimento em Inglês, a língua nativa das próximas 5 
videoaulas. Ao informar sua proficiência, você nos ajudará a conduzir uma análise mais 
precisa e abrangente sobre a qualidade das transcrições automáticas.

\noindent
\textbf{Qual é o seu nível de proficiência em Inglês?}

\noindent
As alternativas simplificam o Quadro Europeu Comum de Referência para Línguas (QECR): ``Básico'' (A1 e A2); ``Intermediário'' (B1 e B2) e ``Avançado'' (C1 e C2). Por outro lado, caso não tenha conhecimento algum em Inglês, selecione a opção "Sem Proficiência" para não avaliar as videoaulas nesse idioma.

\noindent
\textit{[Grupo de Campos para Seleção Única]}

\noindent
\textit{[``Básico'', ``Intermediário'', ``Avançado'' e ``Sem Proficiência (Pular Vídeos em Inglês)'']}

\subsection{Vídeo 1/5 - O Que é Um Visto de Trabalho (en-US)}

\noindent
Transcrições Automáticas da videoaula \url{http://youtube.com/watch?v=VzhhqsIB_x0}:

\begin{itemize}
    \item Provedor 1: Transit visa is going to be a visa that allows you to go in transit between countries, but you're not gonna stay in that country. For example, let's suppose you want to travel. Your final destination is Canada, but you have to stop in the United States. So you need the transit visa to get inside the country while you wait for the next flight to go to your final destination.
    \item Provedor 2: Trails it visa is gonna be a visa that allows you to go in transit between countries but you're not going to stay in that country for example that suppose you want to travel your final destination is canada but you have to stop in the united states so you need a transit visa to get inside the country while you wait for the next flight to go to your final destination.
    \item Provedor 3: Transit visa is going to be a visa that allows you to go in transit between countries but you're not going to stay in that country, for example, let's suppose you want to travel, your final destination is Canada but you have to stop in the United States so you need a transit visa to get inside the country while you wait for the next flight to go to your final destination.
    \item Provedor 4: Transit visa is going to be a visa that allows you to go in transit between countries, but you're not going to stay in that country. For example, let's suppose you want to travel, your final destination is Canada, but you have to stop in the United States. So you need a transit visa to get inside the country while you wait for the next flight to go to your final destination.
    \item Provedor 5: Transit Visa is going to be a Visa that allows you to go in transit between countries but you're not going to stay in that country for example that supposed wants to travel your final destination is Canada but you have to stop in the United States do you need a transit Visa to get inside the country while you wait for the next flight to go to your final destination.
\end{itemize}

\noindent
\textbf{Qual é o nível de coerência das transcrições automáticas apresentadas?}

\noindent
\textit{[Grupo de Campos para Seleção Única (Escala) para Cada Provedor]}

\noindent
\textit{[Provedor 1: ``Muito Incoerente'', ``Incoerente'', ``Útil'', ``Coerente'' e ``Muito Coerente'']}

\noindent
\textit{[Provedor 2: ``Muito Incoerente'', ``Incoerente'', ``Útil'', ``Coerente'' e ``Muito Coerente'']}

\noindent
\textit{[Provedor 3: ``Muito Incoerente'', ``Incoerente'', ``Útil'', ``Coerente'' e ``Muito Coerente'']}

\noindent
\textit{[Provedor 4: ``Muito Incoerente'', ``Incoerente'', ``Útil'', ``Coerente'' e ``Muito Coerente'']}

\noindent
\textit{[Provedor 5: ``Muito Incoerente'', ``Incoerente'', ``Útil'', ``Coerente'' e ``Muito Coerente'']}

\subsection{Vídeo 2/5 - Entrevista de Emprego (en-US)}

\noindent
Transcrições Automáticas da videoaula \url{http://youtube.com/watch?v=VwHhRoHfyAM}:

\begin{itemize}
    \item Provedor 1: So now tell me a little bit about the technologies that you've been mastering during these years. In the recent years I've worked as a software architect mainly in projects in the Java programming language. In this context, my role is to design safe and scalable solutions in my banking domain.
    \item Provedor 2: So now coming little bit about the technology that you've been master in general. In the reason years i worked at a software architect mainly in projects in the job of our running language in this contacts my role is designed safe and scaleable solutions and banking.
    \item Provedor 3: So now tell me a little bit about the technologies that you've been mastering during these years. In the recent years I've worked as a software architect, mainly in projects in the Java programming language. In this context my role is to design safe and scalable solutions in a banking domain.
    \item Provedor 4: So now tell me a little bit about the technologies that you've been mastering during these years. In the recent years, I've worked as a software architect mainly in projects in the Java programming language. Uh in this context, my role is to design safe and scalable solutions in the banking domain.
    \item Provedor 5: Send me little bit about the technology. In the recent years I worked as a software architect mainly in projects in the Java programming language.
\end{itemize}

\noindent
\textbf{Qual é o nível de coerência das transcrições automáticas apresentadas?}

\noindent
\textit{[Grupo de Campos para Seleção Única (Escala) para Cada Provedor]}

\noindent
\textit{[Provedor 1: ``Muito Incoerente'', ``Incoerente'', ``Útil'', ``Coerente'' e ``Muito Coerente'']}

\noindent
\textit{[Provedor 2: ``Muito Incoerente'', ``Incoerente'', ``Útil'', ``Coerente'' e ``Muito Coerente'']}

\noindent
\textit{[Provedor 3: ``Muito Incoerente'', ``Incoerente'', ``Útil'', ``Coerente'' e ``Muito Coerente'']}

\noindent
\textit{[Provedor 4: ``Muito Incoerente'', ``Incoerente'', ``Útil'', ``Coerente'' e ``Muito Coerente'']}

\noindent
\textit{[Provedor 5: ``Muito Incoerente'', ``Incoerente'', ``Útil'', ``Coerente'' e ``Muito Coerente'']}

\subsection{Vídeo 3/5 - A Importância da Resiliência (en-US)}

\noindent
Transcrições Automáticas da videoaula \url{http://youtube.com/watch?v=w5ZUk6HUBNk}:

\begin{itemize}
    \item Provedor 1: And don't be scared. You will receive a lot of notes. I've received so many notes. I received hundreds of nose before I got to Sweden. So don't be, you know, disappointed. Don't be like, get a lot of strength and keep going because somebody will give you a chance.
    \item Provedor 2: And don't be scared you will receive a lot of nose i've received so many nose i received hundreds of nose before i got to switten so don't be you know disappointed don't be like get a lot of strength and keep going because somebody will give you a chance.
    \item Provedor 3: And don't be scared. You will receive a lot of no's. I've received so many no's. I received hundreds of no's before I got to Sweden. So don't be, you know, disappointed. Don't be like, get a lot of strength and keep going because somebody will give you a chance.
    \item Provedor 4: And don't be scared. You will receive a lot of nos I've received so many nos I received hundreds of nos before I got to Sweden. So don't be, you know, disappointed, don't be like get a lot of strength and keep going because somebody will give you a chance.
    \item Provedor 5: And don't be scared you will receive a lot of knows I've received so many knows I received hundreds of knows before I got to Sweden so don't be you know disappointed don't be like get a lot of strength and keep going because somebody will give you a chance.
\end{itemize}

\noindent
\textbf{Qual é o nível de coerência das transcrições automáticas apresentadas?}

\noindent
\textit{[Grupo de Campos para Seleção Única (Escala) para Cada Provedor]}

\noindent
\textit{[Provedor 1: ``Muito Incoerente'', ``Incoerente'', ``Útil'', ``Coerente'' e ``Muito Coerente'']}

\noindent
\textit{[Provedor 2: ``Muito Incoerente'', ``Incoerente'', ``Útil'', ``Coerente'' e ``Muito Coerente'']}

\noindent
\textit{[Provedor 3: ``Muito Incoerente'', ``Incoerente'', ``Útil'', ``Coerente'' e ``Muito Coerente'']}

\noindent
\textit{[Provedor 4: ``Muito Incoerente'', ``Incoerente'', ``Útil'', ``Coerente'' e ``Muito Coerente'']}

\noindent
\textit{[Provedor 5: ``Muito Incoerente'', ``Incoerente'', ``Útil'', ``Coerente'' e ``Muito Coerente'']}

\subsection{Vídeo 4/5 - Liderança Servidora (en-US)}

\noindent
Transcrições Automáticas da videoaula \url{http://youtube.com/watch?v=QJrnDA4mr_8}:

\begin{itemize}
    \item Provedor 1: A servant leader focuses on serving others first, listening and responding to people's needs, involving them in decision making and committing to people's growth and persuading them with empathy.
    \item Provedor 2: A servant leader focuses on serving others first. Listening and responding to people's means involving them a decision making committing to people's growth and persuading them with empathy.
    \item Provedor 3: A servant leader focuses on serving others first, listening and responding to people's needs, involving them in decision making, committing to people's growth and persuading them with empathy.
    \item Provedor 4: A servant leader focuses on serving others first listening and responding to people's needs involving them in decision making, um committing to people's growth and persuading them with empathy.
    \item Provedor 5: A servant leader focuses on serving others first listening and responding to people's need involving them and decision-making committing people's growth and persuading them with empathy.
\end{itemize}

\noindent
\textbf{Qual é o nível de coerência das transcrições automáticas apresentadas?}

\noindent
\textit{[Grupo de Campos para Seleção Única (Escala) para Cada Provedor]}

\noindent
\textit{[Provedor 1: ``Muito Incoerente'', ``Incoerente'', ``Útil'', ``Coerente'' e ``Muito Coerente'']}

\noindent
\textit{[Provedor 2: ``Muito Incoerente'', ``Incoerente'', ``Útil'', ``Coerente'' e ``Muito Coerente'']}

\noindent
\textit{[Provedor 3: ``Muito Incoerente'', ``Incoerente'', ``Útil'', ``Coerente'' e ``Muito Coerente'']}

\noindent
\textit{[Provedor 4: ``Muito Incoerente'', ``Incoerente'', ``Útil'', ``Coerente'' e ``Muito Coerente'']}

\noindent
\textit{[Provedor 5: ``Muito Incoerente'', ``Incoerente'', ``Útil'', ``Coerente'' e ``Muito Coerente'']}

\subsection{Vídeo 5/5 - Simplicidade das Goroutines (en-US)}

\noindent
Transcrições Automáticas da videoaula \url{http://youtube.com/watch?v=0FBZcEA6HGA}:

\begin{itemize}
    \item Provedor 1: Uh, so goroutines are even more lightweight than threads. And they're extremely simple to write. To the point where you literally just put the word go in front of the function and that function is now a go routine.
    \item Provedor 2: Uh so governors are even more light weight than throws. And they are extremely supportive. To the point where you literally just put the word go in front of the function and that function is now augurity.
    \item Provedor 3: So goroutines are even more lightweight than threads. And they are extremely simple to write. To the point where you literally just put the word go in front of a function, and that function is now a goroutine.
    \item Provedor 4: Uh so go routes are even more lightweight than threads. And they are extremely simple to write to the point where you literally just put the word go in front of a function. And that function is now a go routine.
    \item Provedor 5: Even more lightweight than threats. And they are extremely simple. To the point where you literally just put the word go in front of a function and its function is now a go routine.
\end{itemize}

\noindent
\textbf{Qual é o nível de coerência das transcrições automáticas apresentadas?}

\noindent
\textit{[Grupo de Campos para Seleção Única (Escala) para Cada Provedor]}

\noindent
\textit{[Provedor 1: ``Muito Incoerente'', ``Incoerente'', ``Útil'', ``Coerente'' e ``Muito Coerente'']}

\noindent
\textit{[Provedor 2: ``Muito Incoerente'', ``Incoerente'', ``Útil'', ``Coerente'' e ``Muito Coerente'']}

\noindent
\textit{[Provedor 3: ``Muito Incoerente'', ``Incoerente'', ``Útil'', ``Coerente'' e ``Muito Coerente'']}

\noindent
\textit{[Provedor 4: ``Muito Incoerente'', ``Incoerente'', ``Útil'', ``Coerente'' e ``Muito Coerente'']}

\noindent
\textit{[Provedor 5: ``Muito Incoerente'', ``Incoerente'', ``Útil'', ``Coerente'' e ``Muito Coerente'']}

\subsection{NPS}

\noindent
O NPS é uma métrica de satisfação que utiliza uma escala de 0 a 10, onde: 

\begin{itemize}
    \item Notas 9 e 10 são promotoras e indicam um alto nível de satisfação; 
    \item Notas 7 e 8 são neutras e indicam satisfação moderada e sem entusiasmo; 
    \item Notas de 0 a 6 são detratoras e indicam insatisfação.
\end{itemize}

\noindent
\textbf{Considerando as transcrições dos vídeos em Inglês, o quão satisfeito você está com os provedores de reconhecimento de fala?}

\noindent
\textit{[Grupo de Campos para Seleção Única (Escala de 0 a 10)]}

\noindent
\textbf{Se desejar, compartilhe observações, sugestões ou feedbacks sobre as transcrições automáticas dos vídeos em Inglês.}

\noindent
\textit{[Campo de Texto Longo]}

\section{Transcrições Automáticas em Espanhol}

\noindent
Por favor, indique o seu nível de conhecimento em Espanhol, a língua nativa das próximas 
5 videoaulas. Ao informar sua proficiência, você nos ajudará a conduzir uma análise mais 
precisa e abrangente sobre a qualidade das transcrições automáticas.

\noindent
\textbf{Qual é o seu nível de proficiência em Espanhol?}

\noindent
As alternativas simplificam o Quadro Europeu Comum de Referência para Línguas (QECR): ``Básico'' (A1 e A2); ``Intermediário'' (B1 e B2) e ``Avançado'' (C1 e C2). Por outro lado, caso não tenha conhecimento algum em Espanhol, selecione a opção "Sem Proficiência" para não avaliar as videoaulas nesse idioma.

\noindent
\textit{[Grupo de Campos para Seleção Única]}

\noindent
\textit{[``Básico'', ``Intermediário'', ``Avançado'' e ``Sem Proficiência (Pular Vídeos em Espanhol)'']}

\subsection{Vídeo 1/5 - O Que é Programação (es-AR)}

\noindent
Transcrições Automáticas da videoaula \url{http://youtube.com/watch?v=EPsKQy73H38}:

\begin{itemize}
    \item Provedor 1: En definitiva, de la programación es la acción de poder escribir diferentes tipos de programas que obviamente van a estar y se van a ejecutar en una computadora con el objetivo de poder resolver una problemática en particular.
    \item Provedor 2: En definitiva la programación es la acción de poder escribir diferentes tipos de programas que obviamente van a estar y se van a ejecutar en una computadora con el objetivo de poder resolver una problemática en particular.
    \item Provedor 3: En definitiva, la programación es la acción de poder escribir diferentes tipos de programa que obviamente van a estar y se van a ejecutar en una computadora con el objetivo de poder resolver una problemática en particular.
    \item Provedor 4: En definitiva. La programación es la acción de poder escribir diferentes tipos de programa que obviamente van a estar y se van a ejecutar en una computadora con el objetivo de poder resolver una problemática en particular.
    \item Provedor 5: En definitiva la programación es la acción de poder escribir diferentes tipos de programa cuya mente van a estar y se van a ejecutar en una computadora con el objetivo de poder resolver una problemática en particular.
\end{itemize}

\noindent
\textbf{Qual é o nível de coerência das transcrições automáticas apresentadas?}

\noindent
\textit{[Grupo de Campos para Seleção Única (Escala) para Cada Provedor]}

\noindent
\textit{[Provedor 1: ``Muito Incoerente'', ``Incoerente'', ``Útil'', ``Coerente'' e ``Muito Coerente'']}

\noindent
\textit{[Provedor 2: ``Muito Incoerente'', ``Incoerente'', ``Útil'', ``Coerente'' e ``Muito Coerente'']}

\noindent
\textit{[Provedor 3: ``Muito Incoerente'', ``Incoerente'', ``Útil'', ``Coerente'' e ``Muito Coerente'']}

\noindent
\textit{[Provedor 4: ``Muito Incoerente'', ``Incoerente'', ``Útil'', ``Coerente'' e ``Muito Coerente'']}

\noindent
\textit{[Provedor 5: ``Muito Incoerente'', ``Incoerente'', ``Útil'', ``Coerente'' e ``Muito Coerente'']}

\subsection{Vídeo 2/5 - Linguagens de Programação (es-AR)}

\noindent
Transcrições Automáticas da videoaula \url{http://youtube.com/watch?v=y_lMhROqOR0}:

\begin{itemize}
    \item Provedor 1: Entonces existen muchísimos lenguajes de programación como Python comorco-mo.Net como Java como PHP. Todos tienen la misma lógica del del, desde la mirada de que tiene sus reglas, sus normas. Por eso es que 1 siempre habla de la importancia de aprender a programar y desarrollar la lógica de la programación.
    \item Provedor 2: Entonces existen muchísimos lenguajes de programación como piton como r como punto net como java como p p todos tienen la misma lógica del del desde la mirada de que tiene sus reglas su normas por eso es que uno siempre abra de la importancia de aprender a programar y desarrosar la lógica en la programación.
    \item Provedor 3: Entonces, existen muchísimos lenguajes de programación, como Python, como R, como .NET, como Java, como PHP. Todos tienen la misma lógica desde la mirada de que tiene sus reglas, sus normas. Por eso es que uno siempre habla de la importancia de aprender a programar y desarrollar la lógica de la programación.
    \item Provedor 4: Entonces existen muchísimos lenguajes de programación, como Payton, como R, como punto net como Java, como PHP, todos tienen la misma lógica del del desde la mirada de que tiene sus reglas, normas. Por eso es que uno siempre habla de la importancia de aprender a programar y desarrollar la lógica de la programação.
    \item Provedor 5: Entonces existen muchísimos lenguajes de programación como python como r como puntonet.Com hoja va como php todos tienen la mesma lógica de la mirada de que tiene sus reglas o normas por eso es que uno siempre habla de la importancia de aprender a programar y desarrollar la lógica en la programación.
\end{itemize}

\noindent
\textbf{Qual é o nível de coerência das transcrições automáticas apresentadas?}

\noindent
\textit{[Grupo de Campos para Seleção Única (Escala) para Cada Provedor]}

\noindent
\textit{[Provedor 1: ``Muito Incoerente'', ``Incoerente'', ``Útil'', ``Coerente'' e ``Muito Coerente'']}

\noindent
\textit{[Provedor 2: ``Muito Incoerente'', ``Incoerente'', ``Útil'', ``Coerente'' e ``Muito Coerente'']}

\noindent
\textit{[Provedor 3: ``Muito Incoerente'', ``Incoerente'', ``Útil'', ``Coerente'' e ``Muito Coerente'']}

\noindent
\textit{[Provedor 4: ``Muito Incoerente'', ``Incoerente'', ``Útil'', ``Coerente'' e ``Muito Coerente'']}

\noindent
\textit{[Provedor 5: ``Muito Incoerente'', ``Incoerente'', ``Útil'', ``Coerente'' e ``Muito Coerente'']}

\subsection{Vídeo 3/5 - Tipos de Dados em Python (es-AR)}

\noindent
Transcrições Automáticas da videoaula \url{http://youtube.com/watch?v=V98YCkiTULY}:

\begin{itemize}
    \item Provedor 1: Ahora si hablamos de los tipos de datos que existen, es importante tener en cuenta de que hay diferentes variedades Python, como cualquier lenguaje de programación, nos ofrece muchos tipos de datos con los cuales podemos trabajar.
    \item Provedor 2: Ahora si hablamos de los tipos de datos que existen es importante tener en cuenta de que hay diferentes variedades. Payton como cualquier lenguaje de programación nos ofrece muchos tipos de datos con los cuales podemos trabajar.
    \item Provedor 3: Ahora, si hablamos de los tipos de datos que existen, es importante tener en cuenta de que hay diferentes variedades. Python, como cualquier lenguaje de programación, nos oferece muchos tipos de datos con los cuales podemos trabajar.
    \item Provedor 4: Ahora. Si hablamos de los tipos de datos que existen, es importante tener en cuenta de que hay diferentes variedades Payton. Como cualquier lenguaje de programación, nos ofrece muchos tipos de datos con los cuales podemos trabajar.
    \item Provedor 5: Ahora sí Hablamos de los tipos de datos que existen es importante tener en cuenta de que hay diferentes variedades python como cualquier lenguaje de programación nos ofrece muchos tipos de datos con los cuales podemos trabajar.
\end{itemize}

\noindent
\textbf{Qual é o nível de coerência das transcrições automáticas apresentadas?}

\noindent
\textit{[Grupo de Campos para Seleção Única (Escala) para Cada Provedor]}

\noindent
\textit{[Provedor 1: ``Muito Incoerente'', ``Incoerente'', ``Útil'', ``Coerente'' e ``Muito Coerente'']}

\noindent
\textit{[Provedor 2: ``Muito Incoerente'', ``Incoerente'', ``Útil'', ``Coerente'' e ``Muito Coerente'']}

\noindent
\textit{[Provedor 3: ``Muito Incoerente'', ``Incoerente'', ``Útil'', ``Coerente'' e ``Muito Coerente'']}

\noindent
\textit{[Provedor 4: ``Muito Incoerente'', ``Incoerente'', ``Útil'', ``Coerente'' e ``Muito Coerente'']}

\noindent
\textit{[Provedor 5: ``Muito Incoerente'', ``Incoerente'', ``Útil'', ``Coerente'' e ``Muito Coerente'']}

\subsection{Vídeo 4/5 - "Olá Mundo" e 'Olá Mundo' em Python (es-AR)}

\noindent
Transcrições Automáticas da videoaula \url{http://youtube.com/watch?v=Ctv4to22eLY}:

\begin{itemize}
    \item Provedor 1: Ejecutamos hola mundo hola mundo. Esto es lo que nos lleva a pensar es que no hay inconvenientes con trabajar con comillas simples o dobles en Python, bien, pero recordemos que hay otros lenguajes de programación que esto sí que puede llegar a traer inconvenientes.
    \item Provedor 2: Ejecutamos o la mundo y hola. Esto lo que no se va a pensar es que no hay inconvenientes con trabajar con comisas simples o dobles en python bien pero recordemos que hay otros lenguajes de programación que esto si que puedes llevar a traer inconvenientes.
    \item Provedor 3: Ejecutamos hola mundo y hola mundo. Esto lo que nos lleva a pensar es que no hay inconvenientes con trabajar con comillas simples o dobles en Python, pero recordemos que hay otro lenguaje de programación que esto sí que puede llegar a traer inconvenientes.
    \item Provedor 4: Ejecutamos. Hola, mundo. Hola, mundo. Esto lo que nos lleva a pensar es que no hay inconvenientes con trabajar con comillas simples o dobles en Brighton. Bien, pero recordemos que hay otro lenguaje de programación que esto sí que puede llegar a traer inconvenientes.
    \item Provedor 5: Ejecutamos Hola mundo y Hola mundo Esto es lo que nos lleva a pensar es que no hay inconvenientes con trabajar con comillas simples o dobles en python meme pero recordemos que hay otro lenguaje de programación que estos y que puede llegar a traer inconvenientes.
\end{itemize}

\noindent
\textbf{Qual é o nível de coerência das transcrições automáticas apresentadas?}

\noindent
\textit{[Grupo de Campos para Seleção Única (Escala) para Cada Provedor]}

\noindent
\textit{[Provedor 1: ``Muito Incoerente'', ``Incoerente'', ``Útil'', ``Coerente'' e ``Muito Coerente'']}

\noindent
\textit{[Provedor 2: ``Muito Incoerente'', ``Incoerente'', ``Útil'', ``Coerente'' e ``Muito Coerente'']}

\noindent
\textit{[Provedor 3: ``Muito Incoerente'', ``Incoerente'', ``Útil'', ``Coerente'' e ``Muito Coerente'']}

\noindent
\textit{[Provedor 4: ``Muito Incoerente'', ``Incoerente'', ``Útil'', ``Coerente'' e ``Muito Coerente'']}

\noindent
\textit{[Provedor 5: ``Muito Incoerente'', ``Incoerente'', ``Útil'', ``Coerente'' e ``Muito Coerente'']}

\subsection{Vídeo 5/5 - String Slicing em Python (es-AR)}

\noindent
Transcrições Automáticas da videoaula \url{http://youtube.com/watch?v=iKKu8TFtLZY}:

\begin{itemize}
    \item Provedor 1: Y luego vamos a hacer un print, observemos da. Y, vamos a ver primero la posición cero. Y, cómo podemos identificar la posición cero es la ache. Siempre recordemos que la posición inicial en Python, todos los índices inician en cero. Ahora si nosotros ponemos la posición 1, que era el ejemplo que estaba propuesto en el slide, efectivamente visualizamos que es. El elemento de.
    \item Provedor 2: Y luego vamos a hacer un print observemos de a. Y vamos a ver primero la posición cero y como podemos identificar la posición cero es la h siempre recordemos que la posición inicial en payton todos los índices inician en cero. Ahora si nosotros ponemos la posición uno que era el ejemplo que estaba propuesto nerslife efectivamente visualizamos que es. El elemento de.
    \item Provedor 3: Y luego vamos a hacer un print, observemos, de A, y vamos a ver primero la posición cero, y como podemos identificar, la posición cero es la H. Siempre recordemos que la posición inicial en Python, todos los índices inician en cero. Ahora, si nosotros ponemos la posición uno, que era el ejemplo que estaba propuesto en el slide, efectivamente visualizamos que es el elemento D.
    \item Provedor 4: Y luego vamos a hacer un brindis. Observemos de A y vamos a ver primero la posición cero. Y cómo podemos identificar la posición cero? Es la H. Siempre recordemos que la posición inicial en Payton todos los índices inician en cero. Ahora, si nosotros ponemos la posición uno, que era el ejemplo que estaba propuesto en el isla, hay efectivamente, visualizamos que es el elemento de.
    \item Provedor 5: Y luego vamos a hacer un print observemos de a y vamos a ver primero la posición 0 y Cómo podemos identificar la posición 0 es la H siempre recordemos que la posición inicial en python todos los índices inician en ceros ahora si nosotros ponemos la posición uno que era el ejemplo que estaba propuesto en el slime efectivamente visualizamos Qué es el elemento de.
\end{itemize}

\noindent
\textbf{Qual é o nível de coerência das transcrições automáticas apresentadas?}

\noindent
\textit{[Grupo de Campos para Seleção Única (Escala) para Cada Provedor]}

\noindent
\textit{[Provedor 1: ``Muito Incoerente'', ``Incoerente'', ``Útil'', ``Coerente'' e ``Muito Coerente'']}

\noindent
\textit{[Provedor 2: ``Muito Incoerente'', ``Incoerente'', ``Útil'', ``Coerente'' e ``Muito Coerente'']}

\noindent
\textit{[Provedor 3: ``Muito Incoerente'', ``Incoerente'', ``Útil'', ``Coerente'' e ``Muito Coerente'']}

\noindent
\textit{[Provedor 4: ``Muito Incoerente'', ``Incoerente'', ``Útil'', ``Coerente'' e ``Muito Coerente'']}

\noindent
\textit{[Provedor 5: ``Muito Incoerente'', ``Incoerente'', ``Útil'', ``Coerente'' e ``Muito Coerente'']}

\subsection{NPS}

\noindent
O NPS é uma métrica de satisfação que utiliza uma escala de 0 a 10, onde: 

\begin{itemize}
    \item Notas 9 e 10 são promotoras e indicam um alto nível de satisfação; 
    \item Notas 7 e 8 são neutras e indicam satisfação moderada e sem entusiasmo; 
    \item Notas de 0 a 6 são detratoras e indicam insatisfação.
\end{itemize}

\noindent
\textbf{Considerando as transcrições dos vídeos em Espanhol, o quão satisfeito você está com os provedores de reconhecimento de fala?}

\noindent
\textit{[Grupo de Campos para Seleção Única (Escala de 0 a 10)]}

\noindent
\textbf{Se desejar, compartilhe observações, sugestões ou feedbacks sobre as transcrições automáticas dos vídeos em Espanhol.}

\noindent
\textit{[Campo de Texto Longo]}

\section{Conclusão}

\noindent
Agradecemos por ter chegado até aqui e por dedicar seu tempo a participar deste estudo. 
Você está na última seção e há apenas mais duas perguntas a serem respondidas. Sua 
opinião é fundamental para nós, e valorizamos muito suas contribuições.

\noindent
\textbf{Por favor, compartilhe suas observações, sugestões ou feedbacks sobre este
estudo. Sua percepção é fundamental para aperfeiçoarmos nossas futuras
avaliações.}

\noindent
\textit{[Campo de Texto Longo]}

\noindent
\textbf{Se deseja ser informado(a) sobre iniciativas futuras relacionadas a nossa
pesquisa, por gentileza, deixe seu e-mail e/ou telefone abaixo.}

\noindent
\textit{[Campo de Texto Longo]}

\noindent
Por fim, se você tem interesse em arquitetura de software e deseja acompanhar e participar 
das discussões sobre o progresso desta pesquisa, convidamos você a se juntar ao nosso 
grupo no WhatsApp: \url{https://chat.whatsapp.com/FwOiBx1U3u3BbQ51LxXQnM}

\noindent
Muito obrigado e até breve!

\chapter{Survey Avatares de Libras}
\label{appendix:libras-survey}
\section{Avaliação de Avatares de Libras Integrados a Transcrições Automáticas}

\subsection{Termo de Consentimento Livre e Esclarecido}

Prezado(a) participante: Esta pesquisa é realizada de acordo com as recomendações 
estabelecidas pelo Comitê de Ética da Universidade de São Paulo (USP). Em atendimento às 
normas desse Comitê de Ética e orientações científicas, pedimos que registre sua 
concordância na participação da pesquisa no campo abaixo.

OBSERVAÇÃO: O Comitê de Ética em Pesquisa (CEP/EACH) funciona na Av. Arlindo Béttio, 
1000, Ermelino Matarazzo, São Paulo-SP, tel: (11) 3091-1046, e-mail: \href{mailto:cep-each@usp.br}{cep-each@usp.br}.

Ressaltamos que na divulgação dos resultados desta pesquisa, a identidade dos participantes 
será mantida no mais rigoroso sigilo. Se precisar de mais informações sobre sua participação 
ou sobre a pesquisa, faça contato para esclarecimentos: 

Contato: Venilton FalvoJr \href{mailto:falvojr@usp.br}{falvojr@usp.br}

\subsection{Prezado(a) intérprete de Libras}

Você está sendo convidado(a) a participar deste estudo que visa avaliar a 
qualidade/coerência dos avatares de Libras Hand Talk e VLibras quando integrados a 
transcrições e legendas automáticas de videoaulas.

Para isso, utilizamos o Whisper, um serviço de reconhecimento de fala da OpenAI, 
implementado em uma instância da Arquitetura Speech2Learning, para transcrever uma 
videoaula fornecida pela EdTech DIO. Todo esse processo foi detalhado em nosso artigo 
publicado no HICSS-57 no início deste ano.

Nosso objetivo com esta pesquisa é apoiar e valorizar o trabalho essencial dos 
intérpretes de Libras, não substituí-los. Estamos dedicados a desenvolver soluções que 
ampliem a acessibilidade e facilitem o acesso ao conhecimento para toda a comunidade 
de usuários da Libras.

Sua participação é fundamental para respondermos à seguinte questão de pesquisa: 
“Como as tecnologias de reconhecimento de fala (ASR/STT) podem contribuir para uma 
educação mais acessível para usuários da Libras?”

Agradecemos antecipadamente pela sua colaboração e participação. Se possível, 
compartilhe este estudo com outros intérpretes de Libras.

\subsection{Informações Pessoais}

\begin{itemize}
    \item Qual é o seu nome?
    \item Há quantos anos você atua como intérprete de Libras?
    \item Em quais ambientes você já trabalhou como intérprete de Libras? Por favor, forneça detalhes sobre os ambientes nos quais você atuou como intérprete de Libras. Isso pode incluir, por exemplo, aulas de ensino fundamental em escolas municipais, aulas de pedagogia em universidades privadas ou lives sobre Tecnologia da Informação (TI) no YouTube.
\end{itemize}

\subsection{Avaliação dos Avatares de Libras}

Nesta seção, selecionamos e transcrevemos alguns trechos de uma videoaula da DIO, 
visando suas respectivas sinalizações para a avaliação dos avatares de línguas de sinais 
Hand Talk e VLibras. Essas transcrições foram geradas automaticamente, sem 
intervenção humana. Com isso, esperamos compreender se um serviço de 
reconhecimento de fala preciso, como o Whisper, pode transcender sua qualidade para os 
avatares de Libras.

ATENÇÃO: Se você não estiver familiarizado com os avatares mencionados ou se não 
tiver tido contato com eles recentemente, por favor, dedique alguns minutos para visitar a 
página principal de cada um deles e explorar suas configurações e funcionalidades:

\begin{itemize}
    \item Hand Talk: \url{https://handtalk.me}
    \item VLibras: \url{https://gov.br/governodigital/pt-br/vlibras}
\end{itemize}

\subsection{Escala de Avaliação de Qualidade}

A qualidade da sinalização dos avatares será avaliada utilizando uma escala de 5 
alternativas, as quais serão individualmente definidas a fim de minimizar possíveis 
interpretações incorretas:

\begin{enumerate}
    \item Muito Incoerente: Sinalização sem sentido, tornando a compreensão impossível;
    \item Incoerente: Muitos erros, permitindo apenas uma compreensão parcial;
    \item Útil: Erros presentes, mas sem impedir a compreensão geral;
    \item Coerente: Pequenas imprecisões, mas sem comprometer a compreensão total;
    \item Muito Coerente: Sinalização perfeita, garantindo uma compreensão completa.
\end{enumerate}

\subsection{Transcrição Automática Para Sinalização}

\begin{quote}
“Vocês devem estar se perguntando, mas afinal, o que são IAs generativas? Essa é uma 
excelente pergunta!
IAs generativas são sistemas capazes de criar, adaptar e aprimorar conteúdos de maneira 
autônoma, sempre aprendendo e se aperfeiçoando.
Pensem em um mecanismo capaz de gerar imagens, músicas, textos, vozes e até avatares 
virtuais, como o que vocês veem agora. Tudo isso com uma qualidade que se assemelha ao 
trabalho humano.
Aqui na D.I.O. estamos explorando estas tecnologias para oferecer uma aprendizagem mais 
dinâmica, imersiva e, especialmente, acessível.
Estão prontos? Vamos juntos nessa aventura!”
\end{quote}

A videoaula completa dos trechos transcritos acima pode ser acessada em: 
Revolução das IAs Generativas.

\url{https://handtalk.me/} \\
\url{https://gov.br/governodigital/pt-br/vlibras} \\
\url{https://www.youtube.com/watch?v=FDMCF285vt8}

\subsection{Resultado da Sinalização com Hand Talk}

\url{http://youtube.com/watch?v=1But8SvOv7Q}

\subsection{Resultado da Sinalização com VLibras}

\url{http://youtube.com/watch?v=Xuh7RuuzMYE}

\subsection{Agendamento da Entrevista}

Agora, como última etapa deste formulário, vamos entender como será nossa 
entrevista. As entrevistas têm como objetivo apresentar um Player de Vídeo 
desenvolvido sob o conceito de Desenho Universal. Esse conceito defende a concepção 
de produtos, ambientes, programas e serviços que possam ser usados por todas as 
pessoas, sem exceção. Portanto, essa solução é independente dos avatares de Libras, 
mas está estruturada para integração com esse tipo de solução.

Nossa entrevista se baseará nas suas respostas deste formulário, com foco no Player de 
Vídeo integrado à transcrição automática da mesma videoaula que você avaliou aqui. 
Queremos entender se essa iniciativa pode ajudar a democratizar o acesso a conteúdos 
educacionais audíveis (como audiobooks, podcasts e videoaulas).

Por favor, compartilhe suas observações, sugestões ou percepções sobre
a experiência de uso de Avatares de Libras em transcrições automáticas.

\subsection{Preparação para a Entrevista}

Para a entrevista (que você agendará a seguir na data e hora que preferir), é importante que 
você acesse previamente o Player de Vídeo através do seguinte 
link: \url{https://falvojr.github.io/speech2learning/player}. Esta versão está integrada com o 
VLibras, mas o avatar de Libras será irrelevante para a entrevista, que tem foco no player e 
seus conceitos correlacionados.

A entrevista terá duração máxima de 10 minutos e abordará as seguintes questões:

\begin{enumerate}
    \item Como você avalia o Player de Vídeo sob as perspectivas de Acessibilidade, Usabilidade e Design Universal?
    \item Qual é o impacto de iniciativas como este Player de Vídeo no processo de ensino-aprendizagem para usuários das Línguas de Sinais, em especial os surdos?
    \item Considerando o Player de Vídeo, o quão satisfeito você está com o potencial dessa Tecnologia Assistiva para usuários das Línguas de Sinais? Dê uma nota de 0 a 10 (NPS) e justifique sua resposta, por favor.
\end{enumerate}

\subsection{Agende Sua Entrevista Agora}

IMPORTANTE: Por favor, agende sua entrevista através do seguinte link. Escolha o melhor 
dia e horário para que possamos conversar por no máximo 10 minutos: 
\url{https://calendar.app.google/e5SCuVqSWmtRwiJ7A}

\subsection{Agradecimento}

Agradecemos sua participação e colaboração neste estudo. Se precisar de qualquer 
assistência ou tiver dúvidas, sinta-se à vontade para entrar em contato conosco através do e-mail \href{mailto:falvojr@usp.br}{falvojr@usp.br} ou pelo WhatsApp (16) 99721-8281. 

\chapter{Entrevista Intérpretes de Libras}
\label{appendix:libras-interview}
\section{Introdução}

\noindent
\textbf{Objetivo da Entrevista:} Avaliar uma instância da Arquitetura \textit{Speech2Learning}, especificamente um Player de Vídeo acessível integrado a avatares de Libras, sob as perspectivas de usabilidade, eficácia e contribuição para o aprendizado.

\noindent
\textbf{Método de Elaboração:} As perguntas foram formuladas com base nos requisitos pedagógicos do \textit{ReqML-Catalog} \cite{Soad2017_FIE}, um catálogo focado em aplicações Mobile, mas que possui muitos requisitos genéricos relevantes neste contexto.

\noindent
\textbf{Resultados Esperados:} Percepções sobre a eficácia do Player de Vídeo em termos de acessibilidade, usabilidade e impacto no processo de ensino-aprendizagem para surdos.

\noindent
\textbf{Pré-requisitos:} Para a entrevista, é importante que você acesse previamente o Player de Vídeo através do seguinte link: \url{https://falvojr.github.io/speech2learning/player}. Esta versão está integrada com o VLibras, mas o avatar de Libras será irrelevante para a entrevista, que tem foco no player e seus conceitos correlacionados.

\noindent
\textbf{Duração:} 15 minutos (em média)

\section{Questões da Entrevista}

\begin{enumerate}
\item \textbf{Como você avalia o Player de Vídeo sob as perspectivas de Acessibilidade, Usabilidade e Design Universal?}
\begin{itemize}
    \item \textbf{Acessibilidade:} A qualidade do acesso para qualquer pessoa.
    \item \textbf{Usabilidade:} A facilidade de uso para diferentes perfis de usuários.
    \item \textbf{Design Universal:} A capacidade de ser usado por qualquer pessoa.
\end{itemize}

\item \textbf{Qual é o impacto de iniciativas como este Player de Vídeo no processo de ensino-aprendizagem para usuários das Línguas de Sinais, em especial os surdos?}

\begin{itemize}
    \item Considere a relevância das transcrições/legendas automáticas e sua integração com os Avatares de Libras.
\end{itemize}

\item \textbf{Considerando o Player de Vídeo, o quão satisfeito você está com o potencial dessa Tecnologia Assistiva para usuários das Línguas de Sinais? Dê uma nota de 0 a 10 (NPS) e justifique sua resposta, por favor.}

\begin{itemize}
    \item \textbf{Notas 9 e 10:} Promotoras, indicando um alto nível de satisfação.
    \item \textbf{Notas 7 e 8:} Neutras, indicando satisfação moderada e sem entusiasmo.
    \item \textbf{Notas de 0 a 6:} Detratoras, indicando insatisfação.
\end{itemize}
\end{enumerate}

\section{Conclusão}

\noindent
Agradecemos sua participação e colaboração neste estudo. Se precisar de qualquer assistência ou tiver dúvidas, sinta-se à vontade para entrar em contato conosco através do e-mail \url{falvojr@usp.br} ou pelo WhatsApp (16) 99721-8281.

\end{apendicesenv}
% ---


% ----------------------------------------------------------
% Anexos
% ----------------------------------------------------------

% ---
% Inicia os anexos
% ---
% \begin{anexosenv}

%     \chapter{Páginas interessantes na Internet} 
%     \label{chapter:paginas-interessantes}
%     \input{tex/annex/paginas-interessantes}

% \end{anexosenv}
% ---

\end{document}