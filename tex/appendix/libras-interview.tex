\noindent
\textbf{Objetivo da Entrevista:} Avaliar a instância do \textit{Speech2Learning}, especificamente um Player de Vídeo acessível integrado a avatares de Libras, sob as perspectivas de usabilidade, eficácia e contribuição para o aprendizado.

\noindent
\textbf{Método de Elaboração:} As perguntas foram formuladas com base nos requisitos pedagógicos do \textit{ReqML-Catalog} \cite{Soad2017_FIE}, um catálogo focado em aplicações Mobile, mas que possui muitos requisitos genéricos relevantes neste contexto.

\noindent
\textbf{Resultados Esperados:} Insights sobre a eficácia do Player de Vídeo em termos de acessibilidade, usabilidade e impacto no processo de ensino-aprendizagem para surdos.

\noindent
\textbf{Pré-requisitos:} Para a entrevista, é importante que você acesse previamente o Player de Vídeo através do seguinte link: \url{https://falvojr.github.io/speech2learning/player}. Esta versão está integrada com o VLibras, mas o avatar de Libras será irrelevante para a entrevista, que tem foco no player e seus conceitos correlacionados.

\noindent
\textbf{Duração:} 10 minutos (em média)

\subsubsection*{Questões da Entrevista}

\begin{enumerate}
\item \textbf{Como você avalia o Player de Vídeo sob as perspectivas de Acessibilidade, Usabilidade e Design Universal?}
\begin{itemize}
    \item \textbf{Acessibilidade:} A qualidade do acesso para qualquer pessoa.
    \item \textbf{Usabilidade:} A facilidade de uso para diferentes perfis de usuários.
    \item \textbf{Design Universal:} A capacidade de ser usado por qualquer pessoa.
\end{itemize}

\item \textbf{Qual é o impacto de iniciativas como este Player de Vídeo no processo de ensino-aprendizagem para usuários das Línguas de Sinais, em especial os surdos?}

\begin{itemize}
    \item Considere a relevância das transcrições/legendas automáticas e sua integração com os Avatares de Libras.
\end{itemize}

\item \textbf{Considerando o Player de Vídeo, o quão satisfeito você está com o potencial dessa Tecnologia Assistiva para usuários das Línguas de Sinais? Dê uma nota de 0 a 10 (NPS) e justifique sua resposta, por favor.}

\begin{itemize}
    \item \textbf{Notas 9 e 10:} Promotoras, indicando um alto nível de satisfação.
    \item \textbf{Notas 7 e 8:} Neutras, indicando satisfação moderada e sem entusiasmo.
    \item \textbf{Notas de 0 a 6:} Detratoras, indicando insatisfação.
\end{itemize}
\end{enumerate}

Agradecemos sua participação e colaboração neste estudo. Se precisar de qualquer assistência ou tiver dúvidas, sinta-se à vontade para entrar em contato conosco através do e-mail \url{falvojr@usp.br} ou pelo WhatsApp (16) 99721-8281.