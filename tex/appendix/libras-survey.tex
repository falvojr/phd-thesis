\section{Avaliação de Avatares de Libras Integrados a Transcrições Automáticas}

\subsection{Termo de Consentimento Livre e Esclarecido}

Prezado(a) participante: Esta pesquisa é realizada de acordo com as recomendações 
estabelecidas pelo Comitê de Ética da Universidade de São Paulo (USP). Em atendimento às 
normas desse Comitê de Ética e orientações científicas, pedimos que registre sua 
concordância na participação da pesquisa no campo abaixo.

OBSERVAÇÃO: O Comitê de Ética em Pesquisa (CEP/EACH) funciona na Av. Arlindo Béttio, 
1000, Ermelino Matarazzo, São Paulo-SP, tel: (11) 3091-1046, e-mail: \href{mailto:cep-each@usp.br}{cep-each@usp.br}.

Ressaltamos que na divulgação dos resultados desta pesquisa, a identidade dos participantes 
será mantida no mais rigoroso sigilo. Se precisar de mais informações sobre sua participação 
ou sobre a pesquisa, faça contato para esclarecimentos: 

Contato: Venilton FalvoJr \href{mailto:falvojr@usp.br}{falvojr@usp.br}

\subsection{Prezado(a) intérprete de Libras}

Você está sendo convidado(a) a participar deste estudo que visa avaliar a 
qualidade/coerência dos avatares de Libras Hand Talk e VLibras quando integrados a 
transcrições e legendas automáticas de videoaulas.

Para isso, utilizamos o Whisper, um serviço de reconhecimento de fala da OpenAI, 
implementado em uma instância da Arquitetura Speech2Learning, para transcrever uma 
videoaula fornecida pela EdTech DIO. Todo esse processo foi detalhado em nosso artigo 
publicado no HICSS-57 no início deste ano.

Nosso objetivo com esta pesquisa é apoiar e valorizar o trabalho essencial dos 
intérpretes de Libras, não substituí-los. Estamos dedicados a desenvolver soluções que 
ampliem a acessibilidade e facilitem o acesso ao conhecimento para toda a comunidade 
de usuários da Libras.

Sua participação é fundamental para respondermos à seguinte questão de pesquisa: 
“Como as tecnologias de reconhecimento de fala (ASR/STT) podem contribuir para uma 
educação mais acessível para usuários da Libras?”

Agradecemos antecipadamente pela sua colaboração e participação. Se possível, 
compartilhe este estudo com outros intérpretes de Libras.

\subsection{Informações Pessoais}

\begin{itemize}
    \item Qual é o seu nome?
    \item Há quantos anos você atua como intérprete de Libras?
    \item Em quais ambientes você já trabalhou como intérprete de Libras? Por favor, forneça detalhes sobre os ambientes nos quais você atuou como intérprete de Libras. Isso pode incluir, por exemplo, aulas de ensino fundamental em escolas municipais, aulas de pedagogia em universidades privadas ou lives sobre Tecnologia da Informação (TI) no YouTube.
\end{itemize}

\subsection{Avaliação dos Avatares de Libras}

Nesta seção, selecionamos e transcrevemos alguns trechos de uma videoaula da DIO, 
visando suas respectivas sinalizações para a avaliação dos avatares de línguas de sinais 
Hand Talk e VLibras. Essas transcrições foram geradas automaticamente, sem 
intervenção humana. Com isso, esperamos compreender se um serviço de 
reconhecimento de fala preciso, como o Whisper, pode transcender sua qualidade para os 
avatares de Libras.

ATENÇÃO: Se você não estiver familiarizado com os avatares mencionados ou se não 
tiver tido contato com eles recentemente, por favor, dedique alguns minutos para visitar a 
página principal de cada um deles e explorar suas configurações e funcionalidades:

\begin{itemize}
    \item Hand Talk: \url{https://handtalk.me}
    \item VLibras: \url{https://gov.br/governodigital/pt-br/vlibras}
\end{itemize}

\subsection{Escala de Avaliação de Qualidade}

A qualidade da sinalização dos avatares será avaliada utilizando uma escala de 5 
alternativas, as quais serão individualmente definidas a fim de minimizar possíveis 
interpretações incorretas:

\begin{enumerate}
    \item Muito Incoerente: Sinalização sem sentido, tornando a compreensão impossível;
    \item Incoerente: Muitos erros, permitindo apenas uma compreensão parcial;
    \item Útil: Erros presentes, mas sem impedir a compreensão geral;
    \item Coerente: Pequenas imprecisões, mas sem comprometer a compreensão total;
    \item Muito Coerente: Sinalização perfeita, garantindo uma compreensão completa.
\end{enumerate}

\subsection{Transcrição Automática Para Sinalização}

\begin{quote}
“Vocês devem estar se perguntando, mas afinal, o que são IAs generativas? Essa é uma 
excelente pergunta!
IAs generativas são sistemas capazes de criar, adaptar e aprimorar conteúdos de maneira 
autônoma, sempre aprendendo e se aperfeiçoando.
Pensem em um mecanismo capaz de gerar imagens, músicas, textos, vozes e até avatares 
virtuais, como o que vocês veem agora. Tudo isso com uma qualidade que se assemelha ao 
trabalho humano.
Aqui na D.I.O. estamos explorando estas tecnologias para oferecer uma aprendizagem mais 
dinâmica, imersiva e, especialmente, acessível.
Estão prontos? Vamos juntos nessa aventura!”
\end{quote}

A videoaula completa dos trechos transcritos acima pode ser acessada em: 
Revolução das IAs Generativas.

\url{https://handtalk.me/} \\
\url{https://gov.br/governodigital/pt-br/vlibras} \\
\url{https://www.youtube.com/watch?v=FDMCF285vt8}

\subsection{Resultado da Sinalização com Hand Talk}

\url{http://youtube.com/watch?v=1But8SvOv7Q}

\subsection{Resultado da Sinalização com VLibras}

\url{http://youtube.com/watch?v=Xuh7RuuzMYE}

\subsection{Agendamento da Entrevista}

Agora, como última etapa deste formulário, vamos entender como será nossa 
entrevista. As entrevistas têm como objetivo apresentar um Player de Vídeo 
desenvolvido sob o conceito de Desenho Universal. Esse conceito defende a concepção 
de produtos, ambientes, programas e serviços que possam ser usados por todas as 
pessoas, sem exceção. Portanto, essa solução é independente dos avatares de Libras, 
mas está estruturada para integração com esse tipo de solução.

Nossa entrevista se baseará nas suas respostas deste formulário, com foco no Player de 
Vídeo integrado à transcrição automática da mesma videoaula que você avaliou aqui. 
Queremos entender se essa iniciativa pode ajudar a democratizar o acesso a conteúdos 
educacionais audíveis (como audiobooks, podcasts e videoaulas).

Por favor, compartilhe suas observações, sugestões ou percepções sobre
a experiência de uso de Avatares de Libras em transcrições automáticas.

\subsection{Preparação para a Entrevista}

Para a entrevista (que você agendará a seguir na data e hora que preferir), é importante que 
você acesse previamente o Player de Vídeo através do seguinte 
link: \url{https://falvojr.github.io/speech2learning/player}. Esta versão está integrada com o 
VLibras, mas o avatar de Libras será irrelevante para a entrevista, que tem foco no player e 
seus conceitos correlacionados.

A entrevista terá duração máxima de 10 minutos e abordará as seguintes questões:

\begin{enumerate}
    \item Como você avalia o Player de Vídeo sob as perspectivas de Acessibilidade, Usabilidade e Design Universal?
    \item Qual é o impacto de iniciativas como este Player de Vídeo no processo de ensino-aprendizagem para usuários das Línguas de Sinais, em especial os surdos?
    \item Considerando o Player de Vídeo, o quão satisfeito você está com o potencial dessa Tecnologia Assistiva para usuários das Línguas de Sinais? Dê uma nota de 0 a 10 (NPS) e justifique sua resposta, por favor.
\end{enumerate}

\subsection{Agende Sua Entrevista Agora}

IMPORTANTE: Por favor, agende sua entrevista através do seguinte link. Escolha o melhor 
dia e horário para que possamos conversar por no máximo 10 minutos: 
\url{https://calendar.app.google/e5SCuVqSWmtRwiJ7A}

\subsection{Agradecimento}

Agradecemos sua participação e colaboração neste estudo. Se precisar de qualquer 
assistência ou tiver dúvidas, sinta-se à vontade para entrar em contato conosco através do e-mail \href{mailto:falvojr@usp.br}{falvojr@usp.br} ou pelo WhatsApp (16) 99721-8281. 