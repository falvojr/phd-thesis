% ---
% Agradecimentos
% ---
Primeiramente, agradeço a Deus por me dar força e perseverança necessárias para enfrentar os desafios ao longo desta jornada acadêmica.

Um agradecimento especial à minha esposa, Catherine, por sua presença constante, amor e compreensão, sem os quais esta jornada teria sido muito mais difícil. À minha mãe, Marilda, por toda a vida de ensinamentos e pelo apoio incondicional.

À minha orientadora Ellen, sou grato por seus ensinamentos, conselhos e palavras de incentivo. Agradeço por ter acreditado em meu potencial e por tratar-me como um filho.

Expresso minha gratidão à DIO\footnote{\url{https://dio.me}}, não apenas como meu local de trabalho, mas como uma comunidade que apoiou fervorosamente esta pesquisa, proporcionando recursos e um ambiente rico em conhecimento e inovação. Agradeço também à QS Inclusão\footnote{\url{https://qsinclusao.com.br}}, por abrir portas para um entendimento mais profundo de Libras e suas implicações na inclusão e comunicação.

Meus sinceros agradecimentos a todos os professores, funcionários, alunos e colegas do Instituto de Ciências Matemáticas e de Computação (ICMC). Em especial, agradeço ao Laboratório de Engenharia de Software (LabES) e ao Laboratório de Computação Aplicada à Educação e Tecnologia Social Avançada (CAEd), que proporcionaram um ambiente de pesquisa e desenvolvimento essencial para a realização deste trabalho.

Finalmente, aos meus amigos, por todos os momentos de descontração que foram essenciais para manter o equilíbrio durante os períodos de estresse e trabalho intenso.
