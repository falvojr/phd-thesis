% ---
% Agradecimentos
% ---
Primeiramente, agradeço a Deus por me conceder forças e perseverança ao longo deste doutorado, especialmente em tempos de incerteza, como durante a pandemia, em que a fé se mostrou ainda mais essencial.

Um agradecimento especial à minha esposa, Catherine, por seu amor inabalável, motivação constante e por segurar minha mão nos momentos de desespero e dúvida, permitindo que eu seguisse em frente mesmo nos dias mais difíceis. À minha mãe, Marilda, e a toda a minha família, por uma vida de ensinamentos e por moldarem o homem que sou hoje.

À minha orientadora, Ellen, sou profundamente grato por seus ensinamentos, conselhos e palavras de incentivo. Agradeço por acreditar no meu potencial desde a nossa primeira conversa e por me acolher como um filho ao longo desta jornada.

Expresso minha gratidão à DIO\footnote{\url{https://dio.me}}, não apenas como meu local de trabalho, mas como uma comunidade que apoiou fervorosamente esta pesquisa, proporcionando recursos e um ambiente rico em conhecimento e inovação. Agradeço também à QS Inclusão\footnote{\url{https://qsinclusao.com.br}}, por abrir portas para um entendimento mais profundo de Libras e suas implicações na inclusão e comunicação.

Meus sinceros agradecimentos a todos os professores, funcionários, alunos e colegas do Instituto de Ciências Matemáticas e de Computação (ICMC). Em especial, agradeço ao Laboratório de Engenharia de Software (LabES) e ao Laboratório de Computação Aplicada à Educação e Tecnologia Social Avançada (CAEd), que proporcionaram um ambiente de colaboração e desenvolvimento essencial para a realização deste trabalho.

Gostaria de agradecer também às agências de fomento brasileiras: FAPESP (\#2018/26636-2), CAPES (código financeiro 001) e CNPq, pelo suporte imprescindível ao longo desta pesquisa.

Finalmente, aos meus amigos, por todos os momentos de descontração, que foram essenciais para manter o equilíbrio emocional durante os períodos de estresse e trabalho intenso.
